\documentclass[a4paper,10pt]{scrartcl}
\usepackage[utf8x]{inputenc}
\usepackage[T1]{fontenc}
\usepackage{amsmath,amsfonts,amssymb,amscd,amsthm,xspace}
\usepackage[ngerman]{babel}
\usepackage{listingsutf8}
\usepackage{color}
\usepackage{geometry}
\usepackage{graphicx}
\usepackage{multicol}
\geometry{a4paper, left=2cm,right=2cm,top=2cm,bottom=2cm}

\newcommand{\Authors}{Christian Cikryt (Tut. Di. 14-16), Jakob Pfender (Mi. 14-16)}
\title{H\"ohere Algorithmik - 1. \"Ubungsblatt}
\author{\Authors}
\date{\today}

\newcommand{\changefont}[3]{\fontfamily{#1} \fontseries{#2} \fontshape{#3} \selectfont}

\renewcommand{\thesection}{Aufgabe \arabic{section}}
\renewcommand{\theenumi}{(\alph{enumi})}
\renewcommand{\theenumii}{(\roman{enumii}}

\definecolor{lgray}{gray}{0.95}
\definecolor{purple}{rgb}{0.498,0,0.3333}
\definecolor{identifier}{rgb}{0,0,0.1}
\definecolor{string}{rgb}{0.192,0,1}
\definecolor{comment}{rgb}{0.25,0.5,0.37}

\pagestyle{myheadings}
\oddsidemargin\oddsidemargin
\markright{\Authors}

\lstset{
	tabsize=4, 
	frame=tlrb, 
	basicstyle=\footnotesize\changefont{pcr}{m}{n},
	breaklines=true,
	numbers=left,
	emphstyle=\textit, 
	language=Java,
	keywordstyle=\color{purple}\textbf, 
	identifierstyle=\color{identifier},
	stringstyle=\color{string},
	backgroundcolor=\color{lgray},
	showstringspaces=false,
	commentstyle=\color{comment},
	extendedchars=true,
	inputencoding=utf8/latin1
}

\begin{document}

\maketitle

\section{$\boldsymbol{O}$-Notation}
\begin{itemize}
      \item $n^{\frac{1}{\log n}} = n^{\frac{log_n 2}{log_n n}} \\
	= n^{log_ 2} = 2 < c * \ln n \Rightarrow n^{\frac{1}{log n}} \in O(\ln n)$
      \item $\ln n < \ln n * \ln n = \frac{log_2 n}{log_2 \mathrm{e}} * \frac{log_2 n}{log_2 \mathrm{e}} = c * log_2 (n) * log_2 (n) \\
	\Rightarrow \ln n \in O(\ln^2 n)$
      \item $\log^2 n < \sqrt{n} \Rightarrow log^2 n \in O(\sqrt{n})\\ $
	$e^{n}$ steigt schneller als $n^2 \Rightarrow$ Die Umkehrfunktionen haben umgekehrte Eigenschaft.
      \item $(\sqrt{2})^{\log n} = \sqrt{2^{\log n}} = \sqrt{n} < n < n^2 \Rightarrow \sqrt{2}^{log n} \in O(n^2)$\\
      \item $n^2$
       \item $4^{\log n} = 2^{\log n} * 2^{\log n} = n * n = n^2 \Rightarrow 4^{log n} \in \theta(n^2)$\\
	
	\item $2^n > n^2 \Rightarrow n^2 \in O(2^n)$\\
		Exponential Funktion steigt stärker als Quadratfunktion.
	\item $n^{\log \log n}$\\
	    $n^n > 2^n \Rightarrow 2^n \in O(n^{\log \log n})$
	\item $(\lceil \log (n) \rceil )!$\\
		Aufgrund des Verschachtelten Aufrufs des Logarithmus gilt $n^{\log \log n} \in O((\lceil \log (n) \rceil )!)$.
		Da z.\,B. auch gilt $n! > \frac{n}{2}^{\frac{n}{2}}$
	\item $2^n > n \Leftrightarrow 2^{(2^n)} > 2^n$\\
		Steigt exponentiell exponentiell.
	
	
\end{itemize}
Da $<$ transitiv ist, gilt dies auch für die Wachstumsklassen.
\section{Sammelbilder}
\begin{enumerate}
\item 
	\underline{Zu zeigen:} $E[X] = \sum\limits_{i = 1}^{n} E[X_{i}]$\\
	\underline{Herleitung:} \begin{align*}
		\sum\limits_{i = 1}^{n} E\left[X_{i}\right] &= \sum\limits_{i = 1}^{n}\left(\sum\limits_{j = 1}^{\infty}j * P\left(X_{i} = j\right)\right)\\
		&=\sum\limits_{j = 1}^{\infty}\left(\sum\limits_{i = 1}^{n}j * P\left(X_{i} = j\right)\right)\\
		&=\sum\limits_{j = 1}^{\infty}j * \left(\sum\limits_{i = 1}^{n}P\left(X_{i} = j\right)\right)\\
		&= E\left[\sum\limits_{i = 1}^{n}X_{i}\right]\\
		&= E\left[X\right]
	\end{align*}
	
\item 
	\underline{Gesucht:} $E[X_{i}]$\\
	\underline{Herleitung:} \begin{align*}
		E\left[X_{i}\right] &= \frac{1}{p_{i}}\\
		p_{i} &= \frac{n-i}{n}\\
		\\
		\Rightarrow E\left[X_{i}\right] &= \frac{n}{n-i}	
	\end{align*}
\item 
	\underline{Zu zeigen:} $E[X] = O(n \log n)$\\
	\underline{Herleitung:} \begin{align*}
		E\left[X\right] &= \sum\limits_{i = 1}^{n}E\left[X_{i}\right]\\
		&= \sum\limits_{i = 1}^{n}\frac{n}{n - i}\\
		&= n * \left(1 + \frac{1}{2} + \frac{1}{3} + \cdots + \frac{1}{n}\right)\\
		&= n * \sum\limits_{i = 1}^{n}\frac{1}{i}\\
		&= n \log n
	\end{align*}
\end{enumerate}

\section{Varianten Mergesort}
\begin{enumerate}
\item \begin{description}
       \item[Selection Sort] sortiert die Elemente eines Arrays, indem jeweils das kleinste Element der unsortierten Restarrays mit dem Element nach dem sortierten Teil getauscht wird und der sortierte Anteil somit um 1 wächst.

\textbf{Worst-Case-Szenario:} Im schlimmsten Fall ist die Liste falsch
herum sortiert. Der Algorithmus muss so bei jedem Durchlauf den aktuellen
Suchraum von vorne bis hinten durchsuchen, um das jeweils Minimum zu finden. Um das Minimum einer $n$-elementigen Liste zu finden, werden $n-1$ Vergleiche benötigt. Dies muss $n-1$-mal
berechnet werden, da nach $n-1$ Tauschoperationen das Maximum bereits am Ende des Arrays steht. Für den Worst-Case werden als die folgende Anzahl an Vergleichen durchgeführt:
\[
 \sum\limits_{i = 2}^{n} i - 1 = \sum\limits_{i = 1}^{n - 1} i = \frac{1}{2} \left(n * (n - 1)\right) = \frac{n^2 - n}{2} \in O(n^2).
\]


       \item[Mergesort] sortiert die Elemente eines Arrays, indem das Array in zwei gleichgroße Teile aufteilt und diese rekursiv und anschließend aus den zwei sortierten Teilen eine sortierte Gesamtfolge verschmilzt.

\textbf{Worst-Case-Szenario:} Im schlimmsten Fall sind bei jeder Verschmelzung die Teile gleich groß und die Elemente müssen wechselweise eingefügt werden, also das Verschmelzen wird nicht dadurch abgekürzt, dass ab einem gewissen Punkt
mehr als ein Element eines Teiles übernommen werden kann (z.B. weil die andere Folge bereits erschöpft ist).
Dann werden für das Verschmelzen von zwei $n/2$-elementigen Teilen $n - 1$ Vergleiche benötigt, bei drei $n/3$-elementigen Teilen ebenfalls $n - 1$ usw.
So werden bei jedem Rekursionsschritt $n - 1$ Vergleiche durchgeführt.
Da angenommen wird, dass $n$ eine 2er Potenz ist, wird das Verschmelzen bei einer $n$-elementigen Folge $log_2 n$ mal aufgerufen. Insgesamt werden folgende Anzahl an Vergleichen durchgeführt:
\begin{equation*}
 \sum\limits_{i = 1}^{log_2 n} (n - 1) = (n - 1)  * log_2 n \in O(n \log n)
\end{equation*}

      \end{description}
\item \begin{enumerate}
       \item Es wird angenommen, dass $n$ und $M$ 2er-Potenzen sind.
	     Dann gilt für die Anzahl der Vergleiche, dass $log_2 n - log_2 (M - 1)$-mal verschmolzen wird. Zuvor wurden die Elementar-Folgen der Länge $M/2$ mit SelectionSort sortiert, was $\frac{(M/2)^2 - (M/2)}{2}$ Vergleiche erfordert.
	      Es entfallen allso die Rekursionsschritte, die der Selectionsort nun übernimmt.
	     Insgesamt gilt somit für die Anzahl an Vergleichen:
	     \[
	      (log_2 n - log_2 (M-1)) * (n - 1) + \frac{M^2 - M}{2}
	     \]

       \item Der Vergleich der Laufzeiten und die Berechnung eines optimalen Ms erfolgt durch Auflösung der quadratischen Ungleichung nach M bzw. durch Auffinden eines Minimums des Polynoms.
	    \[
	      (log_2 n - log_2 (M-1)) * (n - 1) + \frac{M^2 - M}{2} < (n - 1) * log_2 n \Leftrightarrow \frac{M^2 - M}{2} - log_2 (M - 1) * (n - 1) < 0
	     \] 
      \end{enumerate}
\item
  Die Auswertung des Algorithmus geschah durch zufällig erzeugte Arrays unterschiedlicher Länge (10, 1000, 50000, 1000000) und unterschiedlichen Werten für M (2, 3, 5, 10, 20, 40, 50, 60).
  Man kann unser Programm mit \lstinline!java -jar merge.jar [2,3,5,10,20,40,50,60] [10,1000,50000,100000]! direkt aufrufen, um es übersichtlich angezeigt zu bekommen.
  Interessanterweise brauchte MMergeSort immer mindestens so viele Vergleich wie MergeSort und zwar auch nur genauso viele wenn mit $M = 2$, SelectionSort außen vorgelassen wurde.
  Bezüglich der Laufzeit sah es schon anders aus, hier wurde das Optimum für die jeweiligen Größen ermittelt, indem bei gleichbleibender Länge die Werte von M variiert wurde und die Laufzeit gemessen wurde.
  Das Optimum lag dabei zwischen 10 und 60 und kommt aufgrund von Schwankungen nicht genauer bestimmt werden. MMergeSort war durch weg um $\frac{1}{3}$ schneller als MergeSort.
  Die Konstante der Laufzeit ist bei den Schwankungen trotz mehrmaliger Durchläufe und Mittelung nicht genau genug bestimmbar, als dass sich die Angabe eines konkreten Wertes Sinn ergäbe.
      \lstinputlisting[caption=Sort.java]{src/Sort.java}
      \lstinputlisting[caption=SelectionSort.java]{src/SelectionSort.java}
      \lstinputlisting[caption=MergeSort.java]{src/MergeSort.java}
      \lstinputlisting[caption=MMergeSort.java]{src/MMergeSort.java}
\end{enumerate}
\end{document}