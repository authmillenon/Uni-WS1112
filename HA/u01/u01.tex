\documentclass[a4paper,10pt]{scrartcl}
\usepackage[utf8x]{inputenc}
\usepackage[T1]{fontenc}
\usepackage{amsmath,amsfonts,amssymb,amscd,amsthm,xspace}
\usepackage[ngerman]{babel}
\usepackage{listingsutf8}
\usepackage{color}
\usepackage{geometry}
\usepackage{graphicx}
\usepackage{multicol}
\usepackage{pst-tree}

\geometry{a4paper, left=2cm,right=2cm,top=2cm,bottom=2cm}

\newcommand{\Authors}{Martin Lenders, Ralf M\"uller-Zimmermann}
\title{H\"ohere Algorithmik - 1. \"Ubungsblatt}
\author{\Authors}
\date{\today}

\newcommand{\changefont}[3]{\fontfamily{#1} \fontseries{#2} \fontshape{#3} \selectfont}

\renewcommand{\thesection}{Aufgabe \arabic{section}}
\renewcommand{\theenumi}{(\alph{enumi})}

\definecolor{lgray}{gray}{0.95}
\definecolor{purple}{rgb}{0.498,0,0.3333}
\definecolor{identifier}{rgb}{0,0,0.1}
\definecolor{string}{rgb}{0.192,0,1}
\definecolor{comment}{rgb}{0.25,0.5,0.37}

\pagestyle{myheadings}
\oddsidemargin\oddsidemargin
\markright{\Authors}

\lstset{
	tabsize=4, 
	frame=tlrb, 
	basicstyle=\footnotesize\changefont{pcr}{m}{n},
	breaklines=true,
	numbers=left,
	emphstyle=\textit, 
	language=Python,
	keywordstyle=\color{purple}\textbf, 
	identifierstyle=\color{identifier},
	stringstyle=\color{string},
	backgroundcolor=\color{lgray},
	showstringspaces=false,
	commentstyle=\color{comment},
	extendedchars=true,
	inputencoding=utf8/latin1
}
\psset{nodesep=2pt,levelsep=2em,treesep=2em}

\begin{document}

\maketitle

\section{$\boldsymbol{O}$-Notation}
\section{Sammelbilder}
\begin{enumerate}
\item 
	\underline{Zu zeigen:} $E[X] = \sum\limits_{i = 1}^{n} E[X_{i}]$\\
	\underline{Herleitung:} \begin{align*}
		\sum\limits_{i = 1}^{n} E\left[X_{i}\right] &= \sum\limits_{i = 1}^{n}\left(\sum\limits_{j = 1}^{\infty}j * P\left(X_{i} = j\right)\right)\\
		&=\sum\limits_{j = 1}^{\infty}\left(\sum\limits_{i = 1}^{n}j * P\left(X_{i} = j\right)\right)\\
		&=\sum\limits_{j = 1}^{\infty}j * \left(\sum\limits_{i = 1}^{n}P\left(X_{i} = j\right)\right)\\
		&= E\left[\sum\limits_{i = 1}^{n}X_{i}\right]\\
		&= E\left[X\right]
	\end{align*}
	
\item 
	\underline{Gesucht:} $E[X_{i}]$\\
	\underline{Herleitung:} \begin{align*}
		E\left[X_{i}\right] &= \frac{1}{p_{i}}\\
		p_{i} &= \frac{n-i}{n}\\
		\\
		\Rightarrow E\left[X_{i}\right] &= \frac{n}{n-i}	
	\end{align*}
\item 
	\underline{Zu zeigen:} $E[X] = O(n\ log\ n)$\\
	\underline{Herleitung:} \begin{align*}
		E\left[X\right] &= \sum\limits_{i = 1}^{n}E\left[X_{i}\right]\\
		&= \sum\limits_{i = 1}^{n}\frac{n}{n - i}\\
		&= n * \left(1 + \frac{1}{2} + \frac{1}{3} + \cdots + \frac{1}{n}\right)\\
		&= n * \sum\limits_{i = 1}^{n}\frac{1}{i}\\
		&= n\ log\ n
	\end{align*}
\end{enumerate}

\section{Varianten Mergesort}
\begin{enumerate}
\item \begin{description}
       \item[Selection Sort] sortiert die Elemente einer Liste, indem er eine
Auswahl (\textit{Selection}), die zu Beginn des Algorithmus leer ist, auff\"ullt,
indem er die Verbleibende Liste nach ihrem kleinsten Element durchsucht und
durch eine Vertauschun in die Auswahl aufnimmt.

\textbf{Worst-Case-Szenario:} Im schlimmsten Fall ist die Liste falsch
herum sortiert, da der Algorithmus so bei jedem Durchlauf den aktuellen
Suchraum von vorne bis hinten durchsuchen muss. Im ersten Durchlauf sind das
also $n$ Elemente, im zweiten $n-1$ usw. Die Laufzeit f\"ur den Worst Case ist
also
\[
 \sum\limits_{i = 1}^{n} i = \frac{n \cdot (n + 1)}{2} \in O(n^2).
\]


       \item[Mergesort] sortiert die Elemente einer Liste, indem er eine Liste
in seine Elemente zerlegt und rekursiv diese wieder zusammensetzt, wobei er
jedes mal beim zusammensetzen die neu entstandenen Teillisten sortiert.

\textbf{Worst-Case-Szenario:} 
      \end{description}
\item 
\item 
\end{enumerate}
\end{document}