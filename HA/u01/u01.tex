\documentclass[a4paper,10pt]{scrartcl}
\usepackage[utf8x]{inputenc}
\usepackage[T1]{fontenc}
\usepackage{amsmath,amsfonts,amssymb,amscd,amsthm,xspace}
\usepackage[ngerman]{babel}
\usepackage{listingsutf8}
\usepackage{color}
\usepackage{geometry}
\usepackage{graphicx}
\usepackage{multicol}
\usepackage{pst-tree}

\geometry{a4paper, left=2cm,right=2cm,top=2cm,bottom=2cm}

\newcommand{\Authors}{Martin Lenders, ...}
\title{Höhere Algorithmik - 1. Übungsblatt}
\author{\Authors}
\date{\today}

\newcommand{\changefont}[3]{\fontfamily{#1} \fontseries{#2} \fontshape{#3} \selectfont}

\renewcommand{\thesection}{Aufgabe \arabic{section}}
\renewcommand{\theenumi}{(\alph{enumi})}

\definecolor{lgray}{gray}{0.95}
\definecolor{purple}{rgb}{0.498,0,0.3333}
\definecolor{identifier}{rgb}{0,0,0.1}
\definecolor{string}{rgb}{0.192,0,1}
\definecolor{comment}{rgb}{0.25,0.5,0.37}

\pagestyle{myheadings}
\oddsidemargin\oddsidemargin
\markright{\Authors}

\lstset{
	tabsize=4, 
	frame=tlrb, 
	basicstyle=\footnotesize\changefont{pcr}{m}{n},
	breaklines=true,
	numbers=left,
	emphstyle=\textit, 
	language=Python,
	keywordstyle=\color{purple}\textbf, 
	identifierstyle=\color{identifier},
	stringstyle=\color{string},
	backgroundcolor=\color{lgray},
	showstringspaces=false,
	commentstyle=\color{comment},
	extendedchars=true,
	inputencoding=utf8/latin1
}
\psset{nodesep=2pt,levelsep=2em,treesep=2em}

\begin{document}

\maketitle

\section{$\boldsymbol{O}$-Notation}
\section{Sammelbilder}
\begin{enumerate}
\item 
\item 
\item 
\end{enumerate}

\section{Varianten Mergesort}
\begin{enumerate}
\item \begin{description}
       \item[Selection Sort] sortiert die Elemente einer Liste, indem er eine
Auswahl (\textit{Selection}), die zu Beginn des Algorithmus leer ist, auffüllt,
indem er die Verbleibende Liste nach ihrem kleinsten Element durchsucht und
durch eine Vertauschun in die Auswahl aufnimmt.

\textbf{Worst-Case-Szenario:} Im schlimmsten Fall ist die Liste falsch
herum sortiert, da der Algorithmus so bei jedem Durchlauf den aktuellen
Suchraum von vorne bis hinten durchsuchen muss. Im ersten Durchlauf sind das
also $n$ Elemente, im zweiten $n-1$ usw. Die Laufzeit für den Worst Case ist
also
\[
 \sum\limits_{i = 1}^{n} i = \frac{n \cdot (n + 1)}{2} \in O(n^2).
\]


       \item[Mergesort] sortiert die Elemente einer Liste, indem er eine Liste
in seine Elemente zerlegt und rekursiv diese wieder zusammensetzt, wobei er
jedes mal beim zusammensetzen die neu entstandenen Teillisten sortiert.

\textbf{Worst-Case-Szenario:} Im schlimmsten Fall ist die Liste falsch
herum sortiert, da der Algorithmus so bei jedem Durchlauf den aktuellen
Suchraum von vorne bis hinten durchsuchen muss. Im ersten Durchlauf sind das
also $n$ Elemente, im zweiten $n-1$ usw. Die Laufzeit für den Worst Case ist
also
\[
 \sum\limits_{i = 1}^{n} i = \frac{n \cdot (n + 1)}{2} \in O(n^2).
\]
      \end{description}
\item 
\item 
\end{enumerate}
\end{document}