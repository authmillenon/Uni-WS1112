\documentclass[a4paper,10pt]{scrartcl}
\usepackage[utf8x]{inputenc}
\usepackage[T1]{fontenc}
\usepackage{amsmath,amsfonts,amssymb,amscd,amsthm,xspace}
\usepackage[ngerman]{babel}
\usepackage{listingsutf8}
\usepackage{color}
\usepackage{algorithmic}
\usepackage{geometry}
\usepackage{graphicx}
\usepackage{multicol}
\geometry{a4paper, left=2cm,right=2cm,top=2cm,bottom=2cm}

\newcommand{\Authors}{Christian Cikryt (Tut. Di. 14-16), Jakob Pfender (Mi. 14-16)}
\title{H\"ohere Algorithmik - 5. \"Ubungsblatt}
\author{\Authors}
\date{\today}

\newcommand{\changefont}[3]{\fontfamily{#1} \fontseries{#2} \fontshape{#3} \selectfont}

\renewcommand{\thesection}{Aufgabe \arabic{section}}
\renewcommand{\theenumi}{(\alph{enumi})}
\renewcommand{\theenumii}{(\roman{enumii}}

\definecolor{lgray}{gray}{0.95}
\definecolor{purple}{rgb}{0.498,0,0.3333}
\definecolor{identifier}{rgb}{0,0,0.1}
\definecolor{string}{rgb}{0.192,0,1}
\definecolor{comment}{rgb}{0.25,0.5,0.37}

\pagestyle{myheadings}
\oddsidemargin\oddsidemargin
\markright{\Authors}

\lstset{
	tabsize=4, 
	frame=tlrb, 
	basicstyle=\footnotesize\changefont{pcr}{m}{n},
	breaklines=true,
	numbers=left,
	emphstyle=\textit, 
	language=Java,
	keywordstyle=\color{purple}\textbf, 
	identifierstyle=\color{identifier},
	stringstyle=\color{string},
	backgroundcolor=\color{lgray},
	showstringspaces=false,
	commentstyle=\color{comment},
	extendedchars=true,
	inputencoding=utf8/latin1
}

\begin{document}

\maketitle

\section{Varianten der Vorlesungsbeispiele}
\begin{enumerate}
 \item
  Der hier vorgestellte Algorithmus ist eine Abwandlung des in der Vorlesung behandelten und geht von folgender Rekursionsgleichung aus:
   \begin{align*}
     E[0,b] &= 0 \qquad \forall b \in \mathbb{N} \\
     E[m,0] &= 0 \qquad \forall m \in \mathbb{N} \\
     E[m,b] &= \begin{cases}
                \max(E[m-1,b], w_m + E[m-1, b-p_m], w_m + E[m, b - p_m]), & \text{wenn } b \geq p_m \\
                E[m-1,b], & \text{sonst}
               \end{cases}
    \end{align*}
  Diese Gleichung berücksichtigt, dass man denselben Gegenstand erneut auswählen kann ($w_m + E[m, b - p_m]$).
  Der abgewandelte Algorithmus im Pseudocode (genau wie in der Vorlesung wird die Bestimmung der tatsächlichen Lösung zunächst beiseite gelassen):
\begin{center}
        \begin{algorithmic}
            \FOR{$m \in \{0, \hdots, n\}$}
                \STATE $E[m,0] \gets 0$
            \ENDFOR
            \FOR{$b \in \{0, \hdots, B\}$}
                \STATE $E[0,b] \gets 0$
            \ENDFOR
            \FOR{$m \in \{1, \hdots, n\}$}
                \FOR{$b \in \{1, \hdots, B\}$}
                    \IF{$p[m] > b$}
                        \STATE $E[m,b] \gets E[m-1,b]$
                    \ELSE
                        \STATE $E[m,b] \gets \max\{E[m-1,b], w[m] + E[m-1,b-p[m]], E[m, b - p[m]]\}$
                    \ENDIF
                \ENDFOR
            \ENDFOR
            \RETURN $E[n, B]$
        \end{algorithmic}
       \end{center}
  Der einzige Unterschied im Vergleich zum Vorlesungsalgorithmus ist der 3. Term bei der Maximumsbestimmung, der in jedem Schritt konstante Zeit mehr beansprucht.
 Somit verändert sich die Wachstumsklasse nicht. Diese ist bestimmt durch die Anzahl der Kästen der Tabelle und somit O(m * B).
 Die letztendliche Lösung kann man aufzeichnen, indem man in jedem Schritt die Entscheidung speichert. Dies erfordert zwar mehr Speicherplatz, verändert aber die asymptotische Laufzeit nicht.

 \item
Gegeben: $n$ Städte $0, 1, \hdots, n - 1$. Genau wie in der Vorlesung auch bezeichne $T[S, m]$ die Länge einer optimalen Tour, die bei 0 anfängt, bei m aufhört und zwischen 0 und m genau die Städte aus
$S$ besucht. \newline
Gesucht: $T[\{1, \hdots, n - 1\}, 0]$ \newline
Zu dem im Folgenden vorgestellten Algorithmus seien einige Bemerkungen gestattet:
\begin{itemize}
 \item $T[\emptyset, m] = d(0, m) \forall m \in \{0, \hdots, n - 1\}$
 \item $d(i, j) = d(j, i)$ ist der Abstand zweier Städte $i$ und $j$.
 \item Sei $m \in S$ und $A \subseteq S \setminus \{0, m\}$. Dann gilt $T[A, m] = min\{d(m,j) + T[A \setminus \{ j \}] \left. \right| j \in A\}$. Denn für jede nicht-leere Menge
 A wird der optimale Weg wie folgt definiert: Teste für alle $j \in A$ den Weg, der von $m$ aus zunächst nach $j$ führt. Für den optimalen Weg gilt, dass auch der Rest des Weges (von $j$ nach $0$) optimal ist.
 Dieser Weg wurde aber genau so definiert: $T[A \setminus \{ j \}, j]$.
 \item Der optimale Weg selbst wird in dem Feld O gespeichert ($O[0] = 0, O[n] = 0$)
\end{itemize}

Algorithmus:


  \begin{algorithmic}
   \FOR{$m \in \{0, \hdots, n - 1\}$}
      \STATE $T[\emptyset, m] \gets d(m,0)$
   \ENDFOR
  \STATE $O[0] \gets 0$, $O[n] \gets 0$
  \FOR{$k \in \{1, \hdots, n - 1\}$}
    \FOR{$A \subseteq S \setminus \{0\} $ mit $ \left| A \right| = k$}
      \FOR{$m \in \{0, \hdots, n - 1\}$}
	\IF{$m \notin A$}
	  \STATE $T[A,m] \gets min\{d(m, j) + T[A \setminus \{j\}, j] \left. \right|  j \in A\}$
	  \STATE $0[k] = j $ mit $ j$ erzielte das Minimum in der vorangehenden Zeile.
	\ENDIF
      \ENDFOR
    \ENDFOR
  \ENDFOR
  \RETURN $T[0,\{1, \hdots, n - 1\}], O$
  \end{algorithmic}

\end{enumerate}

\section{Münzwechseln}
\begin{enumerate}
 \item 

  \begin{enumerate}
  \item Finde geeignete Teilprobleme:
    
  $c[n,i]$ gibt an, auf wieviele Arten $n$ Cent mit den $i$ kleinsten Münzen dargestellt werden können. \newline
  $c[n, 0] = 0$ \newline
  $c[0, i] = 1$ \newline
  Für $c[n, i]$ hat man zwei Möglichkeiten zum Wechseln. Entweder man benutzt die $i$-te Münzart ($c[n - value_i, i]$) oder man wechselt den Betrag ohne die $i$-te Münzart ($c[n, i - 1]$).
  Uns interessiert die Summe der beiden rekursiven Möglichkeiten.
  \item Rekursion:

  $c[n, 0] = 0$ \newline
  $c[0, i] = 1$ \newline
   $c[< 0, i] = 0$ \newline
  $c[n, i] = c[n - value_i, i] + c[n, i - 1]$
  \item Pseudocode zum Ausfüllen der Tabelle (dynamische Programmierung):

  \begin{algorithmic}
   \FOR{$i \in \{1, \hdots, n\}$}
      \STATE $c[i, 0] \gets 0$
   \ENDFOR
   \FOR{$i \in \{0, \hdots, k\}$}
      \STATE $c[0,k] \gets 1$
    \ENDFOR
  \FOR{$i \in \{1, \hdots, k\}$}
    \FOR{$j \in \{1, \hdots, n\}$}
      \IF{$n - v_i < 0$}
	\STATE $c[j, i] = 0 + c[n, i - 1]$
      \ELSE
	\STATE $c[j, i] = c[n - v_i, i] + c[n, i - 1]$
      \ENDIF
    \ENDFOR
  \ENDFOR
  \RETURN $c[n,k]$
  \end{algorithmic}

  \end{enumerate}

 \item Die Laufzeit des Algorithmus wird durch die zwei verschachtelten for-Schleifen bestimmt (Größe der Tabelle), da in der innersten Schleife konstante Zeit gebraucht wird.
  Daraus folgt: Laufzeit $\in O(n*k)$.
 \item
  \lstinputlisting[gobble=4,firstnumber=9,caption={SchoolMethod.java}]{src/ChangingMoney.java}
\end{enumerate}

\section{Versteckte Markov-Modelle}
\begin{enumerate}
 \item 
 \item
  Die Wahrscheinlichkeiten durch die Multiplikationen sehr klein. Um dem exponentiellen Absinken entgegen zu wirken, bietet sich eine logarithmische
  Darstellung an. So können Ungenauigkeiten durch das Rechnen mit kleinsten Zahlen vermieden werden. Insgesamt ist der Einsatz des Logarithmus möglich, da der Logarithmus eine monoton wachsende Funktion ist.
 \item
  Zur Fehlerkorrektur eines deutschsprachigen Textes soll ein Markov-Modell verwendet werden. Dabei sind die tatsächlich gemeinten Buchstaben die \textbf{Zustände}, das \textbf{Ausgabealphabet} sind die
  empfangenen Buchstaben. Die \textbf{Ausgabewahrscheinlichkeit}, dass der empfangene Buchstabe auch gemeint war, ist 90 Prozent. Die \textbf{Ausgangswahrscheinlichkeiten} können durch Analyse von korrekten Deutschen Schriften,
  Wörterbüchern etc. ermittelt werden, um herauszufinden, mit welchem Zeichen ein Satz mit welcher Wahrscheinlichkeit anfängt. Genauso können auch durch Analyse von Wörtern die \textbf{Übergangsverteilungen} bestimmt werden.
  Hilfreich wäre es den Kontext zu wissen, da in einem umgangssprachlichen Chat andere Worte, mitunter nicht-korrekte Wörter und Schreibweisen verwendet werden. Auch die Häufigkeit der Wörter in einem
 historischen Text weicht beispielsweise erheblich von einem zeitgenössischen ab.  Man muss insofern Annahme machen anhand derer das Analysematerial auswählen. Ohne weitere Anhaltspunkte wird das Ergebnis
 nicht überzeugend sein, zumal die Fehlerwahrscheinlichkeit von 10 Prozent recht hoch ist. Die Annahme dass die Fehlerwahrscheinlichkeit bestimmt werden kann ist unrealitisch, sie ist bei einer 
 Übertragung in aller Regel in der Realität nicht konstant verteilt.

Randbemerkung: Dass in der Realität schwierig ist zu berechnen welches Wort gemeint war, zeigen Worterbuchvorschläge von Smartphones beispielsweise. Daran sieht man auch dass man durch Beibringen von
persönlich häufig verwendeten Wörtern (Anpassung des Wortschatzes) das Ergebnis verbessern kann.
\end{enumerate}
\end{document}
