\documentclass[a4paper,10pt]{article}
\usepackage[utf8]{inputenc}
\usepackage[T1]{fontenc}
\usepackage{amsmath,amsfonts,amssymb,amscd,amsthm,xspace}
\usepackage[ngerman]{babel}
\usepackage{listingsutf8}
\usepackage{color}
\usepackage{geometry}
\usepackage{graphicx}
\usepackage{multicol}
\usepackage{pst-tree}
\usepackage{algorithmic}
\usepackage{cancel}

\geometry{a4paper, left=2cm,right=2cm,top=2cm,bottom=2cm}

\newcommand{\Authors}{Martin Lenders (Di. 14-16), Ralf M\"uller-Zimmermann (Di. 14-16)}
\title{H\"ohere Algorithmik - 10. \"Ubungsblatt}
\author{\Authors}
\date{\today}

\newcommand{\changefont}[3]{\fontfamily{#1} \fontseries{#2} \fontshape{#3} \selectfont}

\renewcommand{\thesection}{Aufgabe \arabic{section}:}
\renewcommand{\labelenumi}{(\theenumi)}
\renewcommand{\theenumi}{\alph{enumi}}
\renewcommand{\labelenumii}{(\theenumii)}
\renewcommand{\theenumii}{\roman{enumii}}

\definecolor{lgray}{gray}{0.95}
\definecolor{purple}{rgb}{0.498,0,0.3333}
\definecolor{identifier}{rgb}{0,0,0.1}
\definecolor{string}{rgb}{0.192,0,1}
\definecolor{comment}{rgb}{0.25,0.5,0.37}

\pagestyle{myheadings}
\oddsidemargin\oddsidemargin
\markright{\Authors}

\lstset{
	tabsize=4, 
	basicstyle=\footnotesize\fontfamily{pcr}\fontseries{m}\fontshape{n}\selectfont,
	breaklines=true,
	numbers=left,
	emphstyle=\textit, 
	language=Java,
	keywordstyle=\color{purple}\textbf, 
	identifierstyle=\color{identifier},
	stringstyle=\color{string},
	showstringspaces=false,
    escapeinside={((*}{*))},
	commentstyle=\color{comment},
	extendedchars=true,
	inputencoding=utf8/latin1
}
\psset{nodesep=2pt,levelsep=2em,treesep=2em}

\begin{document}

\maketitle

\section{Hashing mit Verkettung}
\begin{enumerate}
\item   
\item   
\item   
\item   
\item   
\end{enumerate}

\section{Page-Rank}
\begin{enumerate}
\item   \[
            A' = \begin{pmatrix}
                0 & 1 & 0 & 0 \\
                0 & 0 & \frac{1}{2} & \frac{1}{2} \\
                1 & 0 & 0 & 0 \\
                0 & 0 & 1 & 0
            \end{pmatrix}
        \]
\item   
\item   Wir haben den Algorithmus in Python implementiert:
        \lstinputlisting[lastline=35,language=Python]{src/pagerank.py}
        Für das Beispiel (Eingabe: \lstinline[language=Python]!pagerank(Graph(vertices=['a','b','c','d'],edges=[('a','b'),('b','c'),('b','d'),('c','a'),('d','c')]))!) weicht die Matix mit einem Fehler kleiner als $0{,}001$ (mit der im Tutorium vorgeschlagenen Maximumsnorm als Fehlermaß) nach \underline{\underline{12}} Iterationen ab.
\end{enumerate}

\section{Prioritätswarteschlangen}
\begin{enumerate}
\item   
\item   
\item   
\end{enumerate}
\end{document}