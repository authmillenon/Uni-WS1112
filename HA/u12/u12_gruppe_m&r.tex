\documentclass[a4paper,10pt]{article}
\usepackage[utf8]{inputenc}
\usepackage[T1]{fontenc}
\usepackage{amsmath,amsfonts,amssymb,amscd,amsthm,xspace}
\usepackage[ngerman]{babel}
\usepackage{listingsutf8}
\usepackage{color}
\usepackage{geometry}
\usepackage{graphicx}
\usepackage{multicol}
\usepackage{pst-tree}
\usepackage{algorithmic}
\usepackage{cancel}

\geometry{a4paper, left=2cm,right=2cm,top=2cm,bottom=2cm}

\newcommand{\Authors}{Martin Lenders (Di. 14-16), Ralf M\"uller-Zimmermann (Di. 14-16)}
\title{H\"ohere Algorithmik - 12. \"Ubungsblatt}
\author{\Authors}
\date{\today}

\newcommand{\changefont}[3]{\fontfamily{#1} \fontseries{#2} \fontshape{#3} \selectfont}

\renewcommand{\thesection}{Aufgabe \arabic{section}:}
\renewcommand{\labelenumi}{(\theenumi)}
\renewcommand{\theenumi}{\alph{enumi}}
\renewcommand{\labelenumii}{(\theenumii)}
\renewcommand{\theenumii}{\roman{enumii}}

\definecolor{lgray}{gray}{0.95}
\definecolor{purple}{rgb}{0.498,0,0.3333}
\definecolor{identifier}{rgb}{0,0,0.1}
\definecolor{string}{rgb}{0.192,0,1}
\definecolor{comment}{rgb}{0.25,0.5,0.37}

\pagestyle{myheadings}
\oddsidemargin\oddsidemargin
\markright{\Authors}

\lstset{
	tabsize=4, 
	basicstyle=\footnotesize\fontfamily{pcr}\fontseries{m}\fontshape{n}\selectfont,
	breaklines=true,
	numbers=left,
	emphstyle=\textit, 
	language=Java,
	keywordstyle=\color{purple}\textbf, 
	identifierstyle=\color{identifier},
	stringstyle=\color{string},
	showstringspaces=false,
	escapeinside={((*}{*))},
	commentstyle=\color{comment},
	extendedchars=true,
	inputencoding=utf8/latin1
}
\psset{nodesep=2pt,levelsep=2em,treesep=2em}

\begin{document}

\maketitle

\section{Netzwerkfluss}

\section{Flüsse, Paarungen, Knotenüberdeckungen}
\begin{enumerate}
\item
\item Aus der Bipartitheit folgt, dass man den Graph in zwei Teilgraphen $X$ und $Y$ zerlegen kann, so dass es keine Kante gibt, die zwei Knoten aus der gleichen Teilmenge verbindet. Ersetze die ungerichteten Kanten durch gerichtete Kanten von $X$ nach $Y$ mit dem Gewicht 1. Füge dann den Startknoten $s$ zu $X$ und den Zielknoten $t$ zu $Y$. Erstelle von $t$ zu jedem Knoten in $X$  und von jedem Knoten in $Y$ nach $t$ eine gerichtete Kante mit dem Gewicht 1. So hat jeder Knoten in $X$ nur eine eingehende Kante, kann für den maximalen Fluss also auch nur eine ausgehende Kante verwenden. Daher hat jede Kante von $X$ nach $Y$ einen eigenen Knoten in $X$. Analog gilt für jeden Knoten in $Y$, dass sie nur eine ausgehende Kante haben und daher auch nur eine eingehende Kante verwenden können. Somit hat jede Kante von $X$ nach $Y$ auch einen eigenen Knoten in $Y$. Somit hat jede Kante von $X$ nach $Y$ keine gemeinsamen Knoten. $H$ bestimmt sich nun aus den Kanten des maximalen Flusses in diesem Graph ohne die Kanten von $s$ aus und nach $t$.

Die Laufzeit ist $O(|E|)$ für das Ersetzen der ungerichteten Kanten durch gerichtete plus $O(|V|)$ für das Hinzufügen von $s$ und $t$ samt zugehöriger Kanten plus der Laufzeit, den optimalen Fluss zu finden.
\item
\end{enumerate}

\section{Kantenzusammenhang}
Damit wir überhaupt Flüsse in G finden können, muss dieser in einen gerichteten Graphen umgewandelt werden. Dafür ersetzen wir jede (ungerichtete) Kante durch zwei gerichtete Kanten, die die benachbarten Knoten miteinander verbinden. Die Kanten erhalten dabei ein Gewicht von 1. So ist die Kapazität eines Flusses gleichzeitig die minimale Anzahl an Kanten in diesem Fluss. Ebenso kann jede Kante nur einmal verwendet werden, sodass jeder Knoten des Flusses genauso viele eingehende wie ausgehende Kanten hat. Nach Berechnen des maximalen Flusses kann man dessen Kapazität anhand der Anzahl der eingehenden Kanten am Zielknoten direkt bestimmen. Die Kapazität eines maximalen Flusses ist also die maximale Anzahl der Pfade zwischen zwei Knoten, die keine gleichen Kanten verwenden.
Um diese beiden Knoten zusammenhangslos zu machen, muss man mindestens so viele Kanten entfernen, wie die Kapazität des maximalen Flusses zwischen diesen beiden Knoten.

Die Zusammenhangskomponente ist das Minimum aller maximalen Flüsse in dem Graph. Dies sind jedoch $O(|V|^2)$ viele.
Es reicht jedoch aus, sich einen beliebiegen Knoten fest zu wählen und den maximalen Fluss zu jedem anderen Knoten zu bestimmen. Denn dabei findet man einen maximalen Fluss mit minimaler Kapazität. Der maximale Fluss minimaler Kapazität ist durch eine minimale "Verjüngung" im Graphen bestimmt, die den Graph in zwei Teilgraphen unterteilt. Diese Verjüngung bestimmt auch die Zusammenhangskomponente. Der fest gewählte Startknoten liegt nun in einem der beiden Teilgraphen. Da man jeden anderen Knoten einmal als Zielknoten wählt, ist irgendwann ein Knoten aus dem anderen Teilgraph der Zielknoten. Der Fluss zwischen diesen beiden Knoten hat genau die Kapazität wie die Anzahl an Kanten in der Verjüngung. Ist dies nicht der Fall, so ist der Fluss entweder nicht maximal oder er verläuft durch eine schmalere Verjüngung. Beides ist ein Wiederspruch zu den Annahmen, dass der Fluss maximal ist und die Verjüngung die minimalste Verjüngung ist.

Somit kann man einen maximalen Fluss minimaler Kapazität in $O(|V|)$ Netzen finden.
\end{document}