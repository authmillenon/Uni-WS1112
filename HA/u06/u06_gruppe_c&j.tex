\documentclass[a4paper,10pt]{scrartcl}
\usepackage[utf8x]{inputenc}
\usepackage[T1]{fontenc}
\usepackage{amsmath,amsfonts,amssymb,amscd,amsthm,xspace}
\usepackage[ngerman]{babel}
\usepackage{listingsutf8}
\usepackage{color}
\usepackage{algorithmic}
\usepackage{geometry}
\usepackage{graphicx}
\usepackage{multicol}
\geometry{a4paper, left=2cm,right=2cm,top=2cm,bottom=2cm}

\newcommand{\Authors}{Christian Cikryt (Tut. Di. 14-16), Jakob Pfender (Mi. 14-16)}
\title{H\"ohere Algorithmik - 6. \"Ubungsblatt}
\author{\Authors}
\date{\today}

\newcommand{\changefont}[3]{\fontfamily{#1} \fontseries{#2} \fontshape{#3} \selectfont}

\renewcommand{\thesection}{Aufgabe \arabic{section}}
\renewcommand{\theenumi}{(\alph{enumi})}
\renewcommand{\theenumii}{(\roman{enumii}}

\definecolor{lgray}{gray}{0.95}
\definecolor{purple}{rgb}{0.498,0,0.3333}
\definecolor{identifier}{rgb}{0,0,0.1}
\definecolor{string}{rgb}{0.192,0,1}
\definecolor{comment}{rgb}{0.25,0.5,0.37}

\pagestyle{myheadings}
\oddsidemargin\oddsidemargin
\markright{\Authors}

\lstset{
	tabsize=4, 
	frame=tlrb, 
	basicstyle=\footnotesize\changefont{pcr}{m}{n},
	breaklines=true,
	numbers=left,
	emphstyle=\textit, 
	language=Java,
	keywordstyle=\color{purple}\textbf, 
	identifierstyle=\color{identifier},
	stringstyle=\color{string},
	backgroundcolor=\color{lgray},
	showstringspaces=false,
	commentstyle=\color{comment},
	extendedchars=true,
	inputencoding=utf8/latin1
}

\begin{document}

\maketitle

\section{Unabhängige Mengen in Pfaden}
\begin{enumerate}
 \item Sei $P = (V,E)$ ein Pfad mit Knoten $v_1, v_2, v_3$ und Gewichten
 $w_1 = 3$, $w_2 = 5$, $w_3 = 3$. Dann wählt der gierige Algorithmus im
 ersten Schritt $v_2$ als Knoten mit maximalem Gewicht und eliminiert
 dann $v_1$ und $v_3$, so dass die vom Algorithmus bestimmte unabhängige
 Menge nur aus $v_2$ besteht. Es ist aber trivial zu sehen, dass ${v_1,v_3}$ die
 maximale unabhängige Menge ist. Also
 arbeitet der Algorithmus nicht korrekt.
 \item Sei $P = (V,E)$ ein Pfad mit Knoten $v_1, v_2, v_3, v_4$ und
 Gewichten $w_1 = 3, w_2 = 2, w_3 = 1, w_4 = 3$. Dann gibt der
 Algorithmus ${v_2,v_4}$ als größte unabhängige Menge mit Gesamtgewicht
 5 zurück. Die maximale unabhängige Menge ist allerdings ${v_1,v_4}$ mit
 Gesamtgewicht 6.
 \item Sei $U[m,b]$ = Maximale unabhängige Menge, bei der wir nur Knoten
 $v_1,...,v_m$ zur Verfügung haben und die $b$ Knoten hat.
\end{enumerate}
\end{document}
