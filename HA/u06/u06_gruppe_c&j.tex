\documentclass[a4paper,10pt]{scrartcl}
\usepackage[utf8x]{inputenc}
\usepackage[T1]{fontenc}
\usepackage{amsmath,amsfonts,amssymb,amscd,amsthm,xspace}
\usepackage[ngerman]{babel}
\usepackage{listingsutf8}
\usepackage{color}
\usepackage{algorithmic}
\usepackage{geometry}
\usepackage{graphicx}
\usepackage{multicol}
\geometry{a4paper, left=2cm,right=2cm,top=2cm,bottom=2cm}

\newcommand{\Authors}{Christian Cikryt (Tut. Di. 14-16), Jakob Pfender (Mi. 14-16)}
\title{H\"ohere Algorithmik - 6. \"Ubungsblatt}
\author{\Authors}
\date{\today}

\newcommand{\changefont}[3]{\fontfamily{#1} \fontseries{#2} \fontshape{#3} \selectfont}

\renewcommand{\thesection}{Aufgabe \arabic{section}}
\renewcommand{\theenumi}{(\alph{enumi})}
\renewcommand{\theenumii}{(\roman{enumii}}

\definecolor{lgray}{gray}{0.95}
\definecolor{purple}{rgb}{0.498,0,0.3333}
\definecolor{identifier}{rgb}{0,0,0.1}
\definecolor{string}{rgb}{0.192,0,1}
\definecolor{comment}{rgb}{0.25,0.5,0.37}

\pagestyle{myheadings}
\oddsidemargin\oddsidemargin
\markright{\Authors}

\lstset{
	tabsize=4, 
	frame=tlrb, 
	basicstyle=\footnotesize\changefont{pcr}{m}{n},
	breaklines=true,
	numbers=left,
	emphstyle=\textit, 
	language=Java,
	keywordstyle=\color{purple}\textbf, 
	identifierstyle=\color{identifier},
	stringstyle=\color{string},
	backgroundcolor=\color{lgray},
	showstringspaces=false,
	commentstyle=\color{comment},
	extendedchars=true,
	inputencoding=utf8/latin1
}

\begin{document}

\maketitle

\section{Unabhängige Mengen in Pfaden}
\begin{enumerate}
 \item Sei $P = (V,E)$ ein Pfad mit Knoten $v_1, v_2, v_3$ und Gewichten
 $w_1 = 3$, $w_2 = 5$, $w_3 = 3$. Dann wählt der gierige Algorithmus im
 ersten Schritt $v_2$ als Knoten mit maximalem Gewicht und eliminiert
 dann $v_1$ und $v_3$, so dass die vom Algorithmus bestimmte unabhängige
 Menge nur aus $v_2$ besteht. Es ist aber trivial zu sehen, dass
 $\{v_1,v_3\}$ die
 maximale unabhängige Menge ist. Also
 arbeitet der Algorithmus nicht korrekt.
 \item Sei $P = (V,E)$ ein Pfad mit Knoten $v_1, v_2, v_3, v_4$ und
 Gewichten $w_1 = 3, w_2 = 2, w_3 = 1, w_4 = 3$. Dann gibt der
 Algorithmus $\{v_2,v_4\}$ als größte unabhängige Menge mit Gesamtgewicht
 5 zurück. Die maximale unabhängige Menge ist allerdings $\{v_1,v_4\}$ mit
 Gesamtgewicht 6.
 \item Sei $P = (V,E)$ ein Pfad mit Knoten $v_1, v_2, \hdots, v_n$ und entsprechenden Gewichten $w_1, w_2, \hdots, w_n$, dann gilt für das
  maximalen Gewicht einer unabhängigen Menge ($G(i)$) folgende Rekursionsgleichung, die über die Anzahl $i$ der Knoten geht:

  \begin{align*}
            G(1) &= w_1 \\
            G(2) &= \max\{w_1,w_2\} \\
            G(i) &= \max\{G(i-1), G(i-2) + w_i\} 
   \end{align*}
  Die Formeln für die Rekursionsanker gelten offensichtlich. Für ein $i > 2$ hat man die Wahl, entweder den $i$-ten Knoten mit in die Ergebnismenge aufzunehmen oder ihn nicht aufzunehmen.
  Falls man ihn aufnimmt, kann der $i-1$-te Knoten nicht gewählt werden, sondern frühestens der $i - 2$-te. Bei Nichtaufnahme kann auch der $i-1$-te Knote gewählt werden.
  Die Berechnung erfolgt nun durch dynamische Programmierung durch das Fülllen des Arrays $G[]$. Jedes Arrayelement besteht aus einem Tupel: 1. das maximale Gewicht, 2. die unabhängige Menge maximalen Gewichts.
  Mit \verb!first! und \verb!second! greift man auf das 1. bzw. 2. Element des Tupels zu.
\begin{algorithmic}
      \STATE $G[1] \gets (w[1], elem_1)$
      \IF{$w[1] > w[2]$}
	\STATE $G[2] \gets G[1]$
      \ELSE
	\STATE $G[2] \gets (w[2], elem_2)$
      \ENDIF
      \FOR{$i \in \{2, \hdots, n\}$}
	\IF{$G[i-1] > G[i - 2] + w[i]$}
	  \STATE $G[i] = G[i - 1]$
	\ELSE
	  \STATE $G[i] = (G[i - 2].first + w[i], G[i - 2].second + elem_i)$
	\ENDIF
    \ENDFOR
    \RETURN G[n]
  \end{algorithmic}
Die Laufzeit liegt in $O(n)$, da die for-Schleife über die n-Elemente des Arrays einmal läuft und in jedem Schritt nur konstant viel Kosten hat. Der Platzbedarf liegt genau genommen in $O(n^2)$,
da in jedem Eintrag $O(n)$-Elemente gespeichert werden und es $O(n)$-Elemente gibt. Den Platzbedarf könnte man falls kritisch auf $O(n)$ optimieren, indem man sich nur 2 aktuellsten Mengen merkt.
\end{enumerate}
\section{Vorlesungplanung}
\begin{enumerate}
 \item Gegenbeispiel (Algorithmus wählt rotmarkiertes, optimale Lösung grün markiert):
  \vfill

  \item Dieses Kriterium ist korrekt. 
    \paragraph*{Implementierung:}
	Gegeben seien die Vorlesungsintervalle $R = \{I_1, I_2, \hdots, I_n\}$ mit $I_i = (a(i), e(i))$.
	Gesucht wird die größte Teilmenge $A \subseteq R$, so dass sich keine Intervalle überlappen.
	Sei $R[]$ die Menge der Vorlesungen und $A[]$ das zu füllende
	Array mit der größtmöglichen Menge verträglicher Vorlesungen.
	Jedes Element in $R$ und $A$ habe zwei Felder $a$ und $e$, die
	Anfang und Ende der Vorlesung entsprechen.
	\begin{algorithmic}
		\STATE $i \gets 0$
		\WHILE{$R \neq \emptyset$}
			\STATE $m \gets argmax\{r.a | r \in R\}$
			\STATE $A[i++] \gets m$
			\STATE $R \gets R \setminus \{r | r \in R, r.e > m.a\}$
		\ENDWHILE
		\RETURN A
	\end{algorithmic}
	Der Algorithmus hat eine Laufzeit von $O(n)$, da die sortierte
	Liste der Vorlesungen genau einmal durchlaufen wird.

    \paragraph*{Beweis der Korrektheit:}
  Dieser Beweis ist analog zum Beweis der Vorlesung, da spätester Anfang das gleiche ist wie zeitigster Anfang (Rückwärtslesen der Intervalle).
  Sei $S$ eine optimale größtmögliche Teilmenge von $R$. \\
  Wir wollen zeigen: $|A| = |S|$. Seien $i_1, i_2, \hdots, i_k$ und $j_1, j_2, \hdots, j_m$ die Indizes der Intervalle von $A_i$ bzw. $S_j$ in zeitlich geordneter Reihenfolge.\\
  Behauptung: $a(i_r) \ge a(j_r) \forall r = 1, \hdots, n \Rightarrow |A| \ge |S|$\\
  Induktionsbeweis: \\
  IA: $r = 1$ : $a(i_1) \ge a(j_1)$ Trivial, da der Algorithmus immer das Intervall mit dem größten Anfangswert wählt. \\
  IS ($r \rightarrow r + 1$):
  Es gilt $a(i_r) \ge a(j_r)$ nach Induktionsbehauptung.
  Nach Auswahl von $I_{ir}$ steht das Intervall $I_{jr + 1}$ noch zur Verfügung, da $a(i_r) \ge a(j_r) \ge e(j_{r + 1})$
  Das nächste Intervall in $A$ ist eins mit $a(i_{r+1}) \ge a(j_{r+1})$. Also gilt $k \ge m$ und damit $|A| \ge |S|$
    

    
  \item Gegenbeispiel (Algorithmus wählt rotmarkiertes, optimale Lösung grün markiert):
 \vfill

\item Gegenbeispiel (Algorithmus wählt rotmarkiertes, optimale Lösung grün markiert):

 \vfill
\item Gegenbeispiel (Algorithmus wählt rotmarkiertes, optimale Lösung grün markiert):

 \vfill
\end{enumerate}
\vfill
\section{Autobahnfahrt}
Da das Problem unterspezifiziert ist, werden folgende Annahmen gemacht: Zu Beginn der Fahrt ist das Auto vollgetankt. Die Distanz zur nächsten Tankstelle ist maximal $n$ Kilometer.
Zwischen Berlin und Dortmund liegen $m$ Tankstellen: $t_1, t_2, \hdots, t_m$, die aufsteigend nach der Entfernung zu Berlin (Entfernung entlang der Strecke) geordnet sind. Berlin kann auch also 0.
Tankstelle $t_0$ bezeichnet werden.
$d(1)$ bezeichne die Entfernung von Berlin zur 1. Tankstelle, $d(2)$ die Entfernung zwischen der 1. und der 2. Tankstelle, $\hdots$, $d(m + 1)$ die Entfernung von der $m$-ten Tankstelle nach Dortmund.

\paragraph*{Eine minimale Menge $A$ von Tankstellen wird mit folgendem gierigen Algorithmus gefunden:}
Am Ort der Tankstelle $i$ nimm jeweils die am weitesten entfernte, noch
erreichbare Tankstelle ($\le n$ km) in die Zielmenge $A$ auf und suche
nun von dieser aufgenommenen Tankstelle die nächste aus.

\paragraph*{Beweis der Korrektheit:}
Da immer die am weitesten entfernte Tankstelle gewählt wurde, ist die Anzahl der gewählten Tankstellen bis zum Ziel offensichtlich minimal.
Nun ist die Frage, ob man mit dieser Strategie auch ohne liegen zu bleiben ans Ziel kommt. Dass dies prinzipiell möglich ist, wurde vorausgesetzt.

Angenommen von einer ausgewählten Tankstellen gäbe es keine
nachfolgende, die maximal $n$ Kilometer entfernt ist. Die nächste Tankstelle, die somit $> n$ Kilomenter entfernt ist, muss aber laut Voraussetzung eine
Vorgängertankstelle haben, die maximal $n$ Kilometer entfernt ist. Diese
Vorgängertankstelle muss gleichzeitig weiter als $n$ und näher als $n$
an der nächsten Tankstelle sein. Wegen dieses Widerspruchs
gibt es keine ausgewählte Tankstelle aus der Menge $A$, die keine erreichbare Nachfolge-Tankstelle hat.
Der Algorithmus terminiert, da man sich bei der Tankstellensuche stets näher an das Ziel bewegt.


Die Laufzeit beträgt $O(m)$, wobei $m$ die Anzahl der Tankstellen ist,
da in jedem Schritt die Anzahl der Tankstellen um eine Konstante
verkleinert wird. Diese Konstante ist gleich der Anzahl der Tankstellen
zwischen der letzten und der neu ausgewählten Tankstelle. (Die Anzahl der Tankstellen ist auch proportional zur Länge der Strecke).




\end{document}
