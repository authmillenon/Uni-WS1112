\documentclass[a4paper,10pt]{article}
\usepackage[utf8]{inputenc}
\usepackage[T1]{fontenc}
\usepackage{amsmath,amsfonts,amssymb,amscd,amsthm,xspace}
\usepackage[ngerman]{babel}
\usepackage{listingsutf8}
\usepackage{color}
\usepackage{geometry}
\usepackage{graphicx}
\usepackage{multicol}
\usepackage{algorithmic}

\geometry{a4paper, left=2cm,right=2cm,top=2cm,bottom=2cm}

\newcommand{\Authors}{Christian Cikryt (Di. 14-16), Jakob Pfender (Mi. 14-16)}
\title{H\"ohere Algorithmik - 8. \"Ubungsblatt}
\author{\Authors}
\date{\today}

\newcommand{\changefont}[3]{\fontfamily{#1} \fontseries{#2} \fontshape{#3} \selectfont}

\renewcommand{\thesection}{Aufgabe \arabic{section}:}
\renewcommand{\labelenumi}{(\theenumi)}
\renewcommand{\theenumi}{\alph{enumi}}
\renewcommand{\labelenumii}{(\theenumii)}
\renewcommand{\theenumii}{\roman{enumii}}

\definecolor{lgray}{gray}{0.95}
\definecolor{purple}{rgb}{0.498,0,0.3333}
\definecolor{identifier}{rgb}{0,0,0.1}
\definecolor{string}{rgb}{0.192,0,1}
\definecolor{comment}{rgb}{0.25,0.5,0.37}

\pagestyle{myheadings}
\oddsidemargin\oddsidemargin
\markright{\Authors}

\lstset{
	tabsize=4, 
	basicstyle=\footnotesize\fontfamily{pcr}\fontseries{m}\fontshape{n}\selectfont,
	breaklines=true,
	numbers=left,
	emphstyle=\textit, 
	language=Java,
	keywordstyle=\color{purple}\textbf, 
	identifierstyle=\color{identifier},
	stringstyle=\color{string},
	showstringspaces=false,
    escapeinside={((*}{*))},
	commentstyle=\color{comment},
	extendedchars=true,
	inputencoding=utf8/latin1
}

\begin{document}

\maketitle

\section{Das Offline-Minimum-Problem}
\begin{enumerate}
	\item	\begin{math}
			E_1 = 4\\
			E_2 = 3\\
			E_3 = 2\\
			E_4 = 6\\
			E_5 = 8\\
			E_6 = 1
		\end{math}
	\item	\begin{itemize}
			\item 	$m = 6$
			\item	$i = 1$
			\item	$K(j) = \{1,7\}$
			\item	$j = 6 \neq m+1$
			\item	$em[6] = 1$
			\item	$l = 7$
			\item	$K(7) = \{5\}$
			\item	$K(7) = \{1,7,5\}$
			\item	$K(6)$ existiert nicht mehr
			\item	$i = 2$
			\item	$K(j) = \{9,2,6\}$
			\item	$j = 3 \neq m+1$
			\item	$em[3] = 2$
			\item	$l = 4$
			\item	$K(4) = \{\}$
			\item	$K(4) = \{9,2,6\}$
			\item	$K(3)$ existiert nicht mehr
			\item	$i = 3$
			\item	$K(j) = \{3\}$
			\item	$j = 2 \neq m+1$
			\item	$em[2] = 3$
			\item	$l = 4$
			\item	$K(4) = \{9,2,6\}$
			\item	$K(4) = \{9,2,6,3\}$
			\item	$K(2)$ existiert nicht mehr
			\item	$i = 4$
			\item	$K(j) = \{4,8\}$
			\item	$j = 1 \neq m+1$
			\item	$em[1] = 4$
			\item	$l = 4$
			\item	$K(4) = \{9,2,6,3\}$
			\item	$K(4) = \{9,2,6,3,4,8\}$
			\item	$K(1)$ existiert nicht mehr
			\item	$i = 5$
			\item	$K(j) = \{1,7,5\}$
			\item	$j = 7 = m+1$
			\item	$i = 6$
			\item	$K(j) = \{9,2,6,3,4,8\}$
			\item	$j = 4 \neq m+1$
			\item	$em[4] = 6$
			\item	$l = 5$
			\item	$K(5) = \{\}$
			\item	$K(5) = \{9,2,6,3,4,8\}$
			\item	$K(4)$ existiert nicht mehr
			\item	$i = 7$
			\item	$K(j) = \{1,7,5\}$
			\item	$j = 7 = m+1$
			\item	$i = 8$
			\item	$K(j) = \{9,2,6,3,4,8\}$
			\item	$j = 5 \neq m+1$
			\item	$em[5] = 8$
			\item	$l = 7$
			\item	$K(7) = \{5\}$
			\item	$K(7) = \{5,9,2,3,4,8\}$
			\item	$K(5)$ existiert nicht mehr
			\item	$i = 9$
			\item	$K(j) = \{5,9,2,3,4,8\}$
			\item	$j = 7 = m+1$
		\end{itemize}
		Der Algorithmus arbeitet korrekt, weil er die Elemente
		in aufsteigender Reihenfolge durchläuft und für jedes
		Element, das sich in einer Menge $K(j)$ befindet, die
		nicht die $m+1$-te Menge ist, genau eine
		\verb!extract-min!-Operation aus der Menge der
		Operationen entfernt und danach die Menge $K(j)$ mit der
		darauffolgenden Menge $K(l)$ vereinigt. Somit folgt auf
		jede Menge außer der $m+1$-ten immer ein
		\verb!extract-min!-Aufruf, so dass für jeden
		\verb!extract-min!-Aufruf immer das kleinste Element,
		das sich zu diesem Zeitpunkt in der Menge befindet,
		entfernt wird.
	\item
\end{enumerate}
\section{Amortisierte Analyse}
\begin{enumerate}
 \item
Zählt man von $0$ bis $n$ hoch, so ändert sich die Einerstelle im binären Zählwerk in jedem Schritt, die Zweierstelle in jedem zweiten Schritt, die Viererstelle bei jedem vierten Schritt usw.
Somit ändert sich die Einerstelle beim Zählen bis $n$ genau $n$ mal, die Zweierstelle $\lfloor{n/2}\rfloor$ mal usw.
Für die Gesamtkosten gilt somit:
\[ \sum_{i = 0}^{\lceil\log_2 n\rceil} \left\lfloor\frac{n}{2^i}\right\rfloor\]
Die Gesamtkosten liegen somit in $O(n)$. Weil:
\begin{align*}
 \sum_{i = 0}^{\lceil\log_2 n\rceil} \left\lfloor\frac{n}{2^i}\right\rfloor &\leq \sum_{i = 0}^{\lceil\log_2 n\rceil} \frac{n}{2^i} =  n * \sum_{i = 0}^{\lceil\log_2 n\rceil} \frac{1}{2^i} \\
 \leq  n * \sum_{i = 0}^{\infty} \frac{1}{2^i} &= 2 * n \\
\end{align*}
Für die amortisierten Kosten pro Zählvorgang gilt somit: 
\[ 2 * n / n = 2 = O(1) \]

\item
Die Länge des Arrays wird verdoppelt, wenn in ein volles Array eingefügt werden soll. Wird ein Element aus einem halbvollen Array gelöscht so wird, die Größe des Arrays halbiert.
Der Zugriff bzw. das Ersetzen eines Elements ist für unsere Analyse uninteressant, da beides aufgrund der Tatsache, dass es sich um ein Array handelt, konstant ist. Durch das Halbieren wird verhindert, dass unnötiger Speicherplatz belegt wird.
Das Verdoppeln ist nötig, um die Elemente unterzubekommen. Theoretisch könnte auch um andere Faktoren oder Konstanten erweitert bzw. verkleinert werden. Die Abhängigkeit von der momentanen Arraygröße stellt den Trade-off zwischen Laufzeit und Speichernutzung dar.
Wie im Folgenden analysiert wird, kosten die Einfüge- und Löschoperationen amortisiert konstante Zeit. Anmerkung: Natürlich lässt sich immer eine Folge von Lösch- und Einfügeoperationen finden, die Laufzeit linear werden lässt (in unserem Fall abwechselndes Einfügen
und Löschen bei halbvollem/vollem Array).
Sowohl das Vergrößern als auch das Halbieren hat kosten $O(n)$, da ein neues Array der Länge $O(n)$ angelegt werden muss und umkopiert werden muss. Alle davon unbetroffenen Einfüge- und Löschoperationen brauchen konstante Zeit.
Betrachten wir zunächst eine Folge von Einfügeoperationen: 


\end{enumerate}

\end{document}
