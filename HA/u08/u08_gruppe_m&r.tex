\documentclass[a4paper,10pt]{article}
\usepackage[utf8]{inputenc}
\usepackage[T1]{fontenc}
\usepackage{amsmath,amsfonts,amssymb,amscd,amsthm,xspace}
\usepackage[ngerman]{babel}
\usepackage{listingsutf8}
\usepackage{color}
\usepackage{geometry}
\usepackage{graphicx}
\usepackage{multicol}
\usepackage{pst-tree}
\usepackage{algorithmic}
\usepackage{cancel}

\geometry{a4paper, left=2cm,right=2cm,top=2cm,bottom=2cm}

\newcommand{\Authors}{Martin Lenders (Mi. 14-16), Ralf M\"uller-Zimmermann (Di. 14-16)}
\title{H\"ohere Algorithmik - 8. \"Ubungsblatt}
\author{\Authors}
\date{\today}

\newcommand{\changefont}[3]{\fontfamily{#1} \fontseries{#2} \fontshape{#3} \selectfont}

\renewcommand{\thesection}{Aufgabe \arabic{section}:}
\renewcommand{\labelenumi}{(\theenumi)}
\renewcommand{\theenumi}{\alph{enumi}}
\renewcommand{\labelenumii}{(\theenumii)}
\renewcommand{\theenumii}{\roman{enumii}}

\definecolor{lgray}{gray}{0.95}
\definecolor{purple}{rgb}{0.498,0,0.3333}
\definecolor{identifier}{rgb}{0,0,0.1}
\definecolor{string}{rgb}{0.192,0,1}
\definecolor{comment}{rgb}{0.25,0.5,0.37}

\pagestyle{myheadings}
\oddsidemargin\oddsidemargin
\markright{\Authors}

\lstset{
	tabsize=4, 
	basicstyle=\footnotesize\fontfamily{pcr}\fontseries{m}\fontshape{n}\selectfont,
	breaklines=true,
	numbers=left,
	emphstyle=\textit, 
	language=Java,
	keywordstyle=\color{purple}\textbf, 
	identifierstyle=\color{identifier},
	stringstyle=\color{string},
	showstringspaces=false,
    escapeinside={((*}{*))},
	commentstyle=\color{comment},
	extendedchars=true,
	inputencoding=utf8/latin1
}
\psset{nodesep=2pt,levelsep=2em,treesep=2em}

\begin{document}

\maketitle

\section{Das Offline-Minimum-Problem}
\begin{enumerate}
\item   
\item   
\item   
\end{enumerate}

\section{Amortisierte Analyse}
\begin{enumerate}
\item   Betrachten wir den Verlauf des Zählwerkes, stellen wir fest, dass die 1er-Stelle sich bei jedem Zählschritt sich 
        ändert, die 2er-Stelle bei jedem zweiten, die 4er-Stelle bei jedem vierten etc. 
        \paragraph*{Beweis} Bezeichnen wir die Schritte der Inkrementierung jeweils mit der Zahl im Ergebnis (Schritt 1: $0 \to 1$, Schritt 2: $1 \to 10$, Schritt 3: $10 \to 11$, ...), so ist die Trivialität der Behauptung ersichtlich, da ein Umschalten jedesmal stattfindet, wenn das Ergebnis jeweils durch die Wertigkeit der Stelle teilbar ist. Damit ist auch der Schritt durch die Wertigkeit der Stelle teilbar \hfill $\square$
        
        Für $\left|\{0,1\}\right|^{k-1} \leq n < \left|\{0,1\}\right|^{k}$ ($n$ habe also $k \geq 1$ Stellen) ergibt sich als Gesamtstromkosten also
        \[\sum\limits_{i=0}^{k} \left\lfloor\frac{n}{2^i}\right\rfloor\]
        Die amortisierten Kosten pro Zählvorgang ergibt sich aus dem Durchschnitt für jede Einzeloperation unter Betrachtung der Gesamtkosten.
        \begin{align*}
         \sum\limits_{i=0}^{k} \left\lfloor\frac{n}{2^i}\right\rfloor &\leq \sum\limits_{i=0}^{k} \frac{n}{2^i} \\
                                                                      &\leq \sum\limits_{i=0}^{\infty} \frac{n}{2^i} \\
                                                                      &= n \cdot \sum\limits_{i=0}^{\infty} \frac{1}{2^i} \\
                                                                      &= 2n
        \end{align*}
        Der Durchschnitt und die amortisierten Kosten sind damit also $\frac{2n}{n} = 2 = O(1)$.
\item   Wir erstellen einen ADT \texttt{DynArray}, der zur Verwaltung eines solchen dynamisch wachsenden Arrays.
        \texttt{DynArray} besitzt zwei Operationen: 
        \texttt{get(i)} zum Lesen eines Wertes an der Stelle $i \geq 0$ und \texttt{set(i,v)} zum Setzen eines Wertes $v$ an der Stelle $i \geq 0$. 
        Wir gehen davon aus, dass das Array indexweise von 0 ab gefüllt wird, bzw. von hinten indexweise geleert wird (was dem Verhalten z. B. in einem Stack entspräche).
        Das Array hat zunächst eine beliebige feste Größe und alle Zellen sind leer.
        Wird für \texttt{set(i,v)} ein ein $i$, das größer gleich der aktuellen Größe des Arrays ist, gewählt, wird die Größe des Arrays verdoppelt, indem ein neues Array entsprechender Größe erstellt wird und die Daten kopiert werden. 
        Wird \texttt{set(i,v)} durch $v = \mathtt{None}$ mit Index $i = \frac{n}{2}$ geleert, wird die Größe wieder halbiert, indem ein neues Array entsprechender Größe erstellt wird und die Daten kopiert werden.
        Verweist \texttt{get(i)} auf einen Wert größer gleich der aktuelle Größe des Arrays, wird ein Fehler geworfen.
        \begin{lstlisting}[language=Python,numbers=none]
class DynArray:
    a = array[N]
    
    def get(i):
        if i >= len(a):
            raise Fehler!
        return a[i]
    
    def set(i,v):
        if i >= len(a):
            b = array[len(a)*2]
            for a[j] in a:
                b[j] = a[j]
            a = b
        else if i < len(a)/2 and a[len(a)/2+1:] is empty:
            b = array[len(a)/2]
            for a[j] in a:
                b[j] = a[j]
            a = b
        a[i] = v
        \end{lstlisting}
        Zur Berechnung der amortisierten Laufzeit verwenden wir die Buchhalter-Methode und setzen das Guthaben für die \texttt{set}-Operation auf 3.
        Die \texttt{get}-Operation ist für diese Analyse auch nicht von Interesse.
        Fügen wir zunächst $N-1$ Elemente in das Array ein, so erhalten wir ein Kontoguthaben von $2(N-1)$, da wir für jedes Einfügen 3 einzahlen und 1 für das Einfügen selbst wieder entnehmen. Das Verdoppeln  kostet uns dann $N-1$, da wir $N-1$ Elemente umschreiben, es kommt also beim nächsten Einfügen zu Kosten von $N$ (Verdoppeln + normales Einfügen), wir zahlen aber auch hier für die Operation 3 auf das Konto ein. Der Kontostand ist dann
        \[2(N-1) + 3 - N = N + 1.\]
        Füllen wir das Vergrößerte Array, zahlen wir $3(N-2)$ auf das Konto ein, entnehmen $N-2$ für das Einfügen und haben somit einen Kontostand von
        \[N + 1 + 2(N-2) = 3(N-1).\]
        Fügen wir nun wieder ein neues Element hinzu Stabiliert sich der Kontostand:
        \[3(N-1) + 3 - 2N = N + 1\]
        Wir brauchen für das Einfügen von $N$ Elementen also amortisiert immer $O(N)$ Zeit, was für jede Einzeloperation $O(1)$ Zeit bedeutet.
        Für das Halbieren kann genauso verfahren werden.
\end{enumerate}

\section{Wahrscheinlichkeitsrechnung und Hashing}
\begin{enumerate}
\item   Die Wahrscheinlichkeit, dass der $(k+1)$-te der insgesamt $n$ Teilnehmer sein eigenes Paket zieht ist
        \[
            \frac{1}{n-k} \prod\limits_{i=0}^{k-1} \frac{n-i-1}{n-i} = \frac{1-\frac{m}{n}}{n-m} = \frac{1}{n}
        \]
        Damit ist die erwartete Anzahl an Teilnehmern, die ihr eigenes Geschenk ziehen
        \begin{align*}
            E[X] &= \sum\limits_{i = 1}^{n} X_i p_i = \sum\limits_{i = 1}^{n} \frac{1}{n} = 1
        \end{align*}
        Mit Zufallsvariable $X = \{X_i = 1\ |\ i \in \{1, ..., n\}\}$ als Anzahl der gezogenen Pakete jedes Teilnehmers.
\item   Wir wollen die Schlüsselmenge $K$ auf $N$ Einträge verteilen, wobei $|K| \leq (n-1)N + 1 > N$. 
        Nach dem \emph{Verschärften Schubfachprinzip} gibt es also mindestens einen Eintrag, in den $|S| > \frac{|K| - 1}{N}$ Schlüssel aus der Menge $S$ gehasht werden.
        \begin{align*}
            \frac{|K| - 1}{N} &\geq \frac{(n-1)N+1-1}{N} = \frac{(n-1)\cancel{N}}{\cancel{N}} \\
        \end{align*}
        Es also mindestens einen Eintrag, in den $|S| > (n-1) \Leftrightarrow |S| \geq n$ Schlüssel gehasht werden. Für die worst-case-Laufzeit bedeutet das, dass alle Schlüssel in diesem Eintrag durchsucht werden müssen, was sie auf $O(|S|)$ festlegt.
\end{enumerate}
\end{document}