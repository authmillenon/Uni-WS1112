\documentclass[a4paper,10pt]{scrartcl}
\usepackage[utf8x]{inputenc}
\usepackage[T1]{fontenc}
\usepackage{amsmath,amsfonts,amssymb,amscd,amsthm,xspace}
\usepackage[ngerman]{babel}
\usepackage{listingsutf8}
\usepackage{color}
\usepackage{geometry}
\usepackage{graphicx}
\usepackage{multicol}
%\usepackage{pst-tree}

\geometry{a4paper, left=2cm,right=2cm,top=2cm,bottom=2cm}

\newcommand{\Authors}{Christian Cikryt (Di. 14-16), Jakob Pfender (Mi. 14-16)}
\title{H\"ohere Algorithmik - 3. \"Ubungsblatt (Tutor Tillmann Miltzow)}
\author{\Authors}
\date{\today}

\newcommand{\changefont}[3]{\fontfamily{#1} \fontseries{#2} \fontshape{#3} \selectfont}

\renewcommand{\thesection}{Aufgabe \arabic{section}}
\renewcommand{\labelenumi}{(\theenumi)}
\renewcommand{\theenumi}{\alph{enumi}}
\renewcommand{\labelenumii}{(\theenumii)}
\renewcommand{\theenumii}{\roman{enumii}}

\definecolor{lgray}{gray}{0.95}
\definecolor{purple}{rgb}{0.498,0,0.3333}
\definecolor{identifier}{rgb}{0,0,0.1}
\definecolor{string}{rgb}{0.192,0,1}
\definecolor{comment}{rgb}{0.25,0.5,0.37}

\pagestyle{myheadings}
\oddsidemargin\oddsidemargin
\markright{\Authors}

\lstset{
	tabsize=4, 
	frame=tlrb, 
	basicstyle=\footnotesize\changefont{pcr}{m}{n},
	breaklines=true,
	numbers=left,
	emphstyle=\textit, 
	language=Java,
	keywordstyle=\color{purple}\textbf, 
	identifierstyle=\color{identifier},
	stringstyle=\color{string},
	backgroundcolor=\color{lgray},
	showstringspaces=false,
	commentstyle=\color{comment},
	extendedchars=true,
	inputencoding=utf8/latin1
}
%\psset{nodesep=2pt,levelsep=2em,treesep=2em}

\begin{document}

\maketitle

\section{Schnelle Matrizenmultiplikation}
\begin{enumerate}
\item	Da bei der Multiplikation zweier $n \times n$~-Matrizen jedes
	Element der ersten mit $n$ Elementen der zweiten Matrix
	multipliziert werden muss, werden insgesamt $n^2 \cdot n = n^3$
	Multiplikationen sowie $n \cdot n \cdot n-1 = (n-1)n^2 = O(n^3)$
	Additionen durchgeführt.

\item
  TBD

\item
  Es werden 7 Mulitplikationen gebraucht. Jede dieser rekursiven Multiplikationen wird auf Eingaben der Größe $n/2$ durchgeführt. In jedem Schritt müssen die Elemente der Matrix berechnet werdne, was Kosten $\Theta(n^2)$ hat. Es ergibt sich folgende Rekursionsformel:
\[T(n) = 7 * T(n/2) + \Theta(n^2)\]
Das Mastertheorem kann mit $a=7$ und $b=2$ zur Lösung angewendet werden. $\epsilon$ ist in dem Fall $\log_2 7 - 2$, was ungefähr $0,807$ ist.
Es handelt sich um den 1. FAll des MT:
\[ T(n) = \Theta (n^{\log_2 7}) = \Theta(n^{2,8})\]
\end{enumerate}

\section{Rekursionsgleichungen}
\begin{enumerate}
\item	\begin{align}
	T(n) = T(9n/10)+n \\
	a = 1, b = \frac{10}{9}, f(n) = n \\
	\log _b a = log _{10} 1 = 0 \\
	\Rightarrow f(n) = n = \Theta(n^{\log _b a}) \\
	\Rightarrow T(n) = \Theta(n^0 * \log n) = \Theta(\log n) \tag{Fall 2 des Master-Theorems}
	\end{align}
\item	\begin{align}
	T(n) = T(n-a) + T(a) + n, a \geq 1 \\
	\end{align}
	MT nicht anwendbar.
	$T(a)$ sei konstant. Dann gilt:
	\begin{align}
	 T(n) = T(n-a) + T(a) + n = \\
	 = T(n - 2*a) + 0(1) + n - a = \\
	 = T(n - 3*a) + O(1) + n - 2*a + n - a \\
	\end{align}
	Daran sieht man, dass in jedem Schritt O(n) Aufwand dazu kommt. Es gibt insgesamt $\frac{n}{a} = O(n)$ Schritte.
	Somit gilt insgesamt.
	\[ T(n) = O(n^2) \]
\item	\begin{align}
	T(n) = T(\sqrt{n}) + 1 \\
	\end{align}
	Das Mastertheorem ist nicht anwendbar. In jedem Schritt hat konstanten Aufwand.Es gibt $\sqrt(n)$-Schritte. Somit gilt
  \[ T(n) = O(\sqrt(n)\]
\item	\begin{align}
	T(n) = 2T(n/4) + \sqrt{n} \\
	a = 2, b = 4, f(n) = \sqrt{n} \\
	\log _b a = \log _4 2 = \frac{1}{2} \\
	\Rightarrow \exists \varepsilon > 0, f(n) = \sqrt{n} = O(n^{\log
	_b a+\varepsilon}), \exists c < 1, 2f(n/4) \leq cf(n) \\
	\Rightarrow T(n) = \Theta(f(n)) = \Theta(\sqrt{n})\tag{Fall 3 des Master-Theorems}
	\end{align}
\item	\begin{align}
	T(n) = 7T(n/3) + n^2 \\
	a = 7, b = 3, f(n) = n^2 \\
	\log _b a = \log _3 7 \approx 1.77 \\
	\Rightarrow \exists \varepsilon > 0, f(n) = n^2 = O(n^{\log _b a-\varepsilon}) \\
	\Rightarrow T(n) = \Theta(n^{log _3 7}) \tag{Fall 1 des Master-Theorems}
	\end{align}
\item	\begin{align}
	T(n) = 2T(n/2) + n \log n \\
	a = 2, b = 2, f(n) = n \log n \\
	\log _b a = \log _2 2 = 1
	\end{align}
	MT nicht anwendbar, da $\nexists \varepsilon > 0, f(n) = n \log
	n = \Omega(n^{\log _b a+\varepsilon}), \nexists \varepsilon > 0,
	f(n) = n \log n = O(n^{\log _b a-\varepsilon}), f(n) \neq
	\Theta(n^{\log _b a}).$


	Ein gedachter Baum hätte $\log n$ Höhe, da $n$ in jedem Schritt halbiert wird und da in jedem Schritt $2 * n * \log n$ Aufwand verursacht wird, gilt insgesamt:
\[ T(n) = O(n^2 * \log n)\]
\item	\begin{align}
	T(n) = T(n-1) + \frac{1}{n} \\
	a = 1, b = 1, f(n) = \frac{1}{n} \\
	\log _b a = \log _1 1 = 1 \\
	\Rightarrow f(n) = \frac{1}{n} = O(n^{\log _b a}) \\
	\Rightarrow T(n) = \Theta(n^1) \tag{Fall 1 des Master-Theorems}
	\end{align}
\end{enumerate}

\section{Implementierung}
\begin{enumerate}
\item  

        \lstinputlisting[linerange={5-19},gobble=4,firstnumber=9,caption={SchoolMethod.java}]{prog_c&j/src/SchoolMethod.java}
      
        \lstinputlisting[linerange={10-36},gobble=4,firstnumber=6,caption={Karatsuba.java}]{prog_c&j/src/Karatsuba.java}

Der Vergleich der Laufzeit geschieht voll automatisch ausgewertet in der main-Methode.
Dabei kam raus dass die Karatsuba erst ab einer Größe von ca. $2^{600}$ (es gab leichte Schwankungen) effizienter ist.

 \lstinputlisting[linerange={8-26},gobble=4,firstnumber=6,caption={MultiplicationMain.java}]{prog_c&j/src/MultiplicationMain.java}

\item
       \lstinputlisting[linerange={9-14},gobble=4,firstnumber=6,caption={MKaratsuba.java}]{prog_c&j/src/MKaratsuba.java}
  Das Finden das optimalen M geschah auch automatisch. Der Range wurde manuell leicht angepasst (da gesehen wurde, dass er sehr klein ist wurde die Abstände zwischen den Testzahlen nach oben hin immer größer).
\lstinputlisting[linerange={28-38},gobble=4,firstnumber=6,caption={MultiplicationMain.java}]{prog_c&j/src/MultiplicationMain.java}
Es wurde somit $M = 16$ als optimal ermittelt.
Der vollständigen Quellcode findet sich in der E-Mail.

\end{enumerate}
\end{document}
