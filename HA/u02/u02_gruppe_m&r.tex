\documentclass[a4paper,10pt]{scrartcl}
\usepackage[utf8x]{inputenc}
\usepackage[T1]{fontenc}
\usepackage{amsmath,amsfonts,amssymb,amscd,amsthm,xspace}
\usepackage[ngerman]{babel}
\usepackage{listingsutf8}
\usepackage{color}
\usepackage{geometry}
\usepackage{graphicx}
\usepackage{multicol}
\usepackage{pst-tree}

\geometry{a4paper, left=2cm,right=2cm,top=2cm,bottom=2cm}

\newcommand{\Authors}{Martin Lenders (Mi. 14-16), Ralf M\"uller-Zimmermann (Di. 14-16)}
\title{H\"ohere Algorithmik - 1. \"Ubungsblatt}
\author{\Authors}
\date{\today}

\newcommand{\changefont}[3]{\fontfamily{#1} \fontseries{#2} \fontshape{#3} \selectfont}

\renewcommand{\thesection}{Aufgabe \arabic{section}}
\renewcommand{\labelenumii}{(\theenumii)}
\renewcommand{\theenumi}{\alph{enumi}}
\renewcommand{\labelenumii}{(\theenumii)}
\renewcommand{\theenumii}{\roman{enumii}}

\definecolor{lgray}{gray}{0.95}
\definecolor{purple}{rgb}{0.498,0,0.3333}
\definecolor{identifier}{rgb}{0,0,0.1}
\definecolor{string}{rgb}{0.192,0,1}
\definecolor{comment}{rgb}{0.25,0.5,0.37}

\pagestyle{myheadings}
\oddsidemargin\oddsidemargin
\markright{\Authors}

\lstset{
	tabsize=4, 
	frame=tlrb, 
	basicstyle=\footnotesize\changefont{pcr}{m}{n},
	breaklines=true,
	numbers=left,
	emphstyle=\textit, 
	language=Java,
	keywordstyle=\color{purple}\textbf, 
	identifierstyle=\color{identifier},
	stringstyle=\color{string},
	backgroundcolor=\color{lgray},
	showstringspaces=false,
	commentstyle=\color{comment},
	extendedchars=true,
	inputencoding=utf8/latin1
}
\psset{nodesep=2pt,levelsep=2em,treesep=2em}

\begin{document}

\maketitle

\section{Gewichteter Median}
\begin{enumerate}
\item   
\item   
\item   
\end{enumerate}


\section{Analyse des BFPRT-Algorithmus}
Sei $k > 0$ eine ungerade Zahl. So ist die Laufzeit des BFPRT-Algorithmus mit einer Aufteilung in $k$er-Blöcke für $n \geq 100$
\[T(n) \leq cn + T\left(\left\lceil\frac{n}{k}\right\rceil\right) + T\left(\frac{3n}{4}\right)\tag{s. VL}\]
für $n < 100$ wird immer Bruteforce verwendet. Die Laufzeit dafür ist nicht von $k$ abhängig und immer $O(1)$.

\begin{description}
\item[Induktionsbehauptung] $\exists\alpha > 0{:}\ T(n+1) < \alpha (n+1)$
\item[Induktionsanfang] ($n < 100$)
    \begin{itemize}
     \item Für $n < 100$ ist $T(n) = O(1) \Rightarrow$ es lässt sich immer ein $\alpha$ finden, so dass $\alpha n > T(n)$ (da $T(n)$ konstant und damit nicht von $n$ abhängig).
    \end{itemize}
\item[Induktionsvoraussetzung] $\exists\alpha > 0{:}\ T(n) < \alpha n$
\item[Induktionsschritt]
\begin{align*}
 T(n)   &\leq cn + T\left(\left\lceil\frac{n}{k}\right\rceil\right) + T\left(\frac{3n}{4}\right) \\
        &\overset{\text{IA}}{\leq} cn + \alpha \left\lceil\frac{n}{k}\right\rceil + \alpha \frac{3n}{4} \\
        &\leq cn + \alpha \left(\frac{n}{k} + 1\right) + \alpha \frac{3n}{4}\\
        &= cn + \alpha n \left(\frac{1}{k} + \frac{3}{4}\right) + \alpha \\
        &= cn + \alpha n \left(\frac{3k + 4}{4k}\right) + \alpha \\
        &\overset{!}{\leq} \alpha n
\end{align*}
\begin{align*}
 \alpha n &\geq cn + \alpha n \frac{3k + 4}{4k} + \alpha \tag*{| da $k > 0$}\\
 \alpha n \frac{k - 4}{4k} &\geq cn + \alpha \tag*{| $\cdot \frac{4k}{n}$}\\
 \alpha (k - 4) &\geq 4kc + \underbrace{\frac{4k\alpha}{n}}_{\leq \frac{\alpha}{k}}\\
 \Rightarrow 4kc + \frac{\alpha}{k} &\geq 4kc + \frac{4k\alpha}{n}
\end{align*}
\begin{align*}
  \alpha &\geq 4kc + \frac{\alpha}{k} \\
  \Leftrightarrow \alpha &\geq k^2c
\end{align*}
\end{description}

\section{Schmutzige Tricks mit dem Einheitskostenmaß}
\begin{enumerate}
\item   \renewcommand{\labelenumii}{(\theenumii)}
        \renewcommand{\theenumii}{\arabic{enumii}}
        \begin{enumerate}
        \item   $a = b \Leftrightarrow x = y$ ist trivial, da gilt, wenn 
                $x = y \Leftrightarrow \forall i \in \{1, ..., n\}{:}\ x_i = y_i$\\
                \[\Rightarrow \sum\limits_{i=1}^{n} x_{i}u^{n-i+1} = \sum\limits_{i=1}^{n} y_{i}u^{n-i+1}\]
        \item   $a < y \Leftrightarrow x$ ist lexikographisch kleiner als $y$:\\
                (Beweis durch Induktion über die Anzahl der sich unterscheidenden Elemente $m$, $1 \leq m \leq n$)
                \begin{description}
                \item[Induktionsanfang] $(m=1)$ Sei $x_1 < y_1$ und $\forall k \in \{2, ..., n\}{:}\ x_k = y_k$. Dann ist
                    \[x_1 u^{n} < y_1 u^{n}\]
                    während alle anderen Werte nach (1) 
                    $\forall k \in \{2, ..., n\}{:}\ x_k u^{n-k+1} = y_k u^{n-k+1}$
                    \[\Rightarrow \sum\limits_{i=1}^{n} x_{i}u^{n-i+1} < \sum\limits_{i=1}^{n} y_{i}u^{n-i+1}\]
                \item[Induktionsvoraussetzung] $(1 < m \leq n)$ Sei ein Wert $j \in \{1, ..., n\}$, 
                \end{description}
        \end{enumerate}
\item   
\end{enumerate}
\end{document}