\documentclass[a4paper,10pt]{scrartcl}
\usepackage[utf8x]{inputenc}
\usepackage[T1]{fontenc}
\usepackage{amsmath,amsfonts,amssymb,amscd,amsthm,xspace}
\usepackage[ngerman]{babel}
\usepackage{listingsutf8}
\usepackage{color}
\usepackage{geometry}
\usepackage{graphicx}
\usepackage{multicol}
\usepackage{pst-tree}

\geometry{a4paper, left=2cm,right=2cm,top=2cm,bottom=2cm}

\newcommand{\Authors}{Martin Lenders (Mi. 14-16), Ralf M\"uller-Zimmermann (Di. 14-16)}
\title{H\"ohere Algorithmik - 1. \"Ubungsblatt}
\author{\Authors}
\date{\today}

\newcommand{\changefont}[3]{\fontfamily{#1} \fontseries{#2} \fontshape{#3} \selectfont}

\renewcommand{\thesection}{Aufgabe \arabic{section}}
\renewcommand{\labelenumi}{(\theenumi)}
\renewcommand{\theenumi}{\alph{enumi}}
\renewcommand{\labelenumii}{(\theenumii)}
\renewcommand{\theenumii}{\roman{enumii}}

\definecolor{lgray}{gray}{0.95}
\definecolor{purple}{rgb}{0.498,0,0.3333}
\definecolor{identifier}{rgb}{0,0,0.1}
\definecolor{string}{rgb}{0.192,0,1}
\definecolor{comment}{rgb}{0.25,0.5,0.37}

\pagestyle{myheadings}
\oddsidemargin\oddsidemargin
\markright{\Authors}

\lstset{
	tabsize=4, 
	basicstyle=\footnotesize\fontfamily{pcr}\fontseries{m}\fontshape{n}\selectfont,
	breaklines=true,
	numbers=left,
	emphstyle=\textit, 
	language=Python,
	keywordstyle=\color{purple}\textbf, 
	identifierstyle=\color{identifier},
	stringstyle=\color{string},
	showstringspaces=false,
    escapeinside={((*}{*))},
	commentstyle=\color{comment},
	extendedchars=true,
	inputencoding=utf8/latin1
}
\psset{nodesep=2pt,levelsep=2em,treesep=2em}

\begin{document}

\maketitle

\section{Gewichteter Median}
\begin{enumerate}
\item   
\item   
\item   
\end{enumerate}


\section{Analyse des BFPRT-Algorithmus}
Sei $k > 0$ eine ungerade Zahl. So ist die Laufzeit des BFPRT-Algorithmus mit einer Aufteilung in $k$er-Blöcke für $n \geq 100$
\[T_k(n) \leq cn + T_k\left(\left\lceil\frac{n}{k}\right\rceil\right) + T_k\left(\frac{3n}{4}\right)\tag{s. VL}\]
für $n < 100$ wird immer Bruteforce verwendet. Die Laufzeit dafür ist nicht von $k$ abhängig und immer $O(1)$.

\begin{description}
\item[Induktionsbehauptung] $\exists\alpha > 0{:}\ T_k(n) \leq \alpha n$
\item[Induktionsvoraussetzung] $\exists\alpha > 0{:}\ T_k(n-1) \leq \alpha (n-1)$
\item[Induktionsanfang] ($n < 100$)
    \begin{itemize}
     \item Für $n < 100$ ist $T_k(n) = O(1) \Rightarrow$ es lässt sich immer ein $\alpha$ finden, so dass $\alpha n \geq T_k(n)$ (da $T(n)$ konstant und damit nicht von $n$ abhängig).
    \end{itemize}
\item[Induktionsschritt]
\[
\begin{array}{rl|rl}
 T_k(n)   &\leq cn + T_k\left(\left\lceil\frac{n}{k}\right\rceil\right) + T_k\left(\frac{3n}{4}\right) &
                \alpha n &\geq cn + \alpha n \frac{3k + 4}{4k} + \alpha\\
        &\overset{\text{IA}}{\leq} cn + \alpha \left\lceil\frac{n}{k}\right\rceil + \alpha \frac{3n}{4} &
                \alpha n \frac{k - 4}{4k} &\geq cn + \alpha\\
        &\leq cn + \alpha \left(\frac{n}{k} + 1\right) + \alpha \frac{3n}{4} &
                \alpha (k - 4) &\geq 4kc + \underbrace{\frac{4k\alpha}{n}}_{\leq \frac{\alpha}{k}} \qquad \text{(für $k < n$)} \\
        &= cn + \alpha n \left(\frac{1}{k} + \frac{3}{4}\right) + \alpha  &
                \alpha (k - 4) &\geq 4kc + \frac{\alpha}{k}\\
        &= cn + \alpha n \left(\frac{3k + 4}{4k}\right) + \alpha &
                \frac{\alpha (k^2 - 4k) - \alpha}{k} &\geq 4kc\\
        &\overset{!}{\leq} \alpha n &
                \alpha (k^2 - 4k - 1) &\geq 4k^2c\\
        && \alpha &\geq \frac{4k^2}{k^2 - 4k - 1} c
\end{array}
\]
$k$ und $c$ sind in unserem Fall beides Konstanten. Der BFPRT-Algorithmus läuft also für alle gegebenen $k$ (also auch $3$ und $7$) in $O(n)$ Zeit.
\end{description}

\begin{description}
\item[Zu zeigen:] Die Laufzeit des BFPRT-Algorithmus mit 3er-Blöcken $T_3(n)$ liegt in $\Omega(n)$\\
    $\overset{+ \text{Beweis oben}}{\Rightarrow}$ $T_3(n) = \Theta(n)$
\item[Induktionsbehauptung] $\exists\alpha > 0{:}\ T_3(n) \geq \alpha n$
\item[Induktionsvoraussetzung] $\exists\alpha > 0{:}\ T_3(n-1) \geq \alpha (n-1)$
\item[Induktionsanfang] ($n < 100$)
    \begin{itemize}
     \item Für $n < 100$ ist $T_3(n) = \Omega(1) \Rightarrow$ es lässt sich immer ein $\alpha$ finden, so dass $\alpha n \leq T_3(n)$ (da $T_3(n)$ konstant und damit nicht von $n$ abhängig).
    \end{itemize}
\item[Induktionsschritt]
\[
\begin{array}{rl|rl}
 T_3(n) &\geq cn + T_3\left(\left\lceil\frac{n}{3}\right\rceil\right) + T_3\left(\frac{3n}{4}\right) &
            \alpha n &\leq cn + \alpha n \left(\frac{13}{12}\right) \\
        &\overset{\text{IA}}{\geq} cn + \alpha \left\lceil\frac{n}{3}\right\rceil + \alpha \frac{3n}{4} &
            - \frac{1}{12} \alpha n &\leq cn \\
        &\geq cn + \alpha \frac{n}{3} + \alpha \frac{3n}{4} &
            \alpha &\geq -12 c \\
        &= cn + \alpha n \left(\frac{13}{12}\right) \\
        &\overset{!}{\geq} \alpha n
\end{array}
\]
Da $\alpha > 0$ sein soll, können wir alle güligen $\alpha$ wählen $\Rightarrow T_3(n) = \Theta(n)$ 
\end{description}


\section{Schmutzige Tricks mit dem Einheitskostenmaß}
\begin{enumerate}
\item   \renewcommand{\labelenumii}{(\theenumii)}
        \renewcommand{\theenumii}{\arabic{enumii}}
        \begin{enumerate} \newcommand{\lex}[1]{\ensuremath{\underset{\text{lex}}{#1}}}
        \item   $a = b \Leftrightarrow x = y$ ist trivial, da gilt, wenn 
                $x = y \Leftrightarrow \forall i \in \{1, ..., n\}{:}\ x_i = y_i$\\
                \[\Rightarrow \sum\limits_{i=1}^{n} x_{i}u^{n-i+1} = \sum\limits_{i=1}^{n} y_{i}u^{n-i+1}\]
        \item   Die lexikografische Ordung $\lex{<}$ zweier Vektoren ist definiert, wie folgt:
                \[ (x_1,...,x_n) \lex{<} (y_1,...,y_n) \Leftrightarrow
                    \exists l \in \{1, ..., n\}{:}\ \forall k < l, k \in \{1, ..., n\}{:}\ x_k = y_k \land x_l < y_l
                \]
                Seien $0 \leq x_i, y_i \leq M, u > M$.
                \begin{align*}
                    (x_1, ..., x_n) \lex{<} (y_1, ..., y_n) 
                        &\Leftrightarrow \sum\limits_{i=1}^{n} x_{i}u^{n-i+1} < \sum\limits_{i=1}^{n} y_{i}u^{n-i+1}  
                    \tag{Definition von $\lex{<}$}\\
                    \exists l \in \{1, ..., n\}{:}\ \forall k < l, k \in \{1, ..., n\}{:}\ x_k = y_k \land x_l < y_l 
                        &\Leftrightarrow \sum\limits_{i=1}^{n} x_{i}u^{n-i+1} < \sum\limits_{i=1}^{n} y_{i}u^{n-i+1}
                \end{align*}
                Wir teilen die Vektoren bei $l$, so dass wir je zwei Vektoren $(x_1,...,x_{l-1})$ und $(x_l,...,x_n)$ bzw.
                $(y_1,...,y_{l-1})$ und $(y_l,...,y_n)$ erhalten. Für $(x_1, ..., x_{l-1})$ und $(y_1, ..., y_{l-1})$ gilt 
                dann jeweils (1). Damit verbleibt zu beweisen:
                \begin{align*}
                    x_l < y_l 
                        &\Leftrightarrow \sum\limits_{i=l}^{n} x_{i}u^{n-i+1} < \sum\limits_{i=l}^{n} y_{i}u^{n-i+1}
                \end{align*}
                Wir müssen nun zeigen, dass -- egal, was $x_{l+1},...,x_n$ und $y_{l+1},...,y_n$ für Werte haben -- immer
                $x_l$ und $y_l$ über die Größenrelation der Summen zu einander entscheiden. Dies ist der Fall, wenn 
                \[x_l u^{n-l+1} > \sum\limits_{i=l+1}^{n} x_{i}u^{n-i+1}\ \text{bzw.}\ y_l u^{n-l+1} > \sum\limits_{i=l+1}^{n} y_{i}u^{n-i+1}\]
                \begin{description}
                \item[Induktionsbehauptung] Für einen Vektor $(v_1,...,v_n)$ mit $0 \leq v_i \leq M$ gilt für $u > M$ gilt:
                        \[v_1 u^n > v_2 u^{n-1} + ... + v_n u\]
                \item[Induktionsvoraussetzung] Für einen Vektor $(v_1,...,v_{n-1})$ mit $0 \leq v_i \leq M$ gilt für $u > M$ 
                        gilt:
                        \[v_1 u^{n-1} > v_2 u^{n-2} + ... + v_{n-1} u\]
                \item[Induktionsanfang] $(n = 2)$
                        \begin{align*}
                         v_1 u^2 &> v_2 u & &\Longleftrightarrow& u^2 &> \frac{v_2 u}{v_1}
                        \end{align*}
                        Da $0 \leq v_1, v_2 < u$ (und $v_1, v_2 \in \mathbb{Z}$) ist das wahr. \hfill$\square$
                \item[Induktionsschritt]
                        \begin{align*}
                         v_1 u^n &> v_2 u^{n-1} + ... + v_n u \tag{IV + Transl. $>$}\\
                         v_1 u^n > v_2 u^{n-1} &> v_3 u^{n-2} + ... + v_n u\\
                         v_1 u^n &> v_2 n^{n-1} \tag{s. IA}
                        \end{align*}
                        \hfill$\square$
                \end{description}
                Damit ist bewiesen:
                \[(x_1, ..., x_n) \lex{<} (y_1, ..., y_n) 
                        \Leftrightarrow \sum\limits_{i=1}^{n} x_{i}u^{n-i+1} < \sum\limits_{i=1}^{n} y_{i}u^{n-i+1} \]
        \end{enumerate}
\item   Wie in (a1) gezeigt, können wir die Gleichheit zweier Vektoren $x = (x_1, ..., x_n)$ und $y = (y_1, ..., y_n)$ zeigen, wenn 
         $x_1 u^m  + ... + x_m u = y_1 u^m + ... + x_m u$. Der Entwurf unseres Algorithmus nutzt dies aus, in dem er zunächst $x$ mit dem Teilvektor $(y_{n-m+1}, ..., y_n)$ (beide haben Länge $m-1$) vergleicht, indem geprüft wird, ob 
         \[x_1 u^m + ... + x_m u = y_{n-m+1} u^m + ... + y_n u\]
         Ist dies der Fall, wird $n-m+1$ als Position der Teilfolge ausgegeben. Ist dies nicht der Fall dekrementieren wir den Index des ersten Elements des Teilvektors (der aber immer Länge $m-1$ behält) solange, bis wir 1 erreichen. Haben wir 1 erreicht und die Hashes sind immer noch nicht gleich, geben wir $-1$ zurück, andernfalls den Index des ersten Elements des Teilvektors.
         Zur Optimierung der Berechnung passen wir den Hash immer jeweils nur an die aktuelle Situation an.
         Da wir nach "`links"' verschieben, ziehen wir zunächst $u y_{n}$ vom Hash ab, teilen das Ergebnis durch $u$, um den Vektor nach rechts zu "`shiften"' und addieren die neue "`Stelle"' $y_i u^m$.
        \begin{lstlisting}[mathescape=true]
def partof($x = (x_0, ..., x_m)$,$y = (y_0,...,y_n)$,$u$))):
    $a = x_1 u^m + ... + x_m u $
    $b = y_{n-m+1} u^m + ... + y_n u $
    $i = n-m$
    while $a \neq b$:
        if $i < 1$:
            return $-1$
        $b = \frac{b - u y_{m+i}}{u} + y_i u^m$
        $i = i - 1$
    return $i + 1$
        \end{lstlisting}
        Im EKM haben wir zunächst 3 Schritte zur Initialisierung von $a, b$ und $i$. Die \lstinline!while!-Schleife benötigt im besten Fall (die gesuchte Teilfolge ist direkt am Ende der zu durchsuchenden Folge) 1 Schritt (der Vergleich $a \neq b$), im schlechtesten Fall (die gesuchte Teilfolge ist gar nich in der Folge) $5 \cdot (n-m) + 2$ Schritte ($(n-m)$  Schleifendurchläufe + 2 Vergleiche $a \neq b$ und $i < 1$) und 1 Schritt für die Rückgabe. Insgesamt kommen wir so auf eine Laufzeit von max.
        \[5(n-m) + 4 = O(n)\] 
        Im LKM brauchen $a$ und $b$ maximal $M u^m$ Bit. Dadurch ändern sich bei jeder Rechenoperation auch auf maximal $M u^m$ Bit, wodurch die Laufzeit im LKM jeder Rechenoperation $O(\log(Mu^m))$ ist. $u$ kann dabei deutlich größer sein als $m$ und $n$ und ist per Definition $> M$. Für die Laufzeit der Schleife hat dies dramatische Folgen, da sie durch die Neuberechnung von $b$ nun
        in \[O(n \log(Mu^m)) = O(n \log M + n \log u^m) = O(n \log M + n m \log u) = O(n m \log u),\] was für große Zahlen deutlich größer ist als $O(n)$.
\end{enumerate}
\end{document}