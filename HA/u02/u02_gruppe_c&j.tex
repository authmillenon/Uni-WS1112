\documentclass[a4paper,10pt]{scrartcl}
\usepackage[utf8x]{inputenc}
\usepackage[T1]{fontenc}
\usepackage{amsmath,amsfonts,amssymb,amscd,amsthm,xspace}
\usepackage[ngerman]{babel}
\usepackage{listingsutf8}
\usepackage{color}
\usepackage{geometry}
\usepackage{graphicx}
\usepackage{multicol}
%\usepackage{pst-tree}

\geometry{a4paper, left=2cm,right=2cm,top=2cm,bottom=2cm}

\newcommand{\Authors}{Christian Cikryt (Di. 14-16), Jakob Pfender (Mi. 14-16)}
\title{H\"ohere Algorithmik - 2. \"Ubungsblatt}
\author{\Authors}
\date{\today}

\newcommand{\changefont}[3]{\fontfamily{#1} \fontseries{#2} \fontshape{#3} \selectfont}

\renewcommand{\thesection}{Aufgabe \arabic{section}}
\renewcommand{\labelenumi}{(\theenumi)}
\renewcommand{\theenumi}{\alph{enumi}}
\renewcommand{\labelenumii}{(\theenumii)}
\renewcommand{\theenumii}{\roman{enumii}}

\definecolor{lgray}{gray}{0.95}
\definecolor{purple}{rgb}{0.498,0,0.3333}
\definecolor{identifier}{rgb}{0,0,0.1}
\definecolor{string}{rgb}{0.192,0,1}
\definecolor{comment}{rgb}{0.25,0.5,0.37}

\pagestyle{myheadings}
\oddsidemargin\oddsidemargin
\markright{\Authors}

\lstset{
	tabsize=4, 
	frame=tlrb, 
	basicstyle=\footnotesize\changefont{pcr}{m}{n},
	breaklines=true,
	numbers=left,
	emphstyle=\textit, 
	language=Java,
	keywordstyle=\color{purple}\textbf, 
	identifierstyle=\color{identifier},
	stringstyle=\color{string},
	backgroundcolor=\color{lgray},
	showstringspaces=false,
	commentstyle=\color{comment},
	extendedchars=true,
	inputencoding=utf8/latin1
}
%\psset{nodesep=2pt,levelsep=2em,treesep=2em}

\begin{document}

\maketitle

\section{Gewichteter Median}
\begin{enumerate}
\item	Gegeben eine Funktion \texttt{weighted\_median()}, die
	aus einer gegebenen Liste den gewichteten Median bestimmt,
	können wir einen einfachen Divide-and-Conquer-Algorithmus
	anwenden, um den Median zu finden:

	\begin{lstlisting}[numbers=none]
	int median(list) {
	  weighted_median = weighted_median(list);
	  smaller = [e | e <- list, e <= weighted_median];
	  larger  = [e | e <- list, e >  weighted_median];
	  if (size(smaller) > (size(list) / 2))
	    return median(smaller);
	  else if (size(smaller) == (size(list) / 2))
	    return weighted_median;
	  else
	    return median(larger);
	}
	\end{lstlisting}

	Für die Gewichtung kann man einfach jedem Element das Gewicht
	$1/n$ geben, wobei $n$ die Länge der Liste ist. Dadurch ist bei
	einer geraden Anzahl an Elementen der gewichtete Median gleich
	dem normalen Median, bei einer ungeraden Anzahl gleich dem
	Nachfolger des Medians. Mit obenstehendem Algorithmus lässt sich
	der Median also mit Laufzeit $T(n) + O(n)$ lösen.

\item	Gegeben eine Liste mit $n$ gewichteten Elementen und eine
	Sortierfunktion \texttt{sort()}, können wir den gewichteten
	Median wie folgt bestimmen. Als Sortierfunktion wird Mergesort verwendet, da es in jedem Fall eine Laufzeit von $O(n \log n)$ besitzt.

	\begin{lstlisting}[numbers=none]
	int weighted_median(list) {
	  sort(list);
	  weight = 0;
	  for (e : list) {
	    weight += e.weight;
	    if (weight > 0.5)
	      return e;
	  }
	}
	\end{lstlisting}

\item

\end{enumerate}

\section{Analyse des BFPRT-Algorithmus}
Es wird allgemein für ein ungerades $k > 0$ die Laufzeit berechnet.
Die Laufzeit des BFPRT-Algorithmus mit einer Aufteilung in $k$er-Blöcke für $n \geq 100$ ist analog zur Vorlesung:
\[T_k(n) \leq cn + T_k\left(\left\lceil\frac{n}{k}\right\rceil\right) + T_k\left(\frac{3n}{4}\right)\]
für $n < 100$ wird ein Bruteforce-Ansatz verwendet. Die Laufzeit dafür ist somit von $k$ unabhängig und wie ebenfalls in der Vorlesung erläutert $O(1)$.

Im Folgenden wird mittels vollständiger Induktion über $n$ gezeigt, dass für die Laufzeit $O(n)$ gilt und zwar für alle ungeraden $k$ ab 5 (für die 3 nicht). In einem zweiten Schritt wird gezeigt, dass sogar $\Theta(n)$ gilt.
\begin{description}
\item[Induktionsbehauptung] $\exists\alpha > 0{:}\ T_k(n) \leq \alpha n$
\item[Induktionsanfang] $n < 100$
    \begin{itemize}
     \item $T_k(n) = O(1)$, weil sich sich immer ein $\alpha$ finden lässt, so dass $\alpha n \geq T_k(n)$ ($O(1) \subset O(n)$).
    \end{itemize}
\item[Induktionsvoraussetzung] Es gelte für ein $n \in \mathbb{N}$ : $\exists\alpha > 0{:}\ T_k(n-1) \leq \alpha (n-1)$

\item[Induktionsschritt] Es wird nun gezeigt, dass unter der Induktionsvoraussetzung ($T_k\left(\left\lceil\frac{n}{k}\right\rceil\right) \in O(n)$)die Induktionsbehauptung auch für T(n) gilt.
\[
\begin{array}{rl|rl}
 T_k(n)   &\leq cn + T_k\left(\left\lceil\frac{n}{k}\right\rceil\right) + T_k\left(\frac{3n}{4}\right) &
                \alpha n &\geq cn + \alpha n \frac{3k + 4}{4k} + \alpha\\
        &\overset{\text{IA}}{\leq} cn + \alpha \left\lceil\frac{n}{k}\right\rceil + \alpha \frac{3n}{4} &
                \alpha n \frac{k - 4}{4k} &\geq cn + \alpha\\
        &\leq cn + \alpha \left(\frac{n}{k} + 1\right) + \alpha \frac{3n}{4} &
                \alpha (k - 4) &\geq 4kc + \underbrace{\frac{4k\alpha}{n}}_{\leq \frac{\alpha}{k}} \qquad \text{(für $k < n$)} \\
        &= cn + \alpha n \left(\frac{1}{k} + \frac{3}{4}\right) + \alpha  &
                \alpha (k - 4) &\geq 4kc + \frac{\alpha}{k}\\
        &= cn + \alpha n \left(\frac{3k + 4}{4k}\right) + \alpha &
                \frac{\alpha (k^2 - 4k) - \alpha}{k} &\geq 4kc\\
        &\overset{!}{\leq} \alpha n &
                \alpha (k^2 - 4k - 1) &\geq 4k^2c\\
        && \alpha &\geq \frac{4k^2}{k^2 - 4k - 1} c
\end{array}
\]
 Der BFPRT-Algorithmus läuft also für alle  $k \ge 5$ in $O(n)$ Zeit. Für $k = 3$ ist $\alpha$ negativ, was einen Widerspruch zu $\alpha > 0$ darstellt und somit gilt $T_3(n) \in \Theata(n^{\log_3 2})$.
\end{description}

Indem man das Ungleichungszeichen umdreht, kann man auch ein $\alpha > 0$ bestimmen, so dass gilt $T_k(n) \in \Omega(n)$ (für $k$ ab 5). Somit gilt insgesamt $T_k(n) \in \Theta(n)$ $\forall$ ungerade $k \ge 5 \in \mathbb{N}$.
\section{Schmutzige Tricks mit dem Einheitskostenmaß}

\begin{enumerate}
\item   \renewcommand{\labelenumii}{(\theenumii)}
        \renewcommand{\theenumii}{\arabic{enumii}}
        \begin{enumerate} \newcommand{\lex}[1]{\ensuremath{\underset{\text{lex}}{#1}}}
        \item	$a = b \Leftrightarrow x = y$ ist trivial, da dann für
	alle $x_i, y_j$ gilt: $i = j \Leftrightarrow x_i = y_j$ und
	somit\\
                \[\sum\limits_{i=1}^{n} x_{i}u^{n-i+1} = \sum\limits_{i=1}^{n} y_{i}u^{n-i+1}\]
	\item	Ist $x$ lexikographisch kleiner als $y$, so gilt:
                \begin{align*}
                    \exists l \in \{1, ..., n\}{:}\ \forall k < l, k \in \{1, ..., n\}{:}\ x_k = y_k \land x_l < y_l 
                        &\Leftrightarrow \sum\limits_{i=1}^{n} x_{i}u^{n-i+1} < \sum\limits_{i=1}^{n} y_{i}u^{n-i+1}
                \end{align*}
		Wir können nun die Vektoren aufteilen, so dass wir
		insgesamt vier Vektoren $(x_1,...x_{l-1})$,
		$(x_l,...,x_n)$, $(y_1,...y_{l-1})$ und
		$(y_l,...,y_n)$ erhalten. Für $(x_1, ..., x_{l-1})$ und $(y_1, ..., y_{l-1})$ gilt 
                dann jeweils (1). Damit bleibt zu beweisen, dass
                \begin{align*}
                    x_l < y_l 
                        &\Leftrightarrow \sum\limits_{i=l}^{n} x_{i}u^{n-i+1} < \sum\limits_{i=l}^{n} y_{i}u^{n-i+1}.
                \end{align*}
		Somit ist zu beweisen, dass $x_l < y_l$ ausreicht, damit
		$a < b$ gilt. Das ist der Fall, wenn
                \[x_l u^{n-l+1} > \sum\limits_{i=l+1}^{n} x_{i}u^{n-i+1}\ \text{bzw.}\ y_l u^{n-l+1} > \sum\limits_{i=l+1}^{n} y_{i}u^{n-i+1}\].
                \begin{description}
                \item[Induktionsbehauptung] Für einen Vektor $(v_1,...,v_n)$ mit $0 \leq v_i \leq M$ gilt für $u > M$:
                        \[v_1 u^n > v_2 u^{n-1} + ... + v_n u\]
                \item[Induktionsvoraussetzung] Für einen Vektor $(v_1,...,v_{n-1})$ mit $0 \leq v_i \leq M$ gilt für $u > M$:
                        \[v_1 u^{n-1} > v_2 u^{n-2} + ... + v_{n-1} u\]
                \item[Induktionsanfang] $(n = 2)$
                        \begin{align*}
                         v_1 u^2 &> v_2 u & &\Longleftrightarrow& u^2 &> \frac{v_2 u}{v_1}
                        \end{align*}
                        Da $0 \leq v_1, v_2 < u$ (und $v_1, v_2 \in \mathbb{Z}$), ist das wahr. \hfill$\square$
                \item[Induktionsschritt]
                        \begin{align*}
                         v_1 u^n &> v_2 u^{n-1} + ... + v_n u \tag{IV + Transl. $>$}\\
                         v_1 u^n > v_2 u^{n-1} &> v_3 u^{n-2} + ... + v_n u\\
                         v_1 u^n &> v_2 n^{n-1} \tag{s. IA}
                        \end{align*}
                        \hfill$\square$
                \end{description}
                Damit ist bewiesen:
                \[(x_1, ..., x_n) \lex{<} (y_1, ..., y_n) 
                        \Leftrightarrow \sum\limits_{i=1}^{n} x_{i}u^{n-i+1} < \sum\limits_{i=1}^{n} y_{i}u^{n-i+1} \]
	\end{enumerate}

\end{enumerate}

\end{document}
