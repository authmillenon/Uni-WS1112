\subsection{Beispiel: Problem des engsten Punktpaares}
\paragraph*{Gegeben} $n$ Punkte in einer Ebene $P \subseteq \mathbb{R}^2, |P| = n$, alle $x$-Koordinaten sind paarweise verschieden.
\begin{center}
 \begin{pspicture}(0,0)(3,3)
  \psdot(0,2.1)\psdot(0.4,0.5)\psdot(1,1)\psdot(1.2,1.9)
  \psdot(1.6,0.6)\psdot(2.1,1.5)\psdot(2.4,2)
  \psdot(2.5,1.1)\psdot(2.6,0.1)\psdot(3,1.6)
 \end{pspicture}
\end{center}
\paragraph*{Gesucht} $p, q \in P$ mit $p \neq q$ und $S = d(p,q)$ minimal
\begin{itemize}
 \item Naiv: Probiere aller $\binom{n}{2}$ Punktpaare durch, nimm das Minimum
         \[\Theta(n^2)\ \text{Laufzeit}\]
        \begin{itemize}
         \item Problem: Wie $d(p,q) = \sqrt{(p_x - q_x)^2 + (p_y - q_y)^2}$ finden?
         \item Lösing: Wir müssen $d(p,q)$ nicht ausrechnen, wir müssen nur $d(p,q) \leq d(r,s)$ entscheiden können.
                \begin{align*}
                 \sqrt{(p_x - q_x)^2 + (p_y - q_y)^2} &< \sqrt{(r_x - s_x)^2 + (r_y + s_y)^2} \\
                 \Leftrightarrow (p_x - q_x)^2 + (p_y - q_y)^2 &< (r_x - s_x)^2 + (r_y + s_y)^2
                \end{align*}
        \end{itemize}
 \item Schneller mit Devide \& Conquer:
         \begin{itemize}
          \item  Teile die Punktmenge in zwei Hälften
          \item  Sortiere nach $x$-Koordinate
          \item  Nimm den Punkt $q$ mit den $\left\lceil\frac{n}{2}\right\rceil$ größten $x$-Koordinaten. Annahme: alle $x$-Koordinaten verschieden
              \begin{align*}
               P_L &= \{p \in P\ |\ \text{$p$ links von $q$}\} \cup \{q\}\\
               P_R &= \{p \in P\ |\ \text{$p$ rechts von $q$}\}
              \end{align*}
              \[
                  \left.\begin{matrix}
                   \text{$S_L \leftarrow$ Bestimme engstes Paar in $P_L$} \\
                   \text{$S_R \leftarrow$ Bestimme engstes Paar in $P_R$}
                  \end{matrix}
                 \right\}\text{rekursiv}
              \]
              \begin{center}
                $S \to \min(S_L, S_R)$
              \end{center}
              $S$ ist eventuell nicht die Lösung für ganz $p$.
\begin{center}
\begin{pspicture}(0,-0.5)(3,3)
\psframe[linestyle=none,fillstyle=solid,fillcolor=melon](1.2,0)(2.0,3)
\psdot(0,2.1)\psdot(0.4,0.5)\psdot(1,1)\psdot(1.2,1.9)
\psline[linecolor=red](1.6,0)(1.6,3)
\psdot(1.6,0.6)\uput{3px}[45](1.6,0.6){$q$}\psdot(2.1,1.5)\psdot(2.4,2)
\psdot(2.5,1.1)\psdot(2.6,0.1)\psdot(3,1.6)
\psline(1.2,-0.1)(2,-0.1)\psline(1.2,-0.2)(1.2,0)\psline(1.6,-0.2)(1.6,0)\psline(2,-0.2)(2,0)
\uput{3px}[-90](1.4,-0.1){$\delta$}\uput{3px}[-90](1.8,-0.1){$\delta$}
\end{pspicture}
\end{center}
            Es könnte sein, dass das engste Paar zwischen $P_L$ und $P_R$ liegt.
\end{itemize}




