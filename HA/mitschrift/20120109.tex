\Lemma Sei $T_0 =$ \#Bäume vor delete\_min. Dann ist die Laufzeit von delete\_min $O(\log n + T_0+ \text{\#gelöschte Knoten})$
\Bew
\begin{enumerate}
 \item Laufzeit: $O(T_0)$
 \item Laufzeit: $O(\log n)$ wegen Invariante
 \item Laufzeit: $O(\log n + T_0)$\\
         Warum: Laufzeit $O(\log n + \#\text{links})$\\
         $\#\text{links}$ $\leq$ \#links $= O(\log n) + T_0$ denn jeder link zerstört einen Baum
 \item Laufzeit: $O(\text{\#gelöschte Knoten} + \log n)$\hfill$\square$
\end{enumerate}
\paragraph{Analyse:}
\begin{description}
 \item[zu zeigen: ] Jede Folge von $i$ inserts, $d$ decrease\_keys und $e$ delete\_mins hat Laufzeit $O(i + d + e \log n)$, mit $n = \max \#\text{Elemente in der Datenstruktur}$
 \item[Idee: ] Wenn delete\_min lange dauert, muss es vorher viele inserts und decrease\_keys gegeben haben.\\
         $\Rightarrow$ können Knoten umverteilen
 \item Wie formalisiert man das?
     \renewcommand{\theenumi}{\alph{enumi}}
     \begin{enumerate}
      \item Potentialmethode (s. Übung)
      \item Buchhaltermethode.
     \end{enumerate}
     \begin{itemize}
      \item 3 Konten:
         \begin{description}
          \item Knotenkonto: enhält 1 \euro\ pro Knoten
          \item Baumkonto: enhält 2 \euro\ pro Baum
          \item Einzelkinder-Konto: enthält 4 \euro\ pro Knoten mit nur 1 Kind.
         \end{description}
      \item Am Anfang sind alle Konten leer
      \item Wir zeigen:
          \begin{itemize}
          \item Bezahlen wir $O(1) \euro$ pro insert/decrease\_key und $O(\log n) \euro$ pro delete\_min
          \end{itemize}
      \item so können wir sicherstellen, dass
      \renewcommand{\theenumi}{\roman{enumi}}
     \begin{enumerate}
      \item die Konteninvariante immer gilt.
      \item die tatsächlichen Kosten können bezahlt werden.
     \end{enumerate}
     \begin{description}
     \item insert: tatsächliche Kosten $O(1)$
         \begin{itemize}
          \item Müssen 1 \euro\ ins Knotenkonto und 2 \euro\ ins Baumkonto zahlen
         \end{itemize}
         zu zahlende Kosten: $O(1) \euro$
     \item decrease\_key: tatsächliche Kosten $O(1)$
         \begin{itemize}
          \item Müssen $\leq 2 \euro$ ins Baumkonto und $\leq 4 \euro$ ins Einzelkinderkonto einzahlen
         \end{itemize}
         zu zahlende Kosten: $O(1) \euro$
     \item delete\_min: tatsächliche Kosten: $O(\log n + T_0 + \mathcal{D})$ mit $\mathcal{D} = $\#gelöschte Knoten\\
         Buchhaltung:
         \begin{itemize}
         \item Hebe $T_0 \euro$ von Baumkonto ab\\
             $\Rightarrow$ tatsächliche Kosten
         \item Überweise \#links \euro\ auf das Knotenkonto (vom Baumkonto + ggf $O(\log n)$ zusätliche \euro)\\
             $\Rightarrow \leq T_0 \euro$ im Baumkonto
         \item Zahle 2 \euro\ pro Baum nach Konsulidierung auf Baumkonto ein \\
             $\Rightarrow O(\log n) \euro$
         \item Hebe $\mathcal \euro$ im Knotenkonto ab\\
             $\Rightarrow$ tatsächliche Kosten
         \item Bei Erdbeben: Überweise $2n[i] \euro$ von Einzelkinderkonto auf Baumkonto
         \end{itemize}
         zu Zahlen $O(\log n) \euro$
     \end{description}
     \item noch zu zeigen Konteninvariante bleibt erhalten
         \begin{itemize}
          \item Knotenkonto $\checkmark$
          \item Baumkonto $\checkmark$
          \item Einzelkinderkonto:
              \Beh Wenn $n[i+1] > \frac{3}{4} n[i]$, so hat $n[i+1]$ mindestens $\frac{n[i]}{2}$ Knoten mit 1 Kind
              \Bew \begin{align*}
                    n[i+1] &= \underbrace{n_1}_{\text{Knoten mit 1 Kind}} + \underbrace{n_2}_{\text{Knoten mit 2 Kindern}} \\
                    n[i] &= n_1 + n_2 \\
                    \text{wissen } n[i+1] &\geq \frac{3}{4} n[i] \\
                    \Rightarrow n_1 + n_2 &\geq \frac{3}{4} (n_1 + 2 n_2) \\
                    \Rightarrow n_1 + n_2 &\geq \frac{3}{4} n_1 + \frac{3}{4} n_2 \Leftrightarrow n_1 \geq 2n_2 \\
                    \Rightarrow 2n_1 = n_1 + n_2 \geq n_1 + 2 n_2 = n[i]
                   \end{align*}\hfill$\square$

          \end{itemize}

     \end{itemize}
\end{description}
