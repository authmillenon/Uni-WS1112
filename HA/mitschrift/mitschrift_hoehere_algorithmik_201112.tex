\documentclass[a4paper,10pt]{scrbook}

% Seitenlayout
\usepackage[a4paper,top=1.5cm,right=2.0cm,bottom=2cm,left=2.0cm]{geometry}
\usepackage{fancyhdr}

% Zeichen
\usepackage[ngerman]{babel}
\usepackage[utf8]{inputenc}
\usepackage{amsmath, amsthm, amssymb}

% Zeichensatz
\usepackage[colorlinks,%
	    citecolor=black,%
	    filecolor=black,%
	    linkcolor=black,%
	    urlcolor=black,%
	    pdftitle = {Mitschrift - CB},%
	    pdfauthor = {Martin Lenders}%
	]{hyperref}
\usepackage{listings}
\usepackage{algorithmic}
\usepackage{hyperref}
\usepackage{epigraph}

% Grafik
\usepackage[usenames]{pstricks}
\usepackage{pst-plot,pst-node,pstricks-add,pst-tree}

% Color definitions
\definecolor{lgray}{gray}{0.95}
\definecolor{save}{rgb}{0.498,0,0}
\definecolor{identifier}{rgb}{0,0,0.1}
\definecolor{string}{rgb}{0.192,0,1}
\definecolor{comment}{rgb}{0.25,0.5,0.37}
\definecolor{yellow}{rgb}{1,1,0}
\definecolor{sand}{rgb}{1,1,.75}
\definecolor{red}{rgb}{1,0,0}
\definecolor{melon}{rgb}{1,0.6,.5}
\definecolor{green}{rgb}{0,1,0}
\definecolor{lime}{rgb}{.75,1,.75}
\definecolor{blue}{rgb}{0,0,1}
\definecolor{azure}{rgb}{.75,.75,1}

% Einstellungen für Pakete
\lstset{
	tabsize=8,
	frame=,
	basicstyle=\footnotesize\changefont{pcr}{m}{n},
	emphstyle=\textit,
	numberstyle=\tiny\textsf,
	numbersep=5pt,
	numbers=none,
	keywordstyle=\color{save}\textbf,
	identifierstyle=\color{identifier},
	stringstyle=\color{string},
	showstringspaces=false,
	commentstyle=\color{comment},
	extendedchars=true,
	xleftmargin=1em,
	inputencoding=utf8,
	mathescape=true;
}

\psset{%
	algebraic=true,
	angleA=0,%
	angleB=180,%
	unit=1cm,%
	subgriddiv=0,%
	griddots=5,%
	gridlabels=7pt%
}

\renewcommand{\chaptermark}[1]{markboth{#1}{}}
\renewcommand{\sectionmark}[1]{\markright{\thesection\ #1}}
\pagestyle{fancy}
\fancyhf{}
\fancyfoot[LE,RO]{\sffamily\thepage}
\fancyhead[LE]{\footnotesize\sffamily\bfseries\leftmark}
\fancyhead[RO]{\footnotesize\sffamily\rightmark} 

% Eigene Befehle
\newcommand{\changefont}[3]{\fontfamily{#1} \fontseries{#2} \fontshape{#3} \selectfont}
\newcommand{\Defi}{\paragraph*{Definition:}}
\newcommand{\Lemma}{\paragraph*{Lemma:}}
\newcommand{\Satz}{\paragraph*{Satz:}}
\newcommand{\Beh}{\paragraph*{Behauptung:}}
\newcommand{\Bew}{\paragraph*{Beweis:}}
\newcommand{\Bsp}{\paragraph*{Beispiel:}}

% Neudefinitionen
\renewcommand{\thepart}{\Alph{part}}
\renewcommand{\sectionmark}[1]{\markright{\thesection\ #1}}
\renewcommand{\algorithmiccomment}[1]{// #1}

\title{Mitschrift\\{\LARGE Höhere Algorithmik}}
\author{gehalten von Prof. Dr. Wolfgang Mulzer \\ mitgeschrieben von Martin Lenders}
\subject{Achtung: Dieses Dokument ist \emph{nur} eine Mitschrift der Vorlesung "Höhere Algorithmik" WiSe2011/12.
	Sie wurde während der Vorlesung angefertigt. Es wird aber seitens des Autors keine Garantie auf
	Vollständigkeit und Richtigkeit des Inhalts gegeben.}

\begin{document}
\maketitle
\tableofcontents
\chapter{Einleitung}
\section{Klausur}
24.04.2012 10-12 Uhr
\begin{itemize}
 \item Auswahlproblem\\
		Gegeben: 
		\begin{description}
		 \item $S$, Menge, total geordnet, $n$ Elemente paarweise verschieden.
		 \item $k \in \{1, ..., n\}$
		\end{description}
		Gesucht:
		\begin{description}
		 \item Element $s \in S$ mit $\operatorname{Rg}(s) = k$, d. h. $S$ enthält $k -1$ viele Elemente kleiner als $k$
		\end{description}
\end{itemize}
\paragraph{Lemma:} Angenommen es  existiert Funktion $\mathtt{SPLITTER}(S)$ die ein Element $q \in S$ liefert, so dass \[\operatorname{Rg}(q) \in \left\{\left\lceil\frac{1}{4} n\right\rceil, ..., \left\lfloor\frac{3}{4}n\right\rfloor\right\}.\]
Dann kann man das Auswahlproblem in $O(n)$ Zeit läsen, wenn $\mathtt{Splitter}$ nichts kostet.

\paragraph{Wie implementiert man \texttt{SPLITTER}?}
\begin{description}
 \item[1. Möglichkeit:] Wähle den \texttt{SPLITTER} zufällig. Mit Wahrscheinlichkeit $\frac{1}{2}$ erwischen wir einen guten Splitter.
		Im Durchschnitt ist das wohl gut genug. \\
		$\Rightarrow$ Erwartungswert der Laufzeit ausrechnen, etc., \emph{Randomisierte Algorithmen (später)}
 \item[2. Möglichkeit:] BFRPT-Methode:
		\begin{description}
		\item[rekursiv] Wähle eine Stichprobe $S' \subseteq S$ mit $|S'| = \left\lceil\frac{n}{5}\right\rceil$, so dass der Median von $S'$ ein guter Splitter von $S$ ist. Bestimme rekursiv den Median von $S$!
		\item[Wie finden wir $\boldsymbol{S'}$?]\hspace{0cm}\\
			\begin{description}
			\item[1. Versuch:] Nimm die ersten $\left\lceil\frac{n}{5}\right\rceil$. Elsement von $S$. $\Rightarrow$ \emph{klappt nicht.}
			\item[2. Versuch:] Nimm jedes 5. Element. $\Rightarrow$ \emph{auch nicht}
			\end{description}
		\end{description}
 \item[3. Möglichkeit] Unterteile $S$ in 5er Gruppen. Nimm aus jeder 5er Gruppe den Median. $S'$ ist die Menge der Mediane.
		\paragraph{Beispiel}
		\[
		\begin{matrix}
		 S & \rnode{51l}{7} & 1 & 5 & 9 & \rnode{51r}{18} \\&\\
		 S' 
		\end{matrix}\quad
		\begin{matrix}
		 \rnode{52l}{30} & 6 & 11 & 15 & \rnode{52r}{20} \\&\\
		 &
		\end{matrix}
		\ncbox[nodesep=2px]{51l}{51r}7
		\ncbox[nodesep=2px]{52l}{52r}15
		\]
		\paragraph{Lemma:} Der Median von $S'$ ist ein guter Splitter von $S$, wenn $n$ groß genug ist.
		\paragraph{Beweis:} Betrachte $S'$ als von links nach rechts sortiert (für den Beweis) und zugehörige 5er Gruppen.
			
			Die Lage im Bild.
		\begin{center}
		 \begin{pspicture}(0,0)(10,5)
		  \psdot(0,4)\psdot(1,4)\psdot(2,4)\psdot(3,4)\psdot(4,4)\psdot(5,4)\psdot(6,4)\psdot(7,4)\psdot(8,4)
		  \psdot(0,3)\psdot(1,3)\psdot(2,3)\psdot(3,3)\psdot(4,3)\psdot(5,3)\psdot(6,3)\psdot(7,3)\psdot(8,3)\psdot(9,3)
	\rput(-1,2){\color{green}$S'$}	  \psdot[linecolor=green](0,2)\psdot[linecolor=green](1,2)\psdot[linecolor=green](2,2)\psdot[linecolor=green](3,2)\psdot[linecolor=green](4,2)\psdot[linecolor=green](5,2)\psdot[linecolor=green](6,2)\psdot[linecolor=green](7,2)\psdot[linecolor=green](8,2)\psdot[linecolor=green](9,2)
		  \psdot(0,1)\psdot(1,1)\psdot(2,1)\psdot(3,1)\psdot(4,1)\psdot(5,1)\psdot(6,1)\psdot(7,1)\psdot(8,1)\psdot(9,1)
		  \psdot(0,0)\psdot(1,0)\psdot(2,0)\psdot(3,0)\psdot(4,0)\psdot(5,0)\psdot(6,0)\psdot(7,0)\psdot(8,0)
		  \psset{arrows=<-}
		  \psline(0,4)(0,3)\psline(1,4)(1,3)\psline(2,4)(2,3)\psline(3,4)(3,3)\psline(4,4)(4,3)
		  \psline(5,4)(5,3)\psline(6,4)(6,3)\psline(7,4)(7,3)\psline(8,4)(8,3)
		  \psline(0,3)(0,2)\psline(1,3)(1,2)\psline(2,3)(2,2)\psline(3,3)(3,2)\psline(4,3)(4,2)
		  \psline(5,3)(5,2)\psline(6,3)(6,2)\psline(7,3)(7,2)\psline(8,3)(8,2)\psline(9,3)(9,2)
		  \psset{arrows=->}
		  \psline[linecolor=green](0,2)(1,2)\psline[linecolor=green](1,2)(2,2)
		  \psline[linecolor=green](2,2)(3,2)\psline[linecolor=green](3,2)(4,2)
		  \psline[linecolor=green](4,2)(5,2)\psline[linecolor=green](5,2)(6,2)
		  \psline[linecolor=green](6,2)(7,2)\psline[linecolor=green](7,2)(8,2)
		  \psline[linecolor=green](8,2)(9,2)
		  \psline(0,1)(0,2)\psline(1,1)(1,2)\psline(2,1)(2,2)\psline(3,1)(3,2)\psline(4,1)(4,2)
		  \psline(5,1)(5,2)\psline(6,1)(6,2)\psline(7,1)(7,2)\psline(8,1)(8,2)\psline(9,1)(9,2)
		  \psline(0,0)(0,1)\psline(1,0)(1,1)\psline(2,0)(2,1)\psline(3,0)(3,1)\psline(4,0)(4,1)
		  \psline(5,0)(5,1)\psline(6,0)(6,1)\psline(7,0)(7,1)\psline(8,0)(8,1)
		  \psframe[linecolor=red](4.8,1.8)(5.2,2.2)
		  \psset{arrows=-}
		  \psline[linecolor=blue](-0.2,-0.2)(5.2,-0.2)(5.2,1.2)(4.2,1.2)(4.2,2.2)(-0.2,2.2)(-0.2,-0.2)
		  \psline[linecolor=blue](9.2,4.2)(9.2,1.8)(5.8,1.8)(5.8,2.8)(4.8,2.8)(4.8,4.2)(9.2,4.2)
		  \uput{0.3cm}[-45](5,2){\color{red}$q$}
		  \uput{0.3cm}[180](-0.2,1){\color{blue}$< q$}
		  \uput{0.3cm}[0](9.2,3){\color{blue}$> q$}
		 \end{pspicture}
		\end{center}
		\begin{itemize}
		 \item Es sind $\underbrace{\left\lceil\frac{1}{2}\underbrace{\left\lceil\frac{n}{5}\right\rceil}_{|5|}\right\rceil}_{\#\text{Gruppen $m$}} - 3$
		 \item definitiv größer als $q$.
		 \item Ebenso gibt es definitiv $3 \left\lceil\frac{1}{2}\left\lceil\frac{n}{5}\right\rceil\right\rceil - 3$ Elemente kleiner als $q$.
		 \item Es gilt:
				\begin{align*}
				 \left\lceil\frac{1}{2}\left\lceil\frac{n}{5}\right\rceil\right\rceil - 3 &\geq
				3 \cdot \frac{1}{2} \cdot \frac{1}{5} \cdot n - 3 \\
					&= \frac{3}{10} - 3
				\end{align*}
				\begin{center}
				Wir wollen: $\frac{3}{10} n - 3 \overset{!}{\geq} \frac{1}{4} n \Rightarrow n \geq 60$. \hfill$\square$
				\end{center}
			\end{itemize}
		\begin{verbatim}
        Algorithmus: Select(S,K)
            if |S| < 100 then
                brute_force
         /  Unterteile S in 5er Gruppen
Splitter |  S <- {Median jeder 5er-Gruppe}
         \  q <- Select(S', ceil((|S'| + 1))/2)
            S_< <- {s in S for s < q}
            S_> <- {s in S for s > q}
            if |S_<| >= k then
                return SELECT(S_<, k)
            else if |S_<| = k - 1 then
                return q
            else
                return SELECT(S_>, k - |S_<| - 1)
        \end{verbatim}
		\paragraph{Laufzeitanalyse}
		\[T(n) \leq \begin{cases}
		             O(1), & n < 100 \\
					 O(n) + T\left(\left\lceil\frac{n}{5}\right\rceil\right) + T\left(\frac{3}{4} n\right), & \text{sonst}
		            \end{cases}\]
		\paragraph{Behauptung} $T(n) = O(n)$
		\paragraph{Beweis durch Induktion}
			\begin{description}
			 \item[Behauptung] $\exists$ Konstante $\alpha > 0$, so dass $T(n) < \alpha n$ ist.
			 \item für $n < 100$: $\checkmark$
			 \item[Induktionsschritt] 
				\begin{align*}
				 T(n) &\leq cn + T\left(\left\lceil\frac{n}{5}\right\rceil\right) + T\left(\frac{3}{4} n\right) \\
					  &\overset{\text{IA}}{\leq } cn + \alpha \left\lceil\frac{n}{5}\right\rceil + \alpha \frac{3}{4} n\\
					  &\leq cn + \alpha \left(\frac{n}{5} + 1\right) + \alpha \frac{3}{4} n \\
					  &= cn + \alpha \left(\frac{1}{5} +  \frac{3}{4}\right)n + \alpha \\
					  &= cn + \frac{19}{20} \alpha n + \alpha \\
					  &\overset{!}{\leq} \alpha n
				\end{align*}
				\begin{align*}
				 \alpha n &\geq cn + \frac{19}{20} \alpha n + \alpha\\
				 \frac{1}{20} \alpha n & \geq cn + \alpha &&| : \frac{n}{20} \\
				 \alpha & \geq 20c + \underbrace{\frac{20\alpha}{n}}_{\leq \frac{\alpha}{5}}
				\end{align*}
				Es gilt: $20 c + \frac{\alpha}{5} \geq 20c + \frac{20\alpha}{n}$ \\
				D. h., wenn:
					\begin{align*}
					(*)\ \alpha &\geq 20c + \frac{\alpha}{5}, \text{dann} \\
					\alpha &\geq 20c + \frac{20\alpha}{n}
					\end{align*}
					(*) gilt, wenn $\alpha \geq 25c$ ist \hfill$\square$
			\end{description}
		Algorithmus:
		\begin{center}
		 \begin{tabular}{ll}
			\textsc{Blum} & Turing-Award 1995 \\
			\textsc{Floyd} & Turing-Award 1978 \\
			\textsc{Pratt} & --- \\
			\textsc{Rivest} & Turing-Award 2002 \\
			\textsc{Tarjan} & Turing-Award 1986 
		 \end{tabular}
		\end{center}
		\paragraph{Bemerkung:}
		\begin{itemize}
		 \item Algorithmus kurz, elegant, optimal.
		 \item Benutzt nicht triviale Struktur im Problem.
		 \item Laufzeiteigenschaften nicht offensichtlich, brauchen Analyse und Beweis.
		 \item Theoretisches Ergebnis.
		\end{itemize}

\end{description}


\section{Berechnungsmodell}
\begin{itemize}
 \item Bei der Analyse von Algorithmen zählen wir "`elementare Schritte"'
 \item Was ist das?
 \item Berechnungsmodell: abstraktes, mathematisches Modell von Rechnern, um Begriffe \emph{Berechenbarkeit}, \emph{Algorithmus}, \emph{Laufzeit}, \emph{Speicherplatz}, etc. zu definieren
      \paragraph*{Beispiele} Turingmaschine, $\mu$-Rekurion, Game of Live, $\lambda$-Kalkül, Markov-Modelle, ...
 \item Für uns: \textbf{Registermaschine} (\emph{R}andom \emph{A}ccess \emph{M}aschine)
\end{itemize}
\subsection{Registermaschine}
\Defi Für eine \emph{Registermaschine} gilt folgendes:
    \begin{itemize}
    \item $\infty$ viele register $R_0, R_1, R_2, ...$
          \[
           \begin{array}{|c|c|c|ccccc}
            \hline
            R_0 & R_1 & R_2 & \dots & & & & \\
            \hline
           \end{array}
          \]
    \item jedes Register speichert eine ganze Zahl $\in \mathbb{Z}$
    \item \textbf{Programm} endliche Folge von Befehlen\\
		\textbf{Befehlstypen}
		\begin{itemize}
		 \item $A := B \operatorname{op} C$, dabei ist $A,B,C$: 
			\begin{itemize}
			 \item Register $R_i$
			 \item indirekt $(R_i)$
			 \item Konstante $c$
			\end{itemize}
			$\operatorname{op} \in \{+,-,\times,/\}$ ($/$ als ganzzahlige Division)
		 \item $A := B$
         \item $\texttt{GOTO } L$, $L$: Label, Programmzeile (auch indirekt)
         \item $\texttt{GGZ } B, L$: $\texttt{GOTO } L$, wenn $B \geq 0$ 
         \item $\texttt{GLZ } B, L$: $\texttt{GOTO } L$, wenn $B \leq 0:$ 
         \item $\texttt{GZ } B, L$: $\texttt{GOTO } L$, wenn $B = 0$
         \item $\texttt{HALT}$: RAM anhalten
		\end{itemize}
	\item \emph{Variante:} Probalistische RAM
		\begin{itemize}
		 \item $\texttt{RAND } B$: erzeuge zufällige Zahl zwischen 0 und $B$
		\end{itemize}
    \item \emph{Zustand $Z$:}
		\begin{itemize}
		 \item $ip$ Befehlszähler
		 \item Registerinhalt: Funktion $\mathbb{N}_0 \to \mathbb{Z}_0$
		\end{itemize}
	\item jeder Befehl hat einen \emph{Effekt}, der den Zustand ändert (operationelle Semandtik).
	\item ein Programm \emph{berechnet} eine Funktion $f{:}\ \mathbb{Z}^* \to \mathbb{Z}^*$, falls gilt:
		\begin{itemize}
		 \item Bei Eingabe $a_0, a_1, ..., a_{n-1}$ in Register $R_0, R_1, ..., R_{n-1}$ läuft das Programm bis $\texttt{HALT}$
		 \item Danach steht die Ausgabe $f(a_0,...,a_{n-1}) = (b_0,...,b_{m-1})$ in $R_0, ..., R_{m-1}$
		\end{itemize}
    \end{itemize}
	\emph{Church-Turing-These}: intuitive berechenbar = RAM-berechenbar

\subsection{Laufzeit \& Speicherplatz}
\Defi Gegeben ein RAM-Programm, das eine Funktion $f$ berechnet. Sei $x \in \mathbb{Z}^*$ eine Eingabe, dann ist:
\begin{center}
 \begin{tabular}{rp{0.7\textwidth}}
  $T(x)$: & (\emph{Laufzeit}) Gesamtkosten der Arbeitsschritte, bis das Programm \texttt{HALT} bei Eingabe $x$ erreicht. \\
  $S(x)$: & (\emph{Speicherplatz}) Gesamter Platzbedarf, bis das Programm \texttt{HALT} bei Eingabe $x$ erreicht. 
 \end{tabular}
\end{center}
 \paragraph*{Was heißt das konkret?} 2 Interpretationen:
	\begin{itemize}
	 \item \textbf{Einheitskostenmaß} (EKM)
		\begin{itemize}
		\item Jeder Schritt hat Kosten 1.
		\item $T(x)$ = \#Schritte, die bei Eingabe $x$ ausgeführt werden
		\item $S(x)$ = \#\emph{verschiedenen} Register, auf die wir zugreifen
		\end{itemize}
	 \item \textbf{Logarithmisches Kostenmaß} (LKM)
		\begin{itemize}
		 \item Kosten eines Befehls: Gesamtzahl der manipulierten Bits:\\
			z. B.: $R_0 := R_1 + R_2$\\
			Kosten: $\left\lfloor\log(|R_1|+1)\right\rceil + \left\lfloor\log(|R_2|+1)\right\rceil$
		 \item $T(x)$ = Summe der Kosten
		 \item $S(x)$ = Maximum über die Gesamtlänge der Register zu jedem Zeitpunkt
		 \item \textbf{Vorteil:} Realistische bei großen Zahlen
		 \item \textbf{Nachteil:} umständlich
		 \begin{center}
		 \begin{tabular}{rl}
			\begin{minipage}{6cm}
			 \begin{tabular}{r|c|c|c|l}
				\multicolumn{1}{c}{} & \multicolumn{1}{c}{\footnotesize 7} & \multicolumn{1}{c}{\footnotesize 2} & \multicolumn{1}{c}{\footnotesize 3} & \footnotesize= 12\,Bits \\\cline{2-4}
				Schritt 1: & 100 & 2 & 5 & \\\cline{2-4}
				\multicolumn{1}{c}{} & \multicolumn{1}{c}{\footnotesize 3} & \multicolumn{1}{c}{\footnotesize 4} & \multicolumn{1}{c}{\footnotesize 1} & \footnotesize= 8\,Bits \\\cline{2-4}
				Schritt 2: & 5 & 10 & 1 & \\\cline{2-4}
				\multicolumn{1}{c}{} & \multicolumn{1}{c}{\footnotesize 8} & \multicolumn{1}{c}{\footnotesize 1} & \multicolumn{1}{c}{\footnotesize 1} &\footnotesize= 10\,Bits \\\cline{2-4}
				Schritt 3: & 200 & 1 & 1 & \\\cline{2-4}
			 \end{tabular}
			\end{minipage}
		  & $S(x) = 12\,\text{Bits}$
		 \end{tabular}
		 \end{center}
		\end{itemize}
	\end{itemize}
\paragraph*{pragmatische Entscheidung}
\begin{itemize}
\item   EKM normalerweise bei kombinatorischen Algorithmen\\
        $\Rightarrow$ Suchen, Sortieren, Zeichenketten, Graphen
\item   LKM normalerweise bei zahlentheoretischen Algorithmen (Primzahlzest)
\end{itemize}
Vorsicht bei schmutzigen Tricks im EKM!

\paragraph*{Bisher:} Laufzeit für eine feste Eingabe
\paragraph*{Wollen:} Allgemeine Aussage
\begin{itemize}\renewcommand{\labelitemi}{$\hookrightarrow$}
\item   fassen Eingaben nach "`Größe"' zusammen
\item   wie verhält sich der Algorithmus bei bestimmter Eingabegröße?
\end{itemize}
\textbf{Worst-Case-Laufzeit:} schlimmstmögliche Laufzeit für eine Eingabegröße
\begin{align*}
T_{\text{wc}}(n) &= \max T(x) \\
S_{\text{wc}}(n) &= \max S(x)
\end{align*}
$x$ Eingabe, $|x| = n$ $\rightarrow$ Problemabhängig





\begin{itemize}
\item   Oft interesiert uns nur die asymprotische Laufzeit \\
        $\Rightarrow$ $O$-Notation
\end{itemize}

\chapter{Grundlegende Techniken zum Entwurf von Algorithmen}
\section{Teile \& Herrsche / Divide \& Impera}
\begin{itemize}
\item   Geht zurück auf Sun-Tsu (500 v. Chr.) und Neuformulierung durch Machiavelli
\item   Unterteile in kleinere Probleme, löse diese einzeln, setze die Teilergebnisse zusammen.
\item   Beispiele: Mergesort, Quicksort
\item   Bei Divide \& Conquer treten Rekursionsgleichungen auf. Wie löst man diese?
\end{itemize}

\Bsp Multiplizieren von Zahlen.
\begin{itemize}
\item   \textbf{Problem:} Gegeben: $a, b \in \mathbb N$
\item   Gesucht: $a \cdot b$
\item   z. B. $a = 1234, b = 512$\\
        Schulmethode:
        \begin{verbatim}
1234 * 512
----------
      2468
     1234
    6170  
----------
    631808\end{verbatim}
        \begin{itemize}
        \item   Annahme: beide Zahlen haben $n$ Ziffern (zur Not werden vorne an eine Zahl 0 angefügt)
        \item   Dann ist die Anzahl der Multiplikationen und Additionen von Ziffern $\Theta(n^2)$
        \item   Analog im Binärsystem
        \item   Können wir Teile \& Herrsche benutzen um schneller zu multiplizieren
                \[
                    \begin{array}{r|c|c|c|c|c|c|c|c|}
                    \cline{2-9}
                     a & 1 & 0 & 1 & 1 & 0 & 0 & 1 & 0 \\\cline{2-9}
                    \end{array}
                \]
                \[
                    \begin{array}{r|c|c|c|c|c|c|c|c|}
                    \cline{2-9}
                     b & 1 & 1 & 1 & 0 & 1 & 0 & 1 & 1 \\\cline{2-9}
                    \end{array}
                \]
        \item   Idee: Teile $a$ und $b$ in Zahlen mit weniger Ziffern auf, löse rekursiv, setze das ergebnis zusammen
                \[  a\ 
                    \overbrace{
                    \begin{array}{|c|c|c|c|}
                    \cline{1-4}
                     1 & 0 & 1 & 1 \\\cline{1-4}
                    \end{array}}^{a_h}
                    \overbrace{
                    \begin{array}{|c|c|c|c|}
                    \cline{1-4}
                     0 & 0 & 1 & 0 \\\cline{1-4}
                    \end{array}}^{a_l}
                \]
                \[  b\ 
                    \overbrace{
                    \begin{array}{|c|c|c|c|}
                    \cline{1-4}
                     1 & 1 & 1 & 0 \\\cline{1-4}
                    \end{array}
                    }^{b_h}
                    \overbrace{
                    \begin{array}{|c|c|c|c|}
                    \cline{1-4}
                     1 & 0 & 1 & 1 \\\cline{1-4}
                    \end{array}}^{b_l}
                \]
        \item   Schreibe:
                \begin{align*}
                 a &= \overbrace{a_h \cdot 2^{\left\lceil\frac{n}{2}\right\rceil}}^{\text{obere Bits}} + \overbrace{a_l}^{\text{untere $\left\lceil\frac{n}{2}\right\rceil$ Bits}}\\
                 b &= b_h \cdot 2^{\left\lceil\frac{n}{2}\right\rceil} + b_l
                \end{align*}
                \begin{align*}
                 a \cdot b  &= (a_h \cdot 2^{\left\lceil \frac{n}{1} \right\rceil} + a_l) \cdot (b_h \cdot 2^{\left\lceil \frac{n}{1} \right\rceil} + b_l)\\
                            &= \psframebox[linecolor=red]{a_h \cdot b_h} 2^{2 \left\lceil \frac{n}{1} \right\rceil} + 
                                \psframebox[linecolor=blue]{(a_h \cdot b_l + b_h \cdot a_l)} 2^{2 \left\lceil \frac{n}{1} \right\rceil} + \psframebox[linecolor=green]{a_l \cdot b_l}
                \end{align*}
        \end{itemize}
        \paragraph{Algorithmus:}
        \begin{itemize}
        \item   Berechne rekursiv $a_h \cdot b_h$, $a_h \cdot b_l$, $a_l \cdot b_h$, $a_l \cdot b_l$
        \item   Berechne $a \cdot b$ nach Formel mit Shiften und Addieren
        \end{itemize}
        \paragraph{Laufzeitanalyse:}
        \begin{align*}
         T(n) &\leq \begin{cases}
                     O(1), & \text{wenn $n < 3$}\\
                     4 \cdot T\left(\frac{n}{2}\right) + O(n), & \text{sonst}
                    \end{cases}
        \end{align*}
        \begin{description}
        \item[Lösung:]  $T(n) = \Theta(n^2)$
        \item[Problem:] Führen $\underline{4}$ rekursive Multiplikationen durch. Dadurch wird nichts gewonnen.
        \end{description}
        \paragraph*{Genialer Einfall:} (Algorithmus von Karatsuba)
        Betrachte $(a_h + a_l) \cdot (b_h + b_l) = \psframebox[linecolor=red]{a_h \cdot b_h} + \psframebox[linecolor=blue]{a_h \cdot b_l + b_h \cdot a_l} + \psframebox[linecolor=green]{a_l \cdot b_l}$
        \begin{itemize}
        \item   Berechne nach: $\psframebox[linecolor=red]{a_h \cdot b_h}$ und $\psframebox[linecolor=green]{a_l \cdot b_l}$
        \item   Und dann: $(a_h + a_l) \cdot (b_h + b_l) - \psframebox[linecolor=green]{a_l \cdot b_l} - \psframebox[linecolor=red]{a_h \cdot b_h}$
        \end{itemize}
        \paragraph*{Laufzeitanalyse:}
        \begin{align*}
         T(n) &\leq \begin{cases}
                     O(1), & \text{wenn $n < 3$}\\
                     {\color{red}3} \cdot T\left(\frac{n}{2}\right) + O(n), & \text{sonst}
                    \end{cases}
        \end{align*}
        Was kommt heraus?
    \paragraph*{Bemerkungen}
    \begin{itemize}
     \item aktueller Champion der Multiplikationsalgorithmen: M. Fürer (2007/09) $O(n \log n 2^{\log_* n})$
     \item Vermutung: $\Theta(n \log n)$ ist optimal
    \end{itemize}
\end{itemize}

\subsection{Lösen von Rekursionsgleichungen}
\begin{itemize}
\item   Methode 1: Raten \& \textbf{Induktion}
\item   Methode 2: \textbf{Wiederholt einsetzen} \& Muster erkennen
        \begin{align*}
         T(n)   &\leq 3 T\left(\frac{n}{2}\right) + c \cdot n \\
                &\leq 3 \cdot \left(3 \cdot T\left(\frac{n}{4}\right) + c \cdot \frac{n}{2}\right) + c \cdot n \\
                &\leq 3 \cdot \left(3 \cdot  \left(3 \cdot T\left(\frac{n}{8}\right) + c \cdot \frac{n}{4}\right) + c \cdot \frac{n}{2}\right) + c \cdot n\\
                &\leq 3^3 T\left(\frac{n}{8}\right) + c \frac{3^2}{2^2}n + c \frac{3}{2}n + c \cdot n\\
                &\vdots \hspace{1cm}\text{$k$ Schritte}\\
                &\leq 3^k T\left(\frac{n}{2^k}\right) + c \left(\frac{3}{2}\right)^{k-1}n + c\left(\frac{3}{2}\right)^{k-2} + ... + cn\\
                & \text{nach $k = \log n$ Schritten ist $\frac{n}{2}$ konstant.}\\
                &= \sum\limits_{k = 0}^{\log_2 n} \left(\frac{3}{2}\right)^k c \cdot n\\
                &= cn \frac{\left(\frac{3}{2}\right)^{1 + \log_2 n} - 1}{\frac{3}{2} - 1} \tag{geometrische Reihe}\\
                &\leq 2 cn \frac{3}{2} \left(\frac{3}{2}\right)^{1 + \log_2 n}\\
                &= 3 cn \cdot n^{\log_2 \frac{3}{2}} = 3 c \cdot n^{\log_2 3}
        \end{align*}
        \paragraph*{Laufzeit} $\Theta(n^{\log_2 3})$, $\log_2 3 \approx 1{,}5385$
% 2011104 hier angefangen wegen itemize
\item   Methode 3: Bild malen \& Muster erkennen (\textbf{Rekursionsbaummethode})
        \[T(n) \leq 3 T\left(\frac{n}{2} + O(n)\right)\]
        Male einen Knoten für jeden rekursiven Aufruf
        \begin{center}
        \begin{minipage}{0.3\textwidth}
        \psset{levelsep=0.6cm,treesep=0.3cm}
        \pstree{\Tr*{$T(n)$}}{
            \pstree{\Tr*{$T\left(\frac{n}{2}\right)$}}{
                \pstree{\Tr*{$T\left(\frac{n}{4}\right)$}}{
                    \pstree{\Tr*{$T\left(\frac{n}{8}\right)$}}{
                        \pstree{\Tr*{$\vdots$}}{
                            \Tr{$T(1)$}
                        }
                    }
                    \Tr{$\vdots$}\Tr{$\vdots$}
                }
                \Tr{$\vdots$}
                \Tr{$\vdots$}
            }
            \pstree{\Tr*{$T\left(\frac{n}{2}\right)$}}{
                \Tr{$\vdots$}
                \Tr{$\vdots$}
                \Tr{$\vdots$}
            }
            \pstree{\Tr*{$T\left(\frac{n}{2}\right)$}}{
                \Tr{$\vdots$}
                \Tr{$\vdots$}
                \Tr{$\vdots$}
            }
        }
        \end{minipage}
        \vline
        \begin{minipage}{0.3\textwidth}
            \centering
            $c \cdot n$ \\[0.3cm]
            $3 \cdot c \cdot \frac{n}{2}$ \\[0.3cm]
            $9 \cdot c \cdot \frac{n}{4}$ \\[0.3cm]
            $27 \cdot c \cdot \frac{n}{8}$ \\[0.3cm]
            \hspace{0cm} \\[0.3cm]
            $3^k \cdot c \cdot \frac{n}{2^k}$
        \end{minipage}
        \end{center}
        Es gibt $O(\log_2 n)$ Ebenen $\leq \log n$ Ebenen
        \begin{itemize}
        \item Addiere alle Kosten $\sum\limits_{k = 0}^{\log n} \left(\frac{3}{2}\right)^k c \cdot n = \Theta(n^{\log_2 3})$
        \end{itemize}
\item   Methode 4: Allgemeines Rezept: \textbf{Master Theorem}
        \Satz Sei $a \geq 1, b \geq 1, f{:}\ \mathbb{N} \to \mathbb{N}$. Sei $T(n) = a \cdot T\left(\frac{n}{b}\right) + f(n)$ eine Rekursion ($T(n) = O(1)$ für $n \leq 2$). Dann gilt:
        \begin{enumerate}
         \item Wenn $f(n) = O(n^{(\log_b a) - \varepsilon})$ für $\varepsilon > 0$, dann ist 
                \[T(n) = \Theta(n^{\log_b a})\]
         \item Wenn $f(n) = \Theta(n^{\log_b a})$ ist, dann ist
                \[T(n) = \Theta\left(n^{\log_b a} \log n\right)\]
         \item Wenn $f(n) = \Omega(n^{(\log_b a) + \varepsilon})$ für $\varepsilon > 0$ und $\exists c < 1{:}\ a f\left(\frac{n}{b}\right) < c f\left(n\right)$, dann ist
                \[T(n) = \Theta\left(f(n)\right)\]
        \end{enumerate}
        \Bsp    \begin{itemize}
        \item   $T(n) \leq 3 T\left(\frac{n}{2}\right) + O(n)$\\
                MT anwendbar mit $a = 3, b = 2, f(n) = c \cdot n$
                \begin{align*}
                 f(n)   &= O(n^{\log_b a - \varepsilon}) \\
                        &= O(n^{\log_2 3} - n^{1{,}5...}), \varepsilon \approx 0{,}5
                \end{align*}
                Fall 1: $T(n) = \Theta(n^{\log_2 3})$
        \item   $T(n) = 4 T\left(\frac{n}{2}\right) + O(n)$ \\
                MT mit $a = 4, b = 2, f(n) = c \cdot n$
                \begin{align*}
                 f(n) &= O(n^{2 - \varepsilon}) \text{ für $\varepsilon \approx 0{,}5$}
                \end{align*}
                Fall 1: $T(n) = \Theta(n^2)$
        \item   $T(n) = 2 T\left(\frac{n}{2}\right) + O(n)$ \\
                MT anwendbar mit $a = 2, b = 2, f(n) = c \cdot n$
                \begin{align*}
                 n^{\log_b a} &= n\\
                 f(n) &= \Theta(n^{\log_b a})
                \end{align*}
                Fall 2: $T(n) = \Theta(n \log n)$
        \item   $T(n) = T\left(\frac{n}{2}\right) + O(1)$ \\
                MT anwendbar mit $a = 1, b = 2, f(n) = c$
                \begin{align*}
                 n^{\log_b a} &= n^0 = 1\\
                 f(n) &= \Theta(n^{\log_b a})
                \end{align*}
                Fall 2: $T(n) = \Theta(\log n)$
        \item   $T(n) = T\left(\frac{3}{4}n\right) + O(n)$\\
                MT anwendbar mit $a = 1, b = \frac{4}{3}, f(n) = d \cdot n$\\
                \begin{align*}
                 n^{\log_b a} &= n^0 = 1 \\
                 f(n) &= \Omega(n^{0+\varepsilon}), \text{z. N. für $\varepsilon = \dfrac{1}{2}$} \\
                 a \cdot f\left(\frac{n}{b}\right) &\leq 1 \cdot \frac{3}{4} \cdot dn
                    &= \frac{3}{4} d n \leq \frac{3}{4} f(n)
                \end{align*}
                Fall 3: $T(n) = \Theta(n)$
        \end{itemize}
        \Bew
            \begin{center}
             \begin{minipage}{0.4\textwidth}
            \centering
        \psset{levelsep=0.6cm,treesep=0.3cm}
        \pstree{\Tr*{$T(n)$}}{
            \pstree{\Tr*{$T\left(\frac{n}{b}\right)$}}{
                \pstree{\Tr*{$T\left(\frac{n}{b^2}\right)$}}{
                    \pstree{\Tr*{$T\left(\frac{n}{b^3}\right)$}}{
                        \pstree{\Tr*{$\vdots$}}{
                            \Tr{$T(1)$}
                        }
                    }
                    \Tr{$\vdots$}\Tr{$\vdots$}
                }
                \Tr{$\hdots$}
                \Tr{$\vdots$}
            }
            \pstree{\Tr*{$T\left(\frac{n}{b}\right)$}}{
                \Tr{$\vdots$}
                \Tr{$\hdots$}
                \Tr{$\vdots$}
            }
            \pstree{\Tr*{$T\left(\frac{n}{b}\right)$}}{
                \Tr{$\vdots$}
                \Tr{$\hdots$}
                \Tr{$\vdots$}
            }
        }
        (jeweils $a$ Kinder)
        \end{minipage}
        \vline
        \begin{minipage}{0.3\textwidth}
            \centering
            $f(n)$ \\[0.3cm]
            $a \cdot f\left(\frac{n}{b}\right)$ \\[0.3cm]
            $a^2 \cdot f\left(\frac{n}{b^2}\right)$ \\[0.3cm]
            $a^3 \cdot f\left(\frac{n}{b^3}\right)$ \\[0.3cm]
            \hspace{0cm} \\[0.3cm]
            $a^k \cdot f\left(\frac{n}{b^k}\right)$ \\[0.3cm]
        \end{minipage}
        \end{center}
        $\leq \log_b n$ Ebenen, Gesamtkosten: $\sum\limits_{k = 0}^{\log_b n} a^k f\left(\frac{n}{b^k}\right)$\\
        \begin{itemize}
        \item   $T(n) = 2 T \left(\frac{n}{2}\right) + \Theta(n \log n)$\\
                MT nicht anwendbar
        \item   $T(n) = 2 T \left(\frac{n}{3}\right) + T \left(\frac{2}{3} n\right) + \Theta(n)$\\
                MT nicht anwendbar (passt nicht ins Schema)
        \end{itemize}
\end{itemize}

\section{Teile und Herrsche}
\section{Dynamisches Programmieren}
\section{Abschneiden und Suchen}
\section{Gierige Algorithmen}
\section{Lineares Programmieren}
\section{Hashing und Fingerprinting}
\section{untere Schranken}
\chapter{Paradigmen}
\section{Worst-Case Analyse}
\section{Average-Case Analyse}
\section{Probabilistische Algorithmen}
\section{Amortisierte Analyse}
\section{Online-Algorithmen und kompetitive Analyse}
\section{Approximationsalgorithmen}
\section{Map-Reduce ????}
\chapter{Algorithmen und Datenstrukturen}
\section{Suchen, Sortieren und Auswählen}
\section{Viterbi-Algorithmus}
\section{TSP}
\section{Datenstrukturen: Splaybäume, Erdbebenheaps, Disjoint Set Union, Bloom Filter}
\section{Graphenalgorithmen: Flüsse, Paarungen, Schnitte, minimale aufspannende Bäume, kürzeste Wege}
\section{Simplex Algorithmus mit Anwendungen, Ellipsoid Algorithmus}
\section{Internet-Algorithmen: Page-Rank, konsistentes Hashing}
\section{Ski-Rental, Paging}
\chapter{Komplexität}
\section{Klassenzoo: LOGSPACE, P, NP, coNP, PSPACE, etc}
\section{NP-Vollständigkeit, Satz von Cook, Reduktionskonzept}
\section{Umgang mit schweren Problemen}
\end{document}
