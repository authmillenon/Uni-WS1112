\documentclass[a4paper,10pt]{scrbook}

% Seitenlayout
\usepackage[a4paper,top=1.5cm,right=2.0cm,bottom=2cm,left=2.0cm]{geometry}
\usepackage{fancyhdr}

% Zeichen
\usepackage[ngerman]{babel}
\usepackage[utf8]{inputenc}
\usepackage{amsmath, amsthm, amssymb}
\usepackage{marvosym}
\usepackage[gen]{eurosym}

% Zeichensatz
\usepackage[colorlinks,%
	    citecolor=black,%
	    filecolor=black,%
	    linkcolor=black,%
	    urlcolor=black,%
	    pdftitle = {Mitschrift - CB},%
	    pdfauthor = {Martin Lenders}%
	]{hyperref}
\usepackage{listings}
\usepackage{algorithmic}
\usepackage{hyperref}
\usepackage{epigraph}
\usepackage{slashbox}
\usepackage{enumitem}
\usepackage{cancel}

% Grafik
\usepackage[usenames]{pstricks}
\usepackage{pst-plot,pst-node,pstricks-add,pst-tree}

% Color definitions
\definecolor{lgray}{gray}{0.95}
\definecolor{save}{rgb}{0.498,0,0}
\definecolor{identifier}{rgb}{0,0,0.1}
\definecolor{string}{rgb}{0.192,0,1}
\definecolor{comment}{rgb}{0.25,0.5,0.37}
\definecolor{yellow}{rgb}{1,1,0}
\definecolor{sand}{rgb}{1,1,.75}
\definecolor{red}{rgb}{1,0,0}
\definecolor{melon}{rgb}{1,0.6,.5}
\definecolor{green}{rgb}{0,1,0}
\definecolor{lime}{rgb}{.75,1,.75}
\definecolor{blue}{rgb}{0,0,1}
\definecolor{azure}{rgb}{.75,.75,1}

% Einstellungen für Pakete
\lstset{
	tabsize=8,
	frame=,
	basicstyle=\footnotesize\changefont{pcr}{m}{n},
	emphstyle=\textit,
	numberstyle=\tiny\textsf,
	numbersep=5pt,
	numbers=none,
	keywordstyle=\color{save}\textbf,
	identifierstyle=\color{identifier},
	stringstyle=\color{string},
	showstringspaces=false,
	commentstyle=\color{comment},
	extendedchars=true,
	xleftmargin=1em,
	inputencoding=utf8,
	mathescape=true;
}

\psset{%
	algebraic=true,
	angleA=0,%
	angleB=180,%
	unit=1cm,%
	subgriddiv=0,%
	griddots=5,%
	gridlabels=7pt%
}

\renewcommand{\chaptermark}[1]{markboth{#1}{}}
\renewcommand{\sectionmark}[1]{\markright{\thesection\ #1}}
\pagestyle{fancy}
\fancyhf{}
\fancyfoot[LE,RO]{\sffamily\thepage}
\fancyhead[LE]{\footnotesize\sffamily\bfseries\leftmark}
\fancyhead[RO]{\footnotesize\sffamily\rightmark} 

% Eigene Befehle
\newcommand{\changefont}[3]{\fontfamily{#1} \fontseries{#2} \fontshape{#3} \selectfont}
\newcommand{\Defi}{\paragraph*{Definition:}}
\newcommand{\Lemma}{\paragraph*{Lemma:}}
\newcommand{\Satz}{\paragraph*{Satz:}}
\newcommand{\Beh}{\paragraph*{Behauptung:}}
\newcommand{\Geg}{\paragraph*{Gegeben:}}
\newcommand{\Ges}{\paragraph*{Gesucht:}}
\newcommand{\Los}{\paragraph*{Lösung:}}
\newcommand{\Bew}{\paragraph*{Beweis:}}
\newcommand{\Bsp}{\paragraph*{Beispiel:}}
\newcommand{\carsten}{$\square$}

% Neudefinitionen
\renewcommand{\thepart}{\Alph{part}}
\renewcommand{\sectionmark}[1]{\markright{\thesection\ #1}}
\renewcommand{\algorithmiccomment}[1]{// #1}

\makeatletter \newcommand{\greek}[1]{% 
\expandafter\@greek\csname c@#1\endcsname } \newcommand{\@greek}[1]{% 
$\ifcase#1\or\alpha\or\beta\or\gamma\or\delta\or\varepsilon \or\zeta\or\eta\or\theta\or\iota\or\kappa\or\lambda \or\mu\or\nu\or\xi\or o\or\pi\or\varrho\or\sigma \or\tau\or\upsilon\or\phi\or\chi\or\psi\or\omega \else\@ctrerr\fi$ } \makeatother

\title{Mitschrift\\{\LARGE Höhere Algorithmik}}
\author{gehalten von Prof. Dr. Wolfgang Mulzer \\ mitgeschrieben von Martin Lenders}
\subject{Achtung: Dieses Dokument ist \emph{nur} eine Mitschrift der Vorlesung "Höhere Algorithmik" WiSe2011/12.
	Sie wurde während der Vorlesung angefertigt. Es wird aber seitens des Autors keine Garantie auf
	Vollständigkeit und Richtigkeit des Inhalts gegeben.}
\begin{document}
\maketitle
\tableofcontents
\chapter{Einleitung}
\section{Klausur}
24.04.2012 10-12 Uhr
\begin{itemize}
 \item Auswahlproblem\\
		Gegeben: 
		\begin{description}
		 \item $S$, Menge, total geordnet, $n$ Elemente paarweise verschieden.
		 \item $k \in \{1, ..., n\}$
		\end{description}
		Gesucht:
		\begin{description}
		 \item Element $s \in S$ mit $\operatorname{Rg}(s) = k$, d. h. $S$ enthält $k -1$ viele Elemente kleiner als $k$
		\end{description}
\end{itemize}
\paragraph{Lemma:} Angenommen es  existiert Funktion $\mathtt{SPLITTER}(S)$ die ein Element $q \in S$ liefert, so dass \[\operatorname{Rg}(q) \in \left\{\left\lceil\frac{1}{4} n\right\rceil, ..., \left\lfloor\frac{3}{4}n\right\rfloor\right\}.\]
Dann kann man das Auswahlproblem in $O(n)$ Zeit läsen, wenn $\mathtt{Splitter}$ nichts kostet.

\paragraph{Wie implementiert man \texttt{SPLITTER}?}
\begin{description}
 \item[1. Möglichkeit:] Wähle den \texttt{SPLITTER} zufällig. Mit Wahrscheinlichkeit $\frac{1}{2}$ erwischen wir einen guten Splitter.
		Im Durchschnitt ist das wohl gut genug. \\
		$\Rightarrow$ Erwartungswert der Laufzeit ausrechnen, etc., \emph{Randomisierte Algorithmen (später)}
 \item[2. Möglichkeit:] BFRPT-Methode:
		\begin{description}
		\item[rekursiv] Wähle eine Stichprobe $S' \subseteq S$ mit $|S'| = \left\lceil\frac{n}{5}\right\rceil$, so dass der Median von $S'$ ein guter Splitter von $S$ ist. Bestimme rekursiv den Median von $S$!
		\item[Wie finden wir $\boldsymbol{S'}$?]\hspace{0cm}\\
			\begin{description}
			\item[1. Versuch:] Nimm die ersten $\left\lceil\frac{n}{5}\right\rceil$. Elsement von $S$. $\Rightarrow$ \emph{klappt nicht.}
			\item[2. Versuch:] Nimm jedes 5. Element. $\Rightarrow$ \emph{auch nicht}
			\end{description}
		\end{description}
 \item[3. Möglichkeit] Unterteile $S$ in 5er Gruppen. Nimm aus jeder 5er Gruppe den Median. $S'$ ist die Menge der Mediane.
		\paragraph{Beispiel}
		\[
		\begin{matrix}
		 S & \rnode{51l}{7} & 1 & 5 & 9 & \rnode{51r}{18} \\&\\
		 S' 
		\end{matrix}\quad
		\begin{matrix}
		 \rnode{52l}{30} & 6 & 11 & 15 & \rnode{52r}{20} \\&\\
		 &
		\end{matrix}
		\ncbox[nodesep=2px]{51l}{51r}7
		\ncbox[nodesep=2px]{52l}{52r}15
		\]
		\paragraph{Lemma:} Der Median von $S'$ ist ein guter Splitter von $S$, wenn $n$ groß genug ist.
		\paragraph{Beweis:} Betrachte $S'$ als von links nach rechts sortiert (für den Beweis) und zugehörige 5er Gruppen.
			
			Die Lage im Bild.
		\begin{center}
		 \begin{pspicture}(0,0)(10,5)
		  \psdot(0,4)\psdot(1,4)\psdot(2,4)\psdot(3,4)\psdot(4,4)\psdot(5,4)\psdot(6,4)\psdot(7,4)\psdot(8,4)
		  \psdot(0,3)\psdot(1,3)\psdot(2,3)\psdot(3,3)\psdot(4,3)\psdot(5,3)\psdot(6,3)\psdot(7,3)\psdot(8,3)\psdot(9,3)
	\rput(-1,2){\color{green}$S'$}	  \psdot[linecolor=green](0,2)\psdot[linecolor=green](1,2)\psdot[linecolor=green](2,2)\psdot[linecolor=green](3,2)\psdot[linecolor=green](4,2)\psdot[linecolor=green](5,2)\psdot[linecolor=green](6,2)\psdot[linecolor=green](7,2)\psdot[linecolor=green](8,2)\psdot[linecolor=green](9,2)
		  \psdot(0,1)\psdot(1,1)\psdot(2,1)\psdot(3,1)\psdot(4,1)\psdot(5,1)\psdot(6,1)\psdot(7,1)\psdot(8,1)\psdot(9,1)
		  \psdot(0,0)\psdot(1,0)\psdot(2,0)\psdot(3,0)\psdot(4,0)\psdot(5,0)\psdot(6,0)\psdot(7,0)\psdot(8,0)
		  \psset{arrows=<-}
		  \psline(0,4)(0,3)\psline(1,4)(1,3)\psline(2,4)(2,3)\psline(3,4)(3,3)\psline(4,4)(4,3)
		  \psline(5,4)(5,3)\psline(6,4)(6,3)\psline(7,4)(7,3)\psline(8,4)(8,3)
		  \psline(0,3)(0,2)\psline(1,3)(1,2)\psline(2,3)(2,2)\psline(3,3)(3,2)\psline(4,3)(4,2)
		  \psline(5,3)(5,2)\psline(6,3)(6,2)\psline(7,3)(7,2)\psline(8,3)(8,2)\psline(9,3)(9,2)
		  \psset{arrows=->}
		  \psline[linecolor=green](0,2)(1,2)\psline[linecolor=green](1,2)(2,2)
		  \psline[linecolor=green](2,2)(3,2)\psline[linecolor=green](3,2)(4,2)
		  \psline[linecolor=green](4,2)(5,2)\psline[linecolor=green](5,2)(6,2)
		  \psline[linecolor=green](6,2)(7,2)\psline[linecolor=green](7,2)(8,2)
		  \psline[linecolor=green](8,2)(9,2)
		  \psline(0,1)(0,2)\psline(1,1)(1,2)\psline(2,1)(2,2)\psline(3,1)(3,2)\psline(4,1)(4,2)
		  \psline(5,1)(5,2)\psline(6,1)(6,2)\psline(7,1)(7,2)\psline(8,1)(8,2)\psline(9,1)(9,2)
		  \psline(0,0)(0,1)\psline(1,0)(1,1)\psline(2,0)(2,1)\psline(3,0)(3,1)\psline(4,0)(4,1)
		  \psline(5,0)(5,1)\psline(6,0)(6,1)\psline(7,0)(7,1)\psline(8,0)(8,1)
		  \psframe[linecolor=red](4.8,1.8)(5.2,2.2)
		  \psset{arrows=-}
		  \psline[linecolor=blue](-0.2,-0.2)(5.2,-0.2)(5.2,1.2)(4.2,1.2)(4.2,2.2)(-0.2,2.2)(-0.2,-0.2)
		  \psline[linecolor=blue](9.2,4.2)(9.2,1.8)(5.8,1.8)(5.8,2.8)(4.8,2.8)(4.8,4.2)(9.2,4.2)
		  \uput{0.3cm}[-45](5,2){\color{red}$q$}
		  \uput{0.3cm}[180](-0.2,1){\color{blue}$< q$}
		  \uput{0.3cm}[0](9.2,3){\color{blue}$> q$}
		 \end{pspicture}
		\end{center}
		\begin{itemize}
		 \item Es sind $\underbrace{\left\lceil\frac{1}{2}\underbrace{\left\lceil\frac{n}{5}\right\rceil}_{|5|}\right\rceil}_{\#\text{Gruppen $m$}} - 3$
		 \item definitiv größer als $q$.
		 \item Ebenso gibt es definitiv $3 \left\lceil\frac{1}{2}\left\lceil\frac{n}{5}\right\rceil\right\rceil - 3$ Elemente kleiner als $q$.
		 \item Es gilt:
				\begin{align*}
				 \left\lceil\frac{1}{2}\left\lceil\frac{n}{5}\right\rceil\right\rceil - 3 &\geq
				3 \cdot \frac{1}{2} \cdot \frac{1}{5} \cdot n - 3 \\
					&= \frac{3}{10} - 3
				\end{align*}
				\begin{center}
				Wir wollen: $\frac{3}{10} n - 3 \overset{!}{\geq} \frac{1}{4} n \Rightarrow n \geq 60$. \hfill$\square$
				\end{center}
			\end{itemize}
		\begin{verbatim}
        Algorithmus: Select(S,K)
            if |S| < 100 then
                brute_force
         /  Unterteile S in 5er Gruppen
Splitter |  S <- {Median jeder 5er-Gruppe}
         \  q <- Select(S', ceil((|S'| + 1))/2)
            S_< <- {s in S for s < q}
            S_> <- {s in S for s > q}
            if |S_<| >= k then
                return SELECT(S_<, k)
            else if |S_<| = k - 1 then
                return q
            else
                return SELECT(S_>, k - |S_<| - 1)
        \end{verbatim}
		\paragraph{Laufzeitanalyse}
		\[T(n) \leq \begin{cases}
		             O(1), & n < 100 \\
					 O(n) + T\left(\left\lceil\frac{n}{5}\right\rceil\right) + T\left(\frac{3}{4} n\right), & \text{sonst}
		            \end{cases}\]
		\paragraph{Behauptung} $T(n) = O(n)$
		\paragraph{Beweis durch Induktion}
			\begin{description}
			 \item[Behauptung] $\exists$ Konstante $\alpha > 0$, so dass $T(n) < \alpha n$ ist.
			 \item für $n < 100$: $\checkmark$
			 \item[Induktionsschritt] 
				\begin{align*}
				 T(n) &\leq cn + T\left(\left\lceil\frac{n}{5}\right\rceil\right) + T\left(\frac{3}{4} n\right) \\
					  &\overset{\text{IA}}{\leq } cn + \alpha \left\lceil\frac{n}{5}\right\rceil + \alpha \frac{3}{4} n\\
					  &\leq cn + \alpha \left(\frac{n}{5} + 1\right) + \alpha \frac{3}{4} n \\
					  &= cn + \alpha \left(\frac{1}{5} +  \frac{3}{4}\right)n + \alpha \\
					  &= cn + \frac{19}{20} \alpha n + \alpha \\
					  &\overset{!}{\leq} \alpha n
				\end{align*}
				\begin{align*}
				 \alpha n &\geq cn + \frac{19}{20} \alpha n + \alpha\\
				 \frac{1}{20} \alpha n & \geq cn + \alpha &&| : \frac{n}{20} \\
				 \alpha & \geq 20c + \underbrace{\frac{20\alpha}{n}}_{\leq \frac{\alpha}{5}}
				\end{align*}
				Es gilt: $20 c + \frac{\alpha}{5} \geq 20c + \frac{20\alpha}{n}$ \\
				D. h., wenn:
					\begin{align*}
					(*)\ \alpha &\geq 20c + \frac{\alpha}{5}, \text{dann} \\
					\alpha &\geq 20c + \frac{20\alpha}{n}
					\end{align*}
					(*) gilt, wenn $\alpha \geq 25c$ ist \hfill$\square$
			\end{description}
		Algorithmus:
		\begin{center}
		 \begin{tabular}{ll}
			\textsc{Blum} & Turing-Award 1995 \\
			\textsc{Floyd} & Turing-Award 1978 \\
			\textsc{Pratt} & --- \\
			\textsc{Rivest} & Turing-Award 2002 \\
			\textsc{Tarjan} & Turing-Award 1986 
		 \end{tabular}
		\end{center}
		\paragraph{Bemerkung:}
		\begin{itemize}
		 \item Algorithmus kurz, elegant, optimal.
		 \item Benutzt nicht triviale Struktur im Problem.
		 \item Laufzeiteigenschaften nicht offensichtlich, brauchen Analyse und Beweis.
		 \item Theoretisches Ergebnis.
		\end{itemize}

\end{description}


\section{Berechnungsmodell}
\begin{itemize}
 \item Bei der Analyse von Algorithmen zählen wir "`elementare Schritte"'
 \item Was ist das?
 \item Berechnungsmodell: abstraktes, mathematisches Modell von Rechnern, um Begriffe \emph{Berechenbarkeit}, \emph{Algorithmus}, \emph{Laufzeit}, \emph{Speicherplatz}, etc. zu definieren
      \paragraph*{Beispiele} Turingmaschine, $\mu$-Rekurion, Game of Live, $\lambda$-Kalkül, Markov-Modelle, ...
 \item Für uns: \textbf{Registermaschine} (\emph{R}andom \emph{A}ccess \emph{M}aschine)
\end{itemize}
\subsection{Registermaschine}
\Defi Für eine \emph{Registermaschine} gilt folgendes:
    \begin{itemize}
    \item $\infty$ viele register $R_0, R_1, R_2, ...$
          \[
           \begin{array}{|c|c|c|ccccc}
            \hline
            R_0 & R_1 & R_2 & \dots & & & & \\
            \hline
           \end{array}
          \]
    \item jedes Register speichert eine ganze Zahl $\in \mathbb{Z}$
    \item \textbf{Programm} endliche Folge von Befehlen\\
		\textbf{Befehlstypen}
		\begin{itemize}
		 \item $A := B \operatorname{op} C$, dabei ist $A,B,C$: 
			\begin{itemize}
			 \item Register $R_i$
			 \item indirekt $(R_i)$
			 \item Konstante $c$
			\end{itemize}
			$\operatorname{op} \in \{+,-,\times,/\}$ ($/$ als ganzzahlige Division)
		 \item $A := B$
         \item $\texttt{GOTO } L$, $L$: Label, Programmzeile (auch indirekt)
         \item $\texttt{GGZ } B, L$: $\texttt{GOTO } L$, wenn $B \geq 0$ 
         \item $\texttt{GLZ } B, L$: $\texttt{GOTO } L$, wenn $B \leq 0:$ 
         \item $\texttt{GZ } B, L$: $\texttt{GOTO } L$, wenn $B = 0$
         \item $\texttt{HALT}$: RAM anhalten
		\end{itemize}
	\item \emph{Variante:} Probalistische RAM
		\begin{itemize}
		 \item $\texttt{RAND } B$: erzeuge zufällige Zahl zwischen 0 und $B$
		\end{itemize}
    \item \emph{Zustand $Z$:}
		\begin{itemize}
		 \item $ip$ Befehlszähler
		 \item Registerinhalt: Funktion $\mathbb{N}_0 \to \mathbb{Z}_0$
		\end{itemize}
	\item jeder Befehl hat einen \emph{Effekt}, der den Zustand ändert (operationelle Semandtik).
	\item ein Programm \emph{berechnet} eine Funktion $f{:}\ \mathbb{Z}^* \to \mathbb{Z}^*$, falls gilt:
		\begin{itemize}
		 \item Bei Eingabe $a_0, a_1, ..., a_{n-1}$ in Register $R_0, R_1, ..., R_{n-1}$ läuft das Programm bis $\texttt{HALT}$
		 \item Danach steht die Ausgabe $f(a_0,...,a_{n-1}) = (b_0,...,b_{m-1})$ in $R_0, ..., R_{m-1}$
		\end{itemize}
    \end{itemize}
	\emph{Church-Turing-These}: intuitive berechenbar = RAM-berechenbar

\subsection{Laufzeit \& Speicherplatz}
\Defi Gegeben ein RAM-Programm, das eine Funktion $f$ berechnet. Sei $x \in \mathbb{Z}^*$ eine Eingabe, dann ist:
\begin{center}
 \begin{tabular}{rp{0.7\textwidth}}
  $T(x)$: & (\emph{Laufzeit}) Gesamtkosten der Arbeitsschritte, bis das Programm \texttt{HALT} bei Eingabe $x$ erreicht. \\
  $S(x)$: & (\emph{Speicherplatz}) Gesamter Platzbedarf, bis das Programm \texttt{HALT} bei Eingabe $x$ erreicht. 
 \end{tabular}
\end{center}
 \paragraph*{Was heißt das konkret?} 2 Interpretationen:
	\begin{itemize}
	 \item \textbf{Einheitskostenmaß} (EKM)
		\begin{itemize}
		\item Jeder Schritt hat Kosten 1.
		\item $T(x)$ = \#Schritte, die bei Eingabe $x$ ausgeführt werden
		\item $S(x)$ = \#\emph{verschiedenen} Register, auf die wir zugreifen
		\end{itemize}
	 \item \textbf{Logarithmisches Kostenmaß} (LKM)
		\begin{itemize}
		 \item Kosten eines Befehls: Gesamtzahl der manipulierten Bits:\\
			z. B.: $R_0 := R_1 + R_2$\\
			Kosten: $\left\lfloor\log(|R_1|+1)\right\rceil + \left\lfloor\log(|R_2|+1)\right\rceil$
		 \item $T(x)$ = Summe der Kosten
		 \item $S(x)$ = Maximum über die Gesamtlänge der Register zu jedem Zeitpunkt
		 \item \textbf{Vorteil:} Realistische bei großen Zahlen
		 \item \textbf{Nachteil:} umständlich
		 \begin{center}
		 \begin{tabular}{rl}
			\begin{minipage}{6cm}
			 \begin{tabular}{r|c|c|c|l}
				\multicolumn{1}{c}{} & \multicolumn{1}{c}{\footnotesize 7} & \multicolumn{1}{c}{\footnotesize 2} & \multicolumn{1}{c}{\footnotesize 3} & \footnotesize= 12\,Bits \\\cline{2-4}
				Schritt 1: & 100 & 2 & 5 & \\\cline{2-4}
				\multicolumn{1}{c}{} & \multicolumn{1}{c}{\footnotesize 3} & \multicolumn{1}{c}{\footnotesize 4} & \multicolumn{1}{c}{\footnotesize 1} & \footnotesize= 8\,Bits \\\cline{2-4}
				Schritt 2: & 5 & 10 & 1 & \\\cline{2-4}
				\multicolumn{1}{c}{} & \multicolumn{1}{c}{\footnotesize 8} & \multicolumn{1}{c}{\footnotesize 1} & \multicolumn{1}{c}{\footnotesize 1} &\footnotesize= 10\,Bits \\\cline{2-4}
				Schritt 3: & 200 & 1 & 1 & \\\cline{2-4}
			 \end{tabular}
			\end{minipage}
		  & $S(x) = 12\,\text{Bits}$
		 \end{tabular}
		 \end{center}
		\end{itemize}
	\end{itemize}
\paragraph*{pragmatische Entscheidung}
\begin{itemize}
\item   EKM normalerweise bei kombinatorischen Algorithmen\\
        $\Rightarrow$ Suchen, Sortieren, Zeichenketten, Graphen
\item   LKM normalerweise bei zahlentheoretischen Algorithmen (Primzahlzest)
\end{itemize}
Vorsicht bei schmutzigen Tricks im EKM!

\paragraph*{Bisher:} Laufzeit für eine feste Eingabe
\paragraph*{Wollen:} Allgemeine Aussage
\begin{itemize}\renewcommand{\labelitemi}{$\hookrightarrow$}
\item   fassen Eingaben nach "`Größe"' zusammen
\item   wie verhält sich der Algorithmus bei bestimmter Eingabegröße?
\end{itemize}
\textbf{Worst-Case-Laufzeit:} schlimmstmögliche Laufzeit für eine Eingabegröße
\begin{align*}
T_{\text{wc}}(n) &= \max T(x) \\
S_{\text{wc}}(n) &= \max S(x)
\end{align*}
$x$ Eingabe, $|x| = n$ $\rightarrow$ Problemabhängig





\begin{itemize}
\item   Oft interesiert uns nur die asymprotische Laufzeit \\
        $\Rightarrow$ $O$-Notation
\end{itemize}

\chapter{Grundlegende Techniken zum Entwurf von Algorithmen}
\section{Teile \& Herrsche / Divide \& Impera}
\begin{itemize}
\item   Geht zurück auf Sun-Tsu (500 v. Chr.) und Neuformulierung durch Machiavelli
\item   Unterteile in kleinere Probleme, löse diese einzeln, setze die Teilergebnisse zusammen.
\item   Beispiele: Mergesort, Quicksort
\item   Bei Divide \& Conquer treten Rekursionsgleichungen auf. Wie löst man diese?
\end{itemize}

\Bsp Multiplizieren von Zahlen.
\begin{itemize}
\item   \textbf{Problem:} Gegeben: $a, b \in \mathbb N$
\item   Gesucht: $a \cdot b$
\item   z. B. $a = 1234, b = 512$\\
        Schulmethode:
        \begin{verbatim}
1234 * 512
----------
      2468
     1234
    6170  
----------
    631808\end{verbatim}
        \begin{itemize}
        \item   Annahme: beide Zahlen haben $n$ Ziffern (zur Not werden vorne an eine Zahl 0 angefügt)
        \item   Dann ist die Anzahl der Multiplikationen und Additionen von Ziffern $\Theta(n^2)$
        \item   Analog im Binärsystem
        \item   Können wir Teile \& Herrsche benutzen um schneller zu multiplizieren
                \[
                    \begin{array}{r|c|c|c|c|c|c|c|c|}
                    \cline{2-9}
                     a & 1 & 0 & 1 & 1 & 0 & 0 & 1 & 0 \\\cline{2-9}
                    \end{array}
                \]
                \[
                    \begin{array}{r|c|c|c|c|c|c|c|c|}
                    \cline{2-9}
                     b & 1 & 1 & 1 & 0 & 1 & 0 & 1 & 1 \\\cline{2-9}
                    \end{array}
                \]
        \item   Idee: Teile $a$ und $b$ in Zahlen mit weniger Ziffern auf, löse rekursiv, setze das ergebnis zusammen
                \[  a\ 
                    \overbrace{
                    \begin{array}{|c|c|c|c|}
                    \cline{1-4}
                     1 & 0 & 1 & 1 \\\cline{1-4}
                    \end{array}}^{a_h}
                    \overbrace{
                    \begin{array}{|c|c|c|c|}
                    \cline{1-4}
                     0 & 0 & 1 & 0 \\\cline{1-4}
                    \end{array}}^{a_l}
                \]
                \[  b\ 
                    \overbrace{
                    \begin{array}{|c|c|c|c|}
                    \cline{1-4}
                     1 & 1 & 1 & 0 \\\cline{1-4}
                    \end{array}
                    }^{b_h}
                    \overbrace{
                    \begin{array}{|c|c|c|c|}
                    \cline{1-4}
                     1 & 0 & 1 & 1 \\\cline{1-4}
                    \end{array}}^{b_l}
                \]
        \item   Schreibe:
                \begin{align*}
                 a &= \overbrace{a_h \cdot 2^{\left\lceil\frac{n}{2}\right\rceil}}^{\text{obere Bits}} + \overbrace{a_l}^{\text{untere $\left\lceil\frac{n}{2}\right\rceil$ Bits}}\\
                 b &= b_h \cdot 2^{\left\lceil\frac{n}{2}\right\rceil} + b_l
                \end{align*}
                \begin{align*}
                 a \cdot b  &= (a_h \cdot 2^{\left\lceil \frac{n}{1} \right\rceil} + a_l) \cdot (b_h \cdot 2^{\left\lceil \frac{n}{1} \right\rceil} + b_l)\\
                            &= \psframebox[linecolor=red]{a_h \cdot b_h} 2^{2 \left\lceil \frac{n}{1} \right\rceil} + 
                                \psframebox[linecolor=blue]{(a_h \cdot b_l + b_h \cdot a_l)} 2^{2 \left\lceil \frac{n}{1} \right\rceil} + \psframebox[linecolor=green]{a_l \cdot b_l}
                \end{align*}
        \end{itemize}
        \paragraph{Algorithmus:}
        \begin{itemize}
        \item   Berechne rekursiv $a_h \cdot b_h$, $a_h \cdot b_l$, $a_l \cdot b_h$, $a_l \cdot b_l$
        \item   Berechne $a \cdot b$ nach Formel mit Shiften und Addieren
        \end{itemize}
        \paragraph{Laufzeitanalyse:}
        \begin{align*}
         T(n) &\leq \begin{cases}
                     O(1), & \text{wenn $n < 3$}\\
                     4 \cdot T\left(\frac{n}{2}\right) + O(n), & \text{sonst}
                    \end{cases}
        \end{align*}
        \begin{description}
        \item[Lösung:]  $T(n) = \Theta(n^2)$
        \item[Problem:] Führen $\underline{4}$ rekursive Multiplikationen durch. Dadurch wird nichts gewonnen.
        \end{description}
        \paragraph*{Genialer Einfall:} (Algorithmus von Karatsuba)
        Betrachte $(a_h + a_l) \cdot (b_h + b_l) = \psframebox[linecolor=red]{a_h \cdot b_h} + \psframebox[linecolor=blue]{a_h \cdot b_l + b_h \cdot a_l} + \psframebox[linecolor=green]{a_l \cdot b_l}$
        \begin{itemize}
        \item   Berechne nach: $\psframebox[linecolor=red]{a_h \cdot b_h}$ und $\psframebox[linecolor=green]{a_l \cdot b_l}$
        \item   Und dann: $(a_h + a_l) \cdot (b_h + b_l) - \psframebox[linecolor=green]{a_l \cdot b_l} - \psframebox[linecolor=red]{a_h \cdot b_h}$
        \end{itemize}
        \paragraph*{Laufzeitanalyse:}
        \begin{align*}
         T(n) &\leq \begin{cases}
                     O(1), & \text{wenn $n < 3$}\\
                     {\color{red}3} \cdot T\left(\frac{n}{2}\right) + O(n), & \text{sonst}
                    \end{cases}
        \end{align*}
        Was kommt heraus?
    \paragraph*{Bemerkungen}
    \begin{itemize}
     \item aktueller Champion der Multiplikationsalgorithmen: M. Fürer (2007/09) $O(n \log n 2^{\log_* n})$
     \item Vermutung: $\Theta(n \log n)$ ist optimal
    \end{itemize}
\end{itemize}

\subsection{Lösen von Rekursionsgleichungen}
\begin{itemize}
\item   Methode 1: Raten \& \textbf{Induktion}
\item   Methode 2: \textbf{Wiederholt einsetzen} \& Muster erkennen
        \begin{align*}
         T(n)   &\leq 3 T\left(\frac{n}{2}\right) + c \cdot n \\
                &\leq 3 \cdot \left(3 \cdot T\left(\frac{n}{4}\right) + c \cdot \frac{n}{2}\right) + c \cdot n \\
                &\leq 3 \cdot \left(3 \cdot  \left(3 \cdot T\left(\frac{n}{8}\right) + c \cdot \frac{n}{4}\right) + c \cdot \frac{n}{2}\right) + c \cdot n\\
                &\leq 3^3 T\left(\frac{n}{8}\right) + c \frac{3^2}{2^2}n + c \frac{3}{2}n + c \cdot n\\
                &\vdots \hspace{1cm}\text{$k$ Schritte}\\
                &\leq 3^k T\left(\frac{n}{2^k}\right) + c \left(\frac{3}{2}\right)^{k-1}n + c\left(\frac{3}{2}\right)^{k-2} + ... + cn\\
                & \text{nach $k = \log n$ Schritten ist $\frac{n}{2}$ konstant.}\\
                &= \sum\limits_{k = 0}^{\log_2 n} \left(\frac{3}{2}\right)^k c \cdot n\\
                &= cn \frac{\left(\frac{3}{2}\right)^{1 + \log_2 n} - 1}{\frac{3}{2} - 1} \tag{geometrische Reihe}\\
                &\leq 2 cn \frac{3}{2} \left(\frac{3}{2}\right)^{1 + \log_2 n}\\
                &= 3 cn \cdot n^{\log_2 \frac{3}{2}} = 3 c \cdot n^{\log_2 3}
        \end{align*}
        \paragraph*{Laufzeit} $\Theta(n^{\log_2 3})$, $\log_2 3 \approx 1{,}5385$
% 2011104 hier angefangen wegen itemize
\item   Methode 3: Bild malen \& Muster erkennen (\textbf{Rekursionsbaummethode})
        \[T(n) \leq 3 T\left(\frac{n}{2} + O(n)\right)\]
        Male einen Knoten für jeden rekursiven Aufruf
        \begin{center}
        \begin{minipage}{0.3\textwidth}
        \psset{levelsep=0.6cm,treesep=0.3cm}
        \pstree{\Tr*{$T(n)$}}{
            \pstree{\Tr*{$T\left(\frac{n}{2}\right)$}}{
                \pstree{\Tr*{$T\left(\frac{n}{4}\right)$}}{
                    \pstree{\Tr*{$T\left(\frac{n}{8}\right)$}}{
                        \pstree{\Tr*{$\vdots$}}{
                            \Tr{$T(1)$}
                        }
                    }
                    \Tr{$\vdots$}\Tr{$\vdots$}
                }
                \Tr{$\vdots$}
                \Tr{$\vdots$}
            }
            \pstree{\Tr*{$T\left(\frac{n}{2}\right)$}}{
                \Tr{$\vdots$}
                \Tr{$\vdots$}
                \Tr{$\vdots$}
            }
            \pstree{\Tr*{$T\left(\frac{n}{2}\right)$}}{
                \Tr{$\vdots$}
                \Tr{$\vdots$}
                \Tr{$\vdots$}
            }
        }
        \end{minipage}
        \vline
        \begin{minipage}{0.3\textwidth}
            \centering
            $c \cdot n$ \\[0.3cm]
            $3 \cdot c \cdot \frac{n}{2}$ \\[0.3cm]
            $9 \cdot c \cdot \frac{n}{4}$ \\[0.3cm]
            $27 \cdot c \cdot \frac{n}{8}$ \\[0.3cm]
            \hspace{0cm} \\[0.3cm]
            $3^k \cdot c \cdot \frac{n}{2^k}$
        \end{minipage}
        \end{center}
        Es gibt $O(\log_2 n)$ Ebenen $\leq \log n$ Ebenen
        \begin{itemize}
        \item Addiere alle Kosten $\sum\limits_{k = 0}^{\log n} \left(\frac{3}{2}\right)^k c \cdot n = \Theta(n^{\log_2 3})$
        \end{itemize}
\item   Methode 4: Allgemeines Rezept: \textbf{Master Theorem}
        \Satz Sei $a \geq 1, b \geq 1, f{:}\ \mathbb{N} \to \mathbb{N}$. Sei $T(n) = a \cdot T\left(\frac{n}{b}\right) + f(n)$ eine Rekursion ($T(n) = O(1)$ für $n \leq 2$). Dann gilt:
        \begin{enumerate}
         \item Wenn $f(n) = O(n^{(\log_b a) - \varepsilon})$ für $\varepsilon > 0$, dann ist 
                \[T(n) = \Theta(n^{\log_b a})\]
         \item Wenn $f(n) = \Theta(n^{\log_b a})$ ist, dann ist
                \[T(n) = \Theta\left(n^{\log_b a} \log n\right)\]
         \item Wenn $f(n) = \Omega(n^{(\log_b a) + \varepsilon})$ für $\varepsilon > 0$ und $\exists c < 1{:}\ a f\left(\frac{n}{b}\right) < c f\left(n\right)$, dann ist
                \[T(n) = \Theta\left(f(n)\right)\]
        \end{enumerate}
        \Bsp    \begin{itemize}
        \item   $T(n) \leq 3 T\left(\frac{n}{2}\right) + O(n)$\\
                MT anwendbar mit $a = 3, b = 2, f(n) = c \cdot n$
                \begin{align*}
                 f(n)   &= O(n^{\log_b a - \varepsilon}) \\
                        &= O(n^{\log_2 3} - n^{1{,}5...}), \varepsilon \approx 0{,}5
                \end{align*}
                Fall 1: $T(n) = \Theta(n^{\log_2 3})$
        \item   $T(n) = 4 T\left(\frac{n}{2}\right) + O(n)$ \\
                MT mit $a = 4, b = 2, f(n) = c \cdot n$
                \begin{align*}
                 f(n) &= O(n^{2 - \varepsilon}) \text{ für $\varepsilon \approx 0{,}5$}
                \end{align*}
                Fall 1: $T(n) = \Theta(n^2)$
        \item   $T(n) = 2 T\left(\frac{n}{2}\right) + O(n)$ \\
                MT anwendbar mit $a = 2, b = 2, f(n) = c \cdot n$
                \begin{align*}
                 n^{\log_b a} &= n\\
                 f(n) &= \Theta(n^{\log_b a})
                \end{align*}
                Fall 2: $T(n) = \Theta(n \log n)$
        \item   $T(n) = T\left(\frac{n}{2}\right) + O(1)$ \\
                MT anwendbar mit $a = 1, b = 2, f(n) = c$
                \begin{align*}
                 n^{\log_b a} &= n^0 = 1\\
                 f(n) &= \Theta(n^{\log_b a})
                \end{align*}
                Fall 2: $T(n) = \Theta(\log n)$
        \item   $T(n) = T\left(\frac{3}{4}n\right) + O(n)$\\
                MT anwendbar mit $a = 1, b = \frac{4}{3}, f(n) = d \cdot n$\\
                \begin{align*}
                 n^{\log_b a} &= n^0 = 1 \\
                 f(n) &= \Omega(n^{0+\varepsilon}), \text{z. N. für $\varepsilon = \dfrac{1}{2}$} \\
                 a \cdot f\left(\frac{n}{b}\right) &\leq 1 \cdot \frac{3}{4} \cdot dn
                    &= \frac{3}{4} d n \leq \frac{3}{4} f(n)
                \end{align*}
                Fall 3: $T(n) = \Theta(n)$
        \end{itemize}
        \Bew
            \begin{center}
             \begin{minipage}{0.4\textwidth}
            \centering
        \psset{levelsep=0.6cm,treesep=0.3cm}
        \pstree{\Tr*{$T(n)$}}{
            \pstree{\Tr*{$T\left(\frac{n}{b}\right)$}}{
                \pstree{\Tr*{$T\left(\frac{n}{b^2}\right)$}}{
                    \pstree{\Tr*{$T\left(\frac{n}{b^3}\right)$}}{
                        \pstree{\Tr*{$\vdots$}}{
                            \Tr{$T(1)$}
                        }
                    }
                    \Tr{$\vdots$}\Tr{$\vdots$}
                }
                \Tr{$\hdots$}
                \Tr{$\vdots$}
            }
            \pstree{\Tr*{$T\left(\frac{n}{b}\right)$}}{
                \Tr{$\vdots$}
                \Tr{$\hdots$}
                \Tr{$\vdots$}
            }
            \pstree{\Tr*{$T\left(\frac{n}{b}\right)$}}{
                \Tr{$\vdots$}
                \Tr{$\hdots$}
                \Tr{$\vdots$}
            }
        }
        (jeweils $a$ Kinder)
        \end{minipage}
        \vline
        \begin{minipage}{0.3\textwidth}
            \centering
            $f(n)$ \\[0.3cm]
            $a \cdot f\left(\frac{n}{b}\right)$ \\[0.3cm]
            $a^2 \cdot f\left(\frac{n}{b^2}\right)$ \\[0.3cm]
            $a^3 \cdot f\left(\frac{n}{b^3}\right)$ \\[0.3cm]
            \hspace{0cm} \\[0.3cm]
            $a^k \cdot f\left(\frac{n}{b^k}\right)$ \\[0.3cm]
        \end{minipage}
        \end{center}
        $\leq \log_b n$ Ebenen, Gesamtkosten: $\sum\limits_{k = 0}^{\log_b n} a^k f\left(\frac{n}{b^k}\right)$\\
        \begin{itemize}
        \item   $T(n) = 2 T \left(\frac{n}{2}\right) + \Theta(n \log n)$\\
                MT nicht anwendbar
        \item   $T(n) = 2 T \left(\frac{n}{3}\right) + T \left(\frac{2}{3} n\right) + \Theta(n)$\\
                MT nicht anwendbar (passt nicht ins Schema)
        \end{itemize}
\end{itemize}
\subsubsection{Problem des engsten Punktpaares}
Gegeben $n$ Punkte in einer Ebene $P$
\begin{center}
 \begin{pspicture}(0,0)(3,3)
  \psdot(0,2.1)\psdot(0.4,0.5)\psdot(1,1)\psdot(1.2,1.9)
  \psdot(1.6,0.6)\psdot(2.1,1.5)\psdot(2.4,2)
  \psdot(2.5,1.1)\psdot(2.6,0.1)\psdot(3,1.6)
 \end{pspicture}
\end{center}
Gesucht $p, q \in P$ mit $p \neq q$ und $d(p,q)$ minimal
\begin{itemize}
 \item Naiv: probiere alle Punktpaare durch, nimm das Minimum
         \[\Theta(n^2)\ \text{Laufzeit}\]
 \item Schnelle mit Devide \& Conquer:
         \begin{itemize}
          \item  Teile die Punktmenge in zwei Hälften
          \item  Sortiere nach $x$-Koordinate
          \item  Nimm den Punkt $q$ mit den $\left\lceil\frac{n}{2}\right\rceil$ größten $x$-Koordinaten. Annahme: alle $x$-Koordinaten verschieden
              \begin{align*}
               P_L &= \{p \in P\ |\ \text{$p$ links von $q$}\} \cup \{q\}\\
               P_R &= \{p \in P\ |\ \text{$p$ rechts von $q$}\}
              \end{align*}
              \[
                  \left.\begin{matrix}
                   \text{Bestimme engstes Paar in $P_L$} \\
                   \text{Bestimme engstes Paar in $P_R$}
                  \end{matrix}
                 \right\}\text{rekursiv}
              \]
         \end{itemize}
       \begin{pspicture}(0,0)(3,3)
       \psdot(0,2.1)\psdot(0.4,0.5)\psdot(1,1)\psdot(1.2,1.9)
       \psline[linecolor=red](1.6,0)(1.6,3)
       \psdot(1.6,0.6)\uput{3px}[45](1.6,0.6){$q$}\psdot(2.1,1.5)\psdot(2.4,2)
       \psdot(2.5,1.1)\psdot(2.6,0.1)\psdot(3,1.6)
       \end{pspicture}
\end{itemize}



\begin{description}
 \item[Beobachtung 1:] Wenn es ein engeres Paar $p,q$ mit $p \in P_L, q \in P_R$ gibt, so liegt es im $2 \delta$-Streifen um die Mittellinie
 \item[Beobachtung 2:] Wenn es ein engeres Paar gibt, so liegt es in einem $2\delta \times \delta$-Rechteck $R$ zentriert um die Mittellinie
 \item[Volumenargument:] Alle Punkte in $P_L$ haben Mindestabstand $\delta$, alle Punkte in $P_R$ haben Mindestabstand $\delta$.
\end{description}
 \item Wie viele Punkte aus $P_L$ können in dem Rechteck $R$ liegen?
 \item Müssen im Quadrat $R_L$, $R_L$ hat Fläche $\delta\delta = \delta^2$\\
        $\Rightarrow$ jedem Punkt gehören $\frac{1}{4} \cdot \left(\frac{\delta}{2}\right)^2 \cdot \pi \leq \frac{3}{16} \delta^2$
 \item D. h.: $R_L$ kann höchstens $\left\lfloor\frac{16}{3}\right\rfloor = 5$ Punkte aus $P_L$ enthalten. Ebenso in $R_R$.
    \begin{itemize}
    \item   Sortiere Punkte nac $y$-Koordinate. Gehe alle Punkte im $2\delta$-Streifen von unten nach oben durch.
    \item   Für jeden Punkt im $2\delta$-Streifen berechne Abstände zu den 9 Nachfolgern
    \item   Nimm das Minimum
    \item   Vergleiche mit $\delta$ und nimm das globale Minimum.
    \end{itemize}
         \end{itemize}

\paragraph{Laufzeitanalyse:}
\[T(n) = 2 \cdot T \left(\frac{n}{2}\right) + \overbrace{O(n \log n)}^{\text{sortieren + Mittelstreifen + Teilen}}\]
Optimierungsidee:
\begin{itemize}
 \item Wir müssen nicht jedes Mal sortieren\\
        $\Rightarrow$ einmal nach $x$-Koordinate sortieren genügt
 \item Ebenso für die $y$-Koordinate: Sortiere einmal auch $y$-Koordinate
 \item Beim Splitten können wir die Sortierung von $P_L$ und $P_R$ nach $y$-Koordinate in $O(n)$ Zeit erhalten durch umgekehrte Menge.
\end{itemize}
Optimale Laufzeit:
\begin{description}
 \item $O(n \log n)$ Vorverarbeitungszeit
 \item $T(n) = 2 \cdot T\left(\frac{n}{2}\right) + O(n)$ für die Rekursion
 \item $\rightarrow T(n) = O(n \log n)$
 \item $\Rightarrow$ Gesamtlaufzeit $O(n \log n)$ 
\end{description}

\section{Dynamisches Programmieren}
\Bsp Einkaufsproblem / Rucksackproblem
\begin{itemize}
 \item Haben $B = 1{,}60$ EUR
        \begin{center}
            \begin{tabular}{lr|c}
                & Preis $p_i$ & Präferenz $w_i$ \\
                1 Apfel & 40 ct. & 3 \\
                2 WM-Brötchen & 60 ct. & 4 \\
                3 Buttermilch & 1 EUR & 6 \\
                4 Gummibärchen & 80 ct. & 9 \\
                5 Bifi & 60 ct. & 10 \\ 
            \end{tabular}
        \end{center}
 \item \textbf{Problem:} Finde eine Teilmenge $X$ von Einkäufen, mit Gesamtpreis $\leq 1{,}60$ EUR, die die Summe aller Präferenzen maximiert
 \item \textbf{Idee:} Rekursiver Ansatz. \\
    $\Rightarrow$ Schränke Problem ein auf Teilprobleme. Finde eine Rekursionsgleichung, die die Teilprobleme in Beziehung setzt.
 \item \textbf{Hier:}
   \begin{center}
    \begin{tabular}{rp{0.7\linewidth}}
     $E[m,b] =$ & Maximale Präferenz des Einkaufs, bei dem nur Lebensmittel $1,...,m$ zur Verfügung stehen \& bei dem wir $b$ ct. haben
    \end{tabular}
   \end{center}
    Wollen: $E[n,B]$ mit $n = 5, B = 160$ ausrechnen\\
    $\Rightarrow$ Finde Rekursion fpr $E[n,p]$
    \begin{align*}
     E[0,b] &= 0 \qquad \forall b \in \mathbb{N} \\
     E[m,0] &= 0 \qquad \forall m \in \mathbb{N} \\
     E[m,b] &= \begin{cases}
                \max(E[m-1,b], w_m + E[m-1, b-p_m]), & \text{wenn } b \geq p_m \\
                E[m-1,b], & \text{sonst}
               \end{cases}
    \end{align*}
\end{itemize}

\begin{itemize}
 \item Implementiere Rekursion.\\
         Direkte Implementierung ist schlecht, da exponentiell viele Aufrufe vorkommen.
     \begin{itemize}
     \item   Besser: Merke Werte in einer Tabelle
     \item   Noch besser: Fülle Tabelle von unten nach oben aus.
     \end{itemize}
     \psset{arrows=->}
     \[
      \begin{array}{r||c|c|c|c|c|c|c|c|c|c|c|c|c|c|c}
        & 0 & \hdots & 40 & \hdots & 60 & \hdots & 80 & \hdots & 100 & \hdots & 120 & \hdots & 140 & \hdots & 160 \\\hline\hline
       0 & \rnode{0_000}{0} & \rnode{0_020}{\hdots} & \rnode{0_040}{0} & \hdots & \rnode{0_060}{0} & \hdots & \rnode{0_080}{0} & \hdots & \rnode{0_100}{0} & \hdots & \rnode{0_120}{0} & \hdots & \rnode{0_140}{0} & \hdots & \rnode{0_160}{0} \\\hline
       1 & \rnode{1_000}{0} & \hdots & \rnode{1_040}{3} & \hdots & \rnode{1_060}{3} & \hdots & \rnode{1_080}{3} & \hdots & \rnode{1_100}{3} & \hdots & \rnode{1_120}{3} & \hdots & \rnode{1_140}{3} & \hdots & \rnode{1_160}{3}\\\hline
       2 & \rnode{2_000}{0} & \hdots & \rnode{2_040}{3} & \hdots & \rnode{2_060}{4} & \hdots & \rnode{2_080}{4} & \hdots & \rnode{2_100}{7} & \hdots & \rnode{2_120}{7} & \hdots & \rnode{1_140}{7} & \hdots & \rnode{1_160}{7}\\\hline
       3 & \rnode{3_000}{0} & \hdots & \rnode{3_040}{3} & \hdots & \rnode{3_060}{4} & \hdots & \rnode{3_080}{4} & \hdots & \rnode{3_100}{7} & \hdots & \rnode{3_120}{7} & \hdots & \rnode{3_140}{9} & \hdots & \rnode{3_160}{10}\\\hline
       4 & \rnode{4_000}{0} & \hdots & \rnode{4_040}{3} & \hdots & \rnode{4_060}{9} & \hdots & \rnode{4_080}{9} & \hdots & \rnode{4_100}{12} & \hdots & \rnode{4_120}{13} & \hdots & \rnode{4_140}{13} & \hdots & \rnode{4_160}{16}\\\hline
       5 & \rnode{5_000}{0} & \hdots & \rnode{5_040}{3} & \hdots & \rnode{5_060}{9} & \hdots & \rnode{5_080}{10} & \hdots & \rnode{5_100}{12} & \hdots & \rnode{5_120}{13} & \hdots & \rnode{5_140}{19} & \hdots & \circlenode{5_160}{\underline{\underline{19}}}
      \end{array}
     \]\ncline{1_040}{0_020}\ncline{2_040}{1_040}
     Optimaler Wert: 19
     \begin{description}
      \item[Algorithmus:]
       \begin{center}
        \begin{algorithmic}
            \FOR{$m \in \{0, \hdots, n\}$}
                \STATE $E[m,0] \gets 0$
            \ENDFOR
            \FOR{$b \in \{0, \hdots, B\}$}
                \STATE $E[0,b] \gets 0$
            \ENDFOR
            \FOR{$m \in \{1, \hdots, n\}$}
                \FOR{$b \in \{1, \hdots, B\}$}
                    \IF{$p[m] > b$}
                        \STATE $E[m,b] \gets E[m-1,b]$
                    \ELSE
                        \STATE $E[m,b] \gets \max\{E[m-1,b], w[m] + E[m-1,b-p[m]]\}$
                    \ENDIF
                \ENDFOR
            \ENDFOR
            \RETURN $E[n, B]$
        \end{algorithmic}
       \end{center}
      \item[Laufzeit:] $O(nB)$ (Pseudopolynomielle Laufzeit)
      \item[Speicher:] $O(nB)$
     \end{description}
\item Bestimme den optimalen Einkauf:
    \renewcommand{\labelenumi}{(\theenumi)}
    \renewcommand{\theenumi}{\alph{enumi}}
    \begin{enumerate}
     \item Nimm 2. Tabelle $\operatorname{Kaufen}[m, b] \begin{cases}
                                                         \textbf{true}, & \text{wenn wir $m$ kaufen für $E[m,b]$} \\
                                                         \textbf{false}, & \text{sonst}
                                                        \end{cases}$\\
        Verfolge die Rekursion Rückwerts von $E[n,B]$ ausgehend. Benutze den Eintrag in $\operatorname{Kaufen}$ als Wegweiser.
     \item Rekonstruiere $\operatorname{Kaufen}$ anhand von $E$, inem wir die Rekursion nach den Werten nachvollziehen
     \item speichere kompletten Einkauf in jeden Eintrag $B[m,b]$\\
             $\Rightarrow$ müglicherweise mehr Speicher.
    \end{enumerate}
\end{itemize}
\subsection{Zusammenfassung}
\begin{itemize}
 \item Optimierungsproblem
 \item Finde zunächst nur den \emph{Wert} einer optimalen Lösung
 \item Schränke geeignet auf Teilprobleme ein
 \item Finde Rekursion für die Teilprobleme
 \item Implementiere die Rekursion (in der Regel Tabelle)
 \item Rekonstruiere optimale Lösung
\end{itemize}


\subsection{Rundreiseproblem (TSP, Traveling Salesperson)}
\Geg $n$ Städte $0, ..., n-1$. Für 2 Städte $i,j$ Abstand $d(i,j) = d(j,i) \leq 0$
\Ges Permutation $\Pi{:}\ \{0,...,n-1\} \to \{0,...,n-1\}$ mit $\Pi(0) = 0$, so dass $\sum\limits_{i=0}^{n-1} d(\Pi(i), \Pi((i-1) \mod n))$
    \begin{center}
    \begin{pspicture}(0,0)(2,3)
     \psdot(1,3)\uput{3pt}[45](1,3){HH}\psdot(1.1,2.2)\uput{3pt}[45](1.1,2.2){H}\psdot(0,1.8)\uput{3pt}[45](0,1.8){D}\psdot(0.1,0)\uput{3pt}[45](0.1,0){S}\psdot(1.8,0)\uput{3pt}[45](1.8,0){M}\psdot(2,1.6)\uput{3pt}[45](2,1.6){L}\psdot(1.9,2.2)\uput{3pt}[45](1.9,2.2){B}
     \psline(1,3)(1.1,2.2)(0,1.8)(0.1,0)(1.8,0)(2,1.6)(1.9,2.2)(1,3)
    \end{pspicture}
    \end{center}
\Los 
\begin{description}
 \item[Naiv] alle Permutationen $\Pi$ durchprobieren
    \[(n-1)! \text{viele}\]
    $\Theta(n)$ Zeit pro Permutation, um die Gesamtlänge zu bestimmen\\
    $\Rightarrow$ Gesamtlaufzeit $\Theta(n!)$\\
    Sehr schlecht:
    \[\Theta(n!) = 2^{\Theta(n\log n)}\]
    Superexponentiell, Hoffnungslos für $n \geq 10$
 \item[Dynamisches Programmieren] $\Theta(n^2 \cdot 2^n)$
  \begin{enumerate}
   \item Finde geeignete Teilprobleme. 
       \begin{itemize}
        \item Sei $S \subseteq \{1, ..., n-1\}$ Teilmenge von Städten.  
        \item Sei $m \in {0, ..., n-1} \setminus S$
        \[
         \begin{array}{rp{0.7\linewidth}}
          T[S,m] &= \text{Länge einer optimalen Tour, die bei 0 anfängt, beim $m$ aufhört und zwischen $0$ und $m$ genau die Städte aus $S$ besucht}
         \end{array}
        \]
       \end{itemize}
       Ziel: Wollen $T[\{1,...,n-1\},0]$
  \item Finde Rekursion:
      \[ T[\emptyset, m] = d(0,m), \forall m \in \{0, ..., n-1\} \]
      \begin{align*}
       T[S,m] &= \min\limits_{a \in S} T[S \setminus a, a] + d(a,m)
      \end{align*}
      Welches ist die letzte Stadt, die wir vor $m$ besuchen
  \item Fülle Tabelle aus
  \item Finde optimale Lösung
  \end{enumerate}
  (3. und 4. sind Teil einer Übung)
\item[Ergebnis:] Tabelle hat $2^{n-1} \cdot n$ viele Einträge, $O(n)$ Zeit pro Eintrag\\
    $\Rightarrow$ Laufzeit: $O(n^22^n)$, Platz: $O(n2^n)$\\
    Bester bekannter Algorithmus (NP-schweres Problem, d. h. es ist unwahrscheinlicht, dass ein wesentlich besserer Algorithmus existiert).
\end{description}

\subsection{Viterbi-Algorithmus}
\begin{itemize}
 \item Lerne für HA-Klausur
 \item Schließe mich 7 Tage bei geschlossenen Vorhängen zu Hause ein
 \item Wissen nicht, wie das Wetter draußen ist
 \item Aber: Mitbewohner geht aus, jeden Tag, und kommt \emph{nass} oder \emph{trocken} zurück.
\end{itemize}
\Bsp NTTNTNN
\paragraph*{Frage:} Wie war das Wetter?
\begin{itemize}
 \item Bauen ein Modell
    \begin{center}
        \psset{arrows=->}
        \begin{psmatrix}
         [mnode=oval] {\color{red}R}egen & [mnode=oval] {\color{red}S}onne
        \end{psmatrix}\nccurve[angleA=-170,angleB=170,ncurv=4]{1,1}{1,1}\naput{0{,}6}\nccurve[angleA=-10,angleB=10,ncurv=4]{1,2}{1,2}\nbput{0{,}7}
        \nccurve[angleA=-10,angleB=-170]{1,1}{1,2}\nbput{0{,4}}\nccurve[angleA=170,angleB=10]{1,2}{1,1}\nbput{0{,}3}
    \end{center}
    \begin{align*}
     \operatorname{Pr}[\text{es regnet}\ |\ \text{Mitbewohner nass}] &= 0{,}8 \\
     \operatorname{Pr}[\text{es regnet}\ |\ \text{Mitbewohner trocken}] &= 0{,}2 \\
     \operatorname{Pr}[\text{es regnet nicht}\ |\ \text{Mitbewohner nass}] &= 0{,}3 \\
     \operatorname{Pr}[\text{es regnet nicht}\ |\ \text{Mitbewohner trocken}] &= 0{,}7 \\\hline
     \operatorname{Pr}[\text{regnet morgen}\ |\ \text{regnet heute}] &= 0{,}6 \\
     \operatorname{Pr}[\text{regnet morgen nicht}\ |\ \text{regnet heute}] &= 0{,}4 \\
     \operatorname{Pr}[\text{regnet morgen}\ |\ \text{regnet heute nicht} &= 0{,}3 \\
     \operatorname{Pr}[\text{regnet morgen nicht}\ |\ \text{regnet heute nicht}] &= 0{,}7 \\\hline
     \operatorname{Pr}[\text{regnet am Montag}] &= 0{,}7 \\
     \operatorname{Pr}[\text{regnet am Montag nicht}] &= 0{,}3
    \end{align*}
    \paragraph*{Problem} Finde die wahrscheinlichste Erklärung für die Beobachtung, d. h. finde eine Folge von Zuständen, so dass die Wahrscheinlichkeit, dass diese Folge von Zuständen, die die Beobachtung erzeugt, maximal ist.
\end{itemize}
\Defi \emph{Verstecktes Markov Modell} (Hidden Markov Modell, HMM) besteht aus:
\begin{itemize}
 \item Zustandsmenge $Q$ (endlich)
 \item Ausgangsalphabet $\Sigma$ (endlich)
 \item Ausgangsverteilung $a(q), \forall q \in Q, a(q) \in [0,1]; \sum\limits_{q \in Q} a(q) = 1$
 \item Ausgabeverteilung: 
     \[\forall q \in Q, \sigma \in \Sigma{:}\ o(q,\sigma) = \operatorname{Pr}[\text{Modell gibt $\sigma$ aus}\ |\ \text{Modell ist in Zustand $q$}]\]
     $\forall q \in Q{:}\ o(q,\sigma) \in [0,1]; \sum\limits_{r \in Q} o(q,\sigma) = 1$
 \item Übergangsverteilungen:
     \[\forall q,r \in Q{:}\ t(q,r) = \operatorname{Pr}[\text{nächster Zustand ist $r$}\ |\ \text{aktueller Zustand ist $q$}]\]
     $\forall q \in Q{:}\ t(q,r) \in [0,1]; \sum\limits_{r \in Q} a(q,r) = 1$
\end{itemize}
Semantik:
\begin{itemize}
 \item Wähle zufülligen Zustand $q_0$ gemäß Anfangsverteilung
 \item Aktueller Zustand $q_i$:
     \begin{itemize}
     \item Wähle zufällige Ausgabe $\sigma_i \in \Sigma$, gemäß der Ausgabeverteilung fpr $q_i$
     \item Gib $\sigma_i$ aus
     \item Wähle nächsten Zustand $q_{i+1}$ gemäß der Übergangsverteilung für $q_i$
     \end{itemize}
 \item Wechsle zu $q_{i+1}$, wiederhole
\end{itemize}
\Geg Beobachtung $\sigma_0\sigma_1\sigma_2...\sigma_{l-1}$ \\
\Ges Zustandsfolge $q_0...q_{l-1}$, so dass $\operatorname{Pr}[\text{Zustandsfolge $q_0...q_{l-1}$}\ |\ \text{Beobachtung $\sigma_0...\sigma_{l-1}$}]$ maximal.
\[\underset{q_0...q_{l-1}}{\operatorname{argmax}} \operatorname{Pr}[\text{Zustandsfolge $q_0...q_{l-1}$}\ |\ \text{Beobachtung $\sigma_0...\sigma_{l-1}$}]\]





\paragraph*{Lösung mit dynamischen Programmieren}
\begin{enumerate}[start=0]
\item Problem verstehen (\textbf{Satz von Bayes})
     \[\operatorname{Pr}[A|B] = \frac{\operatorname{Pr}[B|A] \cdot \operatorname{Pr}[A]}{\operatorname{Pr}[B]}\]
     Also:
     \begin{align*}
     &\operatorname{Pr}[\underbrace{\text{Zustandsfolge $q_0...q_{l-1}$}}_{A}\ |\ \underbrace{\text{Beobachtung $\sigma_0...\sigma_{l-1}$}}_{B}] \\&= \frac{\operatorname{Pr}[\text{Beobachtung $\sigma_0...\sigma_{l-1}$}\ |\ \text{Zustandsfolge $q_0...q_{l-1}$}] \cdot \operatorname{Pr}[\text{Zustandsfolge $q_0...q_{l-1}$}]}{\operatorname{Pr}[\text{Beobachtung $\sigma_0...\sigma_{l-1}$}]}\\
     &\underset{q_0...q_{l-1}}{\operatorname{argmax}} \frac{\operatorname{Pr}[\text{Beobachtung $\sigma_0...\sigma_{l-1}$}\ |\ \text{Zustandsfolge $q_0...q_{l-1}$}] \cdot \operatorname{Pr}[\text{Zustandsfolge $q_0...q_{l-1}$}]}{\operatorname{Pr}[\text{Beobachtung $\sigma_0...\sigma_{l-1}$}]} \\
     &= \underset{q_0...q_{l-1}}{\operatorname{argmax}} \operatorname{Pr}[\text{Beobachtung $\sigma_0...\sigma_{l-1}$}\ |\ \text{Zustandsfolge $q_0...q_{l-1}$}] \cdot \operatorname{Pr}[\text{Zustandsfolge $q_0...q_{l-1}$}]\\
     &= \underset{q_0...q_{l-1}}{\operatorname{argmax}}\ o(q_0,\sigma_0) \cdot ... \cdot o(q_{l-1},\sigma_{l-1}) \cdot a(q_0) \cdot 
         t(q_0,q_1) \cdot t(q_1,q_2) \cdot ... \cdot t(q_{l-2}, q_{l-1})
     \end{align*}
\item Teilprobleme finden:
\begin{center}
    \begin{tabular}{rp{0.6\linewidth}}
        $q \in Q, j = 0, ..., l-1{:}\ E[q,j] =$&$\max$ Wahrscheinlichkeit einer Zustandsfolge, die in $q$ endet und Länge $j-1$ hat und die Ausgabe $\sigma_0...\sigma_i$ erzeugt.
     \end{tabular}
\end{center}
\begin{align*}
 &= \underset{\underset{q_i = q}{q_0,q_1,...,q_i}}{\max} o(q_0,\sigma_0) \cdot o(q_1,\sigma_1) \cdot ... \cdot o(q_j, \sigma_j)
     \cdot a(q_0) \cdot t(q_0,q_1) \cdot ... \cdot t(q_{j-1}, q_j)
\end{align*}
\paragraph*{Ziel:} Finde $\underset{q \in Q}{\max}\ E[q, l-1]$
\item Rekursion:
 \begin{align*}
  \forall q \in Q{:}\ E[q, 0] &= a(q) \cdot o (q, \sigma_0) \\
                      E[q, j] &= \max\limits_{r \in Q}\ E[r, j-1] \cdot t(r,q) \cdot o(q,\sigma_j)
 \end{align*}
\end{enumerate}

\paragraph*{Zurück zum Beispiel:}
\begin{itemize}
 \item $Q = \{\text{R}, \text{S}\}$
 \item $\Sigma = \{\text{{\color{blue}N}ass},\text{{\color{blue}T}rocken}\}$
\end{itemize}
\begin{center}
 \psset{arrows=->}
 \begin{tabular}{c||c|c}
  & R & S \\\hline\hline
  $\sigma_0 =$ N & \rnode{0R}{0{,}56} & \rnode{0S}{0{,}09} \\\hline
  $\sigma_1 =$ T & \rnode{1R}{0{,}0612} & \rnode{1S}{0{,}1568} \\\hline
  $\sigma_2 =$ T & \rnode{2R}{} & \rnode{2S}{} \\\hline
  $\sigma_3 =$ N & \rnode{3R}{} & \rnode{3S}{} \\\hline
  $\sigma_4 =$ T & \rnode{4R}{} & \rnode{4S}{} \\\hline
  $\sigma_5 =$ N & \rnode{5R}{} & \rnode{5S}{} \\\hline
  $\sigma_6 =$ N & \rnode{6R}{} & \rnode{6S}{}
\end{tabular}\ncline{1R}{0R}\ncline{1S}{0R}
\end{center}
\begin{description}
 \item[Speicherplatz:] $\Theta(|Q| \cdot l)$
 \item[Zeit:] $\Theta(|Q|^2 \cdot l)$
\end{description}



\section[Greedy Algorithmen]{Greedy (dt. gierige) Algorithmen}
Algorithmus triff lokal optimale Entscheidung in der Hoffnung, dass diese Entscheidung zu einer global optimalen Lösung führt.

\subsection{Beispiel: Vorlesungsplanung / Intervall-Auswahl}
\Geg     $n$ Intervalle $R = {I_1, I_2, ..., I_n}$
         \[I_i = (a(i), e(i))\]
\Ges     Größte Teilmenge $A \subseteq R$, so dass sich keine Intervalle überlappen.
         Es gilt also:
         \[ \forall I_i, I_j \in A{:}\ a(i) \geq e(j) \lor a(j) \geq e(i) \]
\paragraph*{Grundidee:} Beginne mit $A \neq \emptyset$ und füge so lange wie müglich Intervalle aus $R$ in $A$ ein, die kein Intervall aus $A$ schneiden.

\paragraph*{Regel zur Auswahl der Intervalle:} Nimm Intervall mit dem zeitigsten Ende
\begin{algorithmic}
 \STATE $A \gets \emptyset$
 \WHILE{$R \neq \emptyset$}
 \STATE wähle aus $R$ das Intervall mit minimalen $e(i)$, füge es in $A$ ein
 \STATE entferne alle Intervalle $I_j$ aus $R$ mit $a(j) < e(i)$
 \ENDWHILE
\end{algorithmic}

\paragraph*{Korrektheit} Sei $S$ eine optimale (größtmögliche) Teilmenge von $R$. Wir wollen zeigen: 
\[|A| = |B|\]
Seien $i_1,...,i_k$ und $j_1,...,j_m$ die Indizes der Intervalle von $A_i$ bzw. $S_i$ in zeitlich geordneter Reihenfolge.

\paragraph*{Behauptung:} $\forall r \in \{1,...,n\}{:}\ e(i_r) \leq e(j_r)$

\paragraph{Induktionsbeweis:}
\begin{description}
 \item[IA] $r = 1$ $e(i_1) \leq e(j_1)$ klar, denn der Greedy-Algorithmus wählt das Intervall mit minmalem $e(i)$
 \item[IS] ($r \Rightarrow r + 1$)\\
     Es gilt $e(i_r) \leq e(j_r)$ (nach Induktionsbehauptung).\\
     Nach Auswahl von $I_{i_r}$ steht das Intervall $I_{j_{r+1}}$ noch zur Verfügung, da $e(i_r) \leq e(j_r) \leq a(j_{r+1})$\\
     $\Rightarrow$ das nächste Intervall in $A$ ist eins mit $e(i_{r+1}) \leq e(j_{r+1})$ \hfill \carsten\\
     Also gilt $k \geq m$ und damit $|A| \geq |S|$.
\end{description}

\paragraph*{Widerspruchsbeweis:} Angenommen $k < m$, dann gilt $e(i_k) \leq e(j_k)$ und es gibt noch ein Intervall $I_{j_{k+1}}$. Also enthält $R$ noch $I_{j_{k+1}}$ nachdem die intervalle $I_{i_n},...,I_{i_k}$ ausgewählt wurden. Das heißt der Algorithmus ist terminiert, bevor $R = \emptyset$

\paragraph*{Konkrete Implementierung:}
\begin{enumerate}
 \item Sortiere $R$ bzgl. $e(i)$, danach gilt $e(1) \leq e(2) \leq ... \leq e(n)$
 \item Bilde Array $B = \{a(1), a(2), ..., a(n)\}$
 \item Aufbau von $A$
 \begin{algorithmic}
  \STATE $A = \emptyset$
  \STATE $i = 1$
  \WHILE{$i \leq n$}
  \STATE füge $I_i$ in $A$ ein
  \STATE $j \gets i + 1$
  \WHILE{$a(j) \leq e(i)$ und $j \leq n$}
  \STATE $j \gets j+1$
  \ENDWHILE
  \STATE $i = j$
  \ENDWHILE
 \end{algorithmic}
\end{enumerate}

\paragraph*{Laufzeit}
\begin{enumerate}
 \item $\Theta(n \log n)$
 \item $O(n)$
 \item $O(n)$
\end{enumerate}
\[T(n) = \Theta(n \log n)\]

\subsection{Beispiel: Intervallunterteilungsproblem}
\Geg $n$ Intervalle $I_1, I_2,...,I_n$
\Ges Minimale Anzahl von Labeln, so dass jedes Intervall ein Label hat und Intervalle nur dann das gleiche Label haben, wenn sie sich nicht überlappen.

\paragraph*{Grundidee:} Teste für jedes Intervall, ob ein bereits vorhandenes Label benutzt werden kann. Wenn nicht füge neues Label in die Labelmenge ein.

\paragraph*{Regel zum Durchlaufen der Intervalle:} Nimm Intervalle mit dem zeitigsten Start
\begin{algorithmic}
 \STATE Sortiere die Intervalle nacht ihrem Startpunkt. Sei die Reihenfolge $I_1, I_2, ..., I_n$
 \STATE $i \gets 1$
 \FOR{$i \in \{1, ...,n\}$}
  \FOR{jedes Intervall $I_k$, das vor $I_j$ startet und $I_j$ überlappt}
      \STATE Schließe Label von $I_k$ für $I_j$ aus
  \ENDFOR
  \IF{es gibt ein Label $\{1,...,i\}$, das nicht ausgeschlossen wurde}
      \STATE weise das Label $I_j$ zu
  \ELSE
      \STATE weise $I_j$ das Label $i+1$ zu
      \STATE $i \gets i + 1$
  \ENDIF 
 \ENDFOR
\end{algorithmic}
\[T(n) = O(n \log n)\]

\paragraph*{Korrektheit:} Sei $d$ die Tiefe der Menge der Intervalle, d. h. die maximale Anzahl an Intervallen, die zu einem bestimmten Zeitpunkt parallel laufen.
\Beh Die Anzahl der benötigten Label ist mindestens $d$.
\Bew Es gibt einen Zeitpunkt zu dem $d$ Intevalle gleichzeitig laufen. Jedes dieser Intervalle benötigt ein eigenes Label.
\Beh Der Algorithmus benutzt höchstens $d$ Label.
\Bew Betrachte ein Intervall $I_j$ für das ein neues Label eingefügt wird. Sei $t$ die Anzahl der Intervalle, die vor $I_j$ starten und $I_j$ überlappen. Es gibt $t + 1 \leq d$, d. h. $t = d - 1$. Die Menge der Label ist höchstens $d$. \hfill $ \square$ 
\subsection{Caching}
Speicherhierarchie
\begin{center}
 \begin{psmatrix}
  \framebox{CPU} & \frame{Cache} & \framebox{Hauptspeicher}
 \end{psmatrix}\ncline{1,1}{1,2} \ncline{1,2}{1,3}\ncline{1,3}{1,4}
\end{center}
\begin{itemize}
 \item Hauptspeicher kann $n$ Wörter speichern
 \item Cache kann $k$ Wörter speichern
 \item $k < n$
\end{itemize}
\Geg Eine Folge
\[d_1, d_2, ..., d_m\ \text{von Zugriffen auf den Hauptspeicher}\]
\Ges für jeden Zugriff:
     \begin{itemize}
      \item soll ein Wort aus dem Hauptspeicher in den Cache geholt werden
      \item wenn ja: welches Wort soll aus dem Cache gelöscht werden
      \item jedes Datenwort muss bei einem Zugriff im Cache vorhanden sein
     \end{itemize}
\paragraph*{Ziel:} Minimiere Anzahl der Zugriffe auf den Hauptspeicher
\Bsp
\begin{itemize}
 \item Speicher: $n$ = 3 Wörter
    \begin{center}
    \begin{tabular}{ccc}\hline
     \multicolumn{1}{|c|}{$\quad$} &
     \multicolumn{1}{|c|}{$\quad$} &
     \multicolumn{1}{|c|}{$\quad$} \\\hline
     1, & 2, & 3
    \end{tabular}
    \end{center}
 \item Cache: $k$ = 2 Wörter 
    \begin{center}
    \begin{tabular}{|c|c|}\hline
     [1] & [2] \\\hline
    \end{tabular} $\rightarrow$
    \begin{tabular}{|c|c|}\hline
     [3] & [2] \\\hline
    \end{tabular} $\rightarrow$
    \begin{tabular}{|c|c|}\hline
     [1] & [2] \\\hline
    \end{tabular}
    \end{center}
    Zugriffsfolge: 1, 2, 3, 2, 3, 1, 2 \\
    2 Zugriffe auf Hauptspeicher 
\end{itemize}
\paragraph*{Beobachtung:} Wir dürfen annehmen, Speicherzugriffe passieren nur bei Cache Misses (= Zugriffe auf ein Wort, das nicht im Cache ist).
\paragraph*{Denn:} Wir können jede Ersetzunggsstrategie so anpassen , dass die Zugriffe erst bei einem Cache Miss passieren, ohne die Anzahl der Speicherzugriffe zu erhöhen. Die nennen wir eine \emph{reduzierte} Strategie.
\subsubsection{Furthest-in-the-Future-Regel}
Bei einem Cache-Miss, entfrene das Element, dessen nächster Zugriff am weitesten in der Zukunft liegt.
\Satz Furthest-in-the-Future ist optimal, das heißt es liefert eine reduzierte Strategie mit minimaler Anzahl von Cache Misses.
\Bew Austauschargument: Nimm eine optimale Lösung $S^*$ her, transformiere $S^*$ sukzessive in die ff-Lösung $S_{\text{ff}}$ ohne die Anzahl der Cache Misses zu erhöhen.
\paragraph{Austauschlemma:} Sei $S$ eine reduzierte Strategie, die in den ersten $j$ Zugriffen mit $S_{\text{ff}}$ übereinstimmt. Dann existiert eine reduzierte Strategie $S'$, die in den ersten $j+1$ Schritten mit $S_{\text{ff}}$ übereinstimmt und die höchstens so viele Cache Misses erfährt wie $S$.
\Bew (Austauschlemma) Sei $d = d_{j+1}$ (der $j+1$-te Zugriff)
\begin{description}
 \item[Fall 1:] $d$ ist im Cache.\\
     Setze $S' = S$, denn $S$ ist reduziert und tut nichts bei $d_{i-1}$.
 \item[Fall 2:] $d$ ist nicht im Cache und $S$ und $S_{\text{ff}}$ entfernen das gleiche Element.\\
     Setze $S' = S$, fertig.
 \item[Fall 3:] $d$ ist nicht im Cache und $S_{\text{ff}}$ entfernt $e$, $S$ entfernt $f$ mit $e \neq f$\\
     Konstruierte $S'$ wie folgt:
     \begin{itemize}
      \item für die ersten $j+1$ Schritte, tue das gleiche wie $S_{\text{ff}}$. D. h. im $j+1$-ten Schritt, entferne $e$.
      \item nach $j+1$-ten Zugriff:
          \begin{itemize}
           \item $S$ hat $e$ im Cache
           \item $S$ hat $f$ im Cache
          \end{itemize}
     \end{itemize}
\end{description}
$S'$ verhält sich genau so wie $S$ bis einder der folgenden Ereignisse eintritt:
\renewcommand{\labelenumi}{(\theenumi)}
\renewcommand{\theenumi}{\roman{enumi}}
\begin{enumerate}
 \item Zugriff auf $g \neq e, f$ mit Cache Miss und $S$ entfernt $e$ aus dem Cache\\
     $\Rightarrow$ $S'$ entfernt $f$ aus dem Cache. Danach sind die Caches fpr $S$ un $S'$ wieder gleich und $S'$ verhält sich wie $S$.
 \item Zugriff auf $f$:
    \begin{itemize}
     \item $S$ hat $f$ nicht im Cache
     \item $S$ entfernt $e'$
     \renewcommand{\labelenumii}{(\theenumii)}
     \renewcommand{\theenumii}{\alph{enumii}}
     \begin{enumerate}
     \item $e' = e$: $S'$ muss nichts machen, danach sind die Caches gleich\\
         $\Rightarrow$ $S'$ verfahren wie $S$.
     \item $e' = e$: 2 Schritte:
     \renewcommand{\labelenumiii}{\theenumiii. Schritt:}
     \renewcommand{\theenumiii}{\arabic{enumiii}}
         \begin{enumerate}
         \item $s'$ entfernt $e'$ und holt $e$, verfährt danach wie $S$\\
                 Diese Strategie ist noch nicht reduziert
         \item Wandle $S'$ in eine reduzierte Strategie um. Das erhöht die Anzahl der Speicherzugriffe nicht und lässt die ersten $j+1$ Ereignisse gleich.
         \end{enumerate}
     \end{enumerate}
    \end{itemize}
 \item Zugriff auf $e$:\\
     kann nicht passieren, weil der nächste Zugriff auf $e$ am weitesten in der Zukunft liegt (also kommt Fall (ii) auf jeden Fall vor Fall (iii))
\end{enumerate}
Beweis des Satzes:
\begin{itemize}
 \item \textbf{Beweis 1:} Wandle eine optimale Lösung Schritt für Schritt mit Austauschlemma zu $S_{\text{ff}}$ um.
 \item \textbf{Beweis 2:} Nimm optimale Lösung $S^*$, so dass $S^*$ und $S_{\text{ff}}$ ein längstes Präfix gemeinsam haben. Nimm an $S^* = S_{\text{ff}}$. Erhalte Widerspruch mit dem Austauschlemma. \hfill$\square$
\end{itemize}



\chapter{Datenstrukturen}
\section{Union-Find-Struktur}
\Geg Menge $S = \{1, ..., n\}$
\begin{itemize}
 \item Speichere Partition $\{S_1,S_2,...,S_k\}$ von $S$ und unterstütze $\operatorname{UNION}(S_i,S_j)$ und $\operatorname{FIND}(S)$.
 \item Darstellung durch Dijoint Set Forest (= Wald)
 \item 2 Heuristiken: \textsc{Union-by-Rank}, Pfadkompression
 \item \textsc{Union-by-Rank} gibt $O(1)$ UNION, $O(\log n)$ FIND
 \item Was ist mit Pfadkompression?
     \begin{itemize}
      \item Hilft nicht im Worst-Case: Erstes FIND braucht evtl immer noch $\Theta(\log n)$ Zeit.
     \end{itemize}
 \item Frage: Gegeben Folge $X$ von $m$ UNION-FIND-Operationen, beginnend mit der Partition $\{\{1\}, \{2\}, ..., \{n\}\}$ mit \textsc{Union-by-Rank} und Pfadkompression. Was ist die Gesamtlaufzeit
 \item Nur \textsc{Union-by-Rank}: $\Omega(m \log n)$ worst-case. Wie hilft Pfadkompression?
     \begin{itemize}
      \item \textbf{1. Schritt:} UNION-Operationen loswerden.
      \Defi Sei $F$ ein Wald.
          \begin{description}
           \item[Elternpfad:] Folge $v_1v_2...v_k$ von Knoten in $F$, so dass $v_{i+1}$ Eltern Knoten von $v_i$ ist für $i = 1,..., k-1$
           \item[Wurzelpfad:] Elternpfad, dessn letzter Knoten die Wurzel ist.
          \end{description}
      \paragraph*{veralgemeinerte Kompression}
      \begin{itemize}
       \item Gegeben ein Elternpfad $P= v_1v_2...v_k$. Hänge alle Knoten unter den Elternknoten von $v_k$
           \begin{center}
            \begin{minipage}{3cm}
            \psset{levelsep=0.5cm}
            \def\dred{\ncline[linecolor=red]}
            \pstree{\Tdot}{
                \pstree{\Tdot}{\Tdot}
                \Tdot
                \pstree{\Tdot}{
                    \pstree{\Tdot}{
                        \Tdot
                        \Tdot
                        \Tdot
                    }
                    \pstree{\Tdot[linecolor=red,edge=\dred]}{
                        \pstree{\Tdot[linecolor=red,edge=\dred]}{
                            \Tdot[linecolor=red,edge=\dred]
                        }
                    }
                }
            }
            \end{minipage}
            {\LARGE $\Rightarrow$}
            \begin{minipage}{3cm}
            \psset{levelsep=0.5cm}
            \def\dred{\ncline[linecolor=red]}
            \pstree{\Tdot}{
                \pstree{\Tdot}{\Tdot}
                \Tdot
                \pstree{\Tdot}{
                    \pstree{\Tdot}{
                        \Tdot
                        \Tdot
                        \Tdot
                    }
                    \Tdot[linecolor=red,edge=\dred]
                    \Tdot[linecolor=red,edge=\dred]
                    \Tdot[linecolor=red,edge=\dred]
                }
            }
            \end{minipage}
           \end{center}
       \item Spezialfall: $f$ ist Wurzelpfad: $v_1v_2...v_k$ sind Wurzeln neuer Bäume
           \begin{center}
            \begin{minipage}{1.5cm}
             \psset{levelsep=0.5cm}
             \def\dred{\ncline[linecolor=red]}
             \pstree{\Tdot[linecolor=red]}{
                 \Tdot
                 \pstree{\Tdot[linecolor=red,edge=\dred]}{
                     \Tdot
                     \pstree{\Tdot[linecolor=red,edge=\dred]}{
                         \pstree{\Tdot}{
                             \Tdot
                         }
                         \Tdot
                     }
                 }
             }
            \end{minipage}
            {\LARGE $\Rightarrow$}
            \begin{minipage}{1.5cm}
             \psset{levelsep=0.5cm}
             \def\dred{\ncline[linecolor=red]}
             \pstree{\Tdot[linecolor=red]}{
                 \Tdot
             }
             \hfill
             \pstree{\Tdot[linecolor=red]}{
                 \Tdot
             }
             \hfill
             \pstree{\Tdot[linecolor=red]}{
                 \pstree{\Tdot}{
                     \Tdot
                 }
                 \Tdot
             }
            \end{minipage}
           \end{center}
       \item Kosten sind
           \begin{itemize}
           \item 0 für Wurzelpfad
           \item $k-1$, sonst
           \end{itemize}
      \item Folge von verallgemeinerten Kompressionen:
              \[C = (P_1,P_2,...,P_l)\]
            Sei $F$ Wald:
            \begin{itemize}
            \item Setze $F_0 := F$
            \item Es muss gelten $P_i$ ist Elternpfad für $F_{i-1}$
            \item Sei $F_i$ der Wald nach Ausführung von $P_i$ in $F_{i-1}$
            \end{itemize}
            \begin{description}
             \item[Kosten:] $\operatorname{cost}(C) = $Summe der Einzelkosten
             \item[Länge:] $\operatorname{l}(C) = $Anzahl der Nichtwurzelpfade
            \end{description}
            \Lemma Gegeben $m$ UNION-FIND-Operationen, wie am Anfang.
            Dann existiert ein Wald $F$ und eine Folge $C$ von vergeallgemeinerten Kompressionen, so dass
            \begin{enumerate}
             \item $\operatorname{l}(C) \leq m$
             \item Laufzeit der Operation ist $O(n + m + \operatorname{cost}(C))$
             \item jede Knoten $v$ von $F$ hat einen Rang $\operatorname{rg}(v)$, so dass:
                 \begin{itemize}
                 \item alle Ränge sind $\leq \log n$
                 \item Rang eines Elternknotens ist echt größer als Rang eines jeden Kindes
                 \item es existieren $\leq \frac{n}{2^k}$ Knoten mit Rang $\geq k$
                 \end{itemize}
            \end{enumerate}
            \Bew Führe nur die UNION-Operationen aus. Erhalte den Wald $F$ mit Rängen an den Knoten, die (iii) erfüllen ($\rightarrow$ letzte VL)\\
            Konstruiere $C$, beginne mit $F$:
            \begin{itemize}
             \item Betrachte nächse $\operatorname{FIND}(S)$-Operation
             \item Sei $P$ der Pfad von $s$ zum Kind des Ergebnisses von $\operatorname{FIND}(S)$ (zum entsprechenden Zeitpunkt)
             \item Füge $P$ zu $C$ hinzu, wenn $P$ nicht leer
             \item Führe verallgemeinerte Kompression $P$ aus.
             \item Wiederhole mit nächsten $\operatorname{FIND}$ und dem neuen Wald.
            \end{itemize}
            Jetzt Folgen (i) und (ii) sofort
      \end{itemize}
     \item \textbf{Schritt 2:} Verstehe verallgemeinerte Kompression Devide-and-Conquer-Style
         \paragraph*{Aufgabe} Gegeben $F, C$ wie im Lemma. Was ist $\operatorname{cost}(C)$?
         \Defi Sei $F = (V,E)$ Wald. Sei $V = V_t \cup V_b$ disjunkte Zerlehung von $V$, so dass $V_t$ nach oben abgeschlossen ist ($v \in V_t \Rightarrow$ Elternknoten in $v \in V_t$). Dann ist: 
         \[F_t = (V_t, E \cap V_t \times V_t) \qquad F_b = (V_b, E \cap V_b \times V_b)\]
         Eine \emph{Zerlegung} von $F$.
         \paragraph*{Hauptlemma} Sei $F,C$ wie in der Aufgabe und sei $F_t = (V_t, E_t)$ und $F_b = (V_b, E_b)$ Zerlegung von $F$. Dann existiert eine verallgemeinerte Kompressionsfolge $C_t$ für $F_t$ un $C_b$ für $F_b$
         \begin{enumerate}
          \item $\operatorname{l}(C_b) + \operatorname{l}(C_t) \leq \operatorname{l}(C)$
          \item $\operatorname{cost}(C) \leq \operatorname{cost}(C_t) + \operatorname{cost}(C_b) + |V_b| + \operatorname{l}(C_t)$
         \end{enumerate}
         \Bew Betrachte ein $P \in C$. Füge $P \cap V_b$ zu $C_b$ hinzu und $P \cap V_t$ zu $C_t$. Dies zählt zu höchstens einen von $\operatorname{l}(C_t)$ und $\operatorname{l}(C_b)$. D. h. (i) gilt.
         \begin{description}
          \item[Kosten] Hänge Knoten aus $V_t$ unter Knoten aus $V_t \to \operatorname{cost}(C_t)$. Hänge Knoten aus $V_b$ unter Knoten aus $V_b \to \operatorname{cost}(C_b)$ 
         \end{description}
         Hänge Knoten aus $V_b$ unter Knoten aus $V_t$:
         \begin{itemize}
          \item beim esten Mal: $|V_b|$
          \item jedes weitere Mal: $\operatorname{l}(C_t)$ \hfill $\square$
         \end{itemize}
     \item \textbf{Schritt 3:} Bootstrapping
         \begin{description}
          \item[Ziel] Finde gute Unterteilung für Hauptlemma $\Rightarrow$ Bringe Ränge ins Spiel.
          \item[Beobachtung] $F$ Wald, $C$ Kompressionsfolge. Wenn alle Knoten in $F$ Rang $\leq r$ haben, so ist $\operatorname{cost}(C) \leq |v| \cdot r$
          \item[Beweis] $\operatorname{cost}(C)$: Wie oft erhält Knoten in $F$ neue Elternknoten. Jeder neue Elternknoten hat höheren Rang als der alte \\
         $\Rightarrow$ Jeder Knoten erhält höchstens $r$ neue Elternknoten
         \end{description}
         \begin{itemize}
         \item Sei $F$ Wald mit Maxrang $r$. Setze $s := \log x$
         \item Sei $V_{>S}$: Knoten in $F$ mit Rang $> S  \rightarrow F_{>S}$
         \item Sei $V_{\leq S}$: Knoten in $F$ mit Rang $\leq S  \rightarrow F_{\leq S}$ 
         \end{itemize}
         Dann ist $|V_{>S}| \leq \frac{n}{2^S} = \frac{n}{r}$
         \paragraph*{Hauptlemma (ii):} $\operatorname{cost}(C) \leq \operatorname{cost}(C_{\leq S}) + \operatorname{cost}(C_{> S}) + |V_{\leq S}| + \operatorname{l}(C_{> S})$
         \begin{itemize}
         \item $\operatorname{cost}(C_{>S}) \overset{\text{Beob.}}{\leq} |V_{>S}| \cdot r \leq \frac{n}{r} \cdot r = n$
         \item $|V_{\leq S}| \leq n$
         \item Hauptlemma (i): $\operatorname{l}(C_{> S}) \leq \operatorname{l}(C) - \operatorname{l}(C_{\leq S})$
         \end{itemize}
         Also:
         \begin{align*}
          \operatorname{cost}(C) &\leq \operatorname{cost}(C_{\leq b}) + 2n + \operatorname{l}(C) - \operatorname{l}(C_{\leq S}) \tag{$C_{>S}$ ist weg!}\\
          \Rightarrow \operatorname{cost}(C) - \operatorname{l}(C) &\leq \operatorname{cost}(C_{\leq S} - \operatorname{l}(C_{\leq S}) + 2n)
         \end{align*}
         Jetzt ist $F_{\leq S}$ Wald mit Maxrang $S = \log r$.
         \begin{itemize}
          \item Setze $S' = \log S = \log \log r$
          \item Unterteile $F_{\leq S}$ in $F_{>S'}$ und $F_{\leq S'}$.
          \item Erhalte: $\operatorname{cost}(C_{\leq S}) - \operatorname{l}(C_{\leq S}) \leq \operatorname{cost}(C_{\leq S'} - \operatorname{l}(C_{\leq S'}) + 2n)$
          \item Wiederhole mit $S'' := \log S' = \log \log \log r$, usw, bis $S'''...' \leq 2$ ist. 
         \end{itemize}
         Anzahl der Wiederholungen ist $\log^{*} r$\\
         Rückwerts einsetzen:
         \begin{align*}
          \operatorname{cost}(C) - \operatorname{l}(C) &\leq \operatorname{cost}(C_{\leq S}) - \operatorname{C_{\leq S}} + 2n \\
                                                       &\leq \operatorname{cost}(C_{\leq S'}) - \operatorname{l}(C_{\leq S'}) + 2n + 2n\\
                                                       &\leq \operatorname{cost}(C_{\leq S''}) - \operatorname{l}{C_{\leq S''}} + 2n + 2n + 2n\\
                                                       & \quad \vdots\\
                                                       &\leq \underbrace{2n + 2n + ... + 2n}_{\log^{*} r} = 2n \log * r
         \end{align*}
         Wissen: $\operatorname{cost}(C) \leq l(C) + 2n \log^{*} r$, 
         Fange von vorne an, aber setze $S := \log \log^{*} r$,
         Unterteile $F$ in $F_{>S}$ ujnd $F_{\leq S}$.
         Erhalten: 
         \[|V_{>S}| \leq \frac{n}{\log^{*} r} \Rightarrow \operatorname{cost}(C_{>S}) \leq \operatorname{l}(C_{>S}) + 2n\]
         Anwendung von Hauptlemma gibt:
         \newcommand{\cost}{\operatorname{cost}}
         \renewcommand{\l}{\operatorname{l}}
         \[\cost(C)- 2 \l(C) \leq \cost(C_{\leq S}) - 2 \l(C_{\leq S}) + 3 n\]
         Rekursion wie vorher liefert:
         \[\cost(C) \leq 2 \l(C) + 3n \log^{**} r\]
         
\begin{itemize}
 \item $\lfloor \log r \rfloor$: Wie oft muss man $\frac{r}{2}$ rechnen, damit Ergebnis $< 2$.
 \item $\log^{*} r$: Wie oft muss man $\log r$ rechnen, damit Ergebnis $< 2$.
 \item $\log^{**} r$: Wie oft muss man $\log^{*} r$ rechnen, damit Ergebnis $< 2$.
 \item $\log^{***} r$: Wie oft muss man $\log^{**} r$ rechnen, damit Ergebnis $< 2$.
 \item $\log^{*(j)} r$: Wie oft muss man $\log^{*(j-1)} r$ rechnen, damit Ergebnis $< 2$.
\end{itemize}
Wenn $\cost(C) \leq j \cdot \l(C) + (j+1)n \log^{*(j)} r$ ist, dass, dann $\cost(C) \leq (j+1) \cdot \l(C) + (j+2)n \log^{*(j+1)} r$\\
    Wissen für $j \geq 0$:
    \[\cost(C) \leq (j - 1) m + (j + 1) n \log^{*(j)} n\]
    Welches $i$ sollen wir wählen? Setze die Summanden gleich:
    \begin{align*}
     (j - 1) m &= (j+1) m \log^{*(j)} n
     \Rightarrow \log^{*(j)} n &= \frac{m}{n} \tag{Nimm an $m \geq n$, o. B. d. A.}
    \end{align*}
    Bestimme $\alpha(m,n) = \min\{ j \in \mathbb{N}\ |\ \log^{*(j)} n \leq \frac{m}{n}\}$ (inverse Ackermann-Funktion)\\
    Setze $\alpha(m,n) \Rightarrow \cost(C) \leq m(\alpha(m,n) + 1) + m(\alpha(m,n) + 1) = O(m \alpha(m,n))$
     \end{itemize}
\Satz (Tarjan) Laufzeit von $m$ UNION-FIND-Operationen mit \textsc{Union-by-Rank} und Pfadkompression ist $\Theta(n + m \alpha(m,n))$
\paragraph*{Bemerkung} 
\begin{itemize}
 \item $\alpha(m,n) \leq 5$ für jedes vernünftige $m, n$
 \item Optional
\end{itemize}
\end{itemize}
% s. 02.12.2011
\begin{itemize}
 \item $\alpha(m,n)$ taucht öftes auf: $n$ Strecken, wieviele Schnittpunkte sieht man von unten $O(n \alpha(m,n))$
 \item Bester MST-Algorithmus: $O(\underbrace{m}_{\text{\# Kosten}}, \alpha(m,\underbrace{n}_{\text{\# Knoten}}))$ Zeit
\end{itemize}

\Defi Sei $D$ Datenstruktur mit Operationen $\operatorname{Op}_1, ..., \operatorname{Op}_k$.
         wenn für \emph{jede} genügend lange Folge $X$ von Operationen gilt: 
         \begin{itemize}
         \item Laufzeit von $X$ ist $a_1 \cdot t_1 + ... + a_k t_k$, wobei $o_1 = \text{\# der Operation $\operatorname{Op}_i$ in $X$}$
         \end{itemize}
         Dann sagen wir: Die \emph{amortisierte Laufzeit} für Op$_i$ ist $t_i$

Bei UNION-FIND: Amortisierte  Laufzeit für UNION und FIND ist $O(\alpha(m,n))$. Genauer (kann man zeigen):
\begin{itemize}
 \item die amortisierte Laufzeit für UNION ist $O(1)$
 \item die amortisierte Laufzeit fpr FIND ist $O(\alpha(m,n))$
\end{itemize}

\section{Hashing}
Wörterbuchproblem:
\begin{description}
 \item $K$: Schlüsselmenge
 \item $V$: Wertemenge
\end{description}
Speichere Menge von Einträgen $(k,v) \in K \times V$. Operation:
\begin{description}
 \item $\operatorname{put}(k,v)$ Setze Eintrag für $k$ auf $v$
 \item $\operatorname{delete}(k)$ Lösche Eintrag für $k$
 \item $\operatorname{get}(k)$ Suche Eintrag für $k$
\end{description}

\subsection{Hashing mit Verkettung}
\begin{itemize}
\item Hashtabelle $T$: Array der Länge $m$.
\item Hashfunktion $h{:} K \to \{0, ..., m-1\}$
\item Speichere Eintrage für $k$ an der Stelle $h(k)$ in $T$
\item \textbf{Kollision:} Schlüssel $k \neq l \in K$ mit $h(k) = h(l)$. In der Regel unvermeidbar
\item \textbf{Kollisionsbehandlung:} Verkettung. Speichere alle Einträge, die auf die selbe Stelle hashen in einer verketteten Liste
\end{itemize}
Laufzeit?
\begin{itemize}
 \item Hängt von $h$ ab
 \item Theorie: $h$ zufällige Funktion $\rightarrow$ gute Laufzeit
 \item Aber: Wie wähle ich $h$ zufällig und speichere es?\\
     \begin{itemize}
      \item Lösung 1: heuristische Hashfunktion
      \item Lösung 2: universelles Hashing
     \end{itemize}
\end{itemize}

\subsection[Universelles Hashing]{Universelles Hashing (Carter-Wegman)}
\Defi Sei $H$ eine Menge von Funktionen $h{:} K \to \{0,...,m-1\}$. $H$ heißt \emph{universell}, wenn 
\[\forall k, l \in K, k \neq l{:} \operatorname{Pr}_{h \in H}\left[h(k) = h(l)\right] \leq \frac{1}{m}\]
\Satz Sei $H$ universell, $h \in H$ zufällig. Sei $T$ Hashtabelle, in der $n$ Schlüssel gespeichert sind. Sei $k \in K$ fest. Dann ist die erwartete Anzahl an Schlüsseln $\neq k$ bei $T[h(k)] \leq \frac{n}{m}$
\Bew Sei $S$ Menge der Schlüssel in $T$.
    \begin{description}
     \item $Y_k$: Anzahe der  Schlüssel in $ \setminus \{k\}$, die auf $h(k)$ gehasht werden.
    \end{description}
    Suchen $E[Y_k]$
    \begin{itemize}
     \item Definiere für $l \in S \setminus \{k\}$
             \[X_{k,l} = \begin{cases}
                          1, & \text{falls $h(k) = h(l)$} \\
                          0, & \text{sonst}
                         \end{cases}\]
     \item Dann ist $Y_k = \sum\limits_{l \in S \setminus \{k\}} X_{k,l}$
     \item Also:
         \begin{align*}
          E[Y_k] &= \sum\limits_{l \in S \setminus \{k\}} E[X_{k,l}] \\
                 &= \sum\limits_{l \in S \setminus \{k\}} \operatorname{Pr}[h(k) = h(l)] \\
                 &\overset{\text{(Definition von $H$)}}{\leq} \sum\limits_{l \in S \setminus \{k\}} \frac{1}{m} \leq \frac{n}{m}
         \end{align*}\hfill$\square$
    \end{itemize}
Bei universellem Hashing ist die erwartete Laufzeit pro Opeation $O\left(1+\underset{\text{Belegungsfaktor}}{\frac{n}{m}}\right)$.\\
Gibt es gute universelle Familien?
\begin{itemize}
 \item Ja. Sei $K = \underbrace{\{0, ..., p-1\}}_{\mathbb{Z}_p}$, $p$ Primzahl
 \item Für $a,b \in \{0, ..., p-1\}, a \neq 0$.
 \item Definiere $h_{a,b}(k) = ((\overbrace{a \cdot k + b}^{\text{Gerade}}) \mod p) \mod m$
 \item Setze $H := \{h_{a,b}\ |\ a,b \in \{0, ..., p-1\}, a \neq 0\}\}$
\end{itemize}
\Satz $H$ ist universell
\Bew Fixiere $k, l \in K, k \in l$. Was ist $\operatorname{Pr}[h(k) = h(l)]$?
\begin{itemize}
 \item Klar: $a \cdot k + l \neq a \cdot l + n \mod p$ gibt es keine Kollision:
     \begin{itemize}
     \item Fixiere $r,s \in \{0, ..., p-1\}, r \neq s$
     \item Behauptung: Es exisitiert genau ein $a, b$. so dass 
         \[a \cdot k + b = r \pmod{p} \qquad a \cdot l + b = s \pmod{p}.\]
         \Bew
             \begin{align*}
              a k + b &= r \pmod{p}\\
              a l + b &= s \pmod{p}\\\hline
              a (k - l) &= r - s \pmod{p} \\
\Leftrightarrow      a  &= (r-s)(k-l)^{-1} \pmod{p} 
             \end{align*}
             Kann $b$ auch ausrechnen $\Rightarrow$ $a, b$ eindeutig.
     \item Kollisionen passieren nur $\mod m$:
             \begin{itemize}
             \item Nimm $r, s \in \{0, ..., p-1\}, r \neq s$. Wie viele solcher $r, s$ gibt es mit $r = s \pmod{m}$
             \item $\underbrace{P}_{\text{Möglichkeiten für $r$}} (\underbrace{\left\lceil\frac{p}{m}\right\rceil - 1}_{\text{Möglichkeiten für $s$}}) \leq \frac{p(p-1)}{m} = \frac{|H|}{m}$
             \item Anzahl der Hashfunktionen, für die $h(k) = h(l) \leq \frac{|H|}{m}$ \\
                 $\Rightarrow$ Wahrscheinlichkeit für Kollision $\leq \frac{1}{m}$
             \end{itemize}
     \end{itemize}
\end{itemize}



\subsection{Perfektes Hashing}
Situation: $S \subseteq K, |S| = n$ fest gegeben. Wollen gute \emph{worst-case} Laufzeit.
\begin{description}
 \item[Idee 1:] Finde Hashfunktion $h{:} K \to \{0, ..., m-1\}$, so dass $h$ \emph{keine} Kollision auf $S$ hat. Geht immer, wenn $m \geq n$ ist.
 \begin{description}
  \item[Problem:] $h$ muss einfach zu berechnen sein, und einfach darstellbar.
  \item[Lösungsidee:] Wähle $h$ aus universeller Familie, so dass $h$ auf $S$ keine Kollision hat.
 \end{description}
 \emph{Wann geht das?} Wahrscheinlich, wenn $m$ groß genug ist. \\
 \emph{Wie groß muss $m$ sein, damit zufälliges universelles $h{:}\ K \to \{0,...,m-1\}$ auf $S$ ($|S| = n$) keine Kollision hat?}
 \begin{description}
  \item[Lemma 1:] Sei $S \subseteq K$, $|S| = n$ fest $H$ universelle Familie von Hashfunktionen. Dann
      \[E_{h \in H}[\#\text{Kollisionen auf $S$}] \leq \binom{n}{2} \cdot \frac{1}{m}\]
  \item[Beweis:] Sei $Y = \#\text{Kollisionen}$. Sei 
      \[k \neq l \in S{:}\ X_{k,l} = \begin{cases}1, & \text{wenn $h(k) = h(l)$}\\ 0, & \text{sonst}\end{cases}\]
      \begin{align*}
      E[Y]   &= \sum\limits_{k \neq l \in S} E[X_{kl}] \\
             &= \sum\limits_{k \neq l \in S} \operatorname{Pr}[h(k) = h(l)] \\
             &\leq \sum\limits_{k \neq l \in S} \frac{1}{m} = \binom{n}{2} \cdot \frac{1}{m}.  
      \end{align*} \hfill $\square$
  \item[Problem 1:] Brauchen $m = n(n-1) = \Theta(n^2)$ Plätze in der Hashtabelle, dann $E[\#\text{Kollision}] \leq \frac{1}{2}$. Das ist schlecht.
  \item[Problem 2:] Wie finden wir ein gutes $h \in H$ effizient?
 \end{description}
 \begin{description}
  \item[Problem 1:] Wie bei Hashing mit Verkettung: 2-stufige Hashtabelle
  \item[1. Stufe:] $T_1$ mit $n$ Positionen ($m_1 = n$).
      \begin{itemize}
       \item Wähle $h{:}\ K \to \{0, ..., n - 1\} \in H$ universell.
       \item Setze $S_i = \{k \in S\ |\ h(k) = i\}$
       \item Lege für Position $i$ eine Hashtabelle der Größe $|S_i|^2$ an.
       \item Wähle $h_i{:}\ K \to \{0, ..., |S_i|^2 - 1\}$ universell
      \end{itemize}
      Was ist der Platzbedarf?
      \begin{description}
       \item[Lemma:] $E_{h \in H}[\text{Platzbedarf}] = O(n)$
       \item[Beweis:] \begin{align*}
                       \text{Platzbedarf} &= \overset{\text{für $T_n$}}{n} + \sum\limits_{i = 0}^{n-1} |S_i|^2 \\
                            &= n + \sum\limits_{i = 0}^{n - 1} \left(\sum\limits_{k \in S_i} 1\right) \cdot \left(\sum\limits_{l \in S_i} 1\right) \\
                            &= n + \sum\limits_{i = 0}^{n - 1} \left(\sum\limits_{k \in S_i} 1 + 2 \cdot \sum\limits_{k \neq l \in S_i} 1\right)\\
                            &= n + \sum\limits_{i = 0}^{n - 1} |S_i| + \sum\limits_{i = 0}^{n - 1} \sum\limits_{k \neq l \in S_i} 2\\
                            &= n + n + \sum_{k \neq l \in S} 2 X_{k,l}, \text{wobei $X_{k,l} = \begin{cases}1, &\text{wenn $h(k) = h(l)$}\\ 0, & \text{sonst}\end{cases}$}
                      \end{align*}
                      \begin{align*}
                       E[\text{Platzbedarf}] &= 2n + 2 \sum\limits_{k \neq l \in S} E[X_{k,l}] \\
                               &\leq 2n + 2 \sum_{k \neq l \in S} \frac{1}{n}
                               &= 2n + 2 \cdot \frac{n(n-1)}{2} \cdot \frac{1}{n}
                               &= 3n - 1 = O(n)
                      \end{align*} \hfill $\square$
      \end{description}
  \item[Problem 2:] [100 Basketballspieler. Durchschnittsgröße 2 Meter, \\$\leq 50$ Basketballspieler haben Größe $\geq 4$m.]
 \end{description}
Markov-Ungleichung: $X \geq 0, t \geq 0$
\begin{itemize}
 \item Es gibt: $Pr[X \geq t] \leq \frac{E[X]}{t}$.
\end{itemize}
In unserer Anwendung:
\begin{align*}
\operatorname{Pr}_{h \in H}[\#\text{Kollision}] &\geq \frac{E[\#\text{Kollision}]}{1} \\
                                               &= \frac{1}{2}
\end{align*}

\begin{align*}
\operatorname{Pr}_{h \in H}[\text{Platzbedarf $\geq 6n$}] &\leq \frac{E[\text{Platzbedarf}]}{6n} \\
                                               &= \frac{3n}{6n} = \frac{1}{2}
\end{align*}
\item[Satz:] Können in $O(n)$ \emph{erwartete} Vorverarbeitungszeit eine Datenstruktur für $S$ bauen, die $O(n)$ Platz benötigt und $O(1)$ worst-case Lookups unterstützt
\item[Beweis:] Wähle $h \in H$, bis zufällig, bis Eigenschaften erfüllt. Erfolgswahrscheinlichkeit hat jeweils $\geq \frac{1}{2} \Rightarrow 2$ Versuche jeweils im Erwartungswert.
\end{description}

\section{Internet-Algorithmen}
\subsection{Page-Rank}
benannt nach Larry \textsc{Page} (aktueller CEO von Google)
\paragraph*{Suchmaschinen-Problem}
\begin{description}
\item[Gegeben:] Suchanfrage $s$ von Benutzer $n$
\item[Gesucht:] Was Benutzer $n$ finden möchte.
\end{description}
Was ist das?
\paragraph*{Erste Ansätze}
\renewcommand{\labelenumi}{(\theenumi)}
\renewcommand{\theenumi}{\arabic{enumi}}
\begin{enumerate}
 \item Yahoo: Manueller Katalog
 \item AltaVista: Textbasierte Suchmaschine
     \begin{itemize}
     \item automatisches Katalogisieren des ganzen Webs (Bots)
     \item Speichern eines großen Index
     \item Bei Suchanfrage $s$: durchsuche Index nach $s$, fiefere die Sites mit $s$ zurück
     \end{itemize}
\end{enumerate}
\paragraph*{Problem:} Wie bewertet man die gefundenen Seiten bei einer breiten Suchanfrage?
\paragraph*{Idee} (Kleinberg, Page) aus Bibliographieanalyse:
Verwende Linkstruktur des www, um Seiten zu bewerten (nutze das Wissen der Benutzer)
\paragraph*{Page-Rank}
\begin{center}
 \begin{psmatrix}[mnode=circle]
  GB & & L \\
     & ZS \\
  \  & \ & \ 
 \end{psmatrix}
 \ncline{1,1}{1,3}\ncline{1,1}{2,2}\ncline{2,2}{3,3}\ncline{3,1}{3,2}
 \ncline{2,2}{3,2}\ncline{3,1}{2,2}\ncline{1,3}{3,3}
\end{center}
Der Page-Rank einer Webseite ist die Wahrscheinlichkeit, dass ein zufälliger Surfer bei ihr vorbeikommt.
\paragraph*{Wie formalisieren?}
www: gerichteter Graph
\[G = (V, E) \qquad |V| = N\]
\begin{center}
 \begin{psmatrix}[mnode=circle]
   & \ \\
   \ & \ & \ \\
   \ & \ 
 \end{psmatrix}
 \psset{arrows=->}
 \ncline{1,2}{2,3}\ncline{1,2}{2,2}
 \ncline{2,1}{2,2}\ncline{2,2}{2,3}
 \ncline{3,1}{2,1}\ncline{2,2}{3,1}\ncline{2,2}{3,2}
\end{center}
Berechne in jedem Schritt eine Wahrscheinlichkeitsverteilung auf den Knoten $\Pi_i{:} V \to [0,1]$, $\sum\limits_{v \in V} \Pi_i(v) = 1$.
\begin{itemize}
 \item Am Anfang: $\forall v \in V{:}\ \Pi_0(v) = \frac{1}{N}$
 \item $\Pi_{i+1}(v) = \sum\limits_{w \in V} \frac{1}{\operatorname{outdeg}(w)} \cdot \Pi_i(w) \cdot A_{wv}$, wobei
         $A_{wv} = \begin{cases} 1, & \text{falls $(w,v) \in E$} \\ 0, & \text{sonst} \end{cases}$
\end{itemize}
Schreibe Gleichungen kompakt als Matrix:
\[\underbrace{\Pi_{i+1}}_{\text{Zeilenvektor}} = \Pi_i \cdot A'\]
wobei
\[A' = N \times N\text{-Matrix mit } A'_{wv} = \begin{cases} 
                                                    \frac{1}{\operatorname{outdeg}(w)}, & \text{falls $(w,v) \in E$} \\
                                                    0, \text{sonst}
                                               \end{cases}
\]
\begin{center}
 \begin{psmatrix}[mnode=circle]
  $a$ & $b$ \\
  $d$ & $c$
 \end{psmatrix}
 \psset{arrows=->}
 \ncline{1,1}{1,2}\ncline[offset=2px]{1,1}{2,2}
 \ncline{1,2}{2,2}
 \ncline{2,1}{1,1}
 \ncline{2,2}{2,1}\ncline[offset=2px]{2,2}{1,1}
\end{center}

\[
A' = \begin{pmatrix}
 0 & \frac{1}{2} & \frac{1}{2} & 0 \\
 0 & 0 & 1 & 0 \\
 \frac{1}{2} & 0 & 0 & \frac{1}{2} \\
 1 & 0 & 0 & 0
\end{pmatrix}
\]
$A'$ heißt modifizierte Adjazenzmatrix
\begin{align*}
 \Pi_{i+1} &= \Pi_i A' \\
 \Pi_i &= \underbrace{\begin{pmatrix}\frac{1}{4} & \frac{1}{4} & \frac{1}{4} & \frac{1}{4}\end{pmatrix}}_{\Pi_0} \cdot
 \begin{pmatrix}
 0 & \frac{1}{2} & \frac{1}{2} & 0 \\
 0 & 0 & 1 & 0 \\
 \frac{1}{2} & 0 & 0 & \frac{1}{2} \\
 1 & 0 & 0 & 0
\end{pmatrix}
 &= \begin{pmatrix}\frac{3}{8} & \frac{1}{8} & \frac{3}{8} & \frac{1}{8}\end{pmatrix}
\end{align*}
Hoffnung: Nach genügend vielen Schritten ändert sich $\Pi$, nicht mehr (Konvergenz, Fixpunktiteration)
\Defi Ein Wahrscheinlichkeitsvektor $\Pi^*$ heißt \emph{stationäre Verteilung} für $A'$, wenn gilt
    \[\Pi^* = \Pi^* \cdot A'\]
    $\Pi^*$ ist ein Linkseigenvektor zum Eigenwert 1.
\paragraph*{Probleme}
\begin{enumerate}
 \item Konvergiert $\Pi_i$? Im Allgemeinen: Nein
 \begin{center}
  \begin{psmatrix}[mnode=circle]
   \color{red}$\frac{1}{3}$ & \color{red}$\frac{1}{3}$ & \color{red}$\frac{1}{3}$
  \end{psmatrix}
  \psset{arrows=->}
  \ncline[offset=2px]{1,1}{1,2}\ncline[offset=2px]{1,2}{1,3}
  \ncline[offset=2px]{1,2}{1,1}\ncline[offset=2px]{1,3}{1,2}
 \end{center}
 \begin{center}
  \begin{psmatrix}[mnode=circle]
   \color{red}$\frac{1}{6}$ & \color{red}$\frac{2}{3}$ & \color{red}$\frac{1}{6}$
  \end{psmatrix}
  \psset{arrows=->}
  \ncline[offset=2px]{1,1}{1,2}\ncline[offset=2px]{1,2}{1,3}
  \ncline[offset=2px]{1,2}{1,1}\ncline[offset=2px]{1,3}{1,2}
 \end{center}
 Das System oszilliert
 \item Existiert $\Pi^*$?
 \begin{center}
  \begin{psmatrix}[mnode=circle]
   \color{red}$\frac{1}{2}$ & \color{red}$\frac{1}{2}$
  \end{psmatrix}
  \psset{arrows=->}
  \ncline{1,1}{1,2}
 \end{center}
 \begin{center}
  \begin{psmatrix}[mnode=circle]
   \color{red}$0$ & \color{red}$\frac{1}{2}$
  \end{psmatrix}
  \psset{arrows=->}
  \ncline{1,1}{1,2}
 \end{center}
 \begin{center}
  \begin{psmatrix}[mnode=circle]
   \color{red}$0$ & \color{red}$0$
  \end{psmatrix}
  \psset{arrows=->}
  \ncline{1,1}{1,2}
 \end{center}
\end{enumerate}
\paragraph*{Neustart/Dämpfung}
In jedem Schritt fängt ein bestimmter Anteil der Surfer auf eine zufälligen Seite neu an.\\
Schreibe:
\[ A'' := (1 - d) A' + \frac{d}{N} 
            \begin{pmatrix}
                1 & 1 & \cdots & 1 \\
                1 & 1 & \cdots & 1 \\
                \vdots & \vdots & \ddots & \vdots \\
                1 & 1 & \cdots & 1
            \end{pmatrix}\]
            $d \approx 0{,}15$ Dämpfungsfaktor
\begin{center}
\framebox{Füge Schleifen zu $G$ hinzu. \circlenode{i}{\ }\nccurve[ncurv=4,angleA=70,angleB=110]{->}{i}{i}}
\end{center}
Modifikation verhinder Oszillation und sagt für Existenz von $\Pi^*$.
\Satz (Perron-Frobenius): Es existiert genau eine stationäre Verteilung $\Pi^*$ für $A''$. Die Iteration $\Pi_{i+1} = \Pi_{i} \cdot A''$ konvergiert gegen $\Pi^*$. $\Pi^*$ ist der Pagerank-Score-Vektor des www.
\subsection{Prioritätswarteschlangen}
\begin{description}
 \item[Gegeben:] Elemente $X$, Prioritäten/Schlüssel $K$ (total geordnet)
 \item[Ziel:] Speichere Teilmenge $S \subseteq X \times K$
 \item[Operationen:] \hfill
    \begin{center}
        \begin{tabular}{rp{0.8\linewidth}}
            insert($x,k$): & Füge $x$ mit Priorität $k$ zu zu $S$ hinzu. \\
            delete\_min(): & liefere Element mit kleinster Priorität, entferne aus $S$ \\
            decrease\_key($x,k$): & erniedrige den Schlüssel von $x$ zu $k$
        \end{tabular}
    \end{center}
\end{description}
Klassische Anwendung: Dijkstras ALgorithmus
\begin{description}
 \item[Gegeben:]
     \begin{itemize}
      \item Graph $G = (V,E)$, gerichtet, gewichtet
      \item Gewichtsfunktion: weight: $E \to \mathbb{R}^+_0$
      \item Startknoten $s \in V$
     \end{itemize}
 \item[Gesucht:] kürzeste Wege von $s$ zu allen anderen Knoten (Shortest Path Tree)
\end{description}
\begin{algorithmic}
\FOR{$v \in V$}
    \STATE $v.d \gets \infty$; $v.\pi \gets \perp$
    \STATE $Q.\text{insert}(v,v.d)$
\ENDFOR
\STATE $s.d \gets 0$; $Q.\text{decrease\_key}(s,0)$
\WHILE{$Q \neq \emptyset$}
    \STATE $v \gets Q.\text{delete\_min}()$
    \FOR{$e = (v, w) \in E$}
        \IF{$v.d + e.\text{weight} < w.d$}
            \STATE $w.d \gets v.d + e.\text{width}$
            \STATE $w.\pi \gets v$
            \STATE $Q.\text{decrease\_key}(w,w.d)$
        \ENDIF
    \ENDFOR
\ENDWHILE
\end{algorithmic}
\paragraph*{Laufzeit:} Hängt von $Q$ ab.
\begin{itemize}
 \item $|V|$ inserts und delete\_mins
 \item $\leq |E| + 1$ decrease\_key 
\end{itemize}
Klassische Implementierung: Binärheap
\begin{itemize}
\item   Laufzeit: $O(\log n)$ pro Operation\\
        $\Rightarrow$ Dijkstra läuft in $O(|V|  \log |V| + |E| \log |V|)$ Zeit.
\end{itemize}
Frage: Geht es besser? Können wir eine bessere Implementierung von $Q$ finden?
\begin{itemize}
 \item $|V| \log |V|$ lässt sich wahrscheinlich nicht verbessern
 \item Gibt es Prioritätswarteschlange mit $O(1)$ decrease\_key
 \item 1987: Fredman-Tarjan: Ja: \\
     \textbf{Fibonacci-Heaps}
         \begin{itemize}
          \item $O(\log n)$ insert \& delete\_min
          \item $O(1)$ decrease\_key
         \end{itemize}
        (amortisierte Laufzeit)
 \item seitdem: viele Varianten und Alternativen:
     \begin{itemize}
     \item entspannte Heaps
     \item linksgerichtete Heaps
     \item dünne Heaps
     \item dicke Heaps
     \item $\vdots$
     \item Erdbeben-Heaps (\emph{Quake Heaps})
     \end{itemize}
\end{itemize}
\subsection{Quake-Heaps}
T.M. Chan, 2009
\begin{itemize}
 \item delete\_min in $O(\log n)$ amortisiert
 \item insert \& decrease\_key in $O(1)$ amortisierte
\end{itemize}
$\Rightarrow$ Dijkstra in $O(|V| \log |V| + |E|)$
\begin{itemize}
 \item Ein Quake-Heap speichert die Elemente in Turnierbäumen
 \item Turierbaum: Wurzelbaum
 \begin{itemize}
 \item Speichert Elemente in den Blättern
 \item Alle Blätter haben die gleiche Tiefe (= Länge des Pfades zur Wurzel)
 \item Innere Knoten haben 1 oder 2 Kinder
 \item Innerer Knoten speichert das Minimum seiner Kinder
 \item Jedes Blatt speichert einen Zeiger auf den höchsten Knoten, der den gleichen Schlüssel enthält 
 \end{itemize}
 \begin{center}
 \psset{levelsep=0.7cm,treesep=0.5cm,nodesep=2pt}
  \pstree{\Tr{0}}{
      \pstree{\Tr{4}}{\pstree{\Tr{4}}{\pstree{\Tr{4}}{\Tr{4}}}}
      \pstree{\Tr{0}}{
          \pstree{\Tr{2}}{
              \pstree{\Tr{2}}{
                  \Tr{2}\Tr{3}
              }
          }
          \pstree{\Tr{0}}{
              \pstree{\Tr{0}}{
                  \Tr{1}\Tr{0}
              }
              \pstree{\Tr{7}}{
                  \Tr{7}
              }
          }
      }
  }
 \end{center}
 \item Operationen auf Turnierbäume:
 \begin{enumerate}
  \item link($T_1, T_2$): $T_1, T_2$: Turnierbäume gleicher Höhe \\
      $\Rightarrow$ Turnierbaum mit Höhe + 1
      \begin{center}     
 \psset{levelsep=0.7cm,treesep=0.5cm,nodesep=2pt} 
       \begin{minipage}{3cm}
        \centering
        \pstree{\Tr{5}}{
          \pstree{\Tr{10}}{
              \Tr{10}
          }
          \pstree{\Tr{5}}{
              \Tr{5}\Tr{8}
          }
      }
       \end{minipage}
       \hspace{0.5em}
       \begin{minipage}{3cm}
        \centering
        \pstree{\Tr{2}}{
          \pstree{\Tr{2}}{
              \Tr{2}\Tr{7}
          }
          \pstree{\Tr{4}}{
              \Tr{9}\Tr{4}
          }
      }
       \end{minipage}
       \hspace{0.2cm}
       $\Rightarrow$
       \hspace{0.2cm}
       \begin{minipage}{6cm}
        \centering
        \pstree{\Tr{2}}{
            \pstree{\Tr{5}}{
          \pstree{\Tr{10}}{
              \Tr{10}
          }
          \pstree{\Tr{5}}{
              \Tr{5}\Tr{8}
          }
      }
        \pstree{\Tr{2}}{
          \pstree{\Tr{2}}{
              \Tr{2}\Tr{7}
          }
          \pstree{\Tr{4}}{
              \Tr{9}\Tr{4}
          }
      }
        }
       \end{minipage}
      \end{center}
      Lege neue Wurzel an, speichere in der alten Wurzeln in der neuen Wurzel. $O(1)$ Zeit.
  \item cut($v$): $v$ Knoten, der einen anderen Schlüssel enthält als sein Elternknoten\\
      $\Rightarrow$ zwei Turnierbäume, einen mit Wurzel $v$
      \begin{center}     
 \psset{levelsep=0.7cm,treesep=0.5cm,nodesep=2pt} 
       \begin{minipage}{6cm}
        \centering
        \pstree{\Tr{2}}{
            \pstree{\Tcircle[linecolor=red]{5}\nput{160}{\pssucc}{\color{red}$v$}}{
          \pstree{\Tr{10}}{
              \Tr{10}
          }
          \pstree{\Tr{5}}{
              \Tr{5}\Tr{8}
          }
      }
        \pstree{\Tr{2}}{
          \pstree{\Tr{2}}{
              \Tr{2}\Tr{7}
          }
          \pstree{\Tr{4}}{
              \Tr{9}\Tr{4}
          }
      }
        }
       \end{minipage}
       \hspace{0.2cm}
       $\Rightarrow$
       \hspace{0.2cm}
       \begin{minipage}{3cm}
        \centering
        \pstree{\Tr{5}}{
          \pstree{\Tr{10}}{
              \Tr{10}
          }
          \pstree{\Tr{5}}{
              \Tr{5}\Tr{8}
          }
      }
       \end{minipage}
       \hspace{0.5em}
       \begin{minipage}{3cm}
        \centering
        \pstree{\Tr{2}}{
          \pstree{\Tr{2}}{
              \Tr{2}\Tr{7}
          }
          \pstree{\Tr{4}}{
              \Tr{9}\Tr{4}
          }
      }
       \end{minipage}
      \end{center}
      cut löscht Kante von $v$ zu Elternknoten. $O(1)$ Zeit.
 \end{enumerate}
\end{itemize}

Quake-Heaps speichern
\begin{itemize}
 \item eine Liste von Turnierbäumen
 \item noch mehr (später)
 \item insert($x,k$)
    \begin{itemize}
    \item Lege neuen Turnierbaum für $(x,k)$ an, füge ihn zur Liste hinzu. \\
    insert($x, 15$)
    \end{itemize}
    $O(1)$-Zeit
 \item decrease\_key($y,k$): Annahme: Haben/Bekommen. Zeiger auf das Blatt, das $y$ speichert.
     \begin{itemize}
     \item decrease\_key($y,-1$)
     \item finde höchsten Knoten $v$, der Schlüssel von $y$ enthält
     \item Führe cut($v$) aus, falls $v$ nicht Wurzel
     \item Erniedrige Schlüssel von $y$ auf $k$
     \end{itemize}
     $O(1)$ Implementierung
 \item delete\_min(): Zusätzliche Information (können beide in insert/decrease\_key in $O(1)$ aktualisiert werden):
     \begin{itemize}
      \item Array $T[i]$: Liste aller Bäume mit Höhe $i$
      \item Array $n[i]$: \# Knoten mit Höhe $i$
      \item Invariante: $\forall i \geq 0{:}\ n[i-1] \leq \frac{3}{4} n[i]$
      \item \textbf{Fakt:} Invariante $\Rightarrow$ Jeder Knoten hat Höhe $O(\log n)$
      \item \textbf{Fakt:} insert und decrease\_key erhalten die Invariante.
     \end{itemize}
     \textbf{delete\_min}
     \begin{enumerate}
      \item Gehe alle Wurzeln durch, finde das Element $x$ mit minimalen Schlüssel
      \item Lösche den Pfad, der $x$ enthält. Lege ggf. neue Bäume an
      \begin{center}
     \psset{levelsep=0.7cm,treesep=0.5cm,nodesep=2pt}
      \begin{minipage}{6cm}
        \centering
        \pstree{\Tr{0}}{
            \pstree{\Tr{4}}{
                \pstree{\Tr{4}}{
                    \pstree{\Tr{4}}{
                        \Tr{4}
                    }
                }
                
            }
            \pstree{\Tr{0}}{
                \pstree{\Tr{2}}{
                    \pstree{\Tr{2}}{
                        \Tr{2}
                        \Tr{3}
                    }
                }
                \pstree{\Tr{0}}{
                    \pstree{\Tr{0}}{
                        \Tr{1}
                        \Tr{0}
                    }
                    \pstree{\Tr{7}}{
                        \Tr{7}
                    }
                }
            }
        }
        \Tr{14}
        \pstree{\Tr{5}}{
            \Tr{8}
            \Tr{5}
        }
        \Tr{15}
       \end{minipage}
       \hspace{0.2cm}
       $\Rightarrow$
       \hspace{0.2cm}
       \begin{minipage}{6cm}
        \centering
        \pstree{\Tr{4}}{
            \pstree{\Tr{4}}{
                \pstree{\Tr{4}}{
                    \Tr{4}
                }
            }
            
        }
        \pstree{\Tr{2}}{
            \pstree{\Tr{2}}{
                \Tr{2}
                \Tr{3}
            }
        }
        \pstree{\Tr{0}}{
            \pstree{\Tr{0}}{
                \Tr{1}
                \Tr{0}
            }
            \pstree{\Tr{7}}{
                \Tr{7}
            }
        }
        \Tr{14}
        \pstree{\Tr{5}}{
            \Tr{8}
            \Tr{5}
        }
        \Tr{15}
       \end{minipage}
     \end{center}
      \item Konsolidierung
          \begin{algorithmic}
           \FOR{$i := 0..O(\log n)$}
               \WHILE{$|T[i]| \geq 2$}
                   \STATE Lösche 2 Bäume $T_1,T_2$ aus $T[i]$
                   \STATE $T^* \gets \text{link}(T_1, T_2)$
                   \STATE $T[i+1] \gets T[i+1] ++ T^*$
               \ENDWHILE
           \ENDFOR
          \end{algorithmic}
          \begin{description}
           \psset{levelsep=0.7cm,treesep=0.5cm,nodesep=2pt}
           \item $T[0]:$ 
               \begin{equation*}
                \cancel{1} \cancel{14} 
                15
               \end{equation*}
           \item $T[1]:$ 
               \begin{equation*}
                \cancel{\pstree{\Tr{7}}{\Tr{7}}} 
                \cancel{\pstree{\Tr{5}}{\Tr{8}\Tr{5}}} 
                \cancel{\pstree{\Tr{1}}{\Tr{1} \Tr{14}}}
                {\color{red}\psset{linecolor=red} \pstree{\Tr{1}}{\Tr{1} \Tr{14}}}
               \end{equation*}
           \item $T[2]:$ 
               \begin{equation*}
                \cancel{\pstree{\Tr{2}}{\pstree{\Tr{2}}{\Tr{2}\Tr{4}}}}
                {\color{orange}\psset{linecolor=orange} \pstree{\Tr{5}}{\pstree{\Tr{7}}{\Tr{7}}\pstree{\Tr{5}}{\Tr{6}\Tr{5}}}}
               \end{equation*}
           \item $T[3]:$
               \begin{equation*}
                \cancel{\pstree{\Tr{4}}{\pstree{\Tr{4}}{\pstree{\Tr{4}}{\Tr{4}}}}} 
                {\color{green}\psset{linecolor=green} 
                    \pstree{\Tr{2}}{
                        \pstree{\Tr{4}}{
                            \pstree{\Tr{4}}{
                                \pstree{\Tr{4}}{
                                    \Tr{4}
                                }
                            }
                        }
                        \pstree{\Tr{2}}{
                            \pstree{\Tr{2}}{
                                \pstree{\Tr{2}}{
                                    \Tr{2}\Tr{3}
                                }
                            }
                            \pstree{\Tr{5}}{
                                \pstree{\Tr{7}}{
                                    \Tr{7}
                                }
                                \pstree{\Tr{5}}{
                                    \Tr{6}
                                    \Tr{5}
                                }
                            }
                        }
                    }
                }
               \end{equation*}
               \hrulefill
           \item Ergebnis:
               \begin{equation*}
                   15
                   \pstree{\Tr{1}}{\Tr{1} \Tr{14}}
                    \pstree{\Tr{2}}{
                        \pstree{\Tr{4}}{
                            \pstree{\Tr{4}}{
                                \pstree{\Tr{4}}{
                                    \Tr{4}
                                }
                            }
                        }
                        \pstree{\Tr{2}}{
                            \pstree{\Tr{2}}{
                                \pstree{\Tr{2}}{
                                    \Tr{2}\Tr{3}
                                }
                            }\pstree{\Tr{5}}{
                                \pstree{\Tr{7}}{
                                    \Tr{7}
                                }
                                \pstree{\Tr{5}}{
                                    \Tr{6}
                                    \Tr{5}
                                }
                            }
                        }
                    }
               \end{equation*}
          \end{description}
      \item Erdbeben: Falls Ebene $i$ existiert mit $n[i+1] > \frac{3}{4} n[i]$: 
          \begin{itemize}
          \item Sei $i$ niedrigste solcher Ebenen
          \item Lösche alle Knoten mit Höhe $> i$
          \end{itemize}
          $\rightarrow$ Erzeugt ggf. neue Bäume
     \end{enumerate}
\end{itemize}



\Lemma Sei $T_0 =$ \#Bäume vor delete\_min. Dann ist die Laufzeit von delete\_min $O(\log n + T_0+ \text{\#gelöschte Knoten})$
\Bew
\begin{enumerate}
 \item Laufzeit: $O(T_0)$
 \item Laufzeit: $O(\log n)$ wegen Invariante
 \item Laufzeit: $O(\log n + T_0)$\\
         Warum: Laufzeit $O(\log n + \#\text{links})$\\
         $\#\text{links}$ $\leq$ \#links $= O(\log n) + T_0$ denn jeder link zerstört einen Baum
 \item Laufzeit: $O(\text{\#gelöschte Knoten} + \log n)$\hfill$\square$
\end{enumerate}
\paragraph{Analyse:}
\begin{description}
 \item[zu zeigen: ] Jede Folge von $i$ inserts, $d$ decrease\_keys und $e$ delete\_mins hat Laufzeit $O(i + d + e \log n)$, mit $n = \max \#\text{Elemente in der Datenstruktur}$
 \item[Idee: ] Wenn delete\_min lange dauert, muss es vorher viele inserts und decrease\_keys gegeben haben.\\
         $\Rightarrow$ können Knoten umverteilen
 \item Wie formalisiert man das?
     \renewcommand{\theenumi}{\alph{enumi}}
     \begin{enumerate}
      \item Potentialmethode (s. Übung)
      \item Buchhaltermethode.
     \end{enumerate}
     \begin{itemize}
      \item 3 Konten:
         \begin{description}
          \item Knotenkonto: enhält 1 \euro\ pro Knoten
          \item Baumkonto: enhält 2 \euro\ pro Baum
          \item Einzelkinder-Konto: enthält 4 \euro\ pro Knoten mit nur 1 Kind.
         \end{description}
      \item Am Anfang sind alle Konten leer
      \item Wir zeigen:
          \begin{itemize}
          \item Bezahlen wir $O(1) \euro$ pro insert/decrease\_key und $O(\log n) \euro$ pro delete\_min
          \end{itemize}
      \item so können wir sicherstellen, dass
      \renewcommand{\theenumi}{\roman{enumi}}
     \begin{enumerate}
      \item die Konteninvariante immer gilt.
      \item die tatsächlichen Kosten können bezahlt werden.
     \end{enumerate}
     \begin{description}
     \item insert: tatsächliche Kosten $O(1)$
         \begin{itemize}
          \item Müssen 1 \euro\ ins Knotenkonto und 2 \euro\ ins Baumkonto zahlen
         \end{itemize}
         zu zahlende Kosten: $O(1) \euro$
     \item decrease\_key: tatsächliche Kosten $O(1)$
         \begin{itemize}
          \item Müssen $\leq 2 \euro$ ins Baumkonto und $\leq 4 \euro$ ins Einzelkinderkonto einzahlen
         \end{itemize}
         zu zahlende Kosten: $O(1) \euro$
     \item delete\_min: tatsächliche Kosten: $O(\log n + T_0 + \mathcal{D})$ mit $\mathcal{D} = $\#gelöschte Knoten\\
         Buchhaltung:
         \begin{itemize}
         \item Hebe $T_0 \euro$ von Baumkonto ab\\
             $\Rightarrow$ tatsächliche Kosten
         \item Überweise \#links \euro\ auf das Knotenkonto (vom Baumkonto + ggf $O(\log n)$ zusätliche \euro)\\
             $\Rightarrow \leq T_0 \euro$ im Baumkonto
         \item Zahle 2 \euro\ pro Baum nach Konsulidierung auf Baumkonto ein \\
             $\Rightarrow O(\log n) \euro$
         \item Hebe $\mathcal \euro$ im Knotenkonto ab\\
             $\Rightarrow$ tatsächliche Kosten
         \item Bei Erdbeben: Überweise $2n[i] \euro$ von Einzelkinderkonto auf Baumkonto
         \end{itemize}
         zu Zahlen $O(\log n) \euro$
     \end{description}
     \item noch zu zeigen Konteninvariante bleibt erhalten
         \begin{itemize}
          \item Knotenkonto $\checkmark$
          \item Baumkonto $\checkmark$
          \item Einzelkinderkonto:
              \Beh Wenn $n[i+1] > \frac{3}{4} n[i]$, so hat $n[i+1]$ mindestens $\frac{n[i]}{2}$ Knoten mit 1 Kind
              \Bew \begin{align*}
                    n[i+1] &= \underbrace{n_1}_{\text{Knoten mit 1 Kind}} + \underbrace{n_2}_{\text{Knoten mit 2 Kindern}} \\
                    n[i] &= n_1 + n_2 \\
                    \text{wissen } n[i+1] &\geq \frac{3}{4} n[i] \\
                    \Rightarrow n_1 + n_2 &\geq \frac{3}{4} (n_1 + 2 n_2) \\
                    \Rightarrow n_1 + n_2 &\geq \frac{3}{4} n_1 + \frac{3}{4} n_2 \Leftrightarrow n_1 \geq 2n_2 \\
                    \Rightarrow 2n_1 = n_1 + n_2 \geq n_1 + 2 n_2 = n[i]
                   \end{align*}\hfill$\square$

          \end{itemize}

     \end{itemize}
\end{description}

\section{Graphenalgorithmen}
\begin{description}
\item[Gegeben:] Graph $G = (V,E)$, Zusatzinformationen
\item[Ziel:] Berechne etwas Interessantes über $G$
\item[Beispiele:]
    \begin{itemize}
    \item Durchsuchen von Graphen: BFS, DFS
    \item kürzeste Wege: Dijkstra, Bellman-Ford, Floyd-Warshall, $A^*$, ...
    \item minimale aufgespannte Bäume: Kruskal, Prim-Janik, Borůvka, ...
    \item topologisches Sortieren
    \end{itemize}
\end{description}

\subsection[Flussproblem]{Flussproblem}
\Bsp UdSSR: Problem Transportiere Zement von Nowosibirsk nach Moskau
\begin{center}
    \vspace{1em}
    \begin{psmatrix}[mnode=dot,rowsep=1cm]
          & O & T \\
        N & A & x & M \\
          & J & y
    \end{psmatrix}
    \vspace{1em}
    \nput{90}{1,2}{O}
    \nput{90}{1,3}{T}
    \nput{180}{2,1}{N}
    \nput{45}{2,2}{A}
    \nput{0}{2,4}{M}
    \nput{-90}{3,2}{J}
    \ncarc{->}{1,2}{1,3}\naput{5}
    \ncline{->}{1,2}{2,2}\naput{4}
    \ncarc{->}{1,3}{1,2}\naput{10}
    \ncline{->}{1,3}{2,3}\naput{15}
    \ncline{->}{1,3}{2,4}\naput{10}
    \ncline{->}{2,1}{1,2}\naput{10}
    \ncline{->}{2,1}{2,2}\naput{5}
    \ncline{->}{2,1}{3,2}\nbput{15}
    \ncline{->}{2,2}{2,3}\naput{6}
    \ncline{->}{2,2}{3,2}\naput{6}
    \ncline{->}{2,3}{2,4}\naput{10}
    \ncline{->}{3,3}{2,2}\nbput{6}
    \ncline{->}{3,3}{2,3}\nbput{15}
    \ncline{->}{3,3}{2,4}\nbput{10}
    \ncline{->}{3,2}{3,3}\nbput{30}
\end{center}
\begin{itemize}
\item   Strecken haben \emph{Kapazitäten}
\item   wollen so viel Zement wie möglich von $N$ nach $M$\\
        $\Rightarrow$ Flussproblem
\end{itemize}
USA (Klassenfeind)
\begin{itemize}
\item   Verhindere Transport, schneide $N$ von $M$ ab
\item   Je breiter die Strecke, desto teurer die Unterbrechung\\
    $\Rightarrow$ MIN-$s$-$t$-cut
\end{itemize}
Formalisierung (Flussnetzwerk):
\begin{description}
\item[Gegeben:]
    \begin{itemize}
    \item   Graph $G = (V, E)$ gerichtet
    \item   $s,t \in V, s \neq t$
            \begin{description}
            \item   $s$: Quelle
            \item   $t$: Abfluss/Senke
            \end{description}
    \item   $C{:}\ V \times V \to \mathbb{R}^+_0$, mit $c(u,v) = 0$, wenn $(u,v) \in E$.
    \item   Kapazitäten
    \end{itemize}
\item[Fluss:] Funktion $f{:} V \times V \to \mathbb{R}$, so dass
    \renewcommand{\theenumi}{\roman{enumi}}
    \begin{enumerate}
    \item   $\forall u,v \in V{:}\ f(u,v) \leq c(u,v)$ (Kapazitätseigenschaft)
    \item   $\forall u,v \in V{:}\ f(u,v) = -f(v,u)$ (Antisymmetrie)
    \item   $\forall u \in V \setminus \{s,t\}{:}\ \sum\limits_{v \in V} f(u,v) = 0$
    \end{enumerate}
\item[Bemerkungen:]
    \begin{itemize}
    \item   $f \equiv 0$ ist Fluss
    \item   (i) und (ii) $\Rightarrow f(u,v) = 0$, wenn $(u,v) \in E$ und $(v,u) \in E$
    \item   (ii) $\Rightarrow f(u,u) = 0, \forall u \in V$  
    \end{itemize}
\end{description}
Wert des Flusses $f$
\[|f| := \sum\limits_{v \in V} f(s,v)\]
\textbf{Ziel:} Finde $f$ mit $|f|$ maximal.
\Lemma $|f| = \sum\limits_{v \in V} f(v,t)$
\Bew Betrachte 
\[\sum\limits_{(u,v) \in V \times V} f(u,v) = \sum\limits_{\{u,v\} \in \binom{V}{2}} (f(u,v) - f(v,u)) = 0\]
Also:
\begin{align*}
0 &= \sum\limits_{(u,v) \in V \times V} f(u,v) = \sum\limits_{v \in V} f(s,v) +  \sum\limits_{v \in V} f(t,v)
    + \ovalnode[linecolor=red]{lst}{\sum\limits_{\begin{subarray}(u,v) \in V \times V\\ u \neq s,t\end{subarray}} f(t,v)}\\
  &= \sum\limits_{(u,v) \in V \times V} f(u,v) = \sum\limits_{v \in V} f(s,v) +  \sum\limits_{v \in V} f(t,v)
    + \underbrace{\sum\limits_{u \neq s,t} \sum\limits_{v \in V} f(u,v)}_{= 0} \\
  &= \sum\limits_{(u,v) \in V \times V} f(u,v) = \sum\limits_{v \in V} f(s,v) +  \sum\limits_{v \in V} f(t,v)
\end{align*} \hfill $\square$
\begin{description}
\item[Gegeben:]
    \begin{itemize}
    \item   $G = (V, E)$ Flussnetzwerk
    \item   $s$-$t$-Schnitt $C = (A,B)$
            \[A,B \subseteq V{;}\ A \cap B = \emptyset; A \cup B = V, s \in A, t \in B\]
    \end{itemize}
    Kapazität von $C$:
    \[K(C) := \sum\limits_{u,v \in A \times B} c(u,v)\]
\item[Gesucht:]
    Schnitt $C$ mit minimaler Kapazität.
\end{description}
\Lemma (schwache Dualität)
    Seien $C$ ein Schnitt, $f$ ein Fluss, dann: $|f| \leq K(C)$
\Bew Betrachte
    \[\underbrace{\sum\limits_{(u,v) \in A \times B} f(u,v)}_{\text{Totaler Ausfluss von $A$.}} \leq \sum\limits_{(u,v) \in A \times B} c(u,v) = K(C).\]
\begin{itemize}
\item   noch zu zeigen: $|f| \leq $ totaler Ausfluss aus $A$
\item   Abkürzung: Ersetze $B$ durch Knoten $t'$.
        \[|f| = \sum\limits_{v \in V} f(v,t') \leq \text{Ausfluss aus $A$}\]
\end{itemize}

    
\begin{align*}
    |f| &= \sum\limits_{v \in V} f(s,v) \\
        &= \sum\limits_{v \in V} f(s,v) + \sum\limits_{u \in A \setminus \{s\}} \underbrace{\sum\limits_{v \in V} f(u,v)}_{= 0} \\
        &= \sum\limits_{u \in A}\sum\limits_{v \in V} f(u,v) \\
        &= \underbrace{\sum\limits_{u \in A}\sum\limits_{v \in A} f(u,v)}_{= 0} + \sum\limits_{u \in A}\sum\limits_{v \in B} f(u,v) \\
        &= \sum\limits_{u \in A}\sum\limits_{v \in B} f(u,v) \leq \sum\limits_{u \in A}\sum\limits_{v \in B} c(u,v) = K(C)
\end{align*}
\begin{itemize}
\item   Merke:
\[
    |f| = \sum\limits_{u \in A} \sum\limits_{v \in B} f(u,v)
\]
\item   Folgerung: Falls $\exists$ Schnitt $C$ und ein Fluss $F$ mit $K(C) \leq |F|$. Dann $K(C) = |F|$ und $C$ und $F$ sind optional.
\item   Wie kann man $F$ optimieren?
        \begin{description}
        \item[Idee 1 (Grierig):]
            \begin{itemize}
            \item   Starten mit $F = 0$
            \item   Solange Weg $P$ existiert von $s$ nach $t$ mit freier Kapazität, schicke Fluss entlang von $P$.
            \end{itemize}
            \Bsp
            \begin{center}
                \begin{psmatrix}[mnode=circle,rowsep=0.5]
                        & \ \\
                    $s$ &   & $t$ \\
                        & \ 
                \end{psmatrix}
                \psset{arrows=->}
                \ncline{1,2}{2,3}\naput{{\color{red}0/}10}
                \ncline{1,2}{3,2}\naput{{\color{red}20/}30}
                \ncline{2,1}{1,2}\naput{{\color{red}20/}20}
                \ncline{2,1}{3,2}\nbput{{\color{red}0/}10}
                \ncline{3,2}{2,3}\nbput{{\color{red}20/}20}
            \end{center}
            Nicht optimal!
            \begin{center}
                \begin{psmatrix}[mnode=circle,rowsep=0.5]
                        & \ \\
                    $s$ &   & $t$ \\
                        & \ 
                \end{psmatrix}
                \psset{arrows=->}
                \ncline{1,2}{2,3}\naput{{\color{red}10/}10}
                \ncline{1,2}{3,2}\naput{{\color{red}20/}30}
                \ncline{2,1}{1,2}\naput{{\color{red}20/}20}
                \ncline{2,1}{3,2}\nbput{{\color{red}10/}10}
                \ncline{3,2}{2,3}\nbput{{\color{red}20/}20}
            \end{center}
            Wert 30: Optimal, denn $exists$Schnitt mit Kapazität 30.
        \item[Idee 2 (Ford-Fulkerson)]
            Erlaube es, Fluss zurück zu pumpen.\\
            Sei $G = (V, E)$ Flussnetzwerk. $Ff$ Fluss für $u,v \in V \times V$, definiere Restkapazität (Schlupf) $r(u,v) = c(u,v) - f(u,v)$, $r(u,v) \geq 0$\\
            Restnetzwerk (Schlupfnetzwerk): $R_{G,F} = (V,E')$, mit $G = (V, E)$ und
            \[E' = \{(u,v) \in V \times V\ |\ r_F(u,v) > 0\}\]
            \Bsp
            \begin{center}
                \begin{psmatrix}[mnode=circle,rowsep=0.5]
                        & \ \\
                    $s$ &   & $t$ \\
                        & \ 
                \end{psmatrix}
                \psset{arrows=->}
                \ncarc{1,2}{2,1}\nbput{20}
                \ncarc{1,2}{3,2}\nbput{20}
                \ncarc{3,2}{1,2}\nbput{10}
                \ncarc{2,1}{3,2}\nbput{10}
                \ncarc{2,3}{3,2}\naput{20}
                \ncarc{1,2}{2,3}\naput{10}
            \end{center}
            bzw.
            \begin{center}
                \begin{psmatrix}[mnode=circle,rowsep=0.5]
                        & \ \\
                    $s$ &   & $t$ \\
                        & \ 
                \end{psmatrix}
                \psset{arrows=->}
                \ncarc{1,2}{2,1}\nbput{20}
                \ncarc{1,2}{3,2}\nbput{10}
                \ncarc{3,2}{1,2}\nbput{20}
                \ncarc{2,1}{3,2}\nbput{10}
                \ncarc{2,3}{3,2}\naput{20}
                \ncarc{2,3}{1,2}\nbput{10}
            \end{center}
            \item Beobachtung $\exists$ in $R_{C,F}$ ein Pfad um $s$ nach $t$, dann können wir $|F|$ erhöhen, und zwar um die minimale Restkapazität entlang des Pfades $r_P$. 
            \item $P$ heißt zunehmender (aufgmentierender) Graph.
            \item Ford-Fulkerson
            \begin{algorithmic}
            \STATE Setze $F \equiv 0$
            \REPEAT
                \STATE Konstruiere $R_{G,F}$
                \IF{$\exists$zunehmender Pfad $P$ in $R_{G,F}$}
                    \STATE schicke $r_p$ Einheiten Fluss entlang $P$.
                \ENDIF
            \UNTIL{kein zulässiger Pfad gefunden.}
            \end{algorithmic}
            \Satz $\exists$ zunehmender Pfad in $R_{G,F} \Leftrightarrow |F|$ nicht maximal ist
            \Bew
            \begin{description}
            \item $\Rightarrow$ Wenn zunehmender Pfad nicht existiert, so können wir $|F|$ um $r_p > 0$ erhöhen. Also ist $|F|$ nicht maximal.
            \item $\Leftarrow$ Angenommen $\exists$Kennzeichnenden Pfad in $R_{G,F}$. Definiere: 
                \begin{align*}
                    A &:= \{v \in V\ |\ \text{$s$ kann $v$ in $R_{G,F}$ erreichen}\}
                    B &:= V\setminus A
                \end{align*}
                Annahme $\Rightarrow C= (A,B)$ ist $s$-$t$-Schnitt. Es gilt $\forall (u,v) \in A \times B$: $c(u,v) = 0 \Rightarrow c(u,v) = f(u,v)$
                \[K(C) = \sum\limits_{(u,v) \in A \times B} c(u,v) = \sum\limits_{(u,v) \in A \times B} f(u,v) = |F| \Rightarrow \text{ist optimal}\]
                Folgerung: Wenn Ford-Fulkerson terminiert, so ist $|F|$ optimal
                \begin{align*}
                    \max\limits_{F} |F| &= \min\limits_{C} K(C) \tag{starke Dualität}
                \end{align*}
                Min-Schnitt-Max-Fluss-Satz.
            \end{description}
            \paragraph*{Laufzeit:} Eine Iteration benötigt $O(|E|)$. Wieviele Iterationen?
            \begin{center}
                \begin{psmatrix}[mnode=circle,rowsep=0.5]
                        & \ \\
                    $s$ &   & $t$ \\
                        & \ 
                \end{psmatrix}
                \psset{arrows=->}
                \ncline{1,2}{2,3}\naput{$c$}
                \ncline{1,2}{3,2}\naput{1}
                \ncline{2,1}{1,2}\naput{$c$}
                \ncline{2,1}{3,2}\nbput{$c$}
                \ncline{3,2}{2,3}\nbput{$c$}
            \end{center}
            Wenn Ford-Fulkerson sich blöd anstellt: $O(c)$ Iterationen.\\
            Kann exponentiell in der Eingabelänge sein (brauchen $\Theta(\log c)$) Bits, um $c$ Kodieren $\Rightarrow$ Pseudopolynomiell (Problem)\\
            Reperatur: Wähle $P$ immer mit minimaler $\#$Kanten durch BFS (Edmonds-Karp)
        \end{description}
\end{itemize}



\Satz Edmunds-Karp benötigt $O(|V| \cdot |E|)$ Iterationen
\Bew Sei $R_{G,F}$ Restgraph
\Defi \emph{BFS-Schicht}: Alle Knoten mit Abstand $i$ von $s$.
\begin{itemize}
\item   Zunehmender Weg hat nur Kanten von BFS-Schicht zu $i-1$
\item   Bei Flussvergrößerung:
    \begin{itemize}
    \item   $\geq 1$ Kanten verschwinden aus $R_{G,F}$ (Sättigung)
    \item   neue Kanten können im $R_{G,F}$ auftauchen, aber nur von Schicht $i+1$ zu $i$
    \end{itemize}
\end{itemize}

\paragraph*{Behauptung 1} Durch Flussvergrößerung kann sich die BFS-Schicht eines Knoten nur erhöhen
\Bew Sei $v$ Knoten, dessen BFS-Schicht kleiner wird.
\begin{itemize}
\item   Nehmen an, dass $v$ einziger Knoten auf $Q$ ist, dessen BFS-Schicht kleiner geworden ist.
\item   Sei Vorgänger von $v$ auf $Q$
\item   Die BFS-Schicht von $u$ ist nicht kleiner geworden, also muss die Kante $u_v$ neu sein.
\item   Das kann aber nicht sein, da neue Kanten von BFS-Schicht $i+1$ zu $i$, also die BFS-Schicht von $v$ kleiner als die von $u$ gewesen wäre. 
\end{itemize}
\paragraph*{Behauptung 2} Eine Kante kann höchstens $O(|V|)$ mal gesättigt werden.
\Bew Sei $(u,v)$ Kante.x
\begin{itemize}
 \item Sei $v$ Knoten, dessen BFS-Schicht kleiner wird.
 \item $(u,v)$ kann nur wieder in $R_{G,F}$ auftauchen, wenn entlang $(v,u)$ eine Flussvergrößerung stattfindet.
 \item Das heißt: BFS-Schicht von $u$ muss sich um mindestens $2$ erhöht haben. \\
     $\Rightarrow$ maximal $O(|V|)$ mal \hfill $\square$
\end{itemize}
Es folgt: Können höchstens $O(|V| \cdot |E|)$ Flussvergrößerungen durchführen \hfill $\square$\\
Laufzeit von Edmund-Karp ist $O(|V| \cdot |E|^2)$ (stark polynomiell)

\section{Lineares Programmieren}
Schnaps-Herstellung:
\begin{description}
 \item Zutaten:
     \begin{itemize}
      \item Alkohol
      \item Wasser
      \item Zucker
      \item Geheimzutat $Z_1$
      \item Geheimzutat $Z_2$
     \end{itemize}
 \item 2 Schnapsarten: $S_1$, $S_2$
 \item Gewinne:
     \begin{center}
     \begin{tabular}{ll}
      $S_1$ & $7 \euro/\text{l}$ \\
      $S_2$ & $3 \euro/\text{l}$
     \end{tabular}
     \end{center}
 \item Rezept:
     \begin{center}
        \begin{tabular}{l|ll|r}
                    & $S_1$    & $S_2$    & Vorrat (in l)\\\hline
            Alkohol & $0{,}4$  & $0{,}25$ &  30\\
            Zucker  & $0{,}2$  & $0{,}3$  &  25\\
            $Z_1$   & $0{,}15$ & $0$      &  10\\
            $Z_2$   & $0$      & $0{,}5$  &  20\\
            Wasser  & $0{,}25$ & $0{,}15$ & 100\\\hline
        \end{tabular}
     \end{center}
 \item Sei
     \begin{itemize}
      \item $x_1$ = l für Schnaps 1
      \item $x_2$ = l für Schnaps 2
     \end{itemize}
 \item[Ziel:] maximiere $7x_1 + 3x_2$, wobei
        \begin{align*}
            0{,}4 x_1 + 0{,}25 x_2 &\leq 30 \tag{1} \\
            0{,}2 x_1 + 0{,}3 x_2 &\leq 25 \tag{2} \\
            0{,}15 x_1 &\leq 10 \tag{3}\\
            0{,}3 x_2 &\leq 20  \tag{4}\\
            0{,}25 x_1 + 0{,}15 x_2 &\leq 100 \tag{5} \\
            x_1, x_2 &\geq 0
        \end{align*}
\Defi Ein Punkt $(x_1,x_2)$ heißt \emph{zulässige Lösung}, wenn er alle Nebenbedingungen erfüllt.
 \item Wie sieht die Menge der Zulässigen Lösungen.
 \item Eine Nebenbedingung $a_1 x_1 + a_2 x_2 \leq b$ definiert eine \emph{Halbebene}
 \begin{center}
  \begin{pspicture}(0,0)(5,2)
   \psline[linestyle=none,fillstyle=solid,fillcolor=red](0,0)(0,2)(5,1)(5,0)
   \psline(0,2)(5,1)
   \rput(4,1.6){$a_1 x_1 + a_2 x_2 = b$}
  \end{pspicture}
 \end{center}
Menge der zulässigen Lösungen ist ein Schnitt von Halbebenen.
\Geg Ein Wert $\alpha$, was ist die Menge aller Punkte, für die die Zielfunktion Wert $\alpha$ annimmt?
\[7x_1 + 3x_2 = \alpha\]
Verschiebe Zielfunktionsgerade parallel, bis sie gerade noc den zulässigen Bereich erhöht \\
$\Rightarrow$ Optimale Lösung ist Schnitt mit $g_1$ und $g_3$
\item $\dfrac{200}{3}$ l Schnaps 1 und $\dfrac{40}{3}$ l Schnaps 2.
\end{description}

\begin{itemize}
 \item Lineare Programmierung hat die Form
     \[\max c_1x_1 + c_2x_2 + ... + c_nx_n\]
     wobei
     \begin{align*}
      a_{11}x_1 + a_{12}x_2 + ... + a_{1n}x_n &\leq b_1 \\
      a_{21}x_1 + a_{22}x_2 + ... + a_{2n}x_n &\leq b_2 \\
      \vdots\\
      a_{m1}x_1 + a_{m2}x_2 + ... + a_{mn}x_n &\leq b_m \\
     \end{align*}
 \item Kompakte Schreibweise:
     \[c^T \cdot x, \text{wobei}\ A \cdot x \leq b, c \in \mathbb{R}^n, A \in \mathbb{R}^{n \times m}, b \in \mathbb{R}^n\]
 \item Fakt: lineare Programme kann man lösen (in Theorie und Praxis)
\end{itemize}
Varianten:
\begin{itemize}
 \item $\min$ statt $\max$: Verwende $-c$ statt $c$
 \item Nebenbedingungen der Form $a - x = b$: Schriebe als $a\cdot x \leq b$ und $-ax \leq -b$
\end{itemize}

\begin{description}
 \item Zulässige Lösung: $x \in \mathbb{R}^n, A \cdot x \leq b$
 \item Zulässiger Bereich: Menge alle zulässigen Lösungen $\Rightarrow$ Schnitt von Halbräumen (konvexes Polyeder)
\end{description}
Fragen
\begin{description}
 \item Existiert eine zulässige Lösung?
 \item Existiert eine optimale Lösung?
 \item Wie findet man eine optimale Lösung?
\end{description}

\Defi Sei $i_1 \leq i_2 \leq ... \leq i_n$, so dass
\[\operatorname{Rg}\begin{pmatrix}
                    a_{i_11} & \hdots & a_{i_1m} \\
                    a_{i_21} & \hdots & a_{i_2m} \\
                    \vdots & \ddots & \vdots \\
                    a_{i_n1} & \hdots & a_{i_nm}
                   \end{pmatrix} = n
\]
und so dass
\[ \tilde x = \begin{pmatrix}
                    a_{i_11} & \hdots & a_{i_1m} \\
                    a_{i_21} & \hdots & a_{i_2m} \\
                    \vdots & \ddots & \vdots \\
                    a_{i_n1} & \hdots & a_{i_nm}
                   \end{pmatrix}^{-1} \cdot \begin{pmatrix}
                    b_{i_11} \\
                    b_{i_21} \\
                    \vdots \\
                    b_{i_n1} 
                   \end{pmatrix}\]
zulässige Lösung ist. Dann heißt $\tilde x$ \emph{Ecke} des zulässigen Bereichs \\
$\Rightarrow$ Eine Ecke ist Schnitt von $n$ Hyperebenen.
\begin{itemize}
 \item Man kann zeigen: Wenn $\operatorname{Rg}(A) = n$ und wenn eine optimale Lösung existiert, dann existiert optimale Lösung, die Ecke ist.
 \item Wie findet man optimale Ecke?
 \item Idee: Beginne bei irgendeiner Ecke.
     \begin{itemize}
      \item Ist die Ecke optimal? $\rightarrow$ fertig
      \item Wenn nicht $\rightarrow$ Gehe zu besseren Nachbarecke. Wiederhole
     \end{itemize}
     $\Rightarrow$ Simplex-Algorithmus
\end{itemize}

\Bsp
\begin{align*}
 \frac{2}{5} x_1 + \frac{1}{4} x_2 &\leq \overset{{\color{green}x_1 \leq 75}}{30} \tag{1} \\
 \frac{1}{5} x_1 + \frac{3}{10} x_2 &\leq \overset{{\color{green}x_1 \leq 125}}{25} \tag{2} \\
 {\color{green} \min \rightarrow\ }\frac{3}{20} x_1 &\leq \overset{{\color{green}x_1 \leq 66{,}\overline{6}}}{10} \tag{3} \\
 \frac{3}{10} x_2 &\leq \overset{{\color{green}x_1 \leq \infty}}{20} \tag{4} \\
 \frac{1}{4} x_1 + \frac{3}{20} x_2 &\leq \overset{{\color{green}x_1 \leq 400}}{100} \tag{5} \\
 x_1 &\geq \overset{{\color{green}x_1 \leq \infty}}{0} \tag{6} \\
 x_2 &\geq \overset{{\color{green}x_1 \leq \infty}}{0} \tag{7} \\
 \max \left(7 x_1 + 3 x_2\right) \tag*{Zielfunktion} 
\end{align*}
\begin{center}
 \psset{unit=0.5,ticks=none,labels=none}
 \begin{pspicture}(0,-2)(6,6)
  \pspolygon[linestyle=none,fillstyle=hlines,hatchcolor=red](0,0)(0,3)(2,3)(3,2.5)(4,1)(4,0)
  \psline{*-*}(0,0)(0,3)\uput[180](0,1.5){$g_6$}
  \psline{*-*}(0,3)(2,3)\uput[90](1,3){$g_4$}
  \psline{*-*}(2,3)(3,2.5)\uput[80](2.5,2.75){$g_2$}
  \psline{*-*}(3,2.5)(4,1)\uput[20](3.5,1.75){$g_1$}
  \psline{*-*}(4,1)(4,0)\uput[0](4,0.5){$g_3$}
  \psline{*-*}(4,0)(0,0)\uput[-90](2,0){$g_7$}
  \psaxes{->}(0,0)(0,0)(6,6)
  \psline[linecolor=blue](-1,2)(1,-2)
  \psline[linecolor=blue]{->}(0,0)(.5,.25)
 \end{pspicture}
\end{center}
\begin{itemize}
 \item Startecke $\{(6),(7)\}$
 \item Ist Ecke optimal? Nein, Koeffizienten von $x_1$ und $x_2$ in Zielfunktion sind $> 0$\\
     $\rightarrow$ Erhöhe $x_1$. Wie weit? Überprüfe andere Nebenbedingungen
 \item (3) wird als erstes fest
 \item Neue Ecke: $\{(3),(7)\}$
 \item Wie können wir feststellen, ob neue Ecke optimal ist? Ändere Koordinatensystem, so dass neue Ecke die Form $y_1 \geq 0, y_2 \geq 0$ hat.\\
 Also:
 \begin{align*}
  y_1 &:= 10 - \frac{3}{20} x_1 & \mapsto x_1 &= \frac{200}{30} - \frac{20}{3} y_1 \\
  y_2 &:= x_2 & \mapsto x_2 &= y_2
 \end{align*}
 Erhalten:
\begin{align*}
 {\color{green} \min \rightarrow\ }-\frac{8}{3} y_1 + \frac{1}{4} y_2 &\leq \frac{10}{3} {\color{green}\ \frac{40}{3}} \tag{1} \\
 -\frac{4}{3} y_1 + \frac{3}{10} y_2 &\leq \frac{35}{3} {\color{green}\ \frac{350}{9}} \tag{2} \\
 y_1&\geq 0 {\color{green}\ \infty} \tag{3} \\
 \frac{3}{10} y_2 &\leq 20 {\color{green}\ \frac{200}{9}} \tag{4} \\
 -\frac{5}{3} y_1 + \frac{3}{20} y_2 &\leq \frac{250}{3} {\color{green}\ \frac{5000}{9}} \tag{5} \\
 x_1 &\geq 0 {\color{green}\ \infty} \tag{6} \\
 x_2 &\geq 0 {\color{green}\ \infty} \tag{7} \\
 \max \left(\frac{1400}{3} - \frac{140}{3} y_1 + 3 y_2\right) \tag*{Zielfunktion} 
\end{align*}
\item nicht optimal! Können $y_2$ erhöhen! Wie weit?
\item Neue Ecke: $\{(3),(1)\}$
\item Koordinatentransformation $z_1 := y_1 \Leftrightarrow y_1 = z_1$
    \begin{align*}
     z_1 &:= y_1 \\
     z_2 &:= \frac{10}{3} + \frac{8}{3} y_1 - \frac{1}{4} y_2 \ \Leftrightarrow y_2 = \frac{40}{3} + \frac{32}{3} z_1 + 4 z_2
    \end{align*}
    Erhalte:
\begin{align*}
 z_2 &\leq 0 \tag{1} \\
 \frac{28}{15} z_1 + \frac{6}{5} z_2 &\leq \frac{23}{3}  \tag{2} \\
 z_1&\geq 0 \tag{3} \\
 \frac{16}{10} z_1 - \frac{6}{5} z_2 &\leq 16 \tag{4} \\
 -\frac{1}{15} z_1 + \frac{3}{5} z_2 &\leq \frac{244}{3} \tag{5} \\
 z_1 &\geq 10 \tag{6} \\
 - \frac{32}{3} z_1 + 4 z_2 &\geq \frac{40}{3} \tag{7} \\
 \max \left(\frac{1520}{3} - \frac{41}{3} z_1 + 12 z_2\right) \tag*{Zielfunktion} 
\end{align*}
\end{itemize}
\paragraph*{Simplex}
\begin{itemize}
 \item Wähle Startecke, setze Koordinatensystem in Startecke (Nebenbedingungen $x_1 \geq 0, x_2 \geq 0, ..., x_n \geq 0$ für Startecke)
 \item Wenn alle Koeffizienten der Zielfunktion $\leq 0$ sind $\rightarrow$ fertig (Ecke optional)
 \item Wenn nicht: wähle Variable $x_i$ mit positiven Koeffizienten in der Zielfunktion
     \begin{itemize}
      \item Erhöhe $x_i$ so weit wie möglich. Finde Nebenbedingung, die als erstes fest wird
      \item Ersetze Nebenbedingung $x_i \geq 0$ durch neue Nebenbedingung in der Ecke, transformiere Koordinaten.
      \item springe zum zweiten Punkt
     \end{itemize}
\end{itemize}
Wenn Simplex terminiert, so haben wir eine optimale Lösung.



\paragraph*{Terminiert Simplex?}
\begin{itemize}
 \item Es gibt nut endlich viele Ecken $\left(\leq \dbinom{m}{n}, \text{mit $m$: \#Nebenbedingung, $n$: \#Variablen}\right)$
 \item "`In jedem Schritt erhöht sich die Zielfunktion"' 
 \begin{itemize}
  \item Wirklich? 
  \item Nein! Problem: degeneracies (spezielle Lage): mehr als $n$ Bedingungen/Hyperebenen schneiden sich in einer Ecke
  \item Kommen beim Pivotschritt nicht vorwärts (Stalling/Leerlauf)
  \item Problem bei ungeschickter Wahl der neuen Hyperebene kann es zu endlosschleifen kommen (Cycling)
  \item Aber: es gibt Pivot-Regeln (z. B. Bland-Regel), die Cycling verhindern
 \end{itemize}
\end{itemize}

\paragraph*{Laufzeit}
\begin{itemize}
 \item Für viele bekannte Pivot-Regeln existieren, die $2^n$ Schritte benötigen (Klee-Minty-Würfel)
 \item Frage: $\exists$Pivot-Regel, die polynomielle Laufzeit garanziert? Unbekannt
 \item Frage: Ist das prinzipiell möglich? Unbekannt (Hirsch-Vermutung)
 \item In der Praxis hat Simplex sehr gute Laufzeit\\
     Warum? Smoothes Analysis (Spielman-Teng)
     \begin{description}
      \item Average case analyse von Simplex 
      \item Wackeln liefert erwartete Polynomialzeit
      \item worst-case Eingaben sind selten und dünn gesät
     \end{description}
 \item Geht LP in Polynomialzeit? Ja! 1979 (Khachian): Ellipsoid-Algorithmus
     \begin{itemize}
      \item pollynomiell in \#Ungleichungen, \#Variablen, \#Bits für die Zahlen
      \item Karmarkar: innere Punkt-Methode
     \end{itemize}
 \item Ist LP stark polynomiell lösbar? Unbekannt
\end{itemize}

\paragraph*{Woher kommt die Startecke?}
Durch ein LP!
\begin{itemize}
 \item Schreibe LP als
end{} \[(*) Ax \leq b \tag{ersetze $x_i$ durch $x_i^+ - x_i^-, x_i^+, x_i^- \geq 0$}\]
 \item Betrachte LP: $\min y_1 + y_2 + ... + y_m$, wobei
 \[Ax - \begin{pmatrix} 1 & 0 & \hdots & 0 & 0 \\ 0 & 1 & \hdots & 0 & 0 \\ \vdots & \vdots & \ddots & \vdots & \vdots \\ 0 & 0 & \hdots & 1 & 0 \\ 0 & 0 & \hdots & 0 & 1 \end{pmatrix} \cdot \begin{pmatrix}y_1 \\ y_2 \\ \vdots \\ y_m\end{pmatrix} \leq n, x \geq 0, y_i \geq 0 \tag{**}\]
 \item Startecke: $x \equiv 0; y_i = \max\{0, -b_i\}$.
 \item Löse (**)
     \begin{align*}
      \text{OPT}_{**} > 0 & \Rightarrow \text{$(x)$ ist zuverlässig} \\
      \text{OPT}_{**} = 0 & \Rightarrow \text{nimm optimale Ecke von (**) ohne $y$-Koordinaten als Startecke (*).}
     \end{align*}
\end{itemize}

\subsection{Dualität}
Betrachte folgendes LP $\max 7x_1 + x_2 + 5x_3$, wobei
\begin{align*}
 8x_1 + 2x_2 + 6x_3 &\leq 10 \tag{1} \\
 -x_1 - x_3 &\leq -1 \tag{2} \\
 x_1, x_2, x_3 &\geq 0
\end{align*}
Was können wir über $\text{OPT}_\approx$ sagen?
\paragraph*{Beobachtung:} $\text{OPT}_\approx \leq 10$
\[\underbrace{7x_1 + x_2 + 5x_3}_{\text{Zf.}} \leq 8x_1 + 2x_2 + 6x_3 \overset{(1)}{\leq} 10\]
Besser:
\begin{align*}
 \underbrace{7x_1 + x_2 + 5x_3}_{\text{Zf.}} &\leq (8x_1 + 2x_2 + 6x_3) + (-x_1 - x_3) \\
         &\overset{(1),(2)}{\leq} 10 - 1 \leq 9
\end{align*}
Geht es noch besser? Idee: Multipliziere Nebenbedingung mit Faktoren $\geq 0$. Suche bestmögliche obere Schranke für Zf.
\begin{align*}
 7x_1 - x_2 + 5x_3 &\leq y_1 \cdot \underbrace{(8x_1 + 2x_2 + 6x_3)}_{(1)} + y_2\underbrace{(-x_1 - x_3)}_{(2)}
     &= (8y_1 + y_2)x_1 + 2y_1x_2 + (6y_1 - y_2) x_3 \\
     &\leq x_1 \cdot 10 + y_2 (-1)
\end{align*}
Wollen: $\min 19y_1 - y_2$ wobei
\begin{align*}
 8y_1 - y_2 \geq 7 \\
 2y_1 \geq 1 \\
 6y_1 - y_2 &\geq 5 \\
 y_1, y_2, y_3 &\geq 0
\end{align*}
\Defi Sei $\begin{matrix}\max c^T x \\ Ax \leq b \\ x \geq 0\end{matrix} (*)$ lineqres Programm. \\
Dann heißt: $\min b^Ty$, wobei $A_Ty \geq c, y \geq 0$ orales LP zu (*).
\Lemma Es gilt immer $OPT_P \leq OPT_D$


\section{Fingerprinting}
\section{untere Schranken}
\chapter{Paradigmen}
\section{Worst-Case Analyse}
\section{Average-Case Analyse}
\section{Probabilistische Algorithmen}
\section{Amortisierte Analyse}
\section{Online-Algorithmen und kompetitive Analyse}
\section{Approximationsalgorithmen}
\section{Map-Reduce ????}
\chapter{Algorithmen und Datenstrukturen}
\section{Suchen, Sortieren und Auswählen}
%\section{Viterbi-Algorithmus}
%\section{TSP}
%\section{Datenstrukturen: Splaybäume, Erdbebenheaps, Disjoint Set Union, Bloom Filter}
%\section{Graphenalgorithmen: Flüsse, Paarungen, Schnitte, minimale aufspannende Bäume, kürzeste Wege}
\section{Simplex Algorithmus mit Anwendungen, Ellipsoid Algorithmus}
\section{Ski-Rental, Paging}
\chapter{Komplexität}
\section{Klassenzoo: LOGSPACE, P, NP, coNP, PSPACE, etc}
\section{NP-Vollständigkeit, Satz von Cook, Reduktionskonzept}
\section{Umgang mit schweren Problemen}
\end{document}
