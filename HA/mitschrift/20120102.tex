\section{Internet-Algorithmen}
\subsection{Page-Rank}
benannt nach Larry \textsc{Page} (aktueller CEO von Google)
\paragraph*{Suchmaschinen-Problem}
\begin{description}
\item[Gegeben:] Suchanfrage $s$ von Benutzer $n$
\item[Gesucht:] Was Benutzer $n$ finden möchte.
\end{description}
Was ist das?
\paragraph*{Erste Ansätze}
\renewcommand{\labelenumi}{(\theenumi)}
\renewcommand{\theenumi}{\arabic{enumi}}
\begin{enumerate}
 \item Yahoo: Manueller Katalog
 \item AltaVista: Textbasierte Suchmaschine
     \begin{itemize}
     \item automatisches Katalogisieren des ganzen Webs (Bots)
     \item Speichern eines großen Index
     \item Bei Suchanfrage $s$: durchsuche Index nach $s$, fiefere die Sites mit $s$ zurück
     \end{itemize}
\end{enumerate}
\paragraph*{Problem:} Wie bewertet man die gefundenen Seiten bei einer breiten Suchanfrage?
\paragraph*{Idee} (Kleinberg, Page) aus Bibliographieanalyse:
Verwende Linkstruktur des www, um Seiten zu bewerten (nutze das Wissen der Benutzer)
\paragraph*{Page-Rank}
\begin{center}
 \begin{psmatrix}[mnode=circle]
  GB & & L \\
     & ZS \\
  \  & \ & \ 
 \end{psmatrix}
 \ncline{1,1}{1,3}\ncline{1,1}{2,2}\ncline{2,2}{3,3}\ncline{3,1}{3,2}
 \ncline{2,2}{3,2}\ncline{3,1}{2,2}\ncline{1,3}{3,3}
\end{center}
Der Page-Rank einer Webseite ist die Wahrscheinlichkeit, dass ein zufälliger Surfer bei ihr vorbeikommt.
\paragraph*{Wie formalisieren?}
www: gerichteter Graph
\[G = (V, E) \qquad |V| = N\]
\begin{center}
 \begin{psmatrix}[mnode=circle]
   & \ \\
   \ & \ & \ \\
   \ & \ 
 \end{psmatrix}
 \psset{arrows=->}
 \ncline{1,2}{2,3}\ncline{1,2}{2,2}
 \ncline{2,1}{2,2}\ncline{2,2}{2,3}
 \ncline{3,1}{2,1}\ncline{2,2}{3,1}\ncline{2,2}{3,2}
\end{center}
Berechne in jedem Schritt eine Wahrscheinlichkeitsverteilung auf den Knoten $\Pi_i{:} V \to [0,1]$, $\sum\limits_{v \in V} \Pi_i(v) = 1$.
\begin{itemize}
 \item Am Anfang: $\forall v \in V{:}\ \Pi_0(v) = \frac{1}{N}$
 \item $\Pi_{i+1}(v) = \sum\limits_{w \in V} \frac{1}{\operatorname{outdeg}(w)} \cdot \Pi_i(w) \cdot A_{wv}$, wobei
         $A_{wv} = \begin{cases} 1, & \text{falls $(w,v) \in E$} \\ 0, & \text{sonst} \end{cases}$
\end{itemize}
Schreibe Gleichungen kompakt als Matrix:
\[\underbrace{\Pi_{i+1}}_{\text{Zeilenvektor}} = \Pi_i \cdot A'\]
wobei
\[A' = N \times N\text{-Matrix mit } A'_{wv} = \begin{cases} 
                                                    \frac{1}{\operatorname{outdeg}(w)}, & \text{falls $(w,v) \in E$} \\
                                                    0, \text{sonst}
                                               \end{cases}
\]
\begin{center}
 \begin{psmatrix}[mnode=circle]
  $a$ & $b$ \\
  $d$ & $c$
 \end{psmatrix}
 \psset{arrows=->}
 \ncline{1,1}{1,2}\ncline[offset=2px]{1,1}{2,2}
 \ncline{1,2}{2,2}
 \ncline{2,1}{1,1}
 \ncline{2,2}{2,1}\ncline[offset=2px]{2,2}{1,1}
\end{center}

\[
A' = \begin{pmatrix}
 0 & \frac{1}{2} & \frac{1}{2} & 0 \\
 0 & 0 & 1 & 0 \\
 \frac{1}{2} & 0 & 0 & \frac{1}{2} \\
 1 & 0 & 0 & 0
\end{pmatrix}
\]
$A'$ heißt modifizierte Adjazenzmatrix
\begin{align*}
 \Pi_{i+1} &= \Pi_i A' \\
 \Pi_i &= \underbrace{\begin{pmatrix}\frac{1}{4} & \frac{1}{4} & \frac{1}{4} & \frac{1}{4}\end{pmatrix}}_{\Pi_0} \cdot
 \begin{pmatrix}
 0 & \frac{1}{2} & \frac{1}{2} & 0 \\
 0 & 0 & 1 & 0 \\
 \frac{1}{2} & 0 & 0 & \frac{1}{2} \\
 1 & 0 & 0 & 0
\end{pmatrix}
 &= \begin{pmatrix}\frac{3}{8} & \frac{1}{8} & \frac{3}{8} & \frac{1}{8}\end{pmatrix}
\end{align*}
Hoffnung: Nach genügend vielen Schritten ändert sich $\Pi$, nicht mehr (Konvergenz, Fixpunktiteration)
\Defi Ein Wahrscheinlichkeitsvektor $\Pi^*$ heißt \emph{stationäre Verteilung} für $A'$, wenn gilt
    \[\Pi^* = \Pi^* \cdot A'\]
    $\Pi^*$ ist ein Linkseigenvektor zum Eigenwert 1.
\paragraph*{Probleme}
\begin{enumerate}
 \item Konvergiert $\Pi_i$? Im Allgemeinen: Nein
 \begin{center}
  \begin{psmatrix}[mnode=circle]
   \color{red}$\frac{1}{3}$ & \color{red}$\frac{1}{3}$ & \color{red}$\frac{1}{3}$
  \end{psmatrix}
  \psset{arrows=->}
  \ncline[offset=2px]{1,1}{1,2}\ncline[offset=2px]{1,2}{1,3}
  \ncline[offset=2px]{1,2}{1,1}\ncline[offset=2px]{1,3}{1,2}
 \end{center}
 \begin{center}
  \begin{psmatrix}[mnode=circle]
   \color{red}$\frac{1}{6}$ & \color{red}$\frac{2}{3}$ & \color{red}$\frac{1}{6}$
  \end{psmatrix}
  \psset{arrows=->}
  \ncline[offset=2px]{1,1}{1,2}\ncline[offset=2px]{1,2}{1,3}
  \ncline[offset=2px]{1,2}{1,1}\ncline[offset=2px]{1,3}{1,2}
 \end{center}
 Das System oszilliert
 \item Existiert $\Pi^*$?
 \begin{center}
  \begin{psmatrix}[mnode=circle]
   \color{red}$\frac{1}{2}$ & \color{red}$\frac{1}{2}$
  \end{psmatrix}
  \psset{arrows=->}
  \ncline{1,1}{1,2}
 \end{center}
 \begin{center}
  \begin{psmatrix}[mnode=circle]
   \color{red}$0$ & \color{red}$\frac{1}{2}$
  \end{psmatrix}
  \psset{arrows=->}
  \ncline{1,1}{1,2}
 \end{center}
 \begin{center}
  \begin{psmatrix}[mnode=circle]
   \color{red}$0$ & \color{red}$0$
  \end{psmatrix}
  \psset{arrows=->}
  \ncline{1,1}{1,2}
 \end{center}
\end{enumerate}
\paragraph*{Neustart/Dämpfung}
In jedem Schritt fängt ein bestimmter Anteil der Surfer auf eine zufälligen Seite neu an.\\
Schreibe:
\[ A'' := (1 - d) A' + \frac{d}{N} 
            \begin{pmatrix}
                1 & 1 & \cdots & 1 \\
                1 & 1 & \cdots & 1 \\
                \vdots & \vdots & \ddots & \vdots \\
                1 & 1 & \cdots & 1
            \end{pmatrix}\]
            $d \approx 0{,}15$ Dämpfungsfaktor
\begin{center}
\framebox{Füge Schleifen zu $G$ hinzu. \circlenode{i}{\ }\nccurve[ncurv=4,angleA=70,angleB=110]{->}{i}{i}}
\end{center}
Modifikation verhinder Oszillation und sagt für Existenz von $\Pi^*$.
\Satz (Perron-Frobenius): Es existiert genau eine stationäre Verteilung $\Pi^*$ für $A''$. Die Iteration $\Pi_{i+1} = \Pi_{i} \cdot A''$ konvergiert gegen $\Pi^*$. $\Pi^*$ ist der Pagerank-Score-Vektor des www.