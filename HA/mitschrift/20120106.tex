\subsection{Prioritätswarteschlangen}
\begin{description}
 \item[Gegeben:] Elemente $X$, Prioritäten/Schlüssel $K$ (total geordnet)
 \item[Ziel:] Speichere Teilmenge $S \subseteq X \times K$
 \item[Operationen:] \hfill
    \begin{center}
        \begin{tabular}{rp{0.8\linewidth}}
            insert($x,k$): & Füge $x$ mit Priorität $k$ zu zu $S$ hinzu. \\
            delete\_min(): & liefere Element mit kleinster Priorität, entferne aus $S$ \\
            decrease\_key($x,k$): & erniedrige den Schlüssel von $x$ zu $k$
        \end{tabular}
    \end{center}
\end{description}
Klassische Anwendung: Dijkstras ALgorithmus
\begin{description}
 \item[Gegeben:]
     \begin{itemize}
      \item Graph $G = (V,E)$, gerichtet, gewichtet
      \item Gewichtsfunktion: weight: $E \to \mathbb{R}^+_0$
      \item Startknoten $s \in V$
     \end{itemize}
 \item[Gesucht:] kürzeste Wege von $s$ zu allen anderen Knoten (Shortest Path Tree)
\end{description}
\begin{algorithmic}
\FOR{$v \in V$}
    \STATE $v.d \gets \infty$; $v.\pi \gets \perp$
    \STATE $Q.\text{insert}(v,v.d)$
\ENDFOR
\STATE $s.d \gets 0$; $Q.\text{decrease\_key}(s,0)$
\WHILE{$Q \neq \emptyset$}
    \STATE $v \gets Q.\text{delete\_min}()$
    \FOR{$e = (v, w) \in E$}
        \IF{$v.d + e.\text{weight} < w.d$}
            \STATE $w.d \gets v.d + e.\text{width}$
            \STATE $w.\pi \gets v$
            \STATE $Q.\text{decrease\_key}(w,w.d)$
        \ENDIF
    \ENDFOR
\ENDWHILE
\end{algorithmic}
\paragraph*{Laufzeit:} Hängt von $Q$ ab.
\begin{itemize}
 \item $|V|$ inserts und delete\_mins
 \item $\leq |E| + 1$ decrease\_key 
\end{itemize}
Klassische Implementierung: Binärheap
\begin{itemize}
\item   Laufzeit: $O(\log n)$ pro Operation\\
        $\Rightarrow$ Dijkstra läuft in $O(|V|  \log |V| + |E| \log |V|)$ Zeit.
\end{itemize}
Frage: Geht es besser? Können wir eine bessere Implementierung von $Q$ finden?
\begin{itemize}
 \item $|V| \log |V|$ lässt sich wahrscheinlich nicht verbessern
 \item Gibt es Prioritätswarteschlange mit $O(1)$ decrease\_key
 \item 1987: Fredman-Tarjan: Ja: \\
     \textbf{Fibonacci-Heaps}
         \begin{itemize}
          \item $O(\log n)$ insert \& delete\_min
          \item $O(1)$ decrease\_key
         \end{itemize}
        (amortisierte Laufzeit)
 \item seitdem: viele Varianten und Alternativen:
     \begin{itemize}
     \item entspannte Heaps
     \item linksgerichtete Heaps
     \item dünne Heaps
     \item dicke Heaps
     \item $\vdots$
     \item Erdbeben-Heaps (\emph{Quake Heaps})
     \end{itemize}
\end{itemize}
\subsection{Quake-Heaps}
T.M. Chan, 2009
\begin{itemize}
 \item delete\_min in $O(\log n)$ amortisiert
 \item insert \& decrease\_key in $O(1)$ amortisierte
\end{itemize}
$\Rightarrow$ Dijkstra in $O(|V| \log |V| + |E|)$
\begin{itemize}
 \item Ein Quake-Heap speichert die Elemente in Turnierbäumen
 \item Turierbaum: Wurzelbaum
 \begin{itemize}
 \item Speichert Elemente in den Blättern
 \item Alle Blätter haben die gleiche Tiefe (= Länge des Pfades zur Wurzel)
 \item Innere Knoten haben 1 oder 2 Kinder
 \item Innerer Knoten speichert das Minimum seiner Kinder
 \item Jedes Blatt speichert einen Zeiger auf den höchsten Knoten, der den gleichen Schlüssel enthält 
 \end{itemize}
 \begin{center}
 \psset{levelsep=0.7cm,treesep=0.5cm,nodesep=2pt}
  \pstree{\Tr{0}}{
      \pstree{\Tr{4}}{\pstree{\Tr{4}}{\pstree{\Tr{4}}{\Tr{4}}}}
      \pstree{\Tr{0}}{
          \pstree{\Tr{2}}{
              \pstree{\Tr{2}}{
                  \Tr{2}\Tr{3}
              }
          }
          \pstree{\Tr{0}}{
              \pstree{\Tr{0}}{
                  \Tr{1}\Tr{0}
              }
              \pstree{\Tr{7}}{
                  \Tr{7}
              }
          }
      }
  }
 \end{center}
 \item Operationen auf Turnierbäume:
 \begin{enumerate}
  \item link($T_1, T_2$): $T_1, T_2$: Turnierbäume gleicher Höhe \\
      $\Rightarrow$ Turnierbaum mit Höhe + 1
      \begin{center}     
 \psset{levelsep=0.7cm,treesep=0.5cm,nodesep=2pt} 
       \begin{minipage}{3cm}
        \centering
        \pstree{\Tr{5}}{
          \pstree{\Tr{10}}{
              \Tr{10}
          }
          \pstree{\Tr{5}}{
              \Tr{5}\Tr{8}
          }
      }
       \end{minipage}
       \hspace{0.5em}
       \begin{minipage}{3cm}
        \centering
        \pstree{\Tr{2}}{
          \pstree{\Tr{2}}{
              \Tr{2}\Tr{7}
          }
          \pstree{\Tr{4}}{
              \Tr{9}\Tr{4}
          }
      }
       \end{minipage}
       \hspace{0.2cm}
       $\Rightarrow$
       \hspace{0.2cm}
       \begin{minipage}{6cm}
        \centering
        \pstree{\Tr{2}}{
            \pstree{\Tr{5}}{
          \pstree{\Tr{10}}{
              \Tr{10}
          }
          \pstree{\Tr{5}}{
              \Tr{5}\Tr{8}
          }
      }
        \pstree{\Tr{2}}{
          \pstree{\Tr{2}}{
              \Tr{2}\Tr{7}
          }
          \pstree{\Tr{4}}{
              \Tr{9}\Tr{4}
          }
      }
        }
       \end{minipage}
      \end{center}
      Lege neue Wurzel an, speichere in der alten Wurzeln in der neuen Wurzel. $O(1)$ Zeit.
  \item cut($v$): $v$ Knoten, der einen anderen Schlüssel enthält als sein Elternknoten\\
      $\Rightarrow$ zwei Turnierbäume, einen mit Wurzel $v$
      \begin{center}     
 \psset{levelsep=0.7cm,treesep=0.5cm,nodesep=2pt} 
       \begin{minipage}{6cm}
        \centering
        \pstree{\Tr{2}}{
            \pstree{\Tcircle[linecolor=red]{5}\nput{160}{\pssucc}{\color{red}$v$}}{
          \pstree{\Tr{10}}{
              \Tr{10}
          }
          \pstree{\Tr{5}}{
              \Tr{5}\Tr{8}
          }
      }
        \pstree{\Tr{2}}{
          \pstree{\Tr{2}}{
              \Tr{2}\Tr{7}
          }
          \pstree{\Tr{4}}{
              \Tr{9}\Tr{4}
          }
      }
        }
       \end{minipage}
       \hspace{0.2cm}
       $\Rightarrow$
       \hspace{0.2cm}
       \begin{minipage}{3cm}
        \centering
        \pstree{\Tr{5}}{
          \pstree{\Tr{10}}{
              \Tr{10}
          }
          \pstree{\Tr{5}}{
              \Tr{5}\Tr{8}
          }
      }
       \end{minipage}
       \hspace{0.5em}
       \begin{minipage}{3cm}
        \centering
        \pstree{\Tr{2}}{
          \pstree{\Tr{2}}{
              \Tr{2}\Tr{7}
          }
          \pstree{\Tr{4}}{
              \Tr{9}\Tr{4}
          }
      }
       \end{minipage}
      \end{center}
      cut löscht Kante von $v$ zu Elternknoten. $O(1)$ Zeit.
 \end{enumerate}
\end{itemize}

Quake-Heaps speichern
\begin{itemize}
 \item eine Liste von Turnierbäumen
 \item noch mehr (später)
 \item insert($x,k$)
    \begin{itemize}
    \item Lege neuen Turnierbaum für $(x,k)$ an, füge ihn zur Liste hinzu. \\
    insert($x, 15$)
    \end{itemize}
    $O(1)$-Zeit
 \item decrease\_key($y,k$): Annahme: Haben/Bekommen. Zeiger auf das Blatt, das $y$ speichert.
     \begin{itemize}
     \item decrease\_key($y,-1$)
     \item finde höchsten Knoten $v$, der Schlüssel von $y$ enthält
     \item Führe cut($v$) aus, falls $v$ nicht Wurzel
     \item Erniedrige Schlüssel von $y$ auf $k$
     \end{itemize}
     $O(1)$ Implementierung
 \item delete\_min(): Zusätzliche Information (können beide in insert/decrease\_key in $O(1)$ aktualisiert werden):
     \begin{itemize}
      \item Array $T[i]$: Liste aller Bäume mit Höhe $i$
      \item Array $n[i]$: \# Knoten mit Höhe $i$
      \item Invariante: $\forall i \geq 0{:}\ n[i-1] \leq \frac{3}{4} n[i]$
      \item \textbf{Fakt:} Invariante $\Rightarrow$ Jeder Knoten hat Höhe $O(\log n)$
      \item \textbf{Fakt:} insert und decrease\_key erhalten die Invariante.
     \end{itemize}
     \textbf{delete\_min}
     \begin{enumerate}
      \item Gehe alle Wurzeln durch, finde das Element $x$ mit minimalen Schlüssel
      \item Lösche den Pfad, der $x$ enthält. Lege ggf. neue Bäume an
      \begin{center}
     \psset{levelsep=0.7cm,treesep=0.5cm,nodesep=2pt}
      \begin{minipage}{6cm}
        \centering
        \pstree{\Tr{0}}{
            \pstree{\Tr{4}}{
                \pstree{\Tr{4}}{
                    \pstree{\Tr{4}}{
                        \Tr{4}
                    }
                }
                
            }
            \pstree{\Tr{0}}{
                \pstree{\Tr{2}}{
                    \pstree{\Tr{2}}{
                        \Tr{2}
                        \Tr{3}
                    }
                }
                \pstree{\Tr{0}}{
                    \pstree{\Tr{0}}{
                        \Tr{1}
                        \Tr{0}
                    }
                    \pstree{\Tr{7}}{
                        \Tr{7}
                    }
                }
            }
        }
        \Tr{14}
        \pstree{\Tr{5}}{
            \Tr{8}
            \Tr{5}
        }
        \Tr{15}
       \end{minipage}
       \hspace{0.2cm}
       $\Rightarrow$
       \hspace{0.2cm}
       \begin{minipage}{6cm}
        \centering
        \pstree{\Tr{4}}{
            \pstree{\Tr{4}}{
                \pstree{\Tr{4}}{
                    \Tr{4}
                }
            }
            
        }
        \pstree{\Tr{2}}{
            \pstree{\Tr{2}}{
                \Tr{2}
                \Tr{3}
            }
        }
        \pstree{\Tr{0}}{
            \pstree{\Tr{0}}{
                \Tr{1}
                \Tr{0}
            }
            \pstree{\Tr{7}}{
                \Tr{7}
            }
        }
        \Tr{14}
        \pstree{\Tr{5}}{
            \Tr{8}
            \Tr{5}
        }
        \Tr{15}
       \end{minipage}
     \end{center}
      \item Konsolidierung
          \begin{algorithmic}
           \FOR{$i := 0..O(\log n)$}
               \WHILE{$|T[i]| \geq 2$}
                   \STATE Lösche 2 Bäume $T_1,T_2$ aus $T[i]$
                   \STATE $T^* \gets \text{link}(T_1, T_2)$
                   \STATE $T[i+1] \gets T[i+1] ++ T^*$
               \ENDWHILE
           \ENDFOR
          \end{algorithmic}
          \begin{description}
           \psset{levelsep=0.7cm,treesep=0.5cm,nodesep=2pt}
           \item $T[0]:$ 
               \begin{equation*}
                \cancel{1} \cancel{14} 
                15
               \end{equation*}
           \item $T[1]:$ 
               \begin{equation*}
                \cancel{\pstree{\Tr{7}}{\Tr{7}}} 
                \cancel{\pstree{\Tr{5}}{\Tr{8}\Tr{5}}} 
                \cancel{\pstree{\Tr{1}}{\Tr{1} \Tr{14}}}
                {\color{red}\psset{linecolor=red} \pstree{\Tr{1}}{\Tr{1} \Tr{14}}}
               \end{equation*}
           \item $T[2]:$ 
               \begin{equation*}
                \cancel{\pstree{\Tr{2}}{\pstree{\Tr{2}}{\Tr{2}\Tr{4}}}}
                {\color{orange}\psset{linecolor=orange} \pstree{\Tr{5}}{\pstree{\Tr{7}}{\Tr{7}}\pstree{\Tr{5}}{\Tr{6}\Tr{5}}}}
               \end{equation*}
           \item $T[3]:$
               \begin{equation*}
                \cancel{\pstree{\Tr{4}}{\pstree{\Tr{4}}{\pstree{\Tr{4}}{\Tr{4}}}}} 
                {\color{green}\psset{linecolor=green} 
                    \pstree{\Tr{2}}{
                        \pstree{\Tr{4}}{
                            \pstree{\Tr{4}}{
                                \pstree{\Tr{4}}{
                                    \Tr{4}
                                }
                            }
                        }
                        \pstree{\Tr{2}}{
                            \pstree{\Tr{2}}{
                                \pstree{\Tr{2}}{
                                    \Tr{2}\Tr{3}
                                }
                            }
                            \pstree{\Tr{5}}{
                                \pstree{\Tr{7}}{
                                    \Tr{7}
                                }
                                \pstree{\Tr{5}}{
                                    \Tr{6}
                                    \Tr{5}
                                }
                            }
                        }
                    }
                }
               \end{equation*}
               \hrulefill
           \item Ergebnis:
               \begin{equation*}
                   15
                   \pstree{\Tr{1}}{\Tr{1} \Tr{14}}
                    \pstree{\Tr{2}}{
                        \pstree{\Tr{4}}{
                            \pstree{\Tr{4}}{
                                \pstree{\Tr{4}}{
                                    \Tr{4}
                                }
                            }
                        }
                        \pstree{\Tr{2}}{
                            \pstree{\Tr{2}}{
                                \pstree{\Tr{2}}{
                                    \Tr{2}\Tr{3}
                                }
                            }\pstree{\Tr{5}}{
                                \pstree{\Tr{7}}{
                                    \Tr{7}
                                }
                                \pstree{\Tr{5}}{
                                    \Tr{6}
                                    \Tr{5}
                                }
                            }
                        }
                    }
               \end{equation*}
          \end{description}
      \item Erdbeben: Falls Ebene $i$ existiert mit $n[i+1] > \frac{3}{4} n[i]$: 
          \begin{itemize}
          \item Sei $i$ niedrigste solcher Ebenen
          \item Lösche alle Knoten mit Höhe $> i$
          \end{itemize}
          $\rightarrow$ Erzeugt ggf. neue Bäume
     \end{enumerate}
\end{itemize}


