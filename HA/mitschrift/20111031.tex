\begin{itemize}
\item   Oft interesiert uns nur die asymprotische Laufzeit \\
        $\Rightarrow$ $O$-Notation
\end{itemize}

\chapter{Grundlegende Techniken zum Entwurf von Algorithmen}
\section{Teile \& Herrsche / Divide \& Impera}
\begin{itemize}
\item   Geht zurück auf Sun-Tsu (500 v. Chr.) und Neuformulierung durch Machiavelli
\item   Unterteile in kleinere Probleme, löse diese einzeln, setze die Teilergebnisse zusammen.
\item   Beispiele: Mergesort, Quicksort
\item   Bei Divide \& Conquer treten Rekursionsgleichungen auf. Wie löst man diese?
\end{itemize}

\Bsp Multiplizieren von Zahlen.
\begin{itemize}
\item   \textbf{Problem:} Gegeben: $a, b \in \mathbb N$
\item   Gesucht: $a \cdot b$
\item   z. B. $a = 1234, b = 512$\\
        Schulmethode:
        \begin{verbatim}
1234 * 512
----------
      2468
     1234
    6170  
----------
    631808\end{verbatim}
        \begin{itemize}
        \item   Annahme: beide Zahlen haben $n$ Ziffern (zur Not werden vorne an eine Zahl 0 angefügt)
        \item   Dann ist die Anzahl der Multiplikationen und Additionen von Ziffern $\Theta(n^2)$
        \item   Analog im Binärsystem
        \item   Können wir Teile \& Herrsche benutzen um schneller zu multiplizieren
                \[
                    \begin{array}{r|c|c|c|c|c|c|c|c|}
                    \cline{2-9}
                     a & 1 & 0 & 1 & 1 & 0 & 0 & 1 & 0 \\\cline{2-9}
                    \end{array}
                \]
                \[
                    \begin{array}{r|c|c|c|c|c|c|c|c|}
                    \cline{2-9}
                     b & 1 & 1 & 1 & 0 & 1 & 0 & 1 & 1 \\\cline{2-9}
                    \end{array}
                \]
        \item   Idee: Teile $a$ und $b$ in Zahlen mit weniger Ziffern auf, löse rekursiv, setze das ergebnis zusammen
                \[  a\ 
                    \overbrace{
                    \begin{array}{|c|c|c|c|}
                    \cline{1-4}
                     1 & 0 & 1 & 1 \\\cline{1-4}
                    \end{array}}^{a_h}
                    \overbrace{
                    \begin{array}{|c|c|c|c|}
                    \cline{1-4}
                     0 & 0 & 1 & 0 \\\cline{1-4}
                    \end{array}}^{a_l}
                \]
                \[  b\ 
                    \overbrace{
                    \begin{array}{|c|c|c|c|}
                    \cline{1-4}
                     1 & 1 & 1 & 0 \\\cline{1-4}
                    \end{array}
                    }^{b_h}
                    \overbrace{
                    \begin{array}{|c|c|c|c|}
                    \cline{1-4}
                     1 & 0 & 1 & 1 \\\cline{1-4}
                    \end{array}}^{b_l}
                \]
        \item   Schreibe:
                \begin{align*}
                 a &= \overbrace{a_h \cdot 2^{\left\lceil\frac{n}{2}\right\rceil}}^{\text{obere Bits}} + \overbrace{a_l}^{\text{untere $\left\lceil\frac{n}{2}\right\rceil$ Bits}}\\
                 b &= b_h \cdot 2^{\left\lceil\frac{n}{2}\right\rceil} + b_l
                \end{align*}
                \begin{align*}
                 a \cdot b  &= (a_h \cdot 2^{\left\lceil \frac{n}{1} \right\rceil} + a_l) \cdot (b_h \cdot 2^{\left\lceil \frac{n}{1} \right\rceil} + b_l)\\
                            &= \psframebox[linecolor=red]{a_h \cdot b_h} 2^{2 \left\lceil \frac{n}{1} \right\rceil} + 
                                \psframebox[linecolor=blue]{(a_h \cdot b_l + b_h \cdot a_l)} 2^{2 \left\lceil \frac{n}{1} \right\rceil} + \psframebox[linecolor=green]{a_l \cdot b_l}
                \end{align*}
        \end{itemize}
        \paragraph{Algorithmus:}
        \begin{itemize}
        \item   Berechne rekursiv $a_h \cdot b_h$, $a_h \cdot b_l$, $a_l \cdot b_h$, $a_l \cdot b_l$
        \item   Berechne $a \cdot b$ nach Formel mit Shiften und Addieren
        \end{itemize}
        \paragraph{Laufzeitanalyse:}
        \begin{align*}
         T(n) &\leq \begin{cases}
                     O(1), & \text{wenn $n < 3$}\\
                     4 \cdot T\left(\frac{n}{2}\right) + O(n), & \text{sonst}
                    \end{cases}
        \end{align*}
        \begin{description}
        \item[Lösung:]  $T(n) = \Theta(n^2)$
        \item[Problem:] Führen $\underline{4}$ rekursive Multiplikationen durch. Dadurch wird nichts gewonnen.
        \end{description}
        \paragraph*{Genialer Einfall:} (Algorithmus von Karatsuba)
        Betrachte $(a_h + a_l) \cdot (b_h + b_l) = \psframebox[linecolor=red]{a_h \cdot b_h} + \psframebox[linecolor=blue]{a_h \cdot b_l + b_h \cdot a_l} + \psframebox[linecolor=green]{a_l \cdot b_l}$
        \begin{itemize}
        \item   Berechne nach: $\psframebox[linecolor=red]{a_h \cdot b_h}$ und $\psframebox[linecolor=green]{a_l \cdot b_l}$
        \item   Und dann: $(a_h + a_l) \cdot (b_h + b_l) - \psframebox[linecolor=green]{a_l \cdot b_l} - \psframebox[linecolor=red]{a_h \cdot b_h}$
        \end{itemize}
        \paragraph*{Laufzeitanalyse:}
        \begin{align*}
         T(n) &\leq \begin{cases}
                     O(1), & \text{wenn $n < 3$}\\
                     {\color{red}3} \cdot T\left(\frac{n}{2}\right) + O(n), & \text{sonst}
                    \end{cases}
        \end{align*}
        Was kommt heraus?
    \paragraph*{Bemerkungen}
    \begin{itemize}
     \item aktueller Champion der Multiplikationsalgorithmen: M. Fürer (2007/09) $O(n \log n 2^{\log_* n})$
     \item Vermutung: $\Theta(n \log n)$ ist optimal
    \end{itemize}
\end{itemize}

\subsection{Lösen von Rekursionsgleichungen}
\begin{itemize}
\item   Methode 1: Raten \& \textbf{Induktion}
\item   Methode 2: \textbf{Wiederholt einsetzen} \& Muster erkennen
        \begin{align*}
         T(n)   &\leq 3 T\left(\frac{n}{2}\right) + c \cdot n \\
                &\leq 3 \cdot \left(3 \cdot T\left(\frac{n}{4}\right) + c \cdot \frac{n}{2}\right) + c \cdot n \\
                &\leq 3 \cdot \left(3 \cdot  \left(3 \cdot T\left(\frac{n}{8}\right) + c \cdot \frac{n}{4}\right) + c \cdot \frac{n}{2}\right) + c \cdot n\\
                &\leq 3^3 T\left(\frac{n}{8}\right) + c \frac{3^2}{2^2}n + c \frac{3}{2}n + c \cdot n\\
                &\vdots \hspace{1cm}\text{$k$ Schritte}\\
                &\leq 3^k T\left(\frac{n}{2^k}\right) + c \left(\frac{3}{2}\right)^{k-1}n + c\left(\frac{3}{2}\right)^{k-2} + ... + cn\\
                & \text{nach $k = \log n$ Schritten ist $\frac{n}{2}$ konstant.}\\
                &= \sum\limits_{k = 0}^{\log_2 n} \left(\frac{3}{2}\right)^k c \cdot n\\
                &= cn \frac{\left(\frac{3}{2}\right)^{1 + \log_2 n} - 1}{\frac{3}{2} - 1} \tag{geometrische Reihe}\\
                &\leq 2 cn \frac{3}{2} \left(\frac{3}{2}\right)^{1 + \log_2 n}\\
                &= 3 cn \cdot n^{\log_2 \frac{3}{2}} = 3 c \cdot n^{\log_2 3}
        \end{align*}
        \paragraph*{Laufzeit} $\Theta(n^{\log_2 3})$, $\log_2 3 \approx 1{,}5385$
% 2011104 hier angefangen wegen itemize
\item   Methode 3: Bild malen \& Muster erkennen (\textbf{Rekursionsbaummethode})
        \[T(n) \leq 3 T\left(\frac{n}{2} + O(n)\right)\]
        Male einen Knoten für jeden rekursiven Aufruf
        \begin{center}
        \begin{minipage}{0.3\textwidth}
        \psset{levelsep=0.6cm,treesep=0.3cm}
        \pstree{\Tr*{$T(n)$}}{
            \pstree{\Tr*{$T\left(\frac{n}{2}\right)$}}{
                \pstree{\Tr*{$T\left(\frac{n}{4}\right)$}}{
                    \pstree{\Tr*{$T\left(\frac{n}{8}\right)$}}{
                        \pstree{\Tr*{$\vdots$}}{
                            \Tr{$T(1)$}
                        }
                    }
                    \Tr{$\vdots$}\Tr{$\vdots$}
                }
                \Tr{$\vdots$}
                \Tr{$\vdots$}
            }
            \pstree{\Tr*{$T\left(\frac{n}{2}\right)$}}{
                \Tr{$\vdots$}
                \Tr{$\vdots$}
                \Tr{$\vdots$}
            }
            \pstree{\Tr*{$T\left(\frac{n}{2}\right)$}}{
                \Tr{$\vdots$}
                \Tr{$\vdots$}
                \Tr{$\vdots$}
            }
        }
        \end{minipage}
        \vline
        \begin{minipage}{0.3\textwidth}
            \centering
            $c \cdot n$ \\[0.3cm]
            $3 \cdot c \cdot \frac{n}{2}$ \\[0.3cm]
            $9 \cdot c \cdot \frac{n}{4}$ \\[0.3cm]
            $27 \cdot c \cdot \frac{n}{8}$ \\[0.3cm]
            \hspace{0cm} \\[0.3cm]
            $3^k \cdot c \cdot \frac{n}{2^k}$
        \end{minipage}
        \end{center}
        Es gibt $O(\log_2 n)$ Ebenen $\leq \log n$ Ebenen
        \begin{itemize}
        \item Addiere alle Kosten $\sum\limits_{k = 0}^{\log n} \left(\frac{3}{2}\right)^k c \cdot n = \Theta(n^{\log_2 3})$
        \end{itemize}
\item   Methode 4: Allgemeines Rezept: \textbf{Master Theorem}
        \Satz Sei $a \geq 1, b \geq 1, f{:}\ \mathbb{N} \to \mathbb{N}$. Sei $T(n) = a \cdot T\left(\frac{n}{b}\right) + f(n)$ eine Rekursion ($T(n) = O(1)$ für $n \leq 2$). Dann gilt:
        \begin{enumerate}
         \item Wenn $f(n) = O(n^{(\log_b a) - \varepsilon})$ für $\varepsilon > 0$, dann ist 
                \[T(n) = \Theta(n^{\log_b a})\]
         \item Wenn $f(n) = \Theta(n^{\log_b a})$ ist, dann ist
                \[T(n) = \Theta\left(n^{\log_b a} \log n\right)\]
         \item Wenn $f(n) = \Omega(n^{(\log_b a) + \varepsilon})$ für $\varepsilon > 0$ und $\exists c < 1{:}\ a f\left(\frac{n}{b}\right) < c f\left(n\right)$, dann ist
                \[T(n) = \Theta\left(f(n)\right)\]
        \end{enumerate}
        \Bsp    \begin{itemize}
        \item   $T(n) \leq 3 T\left(\frac{n}{2}\right) + O(n)$\\
                MT anwendbar mit $a = 3, b = 2, f(n) = c \cdot n$
                \begin{align*}
                 f(n)   &= O(n^{\log_b a - \varepsilon}) \\
                        &= O(n^{\log_2 3} - n^{1{,}5...}), \varepsilon \approx 0{,}5
                \end{align*}
                Fall 1: $T(n) = \Theta(n^{\log_2 3})$
        \item   $T(n) = 4 T\left(\frac{n}{2}\right) + O(n)$ \\
                MT mit $a = 4, b = 2, f(n) = c \cdot n$
                \begin{align*}
                 f(n) &= O(n^{2 - \varepsilon}) \text{ für $\varepsilon \approx 0{,}5$}
                \end{align*}
                Fall 1: $T(n) = \Theta(n^2)$
        \item   $T(n) = 2 T\left(\frac{n}{2}\right) + O(n)$ \\
                MT anwendbar mit $a = 2, b = 2, f(n) = c \cdot n$
                \begin{align*}
                 n^{\log_b a} &= n\\
                 f(n) &= \Theta(n^{\log_b a})
                \end{align*}
                Fall 2: $T(n) = \Theta(n \log n)$
        \item   $T(n) = T\left(\frac{n}{2}\right) + O(1)$ \\
                MT anwendbar mit $a = 1, b = 2, f(n) = c$
                \begin{align*}
                 n^{\log_b a} &= n^0 = 1\\
                 f(n) &= \Theta(n^{\log_b a})
                \end{align*}
                Fall 2: $T(n) = \Theta(\log n)$
        \item   $T(n) = T\left(\frac{3}{4}n\right) + O(n)$\\
                MT anwendbar mit $a = 1, b = \frac{4}{3}, f(n) = d \cdot n$\\
                \begin{align*}
                 n^{\log_b a} &= n^0 = 1 \\
                 f(n) &= \Omega(n^{0+\varepsilon}), \text{z. N. für $\varepsilon = \dfrac{1}{2}$} \\
                 a \cdot f\left(\frac{n}{b}\right) &\leq 1 \cdot \frac{3}{4} \cdot dn
                    &= \frac{3}{4} d n \leq \frac{3}{4} f(n)
                \end{align*}
                Fall 3: $T(n) = \Theta(n)$
        \end{itemize}
        \Bew
            \begin{center}
             \begin{minipage}{0.4\textwidth}
            \centering
        \psset{levelsep=0.6cm,treesep=0.3cm}
        \pstree{\Tr*{$T(n)$}}{
            \pstree{\Tr*{$T\left(\frac{n}{b}\right)$}}{
                \pstree{\Tr*{$T\left(\frac{n}{b^2}\right)$}}{
                    \pstree{\Tr*{$T\left(\frac{n}{b^3}\right)$}}{
                        \pstree{\Tr*{$\vdots$}}{
                            \Tr{$T(1)$}
                        }
                    }
                    \Tr{$\vdots$}\Tr{$\vdots$}
                }
                \Tr{$\hdots$}
                \Tr{$\vdots$}
            }
            \pstree{\Tr*{$T\left(\frac{n}{b}\right)$}}{
                \Tr{$\vdots$}
                \Tr{$\hdots$}
                \Tr{$\vdots$}
            }
            \pstree{\Tr*{$T\left(\frac{n}{b}\right)$}}{
                \Tr{$\vdots$}
                \Tr{$\hdots$}
                \Tr{$\vdots$}
            }
        }
        (jeweils $a$ Kinder)
        \end{minipage}
        \vline
        \begin{minipage}{0.3\textwidth}
            \centering
            $f(n)$ \\[0.3cm]
            $a \cdot f\left(\frac{n}{b}\right)$ \\[0.3cm]
            $a^2 \cdot f\left(\frac{n}{b^2}\right)$ \\[0.3cm]
            $a^3 \cdot f\left(\frac{n}{b^3}\right)$ \\[0.3cm]
            \hspace{0cm} \\[0.3cm]
            $a^k \cdot f\left(\frac{n}{b^k}\right)$ \\[0.3cm]
        \end{minipage}
        \end{center}
        $\leq \log_b n$ Ebenen, Gesamtkosten: $\sum\limits_{k = 0}^{\log_b n} a^k f\left(\frac{n}{b^k}\right)$\\
        \begin{itemize}
        \item   $T(n) = 2 T \left(\frac{n}{2}\right) + \Theta(n \log n)$\\
                MT nicht anwendbar
        \item   $T(n) = 2 T \left(\frac{n}{3}\right) + T \left(\frac{2}{3} n\right) + \Theta(n)$\\
                MT nicht anwendbar (passt nicht ins Schema)
        \end{itemize}
\end{itemize}