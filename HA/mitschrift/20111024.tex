\paragraph*{Wie implementiert man \texttt{SPLITTER}?}
\begin{description}
 \item[1. Möglichkeit:] Wähle den \texttt{SPLITTER} zufällig. Mit Wahrscheinlichkeit $\frac{1}{2}$ erwischen wir einen guten Splitter.
		Im Durchschnitt ist das wohl gut genug. \\
		$\Rightarrow$ Erwartungswert der Laufzeit ausrechnen, etc., \emph{Randomisierte Algorithmen (später)}
 \item[2. Möglichkeit:] BFRPT-Methode:
		\begin{description}
		\item[rekursiv] Wähle eine Stichprobe $S' \subseteq S$ mit $|S'| = \left\lceil\frac{n}{5}\right\rceil$, so dass der Median von $S'$ ein guter Splitter von $S$ ist. Bestimme rekursiv den Median von $S$!
		\item[Wie finden wir $\boldsymbol{S'}$?]\hspace{0cm}\\
			\begin{description}
			\item[1. Versuch:] Nimm die ersten $\left\lceil\frac{n}{5}\right\rceil$. Elsement von $S$. $\Rightarrow$ \emph{klappt nicht.}
			\item[2. Versuch:] Nimm jedes 5. Element. $\Rightarrow$ \emph{auch nicht}
			\end{description}
		\end{description}
 \item[3. Möglichkeit] Unterteile $S$ in 5er Gruppen. Nimm aus jeder 5er Gruppe den Median. $S'$ ist die Menge der Mediane.
		\Bsp
		\[
		\begin{matrix}
		 S & \rnode{51l}{7} & 1 & 5 & 9 & \rnode{51r}{18} \\&\\
		 S' 
		\end{matrix}\quad
		\begin{matrix}
		 \rnode{52l}{30} & 6 & 11 & 15 & \rnode{52r}{20} \\&\\
		 &
		\end{matrix}
		\ncbox[nodesep=2px]{51l}{51r}7
		\ncbox[nodesep=2px]{52l}{52r}15
		\]
		\Lemma Der Median von $S'$ ist ein guter Splitter von $S$, wenn $n$ groß genug ist.
		\Bew Betrachte $S'$ als von links nach rechts sortiert (für den Beweis) und zugehörige 5er Gruppen.
			
			Die Lage im Bild.
		\begin{center}
		 \begin{pspicture}(0,0)(10,5)
		  \psdot(0,4)\psdot(1,4)\psdot(2,4)\psdot(3,4)\psdot(4,4)\psdot(5,4)\psdot(6,4)\psdot(7,4)\psdot(8,4)
		  \psdot(0,3)\psdot(1,3)\psdot(2,3)\psdot(3,3)\psdot(4,3)\psdot(5,3)\psdot(6,3)\psdot(7,3)\psdot(8,3)\psdot(9,3)
	\rput(-1,2){\color{green}$S'$}	  \psdot[linecolor=green](0,2)\psdot[linecolor=green](1,2)\psdot[linecolor=green](2,2)\psdot[linecolor=green](3,2)\psdot[linecolor=green](4,2)\psdot[linecolor=green](5,2)\psdot[linecolor=green](6,2)\psdot[linecolor=green](7,2)\psdot[linecolor=green](8,2)\psdot[linecolor=green](9,2)
		  \psdot(0,1)\psdot(1,1)\psdot(2,1)\psdot(3,1)\psdot(4,1)\psdot(5,1)\psdot(6,1)\psdot(7,1)\psdot(8,1)\psdot(9,1)
		  \psdot(0,0)\psdot(1,0)\psdot(2,0)\psdot(3,0)\psdot(4,0)\psdot(5,0)\psdot(6,0)\psdot(7,0)\psdot(8,0)
		  \psset{arrows=<-}
		  \psline(0,4)(0,3)\psline(1,4)(1,3)\psline(2,4)(2,3)\psline(3,4)(3,3)\psline(4,4)(4,3)
		  \psline(5,4)(5,3)\psline(6,4)(6,3)\psline(7,4)(7,3)\psline(8,4)(8,3)
		  \psline(0,3)(0,2)\psline(1,3)(1,2)\psline(2,3)(2,2)\psline(3,3)(3,2)\psline(4,3)(4,2)
		  \psline(5,3)(5,2)\psline(6,3)(6,2)\psline(7,3)(7,2)\psline(8,3)(8,2)\psline(9,3)(9,2)
		  \psset{arrows=->}
		  \psline[linecolor=green](0,2)(1,2)\psline[linecolor=green](1,2)(2,2)
		  \psline[linecolor=green](2,2)(3,2)\psline[linecolor=green](3,2)(4,2)
		  \psline[linecolor=green](4,2)(5,2)\psline[linecolor=green](5,2)(6,2)
		  \psline[linecolor=green](6,2)(7,2)\psline[linecolor=green](7,2)(8,2)
		  \psline[linecolor=green](8,2)(9,2)
		  \psline(0,1)(0,2)\psline(1,1)(1,2)\psline(2,1)(2,2)\psline(3,1)(3,2)\psline(4,1)(4,2)
		  \psline(5,1)(5,2)\psline(6,1)(6,2)\psline(7,1)(7,2)\psline(8,1)(8,2)\psline(9,1)(9,2)
		  \psline(0,0)(0,1)\psline(1,0)(1,1)\psline(2,0)(2,1)\psline(3,0)(3,1)\psline(4,0)(4,1)
		  \psline(5,0)(5,1)\psline(6,0)(6,1)\psline(7,0)(7,1)\psline(8,0)(8,1)
		  \psframe[linecolor=red](4.8,1.8)(5.2,2.2)
		  \psset{arrows=-}
		  \psline[linecolor=blue](-0.2,-0.2)(5.2,-0.2)(5.2,1.2)(4.2,1.2)(4.2,2.2)(-0.2,2.2)(-0.2,-0.2)
		  \psline[linecolor=blue](9.2,4.2)(9.2,1.8)(5.8,1.8)(5.8,2.8)(4.8,2.8)(4.8,4.2)(9.2,4.2)
		  \uput{0.3cm}[-45](5,2){\color{red}$q$}
		  \uput{0.3cm}[180](-0.2,1){\color{blue}$< q$}
		  \uput{0.3cm}[0](9.2,3){\color{blue}$> q$}
		 \end{pspicture}
		\end{center}
		\begin{itemize}
		 \item Es sind $\underbrace{\left\lceil\frac{1}{2}\underbrace{\left\lceil\frac{n}{5}\right\rceil}_{|5|}\right\rceil}_{\#\text{Gruppen $m$}} - 3$
		 \item definitiv größer als $q$.
		 \item Ebenso gibt es definitiv $3 \left\lceil\frac{1}{2}\left\lceil\frac{n}{5}\right\rceil\right\rceil - 3$ Elemente kleiner als $q$.
		 \item Es gilt:
				\begin{align*}
				 \left\lceil\frac{1}{2}\left\lceil\frac{n}{5}\right\rceil\right\rceil - 3 &\geq
				3 \cdot \frac{1}{2} \cdot \frac{1}{5} \cdot n - 3 \\
					&= \frac{3}{10} - 3
				\end{align*}
				\begin{center}
				Wir wollen: $\frac{3}{10} n - 3 \overset{!}{\geq} \frac{1}{4} n \Rightarrow n \geq 60$. \hfill$\square$
				\end{center}
			\end{itemize}
		\hspace{2cm}
		\begin{minipage}{0.7\textwidth}
		\begin{algorithmic}
		\STATE \textbf{Algorithm} $\operatorname{Select}(S, k)$
			\IF{$|S| < 100$} %
				\STATE $\operatorname{brute\_force}$
			\ENDIF
			\STATE \rnode{SPLITTER_B}{}\vspace*{-1em}
			\STATE Unterteile $S$ in 5er-Gruppen
			\STATE $S \gets$ \{Median jeder 5er-Gruppe\}
			\STATE \rnode{SPLITTER_E}{}$q \gets \operatorname{Select}\left(S', \left\lceil\frac{|S'| + 1}{2}\right\rceil\right)$
			\STATE $S_< \gets \{s \in S\ |\ s < q\}$
			\STATE $S_> \gets \{s \in S\ |\ s > q\}$
			\IF{$\left|S_<\right| \geq k$} %
				\RETURN $\operatorname{Select}(S_<,k)$
			\ELSIF{$\left|S_<\right| = k - 1$} %
				\RETURN $q$
			\ELSE %
				\RETURN $\operatorname{Select}(S_>,k - \left|S_<\right| - 1)$
			\ENDIF
		\end{algorithmic}
		\end{minipage}
		\psbrace*[rot=180,nodesepA=-1.8cm,nodesepB=0.5em](SPLITTER_B)(SPLITTER_E){SPLITTER}
		\paragraph*{Laufzeitanalyse}
		\[T(n) \leq \begin{cases}
		             O(1), & n < 100 \\
					 O(n) + T\left(\left\lceil\frac{n}{5}\right\rceil\right) + T\left(\frac{3}{4} n\right), & \text{sonst}
		            \end{cases}\]
		\Beh $T(n) = O(n)$
		\paragraph*{Beweis durch Induktion}
			\begin{description}
			 \item[Behauptung] $\exists$ Konstante $\alpha > 0$, so dass $T(n) < \alpha n$ ist.
			 \item für $n < 100$: $\checkmark$
			 \item[Induktionsschritt] 
				\begin{align*}
				 T(n) &\leq cn + T\left(\left\lceil\frac{n}{5}\right\rceil\right) + T\left(\frac{3}{4} n\right) \\
					  &\overset{\text{IA}}{\leq } cn + \alpha \left\lceil\frac{n}{5}\right\rceil + \alpha \frac{3}{4} n\\
					  &\leq cn + \alpha \left(\frac{n}{5} + 1\right) + \alpha \frac{3}{4} n \\
					  &= cn + \alpha \left(\frac{1}{5} +  \frac{3}{4}\right)n + \alpha \\
					  &= cn + \frac{19}{20} \alpha n + \alpha \\
					  &\overset{!}{\leq} \alpha n
				\end{align*}
				\begin{align*}
				 \alpha n &\geq cn + \frac{19}{20} \alpha n + \alpha\\
				 \frac{1}{20} \alpha n & \geq cn + \alpha &&| : \frac{n}{20} \\
				 \alpha & \geq 20c + \underbrace{\frac{20\alpha}{n}}_{\leq \frac{\alpha}{5}}
				\end{align*}
				Es gilt: $20 c + \frac{\alpha}{5} \geq 20c + \frac{20\alpha}{n}$ \\
				D. h., wenn:
					\begin{align*}
					(*)\ \alpha &\geq 20c + \frac{\alpha}{5}, \text{dann} \\
					\alpha &\geq 20c + \frac{20\alpha}{n}
					\end{align*}
					(*) gilt, wenn $\alpha \geq 25c$ ist \hfill$\square$
			\end{description}
		Algorithmus:
		\begin{center}
		 \begin{tabular}{ll}
			\textsc{Blum} & Turing-Award 1995 \\
			\textsc{Floyd} & Turing-Award 1978 \\
			\textsc{Pratt} & --- \\
			\textsc{Rivest} & Turing-Award 2002 \\
			\textsc{Tarjan} & Turing-Award 1986 
		 \end{tabular}
		\end{center}
		\paragraph*{Bemerkung:}
		\begin{itemize}
		 \item Algorithmus kurz, elegant, optimal.
		 \item Technik: \emph{Prune \& Search} (Abschneiden \& Suchen), eliminiert in jedem Schritt Anteil der Kandidaten
		 \item Benutzt nicht triviale Struktur im Problem.
		 \item Laufzeiteigenschaften nicht offensichtlich, brauchen Analyse und Beweis.
		 \item Theoretisches Ergebnis.
		\end{itemize}

\end{description}

