\section{Graphenalgorithmen}
\begin{description}
\item[Gegeben:] Graph $G = (V,E)$, Zusatzinformationen
\item[Ziel:] Berechne etwas Interessantes über $G$
\item[Beispiele:]
    \begin{itemize}
    \item Durchsuchen von Graphen: BFS, DFS
    \item kürzeste Wege: Dijkstra, Bellman-Ford, Floyd-Warshall, $A^*$, ...
    \item minimale aufgespannte Bäume: Kruskal, Prim-Janik, Borůvka, ...
    \item topologisches Sortieren
    \end{itemize}
\end{description}

\subsection[Flussproblem]{Flussproblem}
\Bsp UdSSR: Problem Transportiere Zement von Nowosibirsk nach Moskau
\begin{center}
    \vspace{1em}
    \begin{psmatrix}[mnode=dot,rowsep=1cm]
          & O & T \\
        N & A & x & M \\
          & J & y
    \end{psmatrix}
    \vspace{1em}
    \nput{90}{1,2}{O}
    \nput{90}{1,3}{T}
    \nput{180}{2,1}{N}
    \nput{45}{2,2}{A}
    \nput{0}{2,4}{M}
    \nput{-90}{3,2}{J}
    \ncarc{->}{1,2}{1,3}\naput{5}
    \ncline{->}{1,2}{2,2}\naput{4}
    \ncarc{->}{1,3}{1,2}\naput{10}
    \ncline{->}{1,3}{2,3}\naput{15}
    \ncline{->}{1,3}{2,4}\naput{10}
    \ncline{->}{2,1}{1,2}\naput{10}
    \ncline{->}{2,1}{2,2}\naput{5}
    \ncline{->}{2,1}{3,2}\nbput{15}
    \ncline{->}{2,2}{2,3}\naput{6}
    \ncline{->}{2,2}{3,2}\naput{6}
    \ncline{->}{2,3}{2,4}\naput{10}
    \ncline{->}{3,3}{2,2}\nbput{6}
    \ncline{->}{3,3}{2,3}\nbput{15}
    \ncline{->}{3,3}{2,4}\nbput{10}
    \ncline{->}{3,2}{3,3}\nbput{30}
\end{center}
\begin{itemize}
\item   Strecken haben \emph{Kapazitäten}
\item   wollen so viel Zement wie möglich von $N$ nach $M$\\
        $\Rightarrow$ Flussproblem
\end{itemize}
USA (Klassenfeind)
\begin{itemize}
\item   Verhindere Transport, schneide $N$ von $M$ ab
\item   Je breiter die Strecke, desto teurer die Unterbrechung\\
    $\Rightarrow$ MIN-$s$-$t$-cut
\end{itemize}
Formalisierung (Flussnetzwerk):
\begin{description}
\item[Gegeben:]
    \begin{itemize}
    \item   Graph $G = (V, E)$ gerichtet
    \item   $s,t \in V, s \neq t$
            \begin{description}
            \item   $s$: Quelle
            \item   $t$: Abfluss/Senke
            \end{description}
    \item   $C{:}\ V \times V \to \mathbb{R}^+_0$, mit $c(u,v) = 0$, wenn $(u,v) \in E$.
    \item   Kapazitäten
    \end{itemize}
\item[Fluss:] Funktion $f{:} V \times V \to \mathbb{R}$, so dass
    \renewcommand{\theenumi}{\roman{enumi}}
    \begin{enumerate}
    \item   $\forall u,v \in V{:}\ f(u,v) \leq c(u,v)$ (Kapazitätseigenschaft)
    \item   $\forall u,v \in V{:}\ f(u,v) = -f(v,u)$ (Antisymmetrie)
    \item   $\forall u \in V \setminus \{s,t\}{:}\ \sum\limits_{v \in V} f(u,v) = 0$
    \end{enumerate}
\item[Bemerkungen:]
    \begin{itemize}
    \item   $f \equiv 0$ ist Fluss
    \item   (i) und (ii) $\Rightarrow f(u,v) = 0$, wenn $(u,v) \in E$ und $(v,u) \in E$
    \item   (ii) $\Rightarrow f(u,u) = 0, \forall u \in V$  
    \end{itemize}
\end{description}
Wert des Flusses $f$
\[|f| := \sum\limits_{v \in V} f(s,v)\]
\textbf{Ziel:} Finde $f$ mit $|f|$ maximal.
\Lemma $|f| = \sum\limits_{v \in V} f(v,t)$
\Bew Betrachte 
\[\sum\limits_{(u,v) \in V \times V} f(u,v) = \sum\limits_{\{u,v\} \in \binom{V}{2}} (f(u,v) - f(v,u)) = 0\]
Also:
\begin{align*}
0 &= \sum\limits_{(u,v) \in V \times V} f(u,v) = \sum\limits_{v \in V} f(s,v) +  \sum\limits_{v \in V} f(t,v)
    + \ovalnode[linecolor=red]{lst}{\sum\limits_{\begin{subarray}(u,v) \in V \times V\\ u \neq s,t\end{subarray}} f(t,v)}\\
  &= \sum\limits_{(u,v) \in V \times V} f(u,v) = \sum\limits_{v \in V} f(s,v) +  \sum\limits_{v \in V} f(t,v)
    + \underbrace{\sum\limits_{u \neq s,t} \sum\limits_{v \in V} f(u,v)}_{= 0} \\
  &= \sum\limits_{(u,v) \in V \times V} f(u,v) = \sum\limits_{v \in V} f(s,v) +  \sum\limits_{v \in V} f(t,v)
\end{align*} \hfill $\square$
\begin{description}
\item[Gegeben:]
    \begin{itemize}
    \item   $G = (V, E)$ Flussnetzwerk
    \item   $s$-$t$-Schnitt $C = (A,B)$
            \[A,B \subseteq V{;}\ A \cap B = \emptyset; A \cup B = V, s \in A, t \in B\]
    \end{itemize}
    Kapazität von $C$:
    \[K(C) := \sum\limits_{u,v \in A \times B} c(u,v)\]
\item[Gesucht:]
    Schnitt $C$ mit minimaler Kapazität.
\end{description}
\Lemma (schwache Dualität)
    Seien $C$ ein Schnitt, $f$ ein Fluss, dann: $|f| \leq K(C)$
\Bew Betrachte
    \[\underbrace{\sum\limits_{(u,v) \in A \times B} f(u,v)}_{\text{Totaler Ausfluss von $A$.}} \leq \sum\limits_{(u,v) \in A \times B} c(u,v) = K(C).\]
\begin{itemize}
\item   noch zu zeigen: $|f| \leq $ totaler Ausfluss aus $A$
\item   Abkürzung: Ersetze $B$ durch Knoten $t'$.
        \[|f| = \sum\limits_{v \in V} f(v,t') \leq \text{Ausfluss aus $A$}\]
\end{itemize}

    