\paragraph*{Lösung mit dynamischen Programmieren}
\begin{enumerate}[start=0]
\item Problem verstehen (\textbf{Satz von Bayes})
     \[\operatorname{Pr}[A|B] = \frac{\operatorname{Pr}[B|A] \cdot \operatorname{Pr}[A]}{\operatorname{Pr}[B]}\]
     Also:
     \begin{align*}
     &\operatorname{Pr}[\underbrace{\text{Zustandsfolge $q_0...q_{l-1}$}}_{A}\ |\ \underbrace{\text{Beobachtung $\sigma_0...\sigma_{l-1}$}}_{B}] \\&= \frac{\operatorname{Pr}[\text{Beobachtung $\sigma_0...\sigma_{l-1}$}\ |\ \text{Zustandsfolge $q_0...q_{l-1}$}] \cdot \operatorname{Pr}[\text{Zustandsfolge $q_0...q_{l-1}$}]}{\operatorname{Pr}[\text{Beobachtung $\sigma_0...\sigma_{l-1}$}]}\\
     &\underset{q_0...q_{l-1}}{\operatorname{argmax}} \frac{\operatorname{Pr}[\text{Beobachtung $\sigma_0...\sigma_{l-1}$}\ |\ \text{Zustandsfolge $q_0...q_{l-1}$}] \cdot \operatorname{Pr}[\text{Zustandsfolge $q_0...q_{l-1}$}]}{\operatorname{Pr}[\text{Beobachtung $\sigma_0...\sigma_{l-1}$}]} \\
     &= \underset{q_0...q_{l-1}}{\operatorname{argmax}} \operatorname{Pr}[\text{Beobachtung $\sigma_0...\sigma_{l-1}$}\ |\ \text{Zustandsfolge $q_0...q_{l-1}$}] \cdot \operatorname{Pr}[\text{Zustandsfolge $q_0...q_{l-1}$}]\\
     &= \underset{q_0...q_{l-1}}{\operatorname{argmax}}\ o(q_0,\sigma_0) \cdot ... \cdot o(q_{l-1},\sigma_{l-1}) \cdot a(q_0) \cdot 
         t(q_0,q_1) \cdot t(q_1,q_2) \cdot ... \cdot t(q_{l-2}, q_{l-1})
     \end{align*}
\item Teilprobleme finden:
\begin{center}
    \begin{tabular}{rp{0.6\linewidth}}
        $q \in Q, j = 0, ..., l-1{:}\ E[q,j] =$&$\max$ Wahrscheinlichkeit einer Zustandsfolge, die in $q$ endet und Länge $j-1$ hat und die Ausgabe $\sigma_0...\sigma_i$ erzeugt.
     \end{tabular}
\end{center}
\begin{align*}
 &= \underset{\underset{q_i = q}{q_0,q_1,...,q_i}}{\max} o(q_0,\sigma_0) \cdot o(q_1,\sigma_1) \cdot ... \cdot o(q_j, \sigma_j)
     \cdot a(q_0) \cdot t(q_0,q_1) \cdot ... \cdot t(q_{j-1}, q_j)
\end{align*}
\paragraph*{Ziel:} Finde $\underset{q \in Q}{\max}\ E[q, l-1]$
\item Rekursion:
 \begin{align*}
  \forall q \in Q{:}\ E[q, 0] &= a(q) \cdot o (q, \sigma_0) \\
                      E[q, j] &= \max\limits_{r \in Q}\ E[r, j-1] \cdot t(r,q) \cdot o(q,\sigma_j)
 \end{align*}
\end{enumerate}

\paragraph*{Zurück zum Beispiel:}
\begin{itemize}
 \item $Q = \{\text{R}, \text{S}\}$
 \item $\Sigma = \{\text{{\color{blue}N}ass},\text{{\color{blue}T}rocken}\}$
\end{itemize}
\begin{center}
 \psset{arrows=->}
 \begin{tabular}{c||c|c}
  & R & S \\\hline\hline
  $\sigma_0 =$ N & \rnode{0R}{0{,}56} & \rnode{0S}{0{,}09} \\\hline
  $\sigma_1 =$ T & \rnode{1R}{0{,}0612} & \rnode{1S}{0{,}1568} \\\hline
  $\sigma_2 =$ T & \rnode{2R}{} & \rnode{2S}{} \\\hline
  $\sigma_3 =$ N & \rnode{3R}{} & \rnode{3S}{} \\\hline
  $\sigma_4 =$ T & \rnode{4R}{} & \rnode{4S}{} \\\hline
  $\sigma_5 =$ N & \rnode{5R}{} & \rnode{5S}{} \\\hline
  $\sigma_6 =$ N & \rnode{6R}{} & \rnode{6S}{}
\end{tabular}\ncline{1R}{0R}\ncline{1S}{0R}
\end{center}
\begin{description}
 \item[Speicherplatz:] $\Theta(|Q| \cdot l)$
 \item[Zeit:] $\Theta(|Q|^2 \cdot l)$
\end{description}


