\begin{itemize}
 \item "`effiziente"' Akgorithmen Polynomialzeitalgorithmen
 \item Sind alle Probleme effizient lösbar? Natürlich nicht, z. B. Halteproblem, PKP, ...
 \item Wie kann man das feststellen? Nicht viel ist sicher bekannt, aber es existiert einegroße Klasse von Problemen, die eng zusammenhängen und die wahrscheinlich nicht effizient lösbar sind.
\end{itemize}
\Bsp
\begin{itemize}
\item   \emph{Einkaufsproblem (Knapsack)}
        \Geg $n$ Artikel, Preise $p_1, ..., p_n$, Werte $w_1, ..., w_n$, Budget $B$
        \Ges Teilmenge von Artikeln m it maximalem Wert, die Budget einhält
\item   \emph{TSP}
        \Geg $n$ Städte mit Abstand $d(i,j)$
        \Ges Rundreise minimaler Länge
\item   \emph{Rudrata-Pfad (Hamilton-Pfad)}
        \Geg Graph $G = (V,E)$
        \Ges Pfad, der jeden Knoten genau einmal besucht
\item   \emph{Partition}
        \Geg $n$ Zahlen $x_1,...,x_n$
        \Ges $I \subseteq \{1, ..., n\}$, so dass
            \[\sum\limits_{i \in I} x_i = \sum\limits_{j \in \{1, ..., n\} \setminus I} x_j\]
\item   \emph{Subset-Sum}
        \Geg $n$ Zahlen $x_1,...,x_n$. Weitere Zahl $X$
        \Ges $I \subseteq \{1, ..., n\}$, so dass
            \[\sum\limits_{i \in I} x_i = X\]
\end{itemize}
Alle diese Probleme sind NP-vollständig, d. h. wenn ein effizienter Algorithmus für ein Problem existiert, dann existieren effiziente Algorithmen für alle obigen Probleme.
\par Was heißt das genau?
\begin{enumerate}
 \item NP-Vollständigkeitstheorie betrachtet nur Entscheidungsprobleme, d. h. die Ausgabe ist immer nur ein Bit (\emph{ja} oder \emph{nein}) (= Sprache)
 
 Zu jedem Optimierungsproblem existiert ein zugehöriges Entscheidungsproblem (existiert eine Lösung mit Wert $\gtreqless x$?). Oft ist es so, dass man mit binärer Suche o. ä. das Optimierungsproblem effizient lösen kann, wenn man das Entscheidungsproblem effizient lösen kann.

 Von nun an betrachten wir nur Entscheidungsprobleme.
 \item Für alle obigen Probleme gilt: Wenn eine Antwort gegeben ist, so lässt sie sich effizient nachprüfen
 
 Es gilt: für alle obigen Probleme gibt es einen kurzen Beweis (= Zertifikat), dass die Antwort des Entscheidungsproblems \emph{ja} ist. Das Zertifikat kann man effizient überprüfen
 
 \Defi \emph{NP} (Nichtdeterministisch polynomiell) ist eine Komplexitätsklasse (Menge an Sprachen).
     Sei $L \subseteq \{0,1\}^*$ eine Sprache. $L$ ist in $NP$ gdw. $\exists$ Konstante $c \geq 0$, Turingmaschine. $M$ mit zwei Eingaben, so dass
     \renewcommand{\labelenumii}{(\theenumii)}
     \renewcommand{\theenumii}{\roman{enumii}}
     \begin{enumerate}
     \item Laufzeit von $M(x,y)$ ist $O((|x|+|y|)^c)$
     \item $\forall x \in \Sigma^*{:}\ x \in L \Leftrightarrow \exists y \in \Sigma^*{:}\ |y| \leq |x|^c \land M(x,y) = 1$
     \end{enumerate}
     $M$ ist Verifizierer für $L$.
\end{enumerate}
\begin{itemize}
 \item Offensichtlich: $P \subseteq NP$
 \item Frage: $P = NP$? Wie könnte man das beweisen?
 \item Idee: Finde ein Problem in NP, das "`so schwer wie möglich"' ist. Problem $L \in$ NP, so dass gilt, wenn $L \in P$, so ist $P = NP$
     \par Wie geht das?
     \par Polynomialzeitreduktion (Karp-Reduktion)
     \par Seien $L_1, L_2$ zwei Sprachen.
         Wir sagen $L_1 \leq_P L_2$ ($L_1$ reduziert in Polynomialzeit auf $L_2$) gdw. $\exists$ Funktion $f{:}\ \Sigma^* \to \Sigma^*$, so dass
         \begin{itemize}
          \item $f$ in Polynomialzeit berechenbar ist
          \item $x \in L_1 \Leftrightarrow f(x) \in L_2$
         \end{itemize}
         \textbf{Beobachtungen}
         \begin{enumerate}
          \item $\leq_P$ ist transitiv. Wenn $L_1 \leq_P L_2$ und $L_2 \leq_P L_3$, dann $L_1 \leq_P L_3$
          \item Wenn $L_1 \leq_P L_2$ und $L_2 \in P$, dann $L_1 \in P$
          \item Wenn $L_1 \leq_P L_2$ und $L_1 \notin P$, dann $L_2 \notin P$.
         \end{enumerate}
\end{itemize}
\Bsp IND-SET:
\Geg $G = (V,E), i$
\Ges $\exists i$ Knoten in $G$, die paarweise nicht adjazent sind?\\[1.5em]
\par betrachten Vertex-Cover VERTEX-COVER
\Geg Graph $G = (V,E), c$
\Ges $\exists$ Knoten $c$, so dass jede Kante inzident zu einem Knoten $C$ ist?

\paragraph*{Behauptung} IND-SET $\leq_P$ VERTEX-COVER\\
$f$: Eingabe: $G = (V,E), i$; Ausgabe: $G = (V,E), |V| - i$
\par z. z. $G$ hat unabhängige Menge der Größe $i$ gdw. $G$ hat Vertex-Cover der Größe $|V|-i$
\begin{align*}
 v \supseteq I\text{ ist IND-SET} &\Leftrightarrow \forall e \in E{:}\ |I \cap \{e\}| \leq 1 \\
              &\Leftrightarrow \forall e \in E{:}\ |(V \setminus I) \cap \{e\}| \geq 1 \\
              &\Leftrightarrow V \setminus I\ \text{VEREX-COVER}
\end{align*}\hfill$\square$

\Defi $L \subseteq \{0,1\}^*$ heißt NP-vollständig, wenn
\renewcommand{\labelenumi}{(\theenumi)}
\renewcommand{\theenumi}{\roman{enumi}}
\begin{enumerate}
 \item $L \in$ NP
 \item $\forall L' \in \text{NP}{:}\ L' \leq_P L$
\end{enumerate}
\paragraph*{Bemerkungen}
\begin{itemize}
 \item $L$ NP-vollständig und $L \in \text{P} \Rightarrow \text{P} = \text{NP}$
 \item $L$ NP-vollständig und $L \notin \text{P} \Rightarrow$ Kein NP-vollständiges Problem ist in P, also $\text{P} \neq \text{NP}$
\end{itemize}
\begin{description}
 \item[Frage 1:] Existieren NP-vollständige Probleme?
 \par Ja!
 \begin{description}
  \item[Eingabe] $M$ Turingmaschine. $x \in \{0,1\}^*$, $1^q$
  \item[Ausgabe] $\exists y{:}\ |y| \leq |x|^q$, so dass $M(x,y) = 1$ in $q$ Schritten?
 \end{description}
 \item[Frage 2:] Existieren Probleme, die \emph{nicht} NP-vollständig sind?
 \par Ja! $L = \emptyset$, $L = \{0,1\}$, Halteproblem
 \item[Frage 3:] Existieren \emph{interessante} Probleme, die NP-vollständig sind?
 \par Ja! CIRCUIT-SAT: Erfüllbarkeitsproblem für Schaltnetze
     \begin{description}
      \item[Eingabe] Schaltnetz (Gerichteter, azyklischer Graph mit $n$ Eingaben, 1 Ausgabe, innere Knoten haben Eingrad 1 oder 2 und entsprechen AND-Gatter, OR-Gatter oder NOT-Gatter)
      \item[Frage] Existiert Belegung der Eingaben, so dass der Schaltkreis 1 liefert?
      \item[Satz] (\textsc{Cook}-\textsc{Levin}): CIRCUIT-SAT ist NP-vollständig.
      \item[Beweis] später
     \end{description}
\end{description}
CIRCUIT-SAT eignet sich gut für Reduktion.
\begin{description}
 \item[CNF-SAT] Erfüllbarkeitsproblem für Aussage logischer Formeln in Konjunktiver Normalform = Konjunktion von Disjunktionen von Literalen
 \[(x_1 \lor \neg x_2 \lor x_3) \land (x_2 \lor \neg x_3 \lor x_4) \land (\neg x_1)\]
 Literal: $x_i$ oder $\neg x_i$
 \begin{description}
  \item[Frage] Existiert Belegung er Variablen, so dass Formel wahr ist?
  \item[Satz] CNF-SAT ist NP-vollständig
  \item[Beweis] CNF-SAT $\in$ NP
  \begin{itemize}
   \item Es genügt z. z. CIRCUIT-SAT $\leq_P$ CNF-SAT (denn CIRCUIT-SAT ist NP-vollständig und $\leq_P$ ist transitiv)
   \item Brauchen $f$: Eingabe: Schaltnetz $C$, Ausgabe: $\Phi$ in KNF
   \item Wollen $C$ erfüllbar $\Leftrightarrow$ $\Phi$ erfüllbar
   \item Gegeben: Schaltnetz, führe ein: fpr jede Kante $v_1, v_2, ...$
   \item Klausel n: 1 Klausel für die Ausgabe:
           \[v_m \leftarrow\text{ Variable an der Ausgabekante}\]
       Betrachte jeden Knoten (Gatter)
   \item Betrachte Fall mit nur einer Ausgabe ($v_1$ und $v_2$ gehen durch AND-Gatter in $v_3$)
   \item Menge von Klauseln, die genau dann alle wahr sind, wenn $v_3 \equiv v_1 \land v_2$ ist
       \[
        \begin{array}{ccc|cr}
            v_1 & v_2 & v_3 & v_3 \equiv v_1 \land v_2 \\\cline{1-4}\cline{1-4}
            0   & 0   & 0   & 1\\\cline{1-4}
            0   & 0   & 1   & 0 & (v_1 \lor v_2 \lor \neg v_3) \\\cline{1-4}
            0   & 1   & 0   & 1 \\\cline{1-4}
            0   & 1   & 1   & 0 & \land (v_1 \lor \neg v_2 \lor \neg v_3) \\\cline{1-4} 
            1   & 0   & 0   & 1 \\\cline{1-4}
            1   & 0   & 1   & 0 & \land (\neg v_1 \lor v_2 \lor \neg v_3) \\\cline{1-4} 
            1   & 1   & 0   & 0 & \land (\neg v_1 \lor \neg v_2 \lor v_3) \\\cline{1-4} 
            1   & 1   & 1   & 1 \\\cline{1-4}
        \end{array}
       \]
   \item Wiederhole für jedes Gatter und jeden Ausgang eines Gatters
   \item $\Phi = 1$ alle Klauseln, die so entstanden sind.
   \item Polyzeit: Machen konstante Arbeit per Ausgabekante aus einem Gatter $\Rightarrow$ $O$(\#{}Kanten)
   \item Außerdem: $C$ erfüllbar
       \begin{description}
       \item $\Leftrightarrow$ $\exists$ Bits an den Kanten, die Gatter respektieren und Ausgabe 1 liefern
       \item $\Leftrightarrow$ $\exists$ Belegung der $v_i$, die alle Klauseln wahr macht
       \item $\Leftrightarrow$ $\emptyset$ erfüllbar \hfill $\square$
       \end{description}
  \end{itemize}
 \end{description}
 \item[3-SAT] CNF-SAT, aber alle Klauseln enthalten nur 3 Literale
     \begin{description}
      \item[Satz] 3-SAT ist NP-vollständig
      \item[Beweis]
      \begin{itemize}
       \item 3-SAT $\in$ NP-vollständig (wie bei CNF-SAT)
       \item CNF-SAT $\leq_P$ 3-SAT
       \item Wollen $f:$
           \begin{description}
            \item[Eingabe] $\Phi$ in KNF
            \item[Ausgabe] $\Phi'$ in 3-KNF, si dass $\Phi$ erfüllbar gdw. $\Phi'$ erfüllbar
           \end{description}
       \item $f$ benutzt lokales Ersetzen, ersetze die Klauseln von $\Phi$ einzeln
       \renewcommand{\theenumi}{\arabic{enumi}}
       \begin{enumerate}
        \item Klauseln $\{l\}$, Ersetze $\{l\}$ durch $\{l \lor a \lor b\}\{l \lor \neg a \lor b\}\{l \lor a \lor \neg b\}\{l \lor \neg a \lor \neg b\}$, $a,b$ neue Variablen
        \par Es gilt: $\{l\}$ ist wahr $\Leftrightarrow$ alle neuen Klauseln sind war
        \item Klauseln: $\{l_1 \lor l_2\}$ Ersetze durch $\{l_1 \lor l_2 \lor a\}, \{l_1 \lor l_2 \lor \neg a\}$
        \item Klauseln: $\{l_1 \lor l_2 \lor l_3\}$ nichts zu tun
        \item Klauseln: $\{l_1 \lor l_2 \lor ... \lor l_n\}, k > 3$ Ersetze durch $\{l_1 \lor l_2 \lor a_1\}\{\neg a \lor l_3 \lor a_2\}\{\neg a_1 \lor l_4 \lor a_3\}...$ $a_i$ neu.
       \end{enumerate}
      \end{itemize}
     \end{description}

\end{description}






