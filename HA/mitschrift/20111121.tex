\section[Greedy Algorithmen]{Greedy (dt. gierige) Algorithmen}
Algorithmus triff lokal optimale Entscheidung in der Hoffnung, dass diese Entscheidung zu einer global optimalen Lösung führt.

\subsection{Beispiel: Vorlesungsplanung / Intervall-Auswahl}
\Geg     $n$ Intervalle $R = {I_1, I_2, ..., I_n}$
         \[I_i = (a(i), e(i))\]
\Ges     Größte Teilmenge $A \subseteq R$, so dass sich keine Intervalle überlappen.
         Es gilt also:
         \[ \forall I_i, I_j \in A{:}\ a(i) \geq e(j) \lor a(j) \geq e(i) \]
\paragraph*{Grundidee:} Beginne mit $A \neq \emptyset$ und füge so lange wie müglich Intervalle aus $R$ in $A$ ein, die kein Intervall aus $A$ schneiden.

\paragraph*{Regel zur Auswahl der Intervalle:} Nimm Intervall mit dem zeitigsten Ende
\begin{algorithmic}
 \STATE $A \gets \emptyset$
 \WHILE{$R \neq \emptyset$}
 \STATE wähle aus $R$ das Intervall mit minimalen $e(i)$, füge es in $A$ ein
 \STATE entferne alle Intervalle $I_j$ aus $R$ mit $a(j) < e(i)$
 \ENDWHILE
\end{algorithmic}

\paragraph*{Korrektheit} Sei $S$ eine optimale (größtmögliche) Teilmenge von $R$. Wir wollen zeigen: 
\[|A| = |S|\]
Seien $i_1,...,i_k$ und $j_1,...,j_m$ die Indizes der Intervalle von $A_i$ bzw. $S_i$ in zeitlich geordneter Reihenfolge.

\paragraph*{Behauptung:} $\forall r \in \{1,...,n\}{:}\ e(i_r) \leq e(j_r)$

\paragraph{Induktionsbeweis:}
\begin{description}
 \item[IA] $r = 1$ $e(i_1) \leq e(j_1)$ klar, denn der Greedy-Algorithmus wählt das Intervall mit minmalem $e(i)$
 \item[IS] ($r \Rightarrow r + 1$)\\
     Es gilt $e(i_r) \leq e(j_r)$ (nach Induktionsbehauptung).\\
     Nach Auswahl von $I_{i_r}$ steht das Intervall $I_{j_{r+1}}$ noch zur Verfügung, da $e(i_r) \leq e(j_r) \leq a(j_{r+1})$\\
     $\Rightarrow$ das nächste Intervall in $A$ ist eins mit $e(i_{r+1}) \leq e(j_{r+1})$ \hfill \carsten\\
     Also gilt $k \geq m$ und damit $|A| \geq |S|$.
\end{description}

\paragraph*{Widerspruchsbeweis:} Angenommen $k < m$, dann gilt $e(i_k) \leq e(j_k)$ und es gibt noch ein Intervall $I_{j_{k+1}}$. Also enthält $R$ noch $I_{j_{k+1}}$ nachdem die intervalle $I_{i_n},...,I_{i_k}$ ausgewählt wurden. Das heißt der Algorithmus ist terminiert, bevor $R = \emptyset$

\paragraph*{Konkrete Implementierung:}
\begin{enumerate}
 \item Sortiere $R$ bzgl. $e(i)$, danach gilt $e(1) \leq e(2) \leq ... \leq e(n)$
 \item Bilde Array $B = \{a(1), a(2), ..., a(n)\}$
 \item Aufbau von $A$
 \begin{algorithmic}
  \STATE $A = \emptyset$
  \STATE $i = 1$
  \WHILE{$i \leq n$}
  \STATE füge $I_i$ in $A$ ein
  \STATE $j \gets i + 1$
  \WHILE{$a(j) \leq e(i)$ und $j \leq n$}
  \STATE $j \gets j+1$
  \ENDWHILE
  \STATE $i = j$
  \ENDWHILE
 \end{algorithmic}
\end{enumerate}

\paragraph*{Laufzeit}
\begin{enumerate}
 \item $\Theta(n \log n)$
 \item $O(n)$
 \item $O(n)$
\end{enumerate}
\[T(n) = \Theta(n \log n)\]

\subsection{Beispiel: Intervallunterteilungsproblem}
\Geg $n$ Intervalle $I_1, I_2,...,I_n$
\Ges Minimale Anzahl von Labeln, so dass jedes Intervall ein Label hat und Intervalle nur dann das gleiche Label haben, wenn sie sich nicht überlappen.

\paragraph*{Grundidee:} Teste für jedes Intervall, ob ein bereits vorhandenes Label benutzt werden kann. Wenn nicht füge neues Label in die Labelmenge ein.

\paragraph*{Regel zum Durchlaufen der Intervalle:} Nimm Intervalle mit dem zeitigsten Start
\begin{algorithmic}
 \STATE Sortiere die Intervalle nacht ihrem Startpunkt. Sei die Reihenfolge $I_1, I_2, ..., I_n$
 \STATE $i \gets 1$
 \FOR{$i \in \{1, ...,n\}$}
  \FOR{jedes Intervall $I_k$, das vor $I_j$ startet und $I_j$ überlappt}
      \STATE Schließe Label von $I_k$ für $I_j$ aus
  \ENDFOR
  \IF{es gibt ein Label $\{1,...,i\}$, das nicht ausgeschlossen wurde}
      \STATE weise das Label $I_j$ zu
  \ELSE
      \STATE weise $I_j$ das Label $i+1$ zu
      \STATE $i \gets i + 1$
  \ENDIF 
 \ENDFOR
\end{algorithmic}
\[T(n) = O(n \log n)\]

\paragraph*{Korrektheit:} Sei $d$ die Tiefe der Menge der Intervalle, d. h. die maximale Anzahl an Intervallen, die zu einem bestimmten Zeitpunkt parallel laufen.
\Beh Die Anzahl der benötigten Label ist mindestens $d$.
\Bew Es gibt einen Zeitpunkt zu dem $d$ Intevalle gleichzeitig laufen. Jedes dieser Intervalle benötigt ein eigenes Label.
\Beh Der Algorithmus benutzt höchstens $d$ Label.
\Bew Betrachte ein Intervall $I_j$ für das ein neues Label eingefügt wird. Sei $t$ die Anzahl der Intervalle, die vor $I_j$ starten und $I_j$ überlappen. Es gibt $t + 1 \leq d$, d. h. $t = d - 1$. Die Menge der Label ist höchstens $d$. \hfill $ \square$ 