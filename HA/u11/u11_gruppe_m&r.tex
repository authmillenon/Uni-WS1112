\documentclass[a4paper,10pt]{article}
\usepackage[utf8]{inputenc}
\usepackage[T1]{fontenc}
\usepackage{amsmath,amsfonts,amssymb,amscd,amsthm,xspace}
\usepackage[ngerman]{babel}
\usepackage{listingsutf8}
\usepackage{color}
\usepackage{geometry}
\usepackage{graphicx}
\usepackage{multicol}
\usepackage{pst-tree}
\usepackage{algorithmic}
\usepackage{cancel}

\geometry{a4paper, left=2cm,right=2cm,top=2cm,bottom=2cm}

\newcommand{\Authors}{Martin Lenders (Di. 14-16), Ralf M\"uller-Zimmermann (Di. 14-16)}
\title{H\"ohere Algorithmik - 10. \"Ubungsblatt}
\author{\Authors}
\date{\today}

\newcommand{\changefont}[3]{\fontfamily{#1} \fontseries{#2} \fontshape{#3} \selectfont}

\renewcommand{\thesection}{Aufgabe \arabic{section}:}
\renewcommand{\labelenumi}{(\theenumi)}
\renewcommand{\theenumi}{\alph{enumi}}
\renewcommand{\labelenumii}{(\theenumii)}
\renewcommand{\theenumii}{\roman{enumii}}

\definecolor{lgray}{gray}{0.95}
\definecolor{purple}{rgb}{0.498,0,0.3333}
\definecolor{identifier}{rgb}{0,0,0.1}
\definecolor{string}{rgb}{0.192,0,1}
\definecolor{comment}{rgb}{0.25,0.5,0.37}

\pagestyle{myheadings}
\oddsidemargin\oddsidemargin
\markright{\Authors}

\lstset{
	tabsize=4, 
	basicstyle=\footnotesize\fontfamily{pcr}\fontseries{m}\fontshape{n}\selectfont,
	breaklines=true,
	numbers=left,
	emphstyle=\textit, 
	language=Java,
	keywordstyle=\color{purple}\textbf, 
	identifierstyle=\color{identifier},
	stringstyle=\color{string},
	showstringspaces=false,
	escapeinside={((*}{*))},
	commentstyle=\color{comment},
	extendedchars=true,
	inputencoding=utf8/latin1
}
\psset{nodesep=2pt,levelsep=2em,treesep=2em}

\begin{document}

\maketitle

\section{Quake Heaps: Details}
Quake Heaps werden mit Hilfe von Turnierbäumen (auch Link-/Cut-Böume genannt) implementiert. Sie verwalten dazu einen Wald von Turnierbäumen, in deren Blättern die Einträge des Quake Heaps gespeichert sind. 
Die inneren Knoten und Wurzeln verwalten, wie bei Turnierbäumen üblich, auf die Schlüssel der Einträge, wobei ein Knoten immer auf den kleineren Schlüssel seiner maximal 2 Kinder zeigt.
Die Blätter verweisen dann jeweils auf den letzten Knoten im Pfad zur Wurzel, der den gleichen Schlüsselwert hat, wie das Blatt.
Wird ein neues Element mit \verb!insert!$(k,v)$ eingefügt, wird lediglich ein neuer Turnierbaum, der nur aus dem entsprechendem Blatt mit dem Wert $v$ und dem Schlüssel $k$ besteht dem Wald hinzugefügt.

\verb!decrease_key!$(v,k)$ verringert den Schlüssel von Wert $v$ auf $k$. Dazu wird zunächst das Blatt gesucht, in dem der Wert $v$ gespeichert ist. 
Durch den Verweis auf den höchsten Knoten im Pfad zur Wurzel finden wir so den höchsten Knoten $u$, der auf den Schlüssel $k$ verweißt, dessen Sub-Baum wir dann mit einer \verb!cut!$(u)$-Operation des Turnierbaums aus dem Baum entfernen und den Schlüssel dann verringern. 
Den aus dem Turnierbaum herausgeschnittenen Baum fügen wir dem Wald hinzu und verringern den Schlüssel auf $k$.

Im Detail entfernt \verb!cut!$(u)$ einen Sub-Baum aus einem Turnierbaum, dessen Wurzel $u$ ist und dessen Schlüssel sich von seinem Elternknoten unterscheidet. Dadurch entstehen zwei neue Turnierbäume.
Eine Eigenschaft von Turnierbäumen ist nun, dass der minimale Schlüssel eines Turnierbaums immer in seiner Wurzel steht.
Daher können wir nach dem Herausschneiden den Schlüssel verringern, ohne die Turnierbaum-Eigenschaften zu verletzen.


\section{Quake Heaps: Analyse}
\begin{enumerate}
\item   
\item   
\item   
\end{enumerate}

\section{Potentialfunktionen}
\begin{enumerate}
\item   
\item   
\item   
\end{enumerate}

\section{Potentialfunktionen}


\end{document}
