\documentclass[a4paper,10pt]{scrartcl}
\usepackage[utf8x]{inputenc}
\usepackage[T1]{fontenc}
\usepackage{amsmath,amsfonts,amssymb,amscd,amsthm,xspace}
\usepackage[ngerman]{babel}
\usepackage{listingsutf8}
\usepackage{color}
\usepackage{geometry}
\usepackage{graphicx}
\usepackage{multicol}
\usepackage{pst-tree}

\geometry{a4paper, left=2cm,right=2cm,top=2cm,bottom=2cm}

\newcommand{\Authors}{Christian Cikryt (Di. 14-16), Jakob Pfender (Mi. 14-16)}
\title{H\"ohere Algorithmik - 4. \"Ubungsblatt}
\author{\Authors}
\date{\today}

\newcommand{\changefont}[3]{\fontfamily{#1} \fontseries{#2} \fontshape{#3} \selectfont}

\renewcommand{\thesection}{Aufgabe \arabic{section}}
\renewcommand{\labelenumi}{(\theenumi)}
\renewcommand{\theenumi}{\alph{enumi}}
\renewcommand{\labelenumii}{(\theenumii)}
\renewcommand{\theenumii}{\roman{enumii}}

\definecolor{lgray}{gray}{0.95}
\definecolor{purple}{rgb}{0.498,0,0.3333}
\definecolor{identifier}{rgb}{0,0,0.1}
\definecolor{string}{rgb}{0.192,0,1}
\definecolor{comment}{rgb}{0.25,0.5,0.37}

\pagestyle{myheadings}
\oddsidemargin\oddsidemargin
\markright{\Authors}

\lstset{
	tabsize=4, 
	basicstyle=\footnotesize\fontfamily{pcr}\fontseries{m}\fontshape{n}\selectfont,
	breaklines=true,
	numbers=left,
	emphstyle=\textit, 
	language=Python,
	keywordstyle=\color{purple}\textbf, 
	identifierstyle=\color{identifier},
	stringstyle=\color{string},
	showstringspaces=false,
    escapeinside={((*}{*))},
	commentstyle=\color{comment},
	extendedchars=true,
	inputencoding=utf8/latin1
}
\psset{nodesep=2pt,levelsep=2em,treesep=2em}

\begin{document}

\maketitle

\section{Bestimmen des engsten Paares}
Die Annahme, dass alle $x$-Koordinaten in $P$ verschieden sind, ist nur
von Bedeutung, wenn die Punkte anhand ihrer $x$-Koordinaten sortiert
werden sollen. Bei diesem Schritt können die Punkte, deren
$x$-Koordinaten identisch sind, einfach noch zusätzlich anhand ihrer
$y$-Koordinaten sortiert werden. Dies hat keinen Einfluss auf die
Laufzeit.

\section{Rekursionsgleichungen}
\begin{enumerate}
\item	ist trivial: Da die Beschriftungen der $x_v$ paarweise
	verschieden sind und $\mathbb{R}$ eine geordnete Menge ist, gilt:
	\[\exists i \in {1,...,n}, \forall j \in {1,...,n}, i \neq j:
	x_{v_i} < x_{v_j}\]
	Es gibt also ein globales und damit auch mindestens ein lokales
	Minimum.
\item
\item	Finde das absolute Minimum von Spalte $x = \lceil \frac{n}{2}
	\rceil$. Vergleiche es mit seinen Zeilennachbarn. Sind diese größer,
	haben wir ein lokales Minimum gefunden. Ansonsten wiederhole den Vorgang
	mit allen Spalten auf der Seite des kleineren Knotens.

	Dieser Algorithmus hat die Rekursionsgleichung $T(n) =
	T(\frac{n}{2}) + 2 \cdot n$. Hier kann das Mastertheorem mit
	$a=1, b=2, f(n)=2$ angewendet werden:

	\begin{align*}
		 n^{\log_b a} &= n^{\log_2 1} = n^0 = 1 \\
		 f(n) &= \Omega(n^{0+\varepsilon}) \tag{z. B.: für $\varepsilon = \frac{1}{2}$} \\
		 a \cdot f\left(\frac{n}{b}\right) &\leq c \cdot f\left(n\right) \tag*{$\|\ c = \frac{1}{2}$} \\
		1 \cdot 2 \cdot \frac{n}{2} & \leq \frac{1}{2} \cdot 2 \cdot n
	\end{align*}

Es gilt also Fall 3 des MT: $T(n) = \Theta(n)$.
\end{enumerate}

\section{Matrizenmultiplikation}
\begin{enumerate}
\item \begin{description}
	\item[Rekursionsanker:] $P\left[i, i\right] = 0$
	\item[Rekursionsgleichung:] $P\left[i, j\right] = \min\limits_{i \leq k < j}\left\{P\left[i, k\right] + P\left[k + 1, j\right] + M_{i.height} \cdot M_{j.width} \cdot \left( 2 \cdot M_{k.width} - 1 \right) \right\}$
\end{description} 
\item \begin{description}
	\item[Voraussetzung:] Wir speichern alle Matrixdimensionen im
	Array DIMENSIONS. Dieses hat Länge $n+1$, da die Breite einer
	Matrix gleich der Höhe der nachfolgenden Matrix ist.
	Desweiteren benötigen wir eine Optimierungsmatrix OPT. Diese hat
	$i$ auf der horizontalen und $j$ auf der vertikalen Achse. Ein
	Element benötigt für seine Berechnung die Elemente, die genau rechts oder genau über ihm liegen.
\begin{lstlisting}[mathescape=true,numbers=none]
BRACKETS(){
	for(i = 1; i < n; i++){
		OPT[i, i].Val = 0;
	}
	for(k = 1; k < n; k++){
		for(i = 1; i <= (n - k); i++){
			min = maxInt;
			j = i + k;
			for(t = 1; t <= k; t++){
				val = OPT[i, j - t].Val 
				+ OPT[i + t, j].Value + DIMENSIONS[i] 
				* DIMENSIONS[j + 1] 
				* (2 * DIMENSIONS[j - t + 1] - 1);
				if(val < min){
					min = val;
					minIndex = j - t;
				}
			}
			OPT[i, j].Val = min;
			OPT[i, j].Index = minIndex;
		}
	}
	return OPT[1, n].Val;
}
\end{lstlisting}
	\item[Laufzeit:] $O(n^3)$, da die Matrix $n^2$ Einträge hat,
	welche sich jeweils in in $O(n)$ berechnen lassen.
	\item[Platzbedarf:] $O(n^2)$, da dieser von den Maßen von OPT
	abhängt.
	\end{description}
\item 
\begin{lstlisting}[mathescape=true,numbers=none]
OPTIMAL(i,j){
	if(i=j){
		return "(DIMENSIONS_"+i+"="+DIMENSIONS[i]+"$\times$"+DIMENSIONS[i+1]+")";
	return "("+OPTIMAL(i,OPT[i,j].Index)+"*"+OPTIMAL(OPT[i,j].Index+1,j)+")";
}
\end{lstlisting}
\end{enumerate}
\end{document}
