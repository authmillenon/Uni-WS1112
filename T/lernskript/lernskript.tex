\documentclass[a4paper,10pt]{scrartcl}

% Seitenlayout
\usepackage[a4paper,top=1.5cm,right=2.0cm,bottom=2cm,left=2.0cm]{geometry}
\usepackage{fancyhdr}

% Zeichen
\usepackage[ngerman]{babel}
\usepackage[utf8]{inputenc}
\usepackage{amsmath, amsthm, amssymb}
\usepackage{marvosym}
\usepackage[gen]{eurosym}

% Zeichensatz
\usepackage[colorlinks,%
	    citecolor=black,%
	    filecolor=black,%
	    linkcolor=black,%
	    urlcolor=black,%
	    pdftitle = {Mitschrift - CB},%
	    pdfauthor = {Martin Lenders}%
	]{hyperref}
\usepackage{listings}
\usepackage{hyperref}

% Grafik
\usepackage{xcolor}

% Color definitions
\definecolor{lgray}{gray}{0.95}
\definecolor{save}{rgb}{0.498,0,0}
\definecolor{identifier}{rgb}{0,0,0.1}
\definecolor{string}{rgb}{0.192,0,1}
\definecolor{comment}{rgb}{0.25,0.5,0.37}
\definecolor{yellow}{rgb}{1,1,0}
\definecolor{sand}{rgb}{1,1,.75}
\definecolor{red}{rgb}{1,0,0}
\definecolor{melon}{rgb}{1,0.6,.5}
\definecolor{green}{rgb}{0,1,0}
\definecolor{lime}{rgb}{.75,1,.75}
\definecolor{blue}{rgb}{0,0,1}
\definecolor{azure}{rgb}{.75,.75,1}

% Einstellungen für Pakete
\lstset{
	tabsize=8,
	frame=,
	basicstyle=\changefont{pcr}{m}{n},
	emphstyle=\textit,
	numberstyle=\tiny\textsf,
	numbersep=5pt,
	numbers=none,
	keywordstyle=\color{save}\textbf,
	identifierstyle=\color{identifier},
	stringstyle=\color{string},
	showstringspaces=false,
	commentstyle=\color{comment},
	extendedchars=true,
	xleftmargin=1em,
	inputencoding=utf8,
	mathescape=true;
}

\renewcommand{\sectionmark}[1]{\markright{\thesection\ #1}}
\pagestyle{fancy}
\fancyhf{}
\fancyfoot[LE,RO]{\sffamily\thepage}
\fancyhead[LE]{\footnotesize\sffamily\bfseries\leftmark}
\fancyhead[RO]{\footnotesize\sffamily\rightmark} 

% Eigene Befehle
\newcommand{\changefont}[3]{\fontfamily{#1}\fontseries{#2}\fontshape{#3}\selectfont}

% Neudefinitionen
\renewcommand{\sectionmark}[1]{\markright{\thesection\ #1}}

\setlength{\parindent}{0pt}

\title{Lernskript\\\LARGE{}Telematik}
\author{Martin Lenders}

\begin{document}
\maketitle
\tableofcontents
\section{ISO-/OSI-Modell}

\section{Schicht 1 -- Bitübertragungsschicht}
\subsection{Signalstufen}
\subsubsection{Einheiten}
% bps, baud
\subsubsection{Das Shannon- und Nyquist-Theorem}
\subsection{Codierung}
\subsection{Modulation}

\section{Schicht 2 -- Sicherungsschicht}
\subsection{Fehlererkennung}
% CRC
\subsection{Fehlerkorrektur}
% FEC / ARQ
\subsection{Flusskontrolle}
\subsection{Rahmenerkennung}
% Bit-/Byte-Stuffing, HDLC, PPP, LLC
\subsection{Medium Access}
\subsection{VLANs}

\section{MPLS}

\section{Schicht 3 -- Vermittlungsschicht}
\subsection{Aufbau des Internets}
\subsection{Internet Protocol}
\subsubsection{Adressklassen, Subnetze und CIDR}
\subsubsection{IPv4}
\subsubsection{IPv6}
\subsubsection{Hilfsprotokolle}
\subsection{Routing}
% Routing, Verfahren, Tabellen, BGP

\section{Schicht 4 -- Transportschicht}
\subsection{Grundlagen}
% Ports -> NATs
\subsection{UDP}
\subsection{TCP}
\subsubsection{Verbindungskontrolle}
\subsubsection{Flusskontrolle}
\subsubsection{Staukontrolle}
\subsection{SCTP}
\subsection{DCCP}

\section{Schicht 5-7 -- Anwendungsprotokolle}
% Client-/Server-Architekturen
\subsection{DNS}
\subsection{E-Mail}
\subsection{WWW}
\subsubsection{HTTP}
\subsubsection{Cookies}
\subsubsection{Proxies}
\subsubsection{HTML}
\subsection{SNMP}
\subsection{P2P}
% Skalierbarkeit
\subsubsection{DHT}
\subsubsection{Overlay-Topology}
\subsection{Multimedia}
% Multimedia (1. BestEffort, 2. DiffServ, 3. IntServ, SIG, Vorteile, Nachteile)

\end{document}
