\documentclass[a4paper,10pt]{scrartcl}
\usepackage[utf8x]{inputenc}
\usepackage[T1]{fontenc}
\usepackage{amsmath,amsfonts,amssymb,amscd,amsthm,xspace}
\usepackage[ngerman]{babel}
\usepackage{listingsutf8}
\usepackage{color}
\usepackage{geometry}
\usepackage{graphicx}
\usepackage{multicol}
\usepackage{pst-tree}

\geometry{a4paper, left=2cm,right=2cm,top=2cm,bottom=2cm}

\title{Übersetzerbau - Übungsblatt 4}
\author{Christian Cikryt (4285814; Tutorium: Mo. 16-18 Uhr)\\
  Robert Fels (4210496; Tutorium: Mi. 10-12 Uhr)\\
  Martin Lenders (4206090; Tutorium: Mi. 10-12 Uhr)
  }
\date{\today}

\newcommand{\changefont}[3]{\fontfamily{#1} \fontseries{#2} \fontshape{#3} \selectfont}

\renewcommand{\thesection}{Aufgabe \arabic{section}}
\renewcommand{\labelenumi}{\theenumi)}
\renewcommand{\theenumi}{\alph{enumi}}

\definecolor{lgray}{gray}{0.95}
\definecolor{purple}{rgb}{0.498,0,0.3333}
\definecolor{identifier}{rgb}{0,0,0.1}
\definecolor{string}{rgb}{0.192,0,1}
\definecolor{comment}{rgb}{0.25,0.5,0.37}

\pagestyle{myheadings}
\oddsidemargin\oddsidemargin
\markright{Martin Lenders, Christian Cikryt}

\lstset{
	tabsize=4, 
	frame=tlrb, 
	basicstyle=\footnotesize\changefont{pcr}{m}{n},
	breaklines=true,
	numbers=left,
	emphstyle=\textit, 
	language=Python,
	keywordstyle=\color{purple}\textbf, 
	identifierstyle=\color{identifier},
	stringstyle=\color{string},
	backgroundcolor=\color{lgray},
	showstringspaces=false,
	commentstyle=\color{comment},
	extendedchars=true,
	inputencoding=utf8/latin1
}
\psset{nodesep=2pt,levelsep=2em,treesep=2em}

\makeatletter
\newcommand{\Rm}[1]{\textrm{\expandafter\@slowromancap\romannumeral #1@}}
\makeatother
\newcommand{\print}[1]{\{\texttt{print(\dq #1\dq})\}}
\newcommand{\add}[1]{\{\texttt{s += #1})\}}

\begin{document}

\maketitle

\section{}
\begin{lstlisting}[language=Java]
Token scan() {
    if (peek == '/') {
        peek = next input character;
        if (peek == '/') {
            for(; peek == '/n'; peek = next input character);
        } else if (peek == '*') {
            next = next input character from peak
            for(; peek == '*' && next = '/'; 
                peek = next input character, next = next input character from peak);
        }
    }
    // Entferne Leerzeichen
    // Erkenne Konstanten
    // Erkenne Bezeichner und Schluesselwoerter
}
\end{lstlisting}


\section{}
\begin{lstlisting}[language=Java,mathescape=True]
Token scan() {
    if (peek == '/') {
        peek = next input character;
        if (peek == '/') {
            for(; peek == '/n'; peek = next input character);
        } else if (peek == '*') {
            next = next input character from peak
            for(; peek == '*' && next = '/'; 
                peek = next input character, next = next input character from peak);
        }
    }
    // Entferne Leerzeichen
    // Erkenne Konstanten
    if (peek in ['<', '>', '=', '!']) {
        next = next input character from peak
        if (next != '=' && peak == '<') {
            return token <$<$>
        } else if (next != '=' && peak == '>') {
            return token <$>$>
        } else if (next == '=') {
            if (peak == '<') {
                return token <$\leq$>
            }
            if (peak == '>') {
                return token <$\geq$>
            }
            if (peak == '=') {
                return token <$==$>
            }
            if (peak == '!') {
                return token <$\neq$>
            }
        }
    }
    // Erkenne Bezeichner und Schluesselwoerter
}
\end{lstlisting}

\section{}

\section{}
\begin{lstlisting}[language=Java]
	  three address code of expr;
L1:   if !expr goto L2      	//jump when for-condition is false
      three address code of stmt;
	  three address code of expr;
      goto L1
L2:
\end{lstlisting}
\section{}
\begin{enumerate}
\item   
\item   
\item   
\item   
\item   
\end{enumerate}

\end{document}
