\chapter{Zwischencode-Erzeugung}
;ethode wie bei semantischer Analyse SDD bzw. SDT
\begin{center}
    \begin{psmatrix}[colsep=1.5cm]
        Quellprogramm & \framebox{Lexer} & \framebox{Parser} & \framebox{
                        \begin{minipage}{1.4cm}
                         \centering
                         Static\\Checker
                        \end{minipage}} & \framebox{\begin{minipage}{2.2cm}
                         \centering
                         Zwischencode-Erzeugung
                        \end{minipage}} & ?
    \end{psmatrix}
    \ncline{->}{1,1}{1,2}\naput{String}
    \ncline{->}{1,2}{1,3}\naput{Tokens}
    \ncline{->}{1,3}{1,4}\naput{\begin{minipage}{1.5cm}\centering Parse"-baum\end{minipage}}
    \ncline{->}{1,4}{1,5}\naput{\begin{minipage}{1.5cm}\centering dek. Parse"-baum\end{minipage}}
    \ncline{->}{1,5}{1,6}\naput{\begin{minipage}{1.5cm}\centering Zwi"-schen"-code\end{minipage}}
\end{center}
Standard: Drei-Adress-Code.
\begin{description}
\item[Charakteristika:] Jeder Befehl enthält höchstens 3 Adressen
\end{description}

\begin{description}
\item[Gelegentlich:] Schrittweise Übersetzung im Zwischencode absteigender Ebenen. Höhere Ebene (Syntaxbäume, nahe der Quellsprache) in niedrigerer Ebene (Drei-Adress-Code: nahe Maschinensprache).
\end{description}
\section{Erste Optimierung auf der Ebene der (abstrakten) Syntaxbäume}
\begin{description}
\item[Idee:] Gemeindame Teilausdrücke sollen nur einmal dargestellt werden.
\item[Beispiel:] $a + a * (b - c) + (b - c) * d$
\item[GaG:] (Gerichteter azyklischer Graph) (DAG, Directed acyclic graph)\\
    Verwende zur Erstellung eines GaG's dieselbe SSD. die zur Erzeugung des Syntaxbaumes verwendet wurde, allerdings unter Verwendung modifizierter Hilfsfunktionen. leaf und node anstelle von new Leaf und new Node. \\
    Speichere Blätter und Knoten in einem Array. Die Funktionen leaf und node liefern vorh. Knoten und Blätter zurück.
    \begin{center}
        \begin{tabular}{l|l}
            \textbf{Produktion} & \textbf{semantische Regeln} \\
            $E \to E + T$       & $T.k = \text{node}(\texttt{'+'}, E_1.k, T.k)$\\
            $E \to E - T$       & $E.k = \text{node}(\texttt{'-'}, E_1.k, T.k)$\\
            $E \to T$           & $F.k = T.k$\\
            $T \to T_1 * F$     & $T.k = \text{node}(\texttt{'*'}, T_1.k, F.k)$\\
            $T \to F$           & $T.k = F.k$\\
            $F \to (E)$         & $F.k = T.k$\\
            $F \to i$           & $F.k = \text{leaf}(i,v)$\\
            $F \to n$           & $F.k = \text{leaf}(n, n.v)$
        \end{tabular}
    \end{center}
    $a * b + a$
    \begin{center}
        \psset{treesep=2cm}
        \pstree{\Tr{$E$}\nput{0}{\pssucc}{\color{red}$k = k_4$}}{
            \pstree{\Tr{$E$}\nput{0}{\pssucc}{\color{red}$k = k_3$}}{
                \pstree{\Tr{$T$}\nput{0}{\pssucc}{\color{red}$k = k_3$}}{
                    \pstree{\Tr{$T$}\nput{0}{\pssucc}{\color{red}$k = k_1$}}{
                        \pstree{\Tr{$F$}\nput{0}{\pssucc}{\color{red}$k = k_1$}}{
                            \Tr{$i$}\nput{0}{\pssucc}{\color{red}$v = a$}
                        }
                    }
                    \Tr{*}
                    \pstree{\Tr{$F$}\nput{0}{\pssucc}{\color{red}$k_2$}}{
                        \Tr{$i$}\nput{0}{\pssucc}{\color{red}$v = b$}
                    }
                }
            }
            \Tr{+}
            \pstree{\Tr{$T$}\nput{0}{\pssucc}{\color{red}$k k_1$}}{
                \pstree{\Tr{$F$}\nput{0}{\pssucc}{\color{red}$k = k_1$}}{
                    \Tr{$i$}\nput{0}{\pssucc}{\color{red}$v = a$}
                }
            }
        }
    \end{center}
\end{description}

\section{Implementierung von GaG's}
Wertnummermethode: Verwende Knotennummern anstelle der Verweise.
\Bsp
\begin{center}
    \begin{tabular}{|c|c|c|c|}
     Index & \multicolumn{3}{|c|}{ } \\\hline
     1 & $i$ & \multicolumn{2}{|c|}{$a$} \\\hline
     2 & $i$ & \multicolumn{2}{|c|}{$b$} \\\hline
     3 & * & 1 & 2 \\\hline
     4 & + & 3 & 1 \\\hline
    \end{tabular}
\end{center}

\paragraph*{Drei-Adress-Code}
Adressierungsarten:
\begin{enumerate}
 \item Hilfsvariablen (vom Übersetzer erzeugt)
 \item Variablen (Bez.) aus Quellprogramm
 \item Konstanten (unterschiedliche Typ, aus Quellprogramm, Adressen)
\end{enumerate}
Befehlsarten:
\begin{enumerate}
 \item $x = y \operatorname{op} z$: für alle binären Operationen der Quellsprache und Adressen $x$, $y$ und $z$
 \item $x = \operatorname{op} y$: für alle unären Operationen der Quellsprache und Adressen $x$, $y$ und $z$
 \item $x = y$: Kopierbefehl
 \item $\operatorname{goto} L$: unbedingter Sprung
 \item $\operatorname{if} x \operatorname{goto} L$ und $\operatorname{if} \operatorname{False} x \operatorname{goto} L$: bedingte Sprünge
 \item $\operatorname{if} x \operatorname{relop} y \operatorname{goto} L$: bedingte Sprünge unter Relationen $<, >, <=, ...$
 \item $\operatorname{param} x, \operatorname{call} p, n, \operatorname{return} y$ und return für Prozeduren\\
          z. B.
          \begin{algorithmic}
           \STATE param $x$
           \STATE param $y$
           \STATE param $z$
           \STATE call $p, 3$
          \end{algorithmic}
 \item Indizierte Kopierbefehle
     \[x = y[i] \operatorname{und} x[i] = y\]
 \item Adress- und Zeigeroperationen
     \[x = \&y, x = *y \operatorname{und} *x = y\]
\end{enumerate}


