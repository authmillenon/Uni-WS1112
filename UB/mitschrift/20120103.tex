\subsection{Item-Mengen}
\paragraph*{Problem:} Erkennen den Handle auf der Kellerspitze!
Bottom-Up-Syntaxanalyse, LR-Parser
\paragraph*{Heute:} Wesentliche Grundlage fpr deterministisch einfache LR-Parser (simple LR-Parser). In der Praxis verwendet man Parsergeneratoren
\paragraph*{Erste Idee:} Verwende einen endl. Automaten (hier: LR(0)-Automat) zur Steuerung des Analyseprozesses.
\paragraph*{Zustandsmenge} des LR(0)-Automaten: Erweiterung der kontextfreien Grammatik $G$ um neues Startsymbol $S'$ und zusätzlichen Produktionen.
\begin{description}
 \item[Item:] $S' \to .S$
 \Defi Jede Produktion von $G'$, in die auf der rechten Regelseite ein Punkt eingefügt wurde, heißt \emph{Item}.
 \Bsp Zu einer Produktion der Form $A \to XYZ$ gehören vier Items:
     \[ A \to .XYZ, A \to X.YZ, A \to XY.Z\ \text{und}\ A \to XYZ. \]
\end{description}
Wir Konstruieren geeignete Item-Mengen als Zustände des LR(0)-Automaten. Wir benötigen zwei Funktionen: Closure und Goto.
\Defi (Closure) Sei $I$ eine Menge von Items. Closure($I$) verhält man unter erschöpfender Anwendung folgender Regeln.
\begin{enumerate}
 \item $I \subseteq \text{Closure}(I)$,
 \item Wenn $A \to \alpha.B\beta \in \text{Closure}(I)$ und $B \to \gamma \in P$, dann füge $B \to .\gamma$ zu $\text{Closure}(I)$ hinzu.
\end{enumerate}
\Bsp
\begin{align*}
 E' & \to E \\
 E  & \to E + T\ |\ T \\
 T  & \to T * F\ |\ F \\
 F  & \to ( E )\ |\ i
\end{align*}
$\text{Closure}(\{E' \to .E\}) = \{E' \to .E, E \to .E+T, E \to .T, T \to .T \times F, T \to .F, F \to .(E), F \to i\}$ ist der Anfangszustand des LR(0)-Automaten fpr unsere Beispielgrammatik.
\Defi (Goto) $\text{Goto}(I,X) := \text{Closure}(\{A \to \alpha X.\beta\ |\ A \to \alpha .X\beta \in I\})$
\Defi Zustandsmenge $\mathcal{C}$ (\emph{harmonische Sammlung von Item-Mengen}) des LR(0)-Automaten konstruiert man unter erschöpfender Anwendung folgender Regeln.
\begin{enumerate}
 \item $I_0 := \text{Closure}(\{S' \to .S\}) \in \mathcal{C}$
 \item Füge fpr jedes Symbos $X \in \mathcal{N} \cup \mathcal{T}$ der Grammatik $G$ und jedem $I$ aus $\mathcal{C}$ die Item-Menge $\text{Goto}(I,X)$ zu $\mathcal{C}$ hinzu, falls diese ungleich $\emptyset$ ist.
\end{enumerate}
\begin{center}
[Beispiel eines LR(0)-Automaten]
\end{center}
Lösung der shift/reduce-Konflikte durch shift, falls es einen entsprechend markierten Übergang gibt, reduce sonst.
\Bsp $i*i\$$
\begin{center}
    \begin{tabular}{c|c|c|c|c}
        Stack (Zustände) & Symbole (redundant) & Eingabe & Aktion & Ausgabe \\\hline
        0 & $\$$ & $i*i\$$ & shift & \\
        05 & $\$i$ & $*i\$$ & reduce & $F \to i$ \\
        03 & $\$F$ & $*i\$$ & reduce & $T \to F$ \\
        02 & $\$T$ & $*i\$$ & shift & \\
        027 & $\$T*$ & $i\$$ & shift & \\
        0275 & $\$T*i$ & $\$$ & reduce & $F \to i$ \\
        027\underline{10} & $\$T*F$ & $\$$ & reduce & $T \to T * F$ \\
        02 & $\$T$ & $\$$ & reduce & $E \to T$ \\
        01 & $\$E$ & $\$$ & accept
    \end{tabular}
\end{center}


