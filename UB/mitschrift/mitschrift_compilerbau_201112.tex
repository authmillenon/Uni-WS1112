\documentclass[a4paper,10pt]{scrbook}

% Seitenlayout
\usepackage[a4paper,top=1.5cm,right=2.0cm,bottom=2cm,left=2.0cm]{geometry}
\usepackage{fancyhdr}

% Zeichen
\usepackage[ngerman]{babel}
\usepackage[utf8]{inputenc}
\usepackage{amsmath, amsthm, amssymb}
\usepackage{cancel}

% Zeichensatz
\usepackage[colorlinks,%
	    citecolor=black,%
	    filecolor=black,%
	    linkcolor=black,%
	    urlcolor=black,%
	    pdftitle = {Mitschrift - CB},%
	    pdfauthor = {Martin Lenders}%
	]{hyperref}
\usepackage{listings}
\usepackage{hyperref}
\usepackage{longtable}
\usepackage{multirow}
\usepackage{algorithmic}
\usepackage{textcomp}

% Grafik
\usepackage[usenames]{pstricks}
\usepackage{pst-plot,pst-node,pstricks-add,pst-tree}

% Color definitions
\definecolor{lgray}{gray}{0.95}
\definecolor{save}{rgb}{0.498,0,0}
\definecolor{identifier}{rgb}{0,0,0.1}
\definecolor{string}{rgb}{0.192,0,1}
\definecolor{comment}{rgb}{0.25,0.5,0.37}
\definecolor{yellow}{rgb}{1,1,0}
\definecolor{sand}{rgb}{1,1,.75}
\definecolor{red}{rgb}{1,0,0}
\definecolor{melon}{rgb}{1,0.6,.5}
\definecolor{green}{rgb}{0,1,0}
\definecolor{lime}{rgb}{.75,1,.75}
\definecolor{blue}{rgb}{0,0,1}
\definecolor{azure}{rgb}{.75,.75,1}

% Einstellungen für Pakete
\lstset{
	tabsize=8,
	frame=,
	basicstyle=\footnotesize\fontfamily{pcr}\fontseries{m}\fontshape{n}\selectfont,
	emphstyle=\textit,
	numberstyle=\tiny\textsf,
	numbersep=5pt,
	numbers=none,
	keywordstyle=\color{save}\textbf,
	identifierstyle=\color{identifier},
	stringstyle=\color{string},
	showstringspaces=false,
	commentstyle=\color{comment},
	extendedchars=true,
	xleftmargin=1em,
	inputencoding=utf8,
	mathescape=true,
}

\psset{%
	algebraic=true,
	angleA=0,%
	angleB=180,%
	unit=1cm,%
	subgriddiv=0,%
	griddots=5,%
	gridlabels=7pt%
}

\renewcommand{\chaptermark}[1]{markboth{#1}{}}
\renewcommand{\sectionmark}[1]{\markright{\thesection\ #1}}
\pagestyle{fancy}
\fancyhf{}
\fancyfoot[LE,RO]{\sffamily\thepage}
\fancyhead[LE]{\footnotesize\sffamily\bfseries\leftmark}
\fancyhead[RO]{\footnotesize\sffamily\rightmark} 

% Eigene Befehle
\newcommand{\changefont}[3]{\fontfamily{#1}\fontseries{#2}\fontshape{#3}\selectfont}
\newcommand{\Defi}{\paragraph*{Definition}}
\newcommand{\Lemma}{\paragraph*{Lemma}}
\newcommand{\Satz}{\paragraph*{Satz}}
\newcommand{\Bew}{\paragraph*{Beweis}}
\newcommand{\Bsp}{\paragraph*{Beispiel}}

% Neudefinitionen
\renewcommand{\thepart}{\Alph{part}}
\renewcommand{\sectionmark}[1]{\markright{\thesection\ #1}}

\title{Mitschrift\\{\LARGE Übersetzerbau}}
\author{gehalten von Prof. Dr. Elfriede Fehr \\ mitgeschrieben von Martin Lenders}
\subject{Achtung: Dieses Dokument ist \emph{nur} eine Mitschrift der Vorlesung "Übersetzerbau" WiSe2011/12.
	Sie wurde während der Vorlesung angefertigt. Es wird aber seitens des Autors keine Garantie auf
	Vollständigkeit und Richtigkeit des Inhalts gegeben.}

\begin{document}
\maketitle
\tableofcontents
\chapter{Einführung}
\section{Kontakt}
\paragraph{Sprechstunde:} Di 13:30 -- 14:30 Uhr
\paragraph{E-Mail:}	\href{mailto:fehr@inf.fu-berlin.de}{fehr@inf.fu-berlin.de}

\section{Literatur}
Vorlesung basiert im wesentlichen auf dem Dragon-Book (Aho, Ullman, Sethi, Lam)

\section{Was ist Übersetzerbau?}
\begin{itemize}
 \item Übersetzung: Höhere Programmiersprache \emph{(Quellsprache)} in Maschinensprache \emph{(Zielsprache)}
 \item \emph{Aufgabe des Übersetzerbauers:} Schreibe ein Programm, deren Eingabe ein Programm der Quellsprache als Eingabe erhält und als Ausgabe ein \textbf{äquivalentes} Programm (also mit gleicher Bedeutung, gleicher Semantik) der Zielsprache liefert.
 \item natürliche Sprachen sind "`schillernd"' $\Rightarrow$ es gibt nicht immer eine genaue Übersetzung
 \item formale Sprachen haben meist eine genau spezifizierte Übersetzung der Semantik (mathematisch formalisierbar)
 \item moderne Programmiersprachen haben mitunter diese \emph{mathematische} Präzision nicht \\
		$\Rightarrow$ unterschiedliche Interpretation bei unterschiedlichen Übersetzern
\end{itemize}

\section{Geschichte}
\begin{psmatrix}[rowsep=0.5em]
 1940'er Jahre :-( & \framebox{Maschine} & \\
 1. Generation & & $\Rightarrow$ Assemblersprache & \framebox{\begin{minipage}{3cm} mnemotechnische \\ Bezeichnung \end{minipage}} :-$|$ \\
 & Maschinenbefehle &  Übersetzer (1:1) & \begin{minipage}{3cm} Makros (1:n) \\ mit Parametern \end{minipage}
\ncline{1,2}{3,2}\ncline{2,3}{3,3}\ncline{3,3}{3,4}
\end{psmatrix}\\
\hrule
\begin{psmatrix}[rowsep=0em]
 1950'er Jahre & $\underset{\text{wiss. Berechn.}}{\text{FORTRAN}}$, COBOL, LISP \\
 3. Generation \\
 & C, C++, Java, ...
\end{psmatrix}\\

\begin{psmatrix}[rowsep=0.5em]
 && Impl.-Sprache \\
 Programmierer :-) & [mnode=oval] \begin{minipage}{2cm} \centering höhere Pro"-gra"-mmier"-spra"-che \end{minipage} & \framebox{Übersetzer} & [mnode=oval] \begin{minipage}{2cm} \centering Maschinen-sprache \end{minipage}
 \ncline{1,3}{2,3}
 \ncline{->}{2,1}{2,2}\ncline{->}{2,2}{2,3}\ncline{->}{2,3}{2,4}
\end{psmatrix}

\section{Blick in die Box Übersetzer}
\begin{itemize}
 \item zwei Phasen: 
		\begin{enumerate}
		 \item Analysieren (lexikalisch, syntaktisch, Typüberprüfung, ...)\\
				\begin{psmatrix}[rowsep=0.5em]
					$\underbrace{\mathtt{173}}_{\text{Zahl}}$\verb! * (4 - 93)! \\
 					\psset{levelsep=1cm,nodesep=0.2em}
					\pstree{\Tdia{*}}{%
							\Tr*{173}
							\pstree{\Tdia{-}}{
 								\Tr*{4}
 								\Tr*{93}}}
				\end{psmatrix}
		 \item Synthetisieren:
				\begin{itemize}
				 \item Zwischencode
				 \item Optimierung
				\hrule
				 \item Codeerzeugung
				 \item Optimierung
				\end{itemize}
			\begin{psmatrix}[rowsep=0.5em]
			 [mnode=circle] $P_1$ && \framebox{$M_1$} \\
			 [mnode=circle] $P_2$ && \framebox{$M_2$} \\
			 $\vdots$ && $\vdots$ \\
			 [mnode=circle] $P_n$ && \framebox{$M_m$} \\
			 \ncline{->}{1,1}{1,3}\ncline{->}{1,1}{2,3}\ncline{->}{1,1}{4,3}
			 \ncline{->}{2,1}{1,3}\ncline{->}{2,1}{2,3}\ncline{->}{2,1}{4,3}
			 \ncline{->}{4,1}{1,3}\ncline{->}{4,1}{2,3}\ncline{->}{4,1}{4,3}
			 & $n \cdot m$ Übersetzer &
			\end{psmatrix}\hfill
			\begin{psmatrix}[rowsep=0.5em]
			 [mnode=circle] $P_1$ && \framebox{$M_1$} \\
			 $\vdots$ & \framebox{\begin{minipage}{1cm}abstr. Masch.\end{minipage}} & $\vdots$ \\
			 [mnode=circle] $P_n$ & (Zwischenspr.) & \framebox{$M_n$} \\
			 \ncline{->}{1,1}{2,2}\ncline{->}{2,2}{1,3}
			 \ncline{->}{3,1}{2,2}\ncline{->}{2,2}{3,3}
			 & $n + m$ Übersetzer &
			\end{psmatrix}
			\hrule
			\begin{center}
				Frontend (Quellsprache) $\Leftrightarrow$ Backend (Zielsprache)
			\end{center}
			\begin{description}
			 \item[Frontend] Analyse, Zwischencodesynthese- optimierung
			 \item[Backend] Codeerzeugung- und optimierung
			\end{description}
		\end{enumerate}
\end{itemize}

\begin{psmatrix}[rowsep=1em]
	[name=Quellprogramm] \verb!pos = pos + dauer * 60! & \emph{(Quellprogramm)}\\
	[name=lex_ana] \framebox{lexikalische Analyse} & \tiny\begin{minipage}{5cm}
										Symboltabelle \\
										\begin{tabular}{|r|c|}
											\multicolumn{2}{c}{\vdots}\\\hline
											14 & \verb!pos! \\\hline
											15 & \verb!dauer! \\\hline
											\multicolumn{2}{c}{\vdots}
										\end{tabular}
	                                   \end{minipage}\\
	[name=Tokenstrom] \verb!<id, 14><=><id,14><+><id,15><*><int,60>!& \emph{(Tokenstrom)}\\
	[name=syn_ana] \framebox{Syntaxanalyse} & \tiny\begin{minipage}{5cm}
								 $P \rightarrow D;A$\\
								 \begin{tabular}{|c|c|}
								  \pstree[levelsep=0.5cm, treesep=0.5cm, nodesep=0.2em]{\Tr*{$;$}}{\Tr*{$D$}\Tr*{$A$}} &
								  \pstree[levelsep=0.5cm, treesep=0.5cm, nodesep=0.2em]{\Tr*{$=$}}{\Tr*{$D$}\Tr*{$;$}\Tr*{$A$}} \\
									Syntaxbaum & Parsebaum
								 \end{tabular}
	                             \end{minipage} \\
	[name=Syntaxbaum] \psset{levelsep=1cm, nodesep=0.2em} \pstree{\Tdia{=}}{\Tr{\texttt{<id, 14>}} \pstree{\Tdia{+}}{\Tr{\texttt{<id, 14>}} \pstree{\Tdia{*}}{\Tr{\texttt{<id,15>}} \Tr{\texttt{<int,60>}}}}} & \emph{(Syntaxbaum)}\\
	[name=tmp_gen] \framebox{Zwischencodeerzeugnung} \\
	[name=Zwischencode] \begin{minipage}{5cm}
					\begin{lstlisting}[basicstyle=\tt]
t1 = inttofloat # 60
t2 = id15 * t1
t3 = id14 * t2
id14 = t3
					\end{lstlisting}
	               \end{minipage} & \emph{(Zwischencode)}\\
    [name=tmp_opt] \framebox{Codeoptimierung} \\
	[name=opt_Zwischencode] \begin{minipage}{5cm}
					\begin{lstlisting}[basicstyle=\tt]
t1 = id15 * 60
id14 = id14 + t1
					\end{lstlisting}
	               \end{minipage} & \emph{(opt. Zwischencode)}
	\psset{arrows=->}\ncline{Quellprogramm}{lex_ana}\ncline{lex_ana}{Tokenstrom}\ncline{Tokenstrom}{syn_ana}
	\ncline{syn_ana}{Syntaxbaum}\ncline{Syntaxbaum}{tmp_gen}\ncline{tmp_gen}{Zwischencode}\ncline{Zwischencode}{tmp_opt}
	\ncline{tmp_opt}{opt_Zwischencode}
\end{psmatrix}

\chapter{Ein einfacher syntaxgerichteter Übersetzer (Frontend)}
\paragraph{Definition:} Ein Viertupel $G = (\mathcal{N}, \mathcal{T}, S, \mathcal{P})$ heißt \textbf{kontextfreie Grammatik}, wenn
\begin{enumerate}
 \renewcommand{\theenumi}{(\roman{enumi})}
 \item $\mathcal{N}$ ist ein Alphabet (endl. Menge) von \emph{Nichtterminalen}, 
 \item $\mathcal{T}$ ist ein Alphabet (endl. Menge) von \emph{Terminalen}, 
 \item $S \in \mathcal{N}$ ist ein \emph{Startsymbol},
 \item $\mathcal{P}$ ist eine endliche Menge von \emph{Produktionen} der Form $A \to \alpha$, wobei $A \in \mathcal{N}$ und $\alpha \in (\mathcal{N} \cup \mathcal{T})^*$
\end{enumerate}

\paragraph{Konventionen:} 
\begin{itemize}
 \item $\varepsilon$ bezeichnet das leere Wort.
 \item Der Kopf der ersten Produktion ist das Startsymbol
 \item Wenn nur $\mathcal P$ angegeben wird, lassen sich $\mathcal T$, $\mathcal N$ und $S$ daraus generieren.
\end{itemize}

\paragraph{Beispiel:}
\begin{minipage}[t]{0.2\linewidth}
\begin{align*}
 S &\to aSb \\
 A &\to \varepsilon \\
 B &\to SA
\end{align*}
\end{minipage}

\paragraph{Definition:} Sei $G = (\mathcal{N}, \mathcal{T}, S, \mathcal{P})$ eine kontextfreie Grammatik. 
	Ein Wort $\gamma$ lässt sich in einem Schritt bezüglich $G$ aus $\beta$ \textbf{herleiten}, wenn es Produktionen $A \to \alpha$ gibt mit $\beta = \alpha_1A\alpha_2$ und $\gamma = \alpha_1\alpha\alpha_2$.

Geschrieben:
\begin{minipage}[t]{0.5\linewidth}
\begin{align*}
 \beta &\xrightarrow[G]{} \gamma \\
 \alpha_1A\alpha_2 &\xrightarrow[G]{} \alpha_1\alpha\alpha_2 & \text{$\alpha_1$, $\alpha_2$ heißt Kontext}
\end{align*}
\end{minipage}
\begin{itemize}
 \item $\gamma$ lässt sich bzgl $G$ aus $\beta$ herleiten $\gamma \xrightarrow[G]{*} \beta$
\end{itemize}

\paragraph{Definition:}
$\mathcal{L}(G) := \{w | w \in \mathcal{T}^*\text{ mit } S  \xrightarrow[G]{*} w \}$
ist die \textbf{von $\boldsymbol{G}$ erzeugte Sprache}.

\paragraph{Definition:} Jeder Produktion wird ein elementarer \textbf{Parsebaum zugeordnet}.\\
$A \to X_1X_2\dots X_n$ wird zugeordnet:
\begin{minipage}[t]{0.5\linewidth}
 \pstree{\Tr*{$A$}}{
	\Tr*{$X_1$} \Tr*{$X_2$} \Tr*{$\dots$} \Tr*{$X_n$}
 }
\end{minipage}
Ein Baum mit inneren Knoten aus $\mathcal{N}$ heißt Parsebaum bezüglich $G$, genau dann wenn die Wurzel mit $S$ markiert ist und alle inneren Knoten zusammen mit ihren Kindern elementare Parsebäume bezüglich $G$ sind.

\paragraph{Beispiel:}
\begin{minipage}[t]{0.2\linewidth}
\begin{align*}
 \mathtt{expr} &\to \mathtt{expr} + \mathtt{digit}\ |\ \mathtt{expr} - \mathtt{digit}\ |\ \mathtt{digit} \\
 \mathtt{digit} &\to 0\ |\ 1\ |\ \dots\ |\ 9 \\
\end{align*}
\end{minipage}
\[9 - 5 + 2 \in \mathcal{L}(G)\]
$\mathtt{expr} \xrightarrow[G]{} \mathtt{expr} + \mathtt{digit} \xrightarrow[G]{} \mathtt{expr} + 2 \xrightarrow[G]{} \mathtt{expr} - \mathtt{digit} + 2 \dots 9-5+2$\\
oder $\mathtt{expr} \xrightarrow[G]{} \mathtt{expr} + \mathtt{digit} \xrightarrow[G]{} \mathtt{expr} - \mathtt{digit} + \mathtt{digit} \dots 9-5+2$
\begin{minipage}[t]{0.5\linewidth}
 \psset{levelsep=0.5cm}
 \pstree{\Tr*{\tt expr}}{
	\pstree{\Tr*{\tt expr}}{
		\pstree{\Tr*{\tt expr}}{
			\pstree{\Tr*{\tt digit}}{
				\Tr*{9}
			}
		}
		\Tr*{$-$}
		\pstree{\Tr*{\tt digit}}{
			\Tr*{5}
		}
	}
	\Tr*{$+$}
	\pstree{\Tr*{\tt digit}}{
		\Tr*{2}
	}
 }
\end{minipage}

\paragraph{Einsicht:} Jeder Parsebaum repräsentiert eine \textit{Menge} von Herleitungen.\\
Alternative Grammtik $G_m$:
\begin{align*}
 A &\to D\ |\ A+A\ |\ A-A \\
 D &\to 0\ |\ 1\ |\ \dots\ |\ 9 \\
\end{align*}
\paragraph{Lemma:} $\mathcal{L}(G) = \mathcal{L}(G_m)$\\
Es gibt $w \in \mathcal{L}(G_m)$, so dass es zwei verschiedene Parsebäume zu $w$ gibt.

\paragraph{Definition:} Eine Grammatik $G$ heißt \textbf{mehrdeutig}, wenn ein Wort $\alpha$ aus $S$ hergeleitet werden kann, so dass es zu $\alpha$ mehr als einen Parsebaum gibt.

Zusammen mit einer Tabelle aller Operatoren, geordnet nach Priorität in aufsteigender Reihenfolge und ihren Assoziativitäten, lässt sich eine \emph{eindeutige} Grammatik erzeugen, die äquivalent ist.

\paragraph{Verfahren:}
\begin{itemize}
 \item Ordne jedem Niveau ein eigenes Nichtterminal, sowie ein neues für die Grundbausteine der Grammatik (im Beispiel $\mathtt{factor}$)
	\begin{center}
	\begin{tabular}{c|l|l}
	\textbf{Ops} & \textbf{Assoziativität} & Nichtterminal \\
	\hline\hline
	$+-$ & links & \texttt{expr} \\
	$*/$ & links & \texttt{term} \\
	\hline
	&&\texttt{factor}
	\end{tabular}
	\end{center}
 \item Beginne von unten nach oben und ordne jeweils angemessene Produktionen hinzu
	\begin{align*}
		\mathtt{factor} &\to \mathtt{digit}\ |\ (\mathtt{expr}) \\
		\mathtt{term} &\to \mathtt{term} * \mathtt{factor}\ |\ \mathtt{term} / \mathtt{factor}\ |\ \mathtt{factor} \\
		\mathtt{expr} &\to \mathtt{expr} + \mathtt{term}\  |\  \mathtt{expr} - \mathtt{term}\ |\ \mathtt{term}
	\end{align*}
 \item Drehe das Regelwerk um:
	\begin{align*}
		\mathtt{expr} &\to \mathtt{expr} + \mathtt{term}\  |\  \mathtt{expr} - \mathtt{term}\ |\ \mathtt{term} \\
		\mathtt{term} &\to \mathtt{term} * \mathtt{factor}\ |\ \mathtt{term} / \mathtt{factor}\ |\ \mathtt{factor} \\
		\mathtt{factor} &\to \mathtt{digit}\ |\ (\mathtt{expr}) \\
	\end{align*}
\end{itemize}







\section{Syntaxgerichtete Übersetzungen-Übersetzungsschemata (Kern des Front-End)}
\psset{levelsep=0.7cm,nodesep=0.2em}
\begin{minipage}{0.5\textwidth}
 \centering
 \pstree{\Tr*{$P$}}{
  \Tr*{$D$} \pstree{\Tr*{$A$}}{
   \pstree{\Tr*{$=$}}{
    \Tr*{\texttt{id}} \pstree{\Tr*{\texttt{expr}}}{
     \pstree{\Tr*{\texttt{expr}}}{\Tr*{\texttt{term}}} \Tr*{$+$} \Tr*{\texttt{term}}
    }
   } \Tr*{$;$} \Tr*{\texttt{stmt}}
  }
 }
\end{minipage}
\begin{minipage}{0.5\textwidth}
\begin{align*}
 \Rightarrow t_0 &= x + 3 \\
 t_1 &= t_0 * 2\\
 &\vdots
\end{align*}
\end{minipage}

\paragraph{Grundidee:}
\[\mathtt{expr} \to \mathtt{expr} + \mathtt{term} \tag{Produktion}\]
\[
	\pstree{\Tr*{\mathtt{expr}}}{
	 \Tr*{\mathtt{expr_1}} \Tr*{+} \Tr*{\mathtt{term}}
	} \tag{primitiver Baum}
\]
\begin{verbatim}
trans(expr_1); trans(term); Beachte '+'
-> trans(expr)
\end{verbatim}
\begin{itemize}
 \item Geeignete Attibute sind den Symbolen zugeordnet
 \item Regeln sind den Produktionen zuzuordnen
\end{itemize}
\paragraph{Fingerübung:} (Schreibweise kennenlernen)
\begin{center}
$G_B$
\framebox{
\begin{minipage}{4cm}\begin{align*}
 E &\to E + Z\ |\ Z \\
 Z &\to Z0 | Z1 | 0 | 1
\end{align*}
\end{minipage}} Addition über Binärzahlen\\[0.5em]
$10 + 11 + 1$
\begin{minipage}{5cm}
 \psset{treesep=0.5cm,levelsep=0.7cm}
 \pstree{\Tr*{$E\color{red}.w = 6$}}{
  \pstree{\Tr*{$E\color{red}.w = 5$}}{
   \pstree{\Tr*{$E\color{red}.w = 2$}}{
    \pstree{\Tr*{$Z\color{red}.w = 2$}}{
     \pstree{\Tr*{$Z\color{red}.w = 1$}}{
      \Tr*{1}
     }
     \Tr*{0}
    }
   }
   \Tr*{+}
   \pstree{\Tr*{$Z\color{red}.w = 3$}}{
    \pstree{\Tr*{$Z\color{red}.w = 1$}}{
     \Tr*{1}
    }
    \Tr*{1}
   }
  }
  \Tr*{+}
  \pstree{\Tr*{$Z\color{red}.w = 1$}}{
   \Tr*{1}
  } 
 }
\end{minipage}
\hspace{1cm}
\begin{minipage}{9cm}
 Übersetuzung in zugehörigen Wert in Dezimalschreibweise\\
 Attribut: {\color{red}$w$} für Wert, zugeordnet jedem Nichtterminal \\[3em]
 $\Rightarrow$ Dekorierter Parsebaum
\end{minipage}
\end{center}
\begin{center}
 \begin{tabular}{r|l|l}
  & \textbf{Produktion} & \textbf{semantische Regeln} \\\hline
  1 & $E \to E_1 + Z$ & $E.w = E_1.w + Z.w$ \\
  2 & $E \to Z$ & $E.w = Z.w$ \\
  3 & $Z \to Z_1 0$ & $Z.w = Z_1.w \cdot 2$ \\
  4 & $Z \to Z_1 1$ & $Z.w = Z_1.w \cdot 2 + 1$ \\
  5 & $Z \to 0$ & $Z.w = 0$ \\
  6 & $Z \to 1$ & $Z.w = 1$
 \end{tabular}
\end{center}
\begin{itemize}
 \item Synthetische Attrbute (von unten nach oben)
 \item Depth-first, postorder-Auswertung
\end{itemize}
\paragraph{Motivation}
\begin{align*}
 9 - (5 + 2) \tag{Infix}\\
 9\ 5\ 2 + - \tag{Postfix}
\end{align*}

\section{Infix $\to$ Postfix}
\begin{center}
 \begin{tabular}{r|l|l}
  \#&\textbf{Produktion}	& \textbf{Semantische Regeln} \\\hline\hline
  1& $E \to E_1 + T$		& $E.p = E_1.p\ \|\ T.p\ \|\ \texttt{'+'}$ \\\hline
  2&$E \to E_1 - T$ 		& $E.p = E_1.p\ \|\ T.p\ \|\ \texttt{'-'}$\\\hline
  3&$E \to T$ 				& $E.p = T.p$\\\hline
  4&$T \to D$ 				& $T.p = D.p$\\\hline
  5&$T \to (E)$ 			& $T.p = E.p$\\\hline
  6&$D \to 0$ 				& $D.p = \texttt{'0'}$\\\hline
  $\vdots$&$\vdots$			& $\vdots$\\\hline
  15&$D \to 9$ 				& $D.p = \texttt{'9'}$
 \end{tabular}
\end{center}\pagebreak
für $9 - (5 + 2)$
\begin{center}
\pstree{\Tr*{$E\color{red}.p = \texttt{\dq952+-\dq}$}}{
	\pstree{\Tr*{$E\color{red}.p = \texttt{'9'}$}}{
		\pstree{\Tr*{$T\color{red}.p = \texttt{'9'}$}}{
			\pstree{\Tr*{$D\color{red}.p = \texttt{'9'}$}}{
				\Tr*{$9$}
			}
		}
	}
	\Tr*{$-$}
	\pstree{\Tr*{$T\color{red}.p = \texttt{\dq52+\dq}$}}{
		\Tr*{$($}
		\pstree{\Tr*{$E\color{red}.p = \texttt{\dq52+\dq}$}}{
			\pstree{\Tr*{$E\color{red}.p = \texttt{'5'}$}}{
				\pstree{\Tr*{$T\color{red}.p = \texttt{'5'}$}}{
					\pstree{\Tr*{$D\color{red}.p = \texttt{'5'}$}}{
						\Tr*{$5$}
					}
				}
			}
			\Tr*{$+$}
			\pstree{\Tr*{$T\color{red}.p = \texttt{'2'}$}}{
				\pstree{\Tr*{$D\color{red}.p = \texttt{'2'}$}}{
					\Tr*{$2$}
				}
			}
		}
		\Tr*{$)$}
	}
}
\end{center}
Wähle \emph{ein} Attribut $p$ für Postfixnotation und ordne es allen Nichtterminalen zu.

\section{Abstrakte Stapelmaschine}
\begin{psmatrix}
&&& [mnode=circle,name=plus] +\\
\begin{minipage}{2cm}
\begin{verbatim}
push 0
push 9
add
sub
\end{verbatim}
\end{minipage} & 
\begin{minipage}{1cm}
\begin{tabular}{|c|}
 \\\hline
 $m$ \\\hline
 $n$ \\\hline
 $l$ 
\end{tabular}\\
Stack
\end{minipage} & 
\begin{minipage}{1cm}
\begin{tabular}{|c|}
 \rnode{k}{$k$}\\\hline
 \rnode{m}{$m$}\\\hline
 $n$ \\\hline
 $l$ 
\end{tabular}\\
Stack
\end{minipage} & 
\begin{minipage}{1cm}
\begin{tabular}{|c|}
 \rnode{m_plus_k}{\color{white}$\cdot$}\\
 $m + k$ \\\hline
 $n$ \\\hline
 $l$ 
\end{tabular}\\
Stack
\end{minipage}
\ncline{->}{2,1}{2,2}\naput{top}
\ncline{->}{2,2}{2,3}\naput{\texttt{push k}}
\ncline{->}{2,3}{2,4}\naput{\texttt{add}}
\nccurve{->}{k}{plus}
\nccurve{->}{m}{plus}
\ncline{->}{plus}{m_plus_k}
\end{psmatrix}












\section{Traversierung "`depth-first-postorder"'}
\begin{lstlisting}
proc visit(Node N) {
	foreach (Child C of N, from left to right) {
		visit(C);
	}
	handle Node N;
}
\end{lstlisting}

\paragraph{Kritik:} Unnötiges Kopieren, Platzbedarf
\paragraph{Eigenschaft $\boldsymbol{E}$} Der Attributwert (Übersetzung) eines inneren Knotens ergibt sich aus der Konkatenation der Attributwerte aller Kinderknoten unter Beibehalt der Reihenfolge und mit möglichen Einfügungen.
\paragraph{Definition:} Ein Syntaxgerichtete Definition (SGD) heißt \emph{einfach}, wenn sie die Eigenschaft $E$ besitzt.
\paragraph{Idee:} Erzeuge Übersetzung inkrementell beim Traversieren des Baumes.
\begin{lstlisting}
print('9')
print('5')
print('2')
print('+')
print('-')
\end{lstlisting}
\subsection{Übersetungsschema}
Eine kontextfreie Grammatik, in deren rechte Produktionsseiten Programmfragmente (semantische Aktionen, hier Druckbefehle) eingestreut sind, heißt Übersetungsschema. \hfill$\square$
\paragraph{Konvention:} Programmfragmente seien in $\{\}$ eingeschlossen.

Die erzeugte Sprache bleibt unverändert!

Semantische Aktionen heißen auch \emph{Pseudoterminale}. Im Parsebaum mit gestrichelten Kanten.

\subsubsection*{Beispiel}
\begin{align*}
E &\to E + T \{\texttt{print('+')}\}\ |\ E + T \{\texttt{print('+')}\ |\ T\} \\
T &\to E * F \{\texttt{print('*')}\}\ |\ E / F \{\texttt{print('/')}\ |\ F\} \\
F &\to (E)\ |\ D\\
D &\to \{\texttt{print('0')}\}0\ |\ ...\ |\ \{\texttt{print('9')}\}9 
\end{align*}
\begin{center}
 \def\dedge{\ncline[linestyle=dashed]}
 \pstree{\Tr{$E$}}{
  \pstree{\Tr{$E$}}{
   \pstree{\Tr{$E$}}{
    \pstree{\Tr{$T$}}{
     \pstree{\Tr{$F$}}{
      \pstree{\Tr{$D$}}{
       \Tr[edge=\dedge]{\texttt{print('9')}}
       \Tr{9}
      }
     }
    }
    \Tr{$+$}
    \pstree{\Tr{$T$}}{
     \pstree{\Tr{$F$}}{
      \pstree{\Tr{$F$}}{
       \Tr[edge=\dedge]{\{\texttt{print('5')}\}}
       \Tr{5}
      }
     }
    }
    \Tr[edge=\dedge]{\{\texttt{print('+')}\}}
   }
  }
  \Tr{$-$}
  \pstree{\Tr{$T$}}{
   \pstree{\Tr{$F$}}{
    \pstree{\Tr{$D$}}{
     \Tr[edge=\dedge]{\{\texttt{print('2')}\}}
     \Tr{2}
    }
   }
  }
  \Tr[edge=\dedge]{\{\texttt{print('-')}\}}
 }
\end{center}
Die Übersetzung wird erzeugt durch Ausführung der semantischen Aktionen bei deren Besuch im Rahmen der Depth-First-Traversierung (left-to-right).

Maschinenprogramm der abstrakten Stapelmaschine

{\color{red} \texttt{print(\dq push 9\dq)...\texttt{print(\dq sub\dq)}}}

\subsection{Java Bytecode}
\begin{center}
\begin{tabular}{c|c|c|c|l}
 \textbf{Mnemonic} & \texttt{Op-Code} & Argumente & Stack vorher / nachher & Beschreibung \\\hline\hline
 ipush & \texttt{0x10} & 1 byte & $\to \texttt{value}$ & pushes a byte onto the stack as integer-value \\\hline
 iadd & \texttt{0x60} && $V_1, V_2 \to \texttt{result}$ & add two ints together \\\hline
 isub & \texttt{0x64} && $V_1, V_2 \to \texttt{result}$ & subtracts two ints \\\hline
 imul & \texttt{0x68} && $V_1, V_2 \to \texttt{result}$ & multiply two ints together \\\hline
 idiv & \texttt{0x6c} && $V_1, V_2 \to \texttt{result}$ & divide two ints \\\hline
\end{tabular}
\end{center}

\paragraph{Übersetungsschema:} Arithmetische Austrücke in Java Bytecode.
\begin{align*}
 E &\to E + T \{\texttt{print(\dq 60\dq)}\} \\
  &\vdots \\
 D &\to 0\{\texttt{print(\dq 1000\dq)}\}\\
  &\vdots\\
 D &\to 9\{\texttt{print(\dq 1009\dq)}\}\\
\end{align*}


\chapter{Syntaxanalyse (Parser)}
\begin{description}
    \item[Eingabe] (nach lexik. Analyse): Folge von Symbolen (Terminalsymbole, Tokennamen)
    \item[Ausgabe] Parsebaum
\end{description}
\Bsp $E \to E+0\ |\ E+1\ |\ 0\ |\ 1$
\begin{description}
 \item[Eingabe] $1+0$
 \item[Ausgabe] \pstree{\Tr{$E$}}{
                    \pstree{\Tr{$E$}}{
                        \Tr{1}
                    }
                    \Tr{$+$}
                    \Tr{0}
                }
\end{description}
\Satz Kontextfreie Sprachen werden von \emph{nichtdeterministischen} Kellerautomaten erkannt.
\paragraph*{Beweisidee}
\begin{description}
 \item[Eingabe]
      \begin{center}
           \begin{pspicture}(0,0)(3.5,2.5)
                \psframe(0,2)(2.5,2.5)
                \psline(0.5,2)(0.5,2.5)
                \psline(1,2)(1,2.5)
                \psline(1.5,2)(1.5,2.5)
                \psline(2,2)(2,2.5)
                \psline{->}(0.25,1.75)(0.25,2.25)
                \uput{3pt}[-90](0.25,1.75){Lookahead}
                \psline(3,2.5)(3,0)(3.5,0)(3.5,2.5)
                \psline(3,2)(3.5,2)
                \psline(3,1.5)(3.5,1.5)
                \psline(3,1)(3.5,1)
                \psline(3,0.5)(3.5,0.5)
                \rput(3.25,0.25){$S$}
           \end{pspicture}
     \end{center}

     \begin{description}
     \item[Keller] $\mathcal{K} = \mathcal T \cup \mathcal{N}$
     \item[Startzustand] Zu analysierendes Wort $w = a_1...a_n$ auf Eingabe, Lookahead = $a_1$, Keller enthält nur $S$
     \item[Überführung] $\Delta{:}\ \mathcal T \times \mathcal K \to (\text{Aktion}, \text{Ausgabe})$; Ausgabe: Knoten $S$
         \begin{align*}
          \Delta(a, A) &= (\operatorname{push}(\bar \alpha).{}\operatorname{pop}, \operatorname{mkTree(\alpha)})
              \tag{$A \to \alpha \in P$, nichtdeterministisch aus allen $A$-Produktionen gewählt.}\\
          \Delta(a,a) &= (\operatorname{move}.{}\operatorname{pop}, \operatorname{nextNode})\\
          \Delta(a,b) &= (-, \text{error})
         \end{align*}
     \end{description}
    Erkennung mit leerer Eingabe und leerem Keller.
 \item[Ausgabe] \pstree{\Tr{$S$}}{\Tfan[linestyle=dashed]}
 \item[Laufzeit] $O(n^3)$ (inakzeptabe! + Stackoverflows sind recht einfach herstellbar)
\end{description}
$\Rightarrow$ Transformiere die Grammatik $G$, so dass lineare Laufzeit möglich ist.
\begin{itemize}
    \item Geht nicht immer, aber bei herkömmlichen höheren Programmiersprachen ist es machbar!
\end{itemize}
\paragraph*{Verfahren der Syntaxanalyse} Top-Down vs. Bottom-Up
\begin{center}
    \pstree{\Tr{$E$}}{
        \pstree{\Tr{$E$}}{
            \Tr{1}
        }\Tr{+}\Tr{0}
    }
\end{center}
\begin{center}
 \begin{tabular}{p{0.45\linewidth}|p{0.45\linewidth}}
  \textbf{Top-Down} & \textbf{Bottom-Up} \\\hline\hline
   Gut für Parser, die per Hand geschrieben werden & größere Sprachklasse, Verwendung bei \emph{Generatoren}, z. B. Yacc
 \end{tabular}
\end{center}

Gängiges Top-Down-Verfahren: Rekursiver Abstieg und Lookahead der Länge 1.

\section{Prädiktives Parsen}
\Bsp (Ausschnitt Kontrollstrukturen)
\begin{align*}
    \texttt{stmt} &\to \underline{\texttt{expr};}\ |\ \underline{\texttt{if}}\ (\underline{\texttt{expr}})\ \texttt{stmt}\ |\ \underline{\texttt{for}}\ (\texttt{optexpr}; \texttt{optexpr}; \texttt{optexpr})\ \texttt{stmt}\ |\ \underline{\text{other}} \\
     \texttt{optexpr} &\to \varepsilon\ |\ \underline{\texttt{expr}}
\end{align*}
Hilfsmethode
\begin{lstlisting}[language=Java]
void match(Terminal t) {
    if (t == loojahead) lookahead = nextTerminal;
    else report ("syntax error")
}
\end{lstlisting}
Schreibe zu jedem \emph{Nichtterminal} eine \emph{rekursive Prozedur}, die anhand des lookahead die Regel auswählt und anschließend "`die rechte Regelseite ausführt"'.

Ein Beispielparser für oben genannte Grammatik lautet:
\begin{lstlisting}[keywords={void, case, break},emphstyle=\underline,emph={expr,if,for,other}]
void stmt() {
    case expr:
        match(expr); match('j'); break;
    case if:
        match(if); match('('); match(expr); match(')'); stmt(); break;
    case for:
        match(for); match('('); optexpr(); match(';'); ... ; stmt(); break;
    case other:
        mathch(other); break;
    default:
        report("Syntax error")
}

void optexpr() {
    if (lookahead == expr) math(expr
    // Nichtterminal bedeuter $\varepsilon$-Produktion anwenden.
}
\end{lstlisting}
\begin{description}
\item[Beispieleingabe]\lstinline[keywords={void, case, break},emphstyle=\underline,emph={expr,if,for,other}]!for (;expr;expr) other!
\begin{center}
    \pstree{\Tr{\texttt{stmt}}}{
            \Tr{\underline{\texttt{for}}}
            \Tr{(}
            \Tr{\texttt{optexpr}}
            \Tr{;}
            \Tr{\texttt{optexpr}}
            \Tr{;}
            \Tr{\texttt{optexpr}}
            \Tr{)}
            \Tr{\texttt{stmt}}
        }
\end{center}

\end{description}









\begin{center}
\begin{pspicture}(-3,-1)(3,3)
 \rput(0,3){$S$}
 \psline[linestyle=dotted](0,2.7)(0,2.3)
 \psline[linestyle=dotted](-0.3,2.7)(-3,0.5)
 \psline[linestyle=dotted](0.3,2.7)(3,0.5)
 \rput(0,2){$A$}
  \psline[linestyle=dotted](-0.3,1.7)(-1,0.5)
  \psline[linestyle=dotted](0.3,1.7)(1,0.5)
 \rput[r](-3.1,0.25){Quellprogramm} \psframe(-3,0)(3,0.5)
 \psbrace[rot=90](-3,0)(-1,0){Terminal}
  \psline(-1,0)(-1,0.5)\rput(-0.75,0.25){$\alpha_2$}
  \psline(-2.5,0)(-2.5,0.5)
  \psline(-2,0)(-2,0.5)
  \psline(-1.5,0)(-1.5,0.5)
  \psline(-0.5,0)(-0.5,0.5)
  \psline(1,0)(1,0.5)\rput(0.25,0.25){$...$}
    \psline{->}(-0.25,-0.7)(-0.75,0)
    \psline{->}(-0.25,-0.7)(-0.25,0)
    \rput(-0.25,-1){Lookahead}
        \psline[linestyle=dashed]{->}(0.7,-1)(2,-1)
        \end{pspicture}
        
        \end{center}
\begin{align*}
 A &\to \alpha\ |\ \beta\ |\ \gamma \\
 \text{einfach } A &\to \alpha_1\alpha_2\ |\ \underline{\underline{\alpha_2\alpha_2}}\ |\ \alpha_3\alpha_3 \quad\text{mit }\alpha_1 \neq \alpha_2, \alpha_1 \neq \alpha_3
\end{align*}
\begin{description}
\item[Top-Down] (von links nach rechts)
\item[Verfahren]
    \begin{itemize}
     \item Schreibe rekursive Prozedur, zu jedem Nichtterminal genau eine
     \item Wähle im Prozedurrumpf die geeignete Regel anhand des Lookahead aus
     \item Transformiere entsprechende rechte Regelseite in Folge von match-Aufrufen bzw. rekursiven Aufrufen.
    \end{itemize}
\item[Typisches Problem] Linksrekursion
    \begin{description}
     \item[Beispiel] $E \to E+0\ |\ E+1\ |\ 0\ |\ 1$
     \item[Einsicht] $\texttt{FIRST}(\alpha)$ und $\texttt{FIRST}(\beta)$ soll disjunkt sein, wenn es zwei Regeln der Form $A \to \alpha\ |\ \beta$ gibt
     \item[Definition] $\texttt{FIRST}(\alpha) = \{a \in \mathcal{T}\ |\ \alpha \overset{*}{\mapsto} a\alpha'\} \cup \{\varepsilon\ |\ \alpha \overset{*}{\mapsto} \varepsilon\}$ \psframe(0,0)(1.5,0.5)\psline(0.5,0)(0.5,0.5)\psline(1,0)(1,0.5)\psline(1.5,0)(1.5,0.5)\rput(0.25,0.25){1}\rput(0.75,0.25){+}\rput(1.25,0.25){1}\psline{->}(0.25,-0.5)(0.25,0)
         \begin{align*}
              A &\to BCDEF \\
              B &\to \varepsilon \\
              C &\to GH\\
              D &\to \varepsilon
         \end{align*}
     \item[Einsicht] Sei $A \to A \alpha\ |\ \beta$, wobei $\beta$ nicht mit $A$ beginnt.\\
         Es gilt $\texttt{FIRST}(A \alpha) \supseteq \texttt{FIRST}(\beta)$
             \begin{center}
             \begin{pspicture}(0,0)(2,2)
                 \rput[l](1,2){$\rnode{A}{A}a \overset{*}{\mapsto} \alpha \beta' \alpha$}
                 \psline[nodesep=3pt](A)(0.5,1.2)(1.5,1.2)(A)
                 \rput[l](1,1){$\rnode{B}{\beta}$}
                 \psline[nodesep=3pt](B)(0.3,0.2)(1.7,0.2)(B)
             \end{pspicture}
             \end{center}
     \item[Idee] Transformiere linksrekursive Grammatik $G$ in äquivalente Grammatik $G'$, so dass $\mathcal{L}(G) = \mathcal L(G')$ und 
                 $G'$ ist \emph{nicht} linksekursiv.
     \item[Prinzip] Eliminierung der Linksrekursion:
         \begin{description}
          \item Betrachte linksrekursiven Ausschnitt einer $G$. $A \to A\alpha\ |\ \beta$
          \item Welche Struktur haben Satzformen, die aus $A$ herleitbar sind?
          \begin{center}
              \pstree{\Tr*[name=root]{$A$}}{
                  \pstree{\Tr*{$A$}}{
                      \pstree{\Tr*{$A$}}{
                          \pstree{\Tr*[name=bla]{$A$}}{
                              \pstree{\Tr*{$A$}}{
                                   \Tfan{$\beta$}
                               }
                          }
                          \Tfan[nodesepB=3pt]{$\alpha$}
                      }
                  }
                  \Tfan[nodesepB=3pt]{$\alpha$}
              }
              \psbrace[rot=225,nodesepA=-2cm](root)(bla){$n$-mal $A \to A\alpha$}
              $\beta \alpha^n$
          \end{center}
          \item Ersetze diese beiden Regeln
              \begin{align*}
               A &\to \beta R \\
               R &\to \alpha R\ |\ \varepsilon
              \end{align*}
         \end{description}
     \item[Beispiel] Anwendung arithmetischer Ausdrücke (s. o.)
         \[ E \to E \alpha_1\ |\ E\alpha_2\ |\ \beta_1\ |\ \beta_2\]
         \begin{align*}
             E &\overset{*}{\mapsto} \beta_1 (\alpha_1\ |\ \alpha_2)^n \\
             E &\overset{*}{\mapsto} \beta_2 (\alpha_1\ |\ \alpha_2)^n
         \end{align*}
         \begin{align*}
             E &\to \beta_1 R\ |\ \beta_2 R \\
             R &\to \alpha_1 R\ |\ \alpha_2 R\ |\ \varepsilon
         \end{align*}
         konkret ($G'$):
         \framebox{
             \begin{minipage}{4cm}
                \begin{align*}
                    E &\to 0 R\ |\ 1 R \\
                    R &\to +0 R\ |\ +1R\ |\ \varepsilon
                \end{align*}
             \end{minipage}
         }
         \begin{center}
        \begin{minipage}{0.4\textwidth}
        \pstree{\Tr*{$E$}}{
            \pstree{\Tr*{$E$}}{
                \pstree{\Tr*{$E$}}{
                    \Tr*{1}
                }
                \Tr*{+}
                \Tr*{0}
            }
            \Tr*{+}
            \Tr*{0}
        }
        \end{minipage}
\begin{minipage}{0.4\textwidth}
\pstree{\Tr*{$E$}}{
    \Tr*{1}
    \pstree{\Tr*{$R$}}{
        \Tr*{+}
        \Tr*{0}
        \pstree{\Tr*{$R$}}{            
            \Tr*{+}
            \Tr*{1}
            \pstree{\Tr*{$R$}}{
                \Tr*{$\varepsilon$}
            }
        }
    }
}
\end{minipage}\end{center}
Übersetzungsschema schwierig bis nicht machbar
    \end{description}
    \begin{align*}
     G{:}\ E &\to E + T\ |\ E - T\ |\ T & G'{:}\ E &\to T R \\
     T &\to 0\ |\ 1 & R &\to + T R\ |\ - T R\ |\ \varepsilon \\
     && T &\to 0\ |\ 1
    \end{align*}
    Übersetzungsschema zu $G$:
    \begin{align*}
     E &\to E \pm T \{\texttt{print('+')}\}\ |\ T \\
     T &\to 0\{\texttt{print('0')}\ |\ 1\{\texttt{print('1')}\}\}
    \end{align*}
\item[Idee:] Eliminiere Linksrekursion aus dem Übersetzungsschema unter Einbeziehung der semantischen Aktionen.
\begin{align*}
 E &\to TR \\
  R &\to +T \{\texttt{print('+')}\} R\ |\ -T \{\texttt{print('-')}\} R\ |\ \varepsilon \\
  T &\to ...
\end{align*}
\end{description}
\section{Einfacher Übersetzer}
Prädiktive Syntaxanalyse angereichert um übernommenes Übersetzungsschema.
\begin{lstlisting}[language=Java]
// match-Prozedur, wie gesehen
void E() { T(); R(); }
void R() { 
    if (lookahead == '+') { match('+'); T(); print('+'); R; } 
    else if (lookahead == '-') { match('-'); T(); print('-'); R; } 
}
void T() {
    if (lookahead == '0') { match('0'); print('0'); }
    else if (lookahead == '1') { match('1'); print('1'); }
}
\end{lstlisting}


\chapter{Lexikalische Analyse}
Alles kontextfrei:
\begin{align*}
 N &\to ND\ |\ D \tag{für natürliche Zahlen}\\
 D &\to 0\ |\ ...\ |\ 9
\end{align*}

\begin{align*}
 I &\to IL'\ |\ L \tag{für Bezeichner}\\
 L &\to \texttt{'a'}\ |\ ...\ |\ \texttt{'b'}\\
 L' &\to L\ |\ D\ |\ \texttt{'\_'}
\end{align*}

\begin{enumerate}
 \item \emph{Symboltabellen}
 \item Semantische Analyse (kursorisch) -- Statische Überprüfungen \emph{nicht} kontextfreier Eigenschaften
 \item Zwischencode-Erzeugung (kursorisch)
\end{enumerate}

\begin{enumerate}
 \item Lexikalische Analyse liefert Token bestehend aus Tokenname und optionalen Attributen (Lexem, Index in Syboltabellen, $<$Konstante, Wert$>$, $<=>$, $<$IF$>$, ...)
           \begin{description}
            \item[Problem]  Gültigkeitsbereiche $\rnode{x_dek}{\{\textbf{int } x}; \rnode{y_dek}{\textbf{bool } y}; ... x ... \rnode{y}{y} \{\rnode{y2_dek}{\textbf{float } y}; ... \rnode{x}{x} ... \rnode{y2}{y} ...\} ... \{y ...\}\}$\\[0.2em]
 \ncangle[angleA=-90,angleB=-90,arm=0.5]{->}{x_dek}{x}
  \ncangle[angleA=-90,angleB=-90,arm=0.3]{->}{y_dek}{y}
  \ncangle[angleA=-90,angleB=-90,arm=0.3]{->}{y2_dek}{y2}
            \item[Idee] Strukturierte Symboltabelle
            \item[Erkenntnis] Lexikalische Analyse ist überfordert!
            \item[Ergebnis] Lexer liefert flache Struktur, d. h. eine Tabelle, die Gültigkeitsbereiche \emph{nicht} berücksichtigt.
           \end{description}
           Zusatzaufgabe des Parsers: Erstelle \emph{eigene} Symboltabelle für jeden Block, jedde Klasse, jede Methode, die in der lokalen Deklaration vorkommen. Organisation als Baumstruktur
           \begin{center}
            \pstree{\Tr*{
                \begin{minipage}{2cm}\centering
                 \begin{tabular}{|rl|}
                  \hline
                  $x$: & \textbf{int}\\\hline
                  $y$: & \textbf{bool}\\\hline
                 \end{tabular}
                \end{minipage}
            }}{
                \pstree{\Tr*{
                \begin{minipage}{2cm}\centering
                 \begin{tabular}{|rl|}
                  \hline
                  $y$: & \textbf{float}\\\hline
                 \end{tabular}
                \end{minipage}
                }}{
                    \Tr*{\begin{minipage}{2cm}\centering
                 \begin{tabular}{|c|}
                  \hline
                  ...\\\hline
                 \end{tabular}
                \end{minipage}}
                }
                 \Tr*{\begin{minipage}{2cm}\centering
                 \begin{tabular}{|c|}
                  \hline
                  ...\\\hline
                 \end{tabular}
                \end{minipage}}
            }
           \end{center}
\item  Statisch Überprüfung
    \begin{itemize}
     \item Typüberprüfungen, -herleitung, -konversion
     \item Überprüfung ob jede Variable deklariertt wurde
    \end{itemize}
    \emph{Methode:} Syntaxgerichtete Definition
\item  Zwischencodeerzeugung
    \begin{itemize}
     \item Bisher: Postfix-Notation, Befehle der abstrakten Stack-Maschine
     \item Typische Formen des Zwischencodes:
         \begin{enumerate}
          \item abstrakte Syntaxbäume
          \item Drei-Adress-Code (lineare Struktur, Befehlsfolge)
          Operatorbaum für 2 - (9+5)
          \begin{center}\psset{levelsep=1cm}
\pstree{\Tr{$-$}\nput{20}{\pssucc}{\color{blue}$t_1 = 2 - t_0$}}{
   \Tr{2}\nput{20}{\pssucc}{\color{blue}$2$}
   \pstree{\Tr{+}\nput{20}{\pssucc}{\color{blue}$t_0 = 9 + 5$}}{
   \Tr{9}\nput{20}{\pssucc}{\color{blue}$9$}\Tr{5}\nput{20}{\pssucc}{\color{blue}$5$}
}
}
          \end{center}
         \end{enumerate}
    \lstinline[language=Java]!while(E) S!
    \begin{center}
     \pstree{\Tr{\textbf{while}}}{
     \Ttri{E}\nput{160}{\pssucc}{\color{blue}Code für $E$ in $t$}
     \Ttri{S}\nput{20}{\pssucc}{\color{blue}Code von $S$}
    }
    \begin{minipage}{0.3\textwidth}
     \color{blue}
     \begin{tabular}{|c|l|c|}
      \hline
      \multicolumn{3}{|c|}{}\\\cline{2-2}
      M & Code für E & \\ \cline{2-2}
       & ifFalse t goto L& \\ \cline{2-2}
       & Code von S & \\ \cline{2-2}
       & goto M  &\\ \cline{2-2}
       \multicolumn{3}{|l|}{L}\\\hline
     \end{tabular}
    \end{minipage}
    \end{center}
    \end{itemize}
\item Optimierung\\
    \begin{tabular}{lcl}
     \lstset{language=Java}
     Beispiel: & \lstinline!if (peek = '\n') line = line + 1! & Quellprogramm \\
         & $\Downarrow$ & \\\cline{2-2}
         & \multicolumn{1}{|c|}{lexikalische Analyse} & \\\cline{2-2}
         & $\Downarrow$ & \\
         & $<\textbf{if}><(><\textbf{id},\texttt{\dq peek\dq}><=><\textbf{char}, \texttt{'\textbackslash n'}><)>$ & \\
         & $<\textbf{id},\texttt{\dq line\dq}><=><\textbf{id}, \texttt{\dq line\dq}><+><\textbf{int},1><;>$ & \\
         & \rnode{start}{$\Downarrow$}
    \end{tabular}
    \vspace{10mm}
    \begin{center}
     \begin{tabular}{cc}
\begin{minipage}{0.4\textwidth}
\psset{levelsep=1cm}
\pstree{\Tr[name=var1]{\textbf{if}}}{
    \pstree{\Tr{\textbf{eq}}}{
        \Tr{peek}
        \pstree{\Tr{(\textbf{int})}}{
            \Tr{\lstinline!'\n'!}
        }
    }
    \pstree{\Tr{\textbf{assign}}}{
        \Tr{line}
        \pstree{\Tr{+}}{
            \Tr{line}
            \Tr{1}
        }
    }
}
\end{minipage}    &
     \rnode{var2}{
         \begin{minipage}{0.4\textwidth}
         \texttt{1. t1 = (int)'\textbackslash n'}\\
         \texttt{2. ifFalse (peek == t1) goto 4}\\
         \texttt{3. line = line + 1}\\
         \texttt{4. ...}
         \end{minipage}
     }
     \end{tabular}
     \ncline{->}{start}{var1}\nbput{Variante 1}
     \ncline{->}{start}{var2}\naput{Variante 2 unter Optimierung}
    \end{center}
\end{enumerate}

\psset{colsep=0.5cm,arrows=->}
\begin{psmatrix}
             & & & Analysephase \\
             & & \framebox{\begin{minipage}{3cm}
                               Fehlerbehandlung
                          \end{minipage}} && Start der Übersetzung \\
Quellprogramm & \framebox{Lexer} & \psframebox[linestyle=none,fillstyle=hlines,hatchcolor=red]{Tokenstrom} &
\framebox{Parser} & ... \\
              && \framebox{
                  \begin{minipage}{3cm}
                   Symboltabelle
                  \end{minipage}}
\end{psmatrix}\ncline{3,1}{3,2}\ncline{3,2}{3,3}\ncline{3,3}{3,4}\ncline{3,4}{3,5}
\ncline{<->}{2,3}{3,2}\ncline{<->}{2,3}{3,4}\ncline{<->}{2,3}{3,5}\ncline{<->}{4,3}{3,2}\ncline{<->}{4,3}{3,4}\ncline{<->}{4,3}{3,5}
\nccurve[angleA=-10,angleB=-170,linecolor=red]{3,2}{3,4}\nbput{\color{red} Token}\nccurve[angleA=170,angleB=10,linecolor=red]{3,4}{3,2}\nbput{\color{red} getNextToken}\ncline[doubleline=true]{2,5}{3,4}

\paragraph{Begriffserklärung}
\begin{itemize}
 \item Token
 \item Muster-Berschreibung zulässiger Lexeme
 \item Lexem
\end{itemize}
\begin{center}
 \texttt{\textvisiblespace\textvisiblespace'\textbackslash t'\textvisiblespace'\textbackslash n'} "`Leerzeichen"'
\end{center}
\begin{center}
 $\underbrace{\textvisiblespace\textvisiblespace}_{Leerraum}\underset{\overset{\uparrow}{<\textbf{num}, 34>}}{34} \overset{\underset{\downarrow}{\text{Beginnlexem}}}{+}\underbrace{\textvisiblespace\textvisiblespace}_{\textbf{id}, \texttt{'count'}}\textvisiblespace\textvisiblespace...$
\end{center}

\begin{enumerate}
 \item Folge von Leerzeichen -- Leerräume: werden entfernt
 \item Suche längsten Präfix der verbleibenden Eingabe, der auf eines der Muster, d. h. der Beschreibung von Strukturen passt
 \item Gib Token bestehend aus Tokenname (Terminalsymbol der kontextfreien Grammatik) ggf. zusammen mit Attributwert zurück, falls Suche erfolgreich. Fehlerbehandlung sonst.
\end{enumerate}
\paragraph*{Vorsicht:} Beispiel zu FORTRAN:
\begin{itemize}
 \lstset{language=fortran}
 \item \lstinline!DO 10 I = 1.25! $\rightsquigarrow$ \underline{\lstinline!DO10I!}\lstinline!=!\underline{\lstinline!1.25!} (Bezeichner, Zuweisung, Konstante)
 \item \lstinline!DO 10 I = 1,25! $\rightsquigarrow$ \lstset{language=} \underline{\lstinline!DO!}\lstinline!10!\underline{\lstinline!I!}\lstinline!=!\underline{\lstinline!1!}\lstinline!,!\underline{\lstinline!25!}
\end{itemize}
Scanner entfernt Leerräume. Unschön, in modernen Sprachen ausgemerzt
\begin{itemize}
 \item Leerzeichen als Trennzeichen
 \item Reservierung der Schlüsselwörter
\end{itemize}

\section{Fehlerbehandlung}
Nicht alle Schreibfehler können vom Lexer erkannt werden!
\paragraph*{Beispiel:} \lstinline!fi(x == f(y)) ...! vs \lstinline!if (x == f(y)) ....! \\
$<\textbf{id}, \texttt{'fi'}><(> ....$ kein lexik. Fehlerbehandlung\\
Welche Fehlersituation kann auftreten? \emph{Kein} Präfix der verbleibenden Eingabe passt auf einen der Muster.
\paragraph{Systematik zu Fehlerbehandlung}
\begin{itemize}
 \item "`Panic Recovery"' -- Entfenrne einzelne Zeichen am Anfang der verbleibenden Eingabe
 \item Vier primitive Operation werden angewendet, um korrekte Eingabe herzustellen
     \begin{itemize}
     \item Entferne ein Zeichen aus verbleibender Eingabe
     \item Füge ein Zeichen in verbleibender Eingabe ein
     \item Vertausche zwei benachbarte Zeichen in verbleibender Eingabe
     \item Ersetze ein Zeichen durch ein anderes in verbleibender Eingabe
     \end{itemize}
\end{itemize}

\paragraph{Optimale Strategie} Wende eine minimale Anzahl promitiver Operationen an, so dass verbleibebende Eingabe bis zum Ende erfolgreich lexikalisch analysiert werden kann.

Aus Effizienzgründen \emph{nicht praktikabel}!
\begin{itemize}
 \item Anwendung einer dieser Optionen ist meistens erfolgreich
\end{itemize}

\section{Eingabebehandlung mit zweigeteilten Puffer}
\begin{longtable}{|c|c|c|c|c|c|c|c|c|c||c|c|c|c|c|c|c|c|c|c|}
   &   & & & \multicolumn{4}{l}{beginLexem} & \multicolumn{4}{l}{$\downarrow$ forward} \\
 \hline
   &   & & & c & o & u & n & t & & = & & c & o & u & n & t & + & 1 & eof \\\hline
 0 & 1 & & & & & & & & $N-1$ & 0 & 1 & & & & & & & & $N-1$
\end{longtable}
$N$ ist Blocklänge, mit einem Lesebefehl wird jeweils eine Hälfte gelesen. Typisches $N = 4096$\\
\begin{itemize}
 \item Vereinfachung durch Einführung von Wächtern (Sentinel).
 \item Test auf Pufferende kann mit \emph{Lesen} des nächsten Zeichens verschmolzen werden.
\end{itemize}






\input{20111117}
\section{Automatische Erzeugung von Lexern - Endliche Automaten}
In der Praxis vorhandene Werkzeuge wie \textsc{Lex} oder \textsc{Flex} folgen den Prinzipien regulärer Definitionen $\rightsquigarrow$ endliche nichtdeterministische Automaten $\rightsquigarrow$ deterministischer Automate ( $\rightsquigarrow$ Optimierung ).

\paragraph*{Allgemeiner Prozess am Beispiel Lex:}
\begin{center}\psset{rowsep=0.5cm}
 \begin{psmatrix}
  \texttt{lex.l} & \framebox{lex compiler} & \texttt{lex.yy.c} & \framebox{C compiler} & \texttt{a.out} \\
  & Quellprogramm & \framebox{\texttt{a.out}} & Tokenfolge
 \end{psmatrix}\ncline{1,1}{1,2}\ncline{1,2}{1,3}\ncline{1,3}{1,4}\ncline{1,4}{1,5}\ncline{2,2}{2,3}\ncline{2,3}{2,4}
\end{center}
\begin{itemize}
 \item \texttt{lex.l} ist in \emph{formaler Sprache} lex-language geschriebene reguläre Definition plus Zusatzinfos
 \item Tokenfolge: Jeweils ein Tokenname als Ergebniswert und Attribut in globaler Variable \texttt{yylval}
\end{itemize}

Formale Sprache lex-language am Beispiel aus vorheriger Vorlesung:
\paragraph*{Struktur:}\hspace*{0cm}\\
\begin{lstlisting}[language=,mathescape=true]
%%
Deklarationsteil        /* Variablen, sympolische Konstanten
                         * wie LE, GT, ..., IF, ELSE, ...
                         * regulaere Definitionen
                         */       | %%
                                  | delim {\n\t }
                                  | ws    {delim}+
                                  | ...
                                  | letter ...
                                  | id    {letter}({letter}|{digit})*
%%                                | ...
Uebersetzungsregeln               | %%
Form: Muster {Aktionen}           | {ws}  {/* keine Aktion, keine Rueckgabe */}
                                  | if    {return (IF);}
                                  | {id}  {yylval = (int)installID(); return (ID);}
                                  | "<="  {yylval = LE; return (RELOP);}
%%                                | ...
Hilfsfunktionen, z. B. installID  | %%
                                  | ...
\end{lstlisting}

\subsection{Endliche Automaten}
Nichtdeterministische endliche Automaten (NEA), deterministische endliche Automaten (DEA)
\Defi Ein 5-Tupel der Form $(Z, \Sigma, \delta, Z_0, E)$ heißt \textbf{nichtdeterministischer endlicher Automat}, genau dann wenn
\begin{enumerate}
 \item $Z$ ist eine endl Menge von \emph{Zuständen}
 \item $\Sigma$ ist Eingabealphabet
 \item $\delta{:}\ Z \times (\{\varepsilon\} \cup \Sigma) \to \mathcal{P}(Z)$ Folgezustände zu $z$ und $a$ bzw. $\varepsilon$
 \item $z_0 \in Z$ Anfangszustand
 \item $E \subseteq Z$ Endzustände
\end{enumerate}
\Defi Ein nichtdeterministischer endlicher Automat $A = (Z, \Sigma, \delta, z_0, E)$ \textbf{erkennt die Sprache} $\mathcal{L}(A)$ mit 
     \[\mathcal{L}(A) = \{w  \in \Sigma^*\ |\ \text{es gibt eine Überführung von $z_0$ nach $z \in E$ unter Eingabe von $w$}\}\]
\Defi Ein 5-Tupel der Form $(Z, \Sigma, \delta, Z_0, E)$ heißt \textbf{deterministischer endlicher Automat}, genau dann wenn
\begin{enumerate}
 \item $Z$ ist eine endl Menge von \emph{Zuständen}
 \item $\Sigma$ ist Eingabealphabet
 \item {\color{red} $\delta{:}\ (Z \times \Sigma) \to Z$}
 \item $z_0 \in Z$ Anfangszustand
 \item $E \subseteq Z$ Endzustände
\end{enumerate}
\Defi Ein deterministischer endlicher Automat $A$ \textbf{erkennt die Sprache} $\mathcal{L}(A)$ mit
    \[\mathcal{L}(A) = \{w  \in \Sigma^*\ |\ z_0 \xrightarrow[\delta]{a_1} z_1 \to ... \xrightarrow{a_n} z_n,\ w = a_1...a_n, z_n \in E\}\]
\begin{description}
 \item[Beschreibung endlicher Automaten] (für menschliche Leser) sind Transitionsgraphen, deren Knoten die Zustände sind und es genau dann eine Kante von $s$ nach $t$ mit der Markierung $a$ gibt, wenn $t \in \delta(s, a), s \in Z, a \in \Sigma \cup \{\varepsilon\}$. Markiere Anfangszustand mit $\xrightarrow{\text{start}}$ und Endzustände mit \pscircle[doubleline=true,unit=1em](0.5,0.5){0.5em}\hspace{1em}.
 \item[Zur maschinellen Verarbeitung] daraus abgeleitete Übergangstabellen, deren Zeilen und Zutände markiert sind, Spaltn mit $\Sigma \cup \{\varepsilon\}$ und Einträge aus $\mathcal{P}(Z)$
 \Bsp $A$:
     \begin{center}
      \psset{colsep=1cm,rowsep=1cm,arrows=->}
      \begin{psmatrix}[mnode=circle]
      [mnode=none]$\xrightarrow{\text{start}}$ & 0 & 1 & 2 & [doubleline=true] 3
      \end{psmatrix} erkennt $(a | b)* abb$
      \nccurve[angleA=60,angleB=120,ncurv=4]{1,2}{1,2}\ncput{$a$}
      \nccurve[angleA=-60,angleB=-120,ncurv=4]{1,2}{1,2}\ncput{$b$}
      \ncline{1,2}{1,3}\naput{$a$}
      \ncline{1,3}{1,4}\naput{$b$}
      \ncline{1,4}{1,5}\naput{$b$}
     \end{center}
     Zu gegebener Eingabe z. B. $aabb$ gibt es eine Vielfalt von Überführungen.
     \begin{itemize}\psset{colsep=1cm,rowsep=1cm,arrows=->}
      \item z. B. \begin{psmatrix}[mnode=circle]
                   0 & 0 & 0 & 0 & 0
                  \end{psmatrix}
                  \ncline{1,1}{1,2}\naput{$a$}\ncline{1,2}{1,3}\naput{$a$}\ncline{1,3}{1,4}\naput{$b$}\ncline{1,4}{1,5}\naput{$b$} und
                  \begin{psmatrix}[mnode=circle]
                   0 & 0 & 1 & 2 & 3
                  \end{psmatrix}
                  \naput{$a$}\ncline{1,2}{1,3}\naput{$a$}\ncline{1,3}{1,4}\naput{$b$}\ncline{1,4}{1,5}\naput{$b$}
     \end{itemize}
     Übergangstabelle:
     \begin{center}
      \begin{tabular}{c|c|c|c}
       \textbf{Zustände} & $a$ & $b$ & $\varepsilon$ \\\hline
       0 & \{0,1\} & \{0\} & $\emptyset$ \\
       1 & $\emptyset$ & \{2\} & $\emptyset$ \\
       2 & $\emptyset$ & \{3\} & $\emptyset$ \\
       3 & $\emptyset$ & $\emptyset$ & $\emptyset$ \\
      \end{tabular}
     \end{center}
        Allgemein: Eingabe abgeschlossen mit eof
    \subsection{Transformation von NEA in DEA}
    \paragraph*{Idee:} Potenzmengenautomat! Zustandsmenge des DEA = $\mathcal{P}(Z)$\\
        $\varepsilon$-Hülle($z$), $\varepsilon$-Hülle($T$), move($T, a$)
    \paragraph*{strukturelle Induktion} für reguläre Definitionen $\rightsquigarrow$ endliche nichtdeterministische Automaten
    \paragraph*{Basis:}\psset{colsep=1cm,rowsep=1cm,arrows=->}\hfill\\
    \begin{psmatrix}[mnode=circle]
     [mnode=none] $\varepsilon$ $\xrightarrow{\text{start}}$ & $i$ & [doubleline=true] $f$
    \end{psmatrix}, wobei $i, f$ neue Zustände \ncline{1,2}{1,3}\naput{$\varepsilon$}\\
    \begin{psmatrix}[mnode=circle]
     [mnode=none] $a \in \Sigma$ $\xrightarrow{\text{start}}$ & $i$ & [doubleline=true] $f$
    \end{psmatrix}, wobei $i, f$ neue Zustände \ncline{1,2}{1,3}\naput{$a$}
\paragraph*{Induktionsannahme:} Endliche Automaten $N(s)$ und $N(t)$ seien für geg. $s$ und $t$ konstruiert:
\begin{enumerate}
 \item $r = s|t{:}\ N(r)$ \begin{psmatrix}[mnode=circle,rowsep=0cm]
     & & $\quad$ &[doubleline=true] $\quad$  \\
     [mnode=none] $\xrightarrow{\text{start}}$ & $i$ &&& [doubleline=true] $f$ \\
     & & $\quad$ &[doubleline=true] $\quad$
    \end{psmatrix}\ncline{2,2}{1,3}\naput{$\varepsilon$}\ncline{2,2}{3,3}\naput{$\varepsilon$}\ncbox{1,3}{1,4}\naput{$N(s)$}\ncbox{3,3}{3,4}\naput{$N(r)$}\ncline{1,4}{2,5}\naput{$\varepsilon$}\ncline{3,4}{2,5}\naput{$\varepsilon$}
 \item $r = st{:}\ N(r)$
     \begin{psmatrix}[mnode=circle,rowsep=0.5cm]
      [mnode=none] $\xrightarrow{\text{start}}$ & $i$ & $\quad$ & [doubleline=true] $\quad$ & $\quad$ & [doubleline=true] $\quad$ & [doubleline=true] $f$ 
     \end{psmatrix}\ncline{1,2}{1,3}\naput{$\varepsilon$}\ncbox{1,3}{1,4}\naput{$N(s)$}\ncline{1,4}{1,5}\naput{$\varepsilon$}\ncbox{1,5}{1,6}\naput{$N(r)$}\ncline{1,6}{1,7}\naput{$\varepsilon$}
 \item $r = s^*{:}\ N(r)$ 
     \begin{psmatrix}[mnode=circle,rowsep=0.5cm]
     \\
      [mnode=none] $\xrightarrow{\text{start}}$ & $i$ & $\quad$ & [doubleline=true] $\quad$ & [doubleline=true] $f$ \\
     \end{psmatrix}\ncline{2,2}{2,3}\naput{$\varepsilon$}\ncbox{2,3}{2,4}\naput{$N(s)$}\ncline{2,4}{2,5}\naput{$\varepsilon$}
         \nccurve[angleA=100,angleB=80]{2,4}{2,3}\nbput{$\varepsilon$}
         \nccurve[angleA=-30,angleB=-150]{2,2}{2,5}\nbput{$\varepsilon$}
 \item $r = (s){:}\ N(r) = N((s))$
\end{enumerate}




\end{description}
\chapter{Syntaxanalyse (Parser)}
\begin{itemize}
 \item kontextfreie Grammatiken bieten eine präzise, formale Spezifikation der grammatikalischen Struktur typischer Programmiersprachen; meist gegeben in Form von BNF und EBNF
 \item Gewisse \emph{eingeschränkte} Klassen kontextfreie Sprachen lassen sich effizient parsen und sogar Parsergeneratoreb schreiben.
 \item Grammatikalische Struktur der Programme bieten das \textit{Gerüst}, an dem der Übersetzungsprozess induktiv organisiert werden kann als auch systematische Fehlererkennung erlaubt.
 \item kontextfreie Struktur erlaubt \textit{inkrementelle Entwicklung}.
\end{itemize}
Nicht-kontextfreie Features werden abgesondert und einzeln regelhaft behandelt!
\begin{center}
\begin{psmatrix}[colsep=2cm,rowsep=0.5cm]
& & \framebox{Fehlerbehandlung}\\
Quellsprache & \framebox{Lexer} & \framebox{Parser} &
 \framebox{\begin{minipage}{1.5cm}
            weitere Phasen des Frontends
           \end{minipage}} & ...\\
& & \framebox{Symboltabelle}
\end{psmatrix}\ncline{<->}{1,3}{2,2}\ncline{<->}{1,3}{2,3}\ncline{<->}{1,3}{2,4}\ncline{2,1}{2,2}\ncline[offset=0.5em]{2,2}{2,3}\naput{Token}\ncline[offset=0.5em]{2,3}{2,2}\naput{getNextToken}
\ncline[linestyle=dashed]{2,3}{2,4}\naput{Parsebaum} \ncline{2,4}{2,5}\naput{Zwischen-}\nbput{code}
\ncline{<->}{3,3}{2,2}\ncline{<->}{3,3}{2,3}\ncline{<->}{3,3}{2,4}
\end{center}

\section{Drei Klassen von \emph{Syntaxanalysemethoden}}
\begin{itemize}
 \item universelle: jede kontextfreie Grammatik ist analysierbar ($O(n^3)$, \emph{nicht} akzeptabel)
 \item top-down: effiziente Teilklassen LL-Grammatiken
 \item bottom-up: effiziente Teilklassen LR-Grammatiken
\end{itemize}
L (Eingabe wird von \textbf{l}inks nach rechts gelesen) L (Linksherleitung)\\
L (Eingabe wird von \textbf{l}inks nach rechts gelesen) R (Rechtsherleitung, in umgekehrter Reihenfolge)
\[\text{LL} \subseteq \text{LR}\]
\begin{itemize}
 \item Parser für LL-Grammatiken sind leicht von Hand geschrieben
 \item Parser für LR-Grammatiken werden mit Parsergeneratoren generiert
\end{itemize}

\paragraph*{Beispielgrammatiken} (hier Ausdrücke):
\begin{align*}
 G_1{:} \quad E &\to E + E\ |\ E * E\ |\ \operatorname{id}\ |\ (E)
\end{align*}
Ist mehrdeutig, lässt sich \emph{nicht effektiv parsen}
\begin{align*}
 G_2{:} \quad E &\to E + T\ |\ T\\
 T &\to T * F\ |\ F\\
 F &\to \operatorname{id}\ |\ (E)
\end{align*}
\begin{itemize}
 \item eindeutig
 \item ist LR-Grammatik, d. h. effizient bottom-up zu parsen
\end{itemize}
\begin{align*}
 G_3{:} \quad E &\to TE'\\
 E' &\to + TE'\ |\ \varepsilon\\
 T &\to FT' \\
 T' &\to *FT'\ |\ \varepsilon\\
 F &\to \operatorname{id}\ |\ (E)
\end{align*}
\begin{itemize}
 \item eindeutig
 \item effizient zu parsen
 \item LL-Grammatik
\end{itemize}

\section{Fehlerbehandlung}
\paragraph*{Kriterien}
\begin{itemize}
 \item Präzision (Lokalisierung, Beschreibung)
 \item Qualität (Erkenngun möglichst aller Fehler, schnelle Erholung)
 \item Effizienz
\end{itemize}

\paragraph*{Typische Methoden}
\begin{itemize}
 \item \textbf{Konstruktorientierte (lokale) Methoden:}
  \begin{itemize}
  \item lokale Manipulation beim Auftreten eines Fehlers, z. B. Vertauschen zweier Symbole, Ersetzen eines ein anderes, Weglassen... \item effizient und in typischen Fällen zielführend
  \end{itemize}
 \item \textbf{Panic Mode (Panische Fehlererholung):}
  \begin{itemize}
  \item Streiche alle Symole bis man auf ein Synchronisationssymbol stößt, typische Trenn- und Abschlusssymbol, unter anderen z. B. "`;"'
  \item besonders effizient (überspringt weitere Fehler in fehlerhafter Teilstruktur)
  \end{itemize}
 \item \textbf{Globale Korrektur}
  \begin{itemize}
  \item Manipulation der Tokenfolge durch minimale Anzahl elementarer Operationen; wie
   \begin{itemize}
   \item weglassen
   \item vertauschen
   \item einfügen
   \item ersetzen
   \end{itemize}
  \item Optimal hinsichtlich Qualität; Zyklen möglich
  \end{itemize}
 \item \textbf{Fehlerproduktionen}
  \Bsp $\texttt{stmt} \to \textbf{if}\ \texttt{expr}\ \textbf{then}\ \texttt{stmt}$ (Pascal) \\
    Fehlerproduktion: $\texttt{stmt} \to \textbf{if}\ \texttt{expr}\ \texttt{stmt}$ (Java)
    \begin{itemize}
     \item Reperatur einfach möglich durch Einfügen von \textbf{then}
    \end{itemize}
    Sehr präzise, effizient, nicht allgemein verwendbar
 \item \textbf{Mehrdeutigkeit}
  \begin{itemize}
  \item gewisse Mehrdeutigkeit in der Grammatik enthalten, die durch zusätzliche Regeln aufgelöst wird\\
  \paragraph*{Typisches Beispiel:} \emph{Dangling-Else}
    \begin{verbatim}
S -> if E then S          // bedingte Anweisung
  |  if E then S else S   // Verzweigung
  |  other
    \end{verbatim}
    Methode (fast immer möglich) besteht aus Transformation der Grammatik
    \begin{verbatim}
S -> M                    // (matched-Anwesiung, geschlossen)
  | U                     // (unmatched Anweisung, offen)
M -> if E then M else M
  |  other
U -> if E then S
  |  if E then M else U
    \end{verbatim}
    Eindeutigkeit
  \end{itemize}
\end{itemize}



\section{Top-Down-Analyse}
\paragraph{allgemeines Verfahren} (für beliebige kontextfreie Grammatik $G = (\mathcal{N}, \mathcal{T}, S, \mathcal{P})$). Schreibe zu jedem $X$ aus $\mathcal{N}$ eine Methode wie folgt:
\begin{lstlisting}[language=Java,mathescape=true,morekeywords={to}]
void $X$() {
    waehle eine $X$-Produktion $X \to X_1...X_n$ aus // nichtdeterministisch 
    for ($i$ = 1 to $n$) {
        if ($X_i \in \mathcal{N}$) $X_i$();           // rofe $X_i$-Methode rekursiv auf
        else if ($X_i$ == input) input = nextToken();
        else Fehlermeldung/-behandlung
    }
}

// Top-Down-Parser
void parser() {
    Token input = nextToken();
    $S$();
}
\end{lstlisting}

\begin{itemize}
 \item \emph{nichtdeterministisch}: Inakzeptabel hinsichtlich Effizienz\\
         $\Rightarrow$ Löse Nichtdeterminismus auf!
 \item Auflösung von Nichtdeterminismen nicht immer möglich, aber in der Praxis in der Regel doch
 \item Rezept: Entscheide anhand des aktuellen Inputs (Lookahead)
\end{itemize}
\Bsp
\begin{align*}
 E &\to TE'\\
 E' &\to +TE'\ |\ \varepsilon \\
 T &\to FT' \\
 T' &\to *FT'\ |\ \varepsilon \\
 F &\to (E)\ |\ \operatorname{id}
\end{align*}
\psset{arrows=-}
\begin{description}
 \item[Eingabe] \texttt{id + id * id}
 \item[Syntaxanalyse]
 \begin{center}
    \pstree{\Tr*{$E$}}{
        \pstree{\Tr*{$T$}}{
            \pstree{\Tr*{$F$}}{
                \Tr*{id}
            }
            \pstree{\Tr*{$T'$}}{
                \Tr*{$\varepsilon$}
            }
        }
        \pstree{\Tr*{$E'$}}{
            \Tr*{+}
            \pstree{\Tr*{$T$}}{
                \pstree{\Tr*{$F$}}{
                    \Tr*{id}
                }
                \pstree{\Tr*{$T'$}}{
                    \Tr*{*}
                    \pstree{\Tr*{$F$}}{
                        \Tr*{id}
                    }
                    \pstree{\Tr*{$T'$}}{
                        \Tr*{$\varepsilon$}
                    }
                }
            }
            \Tr*{$E$}
        }
    }
 \end{center}
\end{description}
\subsection{Eliminiertung der Linksrekursion}
\begin{itemize}
 \item geht immer!
 \item Bisher: (direkte Linksrekursion) $A \to A \alpha\ |\ \beta \quad \rightsquigarrow \quad A \to \beta A'; A' \to \alpha A'\ |\ \varepsilon$
 \item mehrere liksrekursive Produktionen: $A \to A\alpha_1\ |\ ...\ |\ A\alpha_n\ |\ \beta_1\ |\ ...\ |\ \beta_m$
         \begin{align*}
          \rightsquigarrow \quad   A &\to \beta_1A'\ |\ ...\ |\ \beta_mA' \\
                              A' &\to \alpha_1A'\ |\ ...\ |\ \alpha_nA'\ |\ \varepsilon
         \end{align*}
\end{itemize}
\Bsp
\begin{align*}
 S &\to Aa\ |\ b \\
 A &\to Ac\ |\ Sd\ |\ \varepsilon
\end{align*}
\begin{description}
 \item[Eingabe:] $bda\$$
 \item[indirekte Linksrekursion:] $\underline{S} \to \underline{A}a \to Sda...$
     \begin{center}
         \pstree{\Tr{$S$}}{
             \pstree{\Tr{$A$}}{
                 \Tr{$S$}
                 \Tr{$d$}
             }
             \Tr{a}
         }
     \end{center}
\end{description}

\subsubsection{Allgemeines Verfahren zur Eliminierung der Linksrekursion}
\begin{lstlisting}[language=Java,mathescape=true,morekeywords={to}]
Sortiere alle Nichtdeterminimale $A_1...A_n$
for ($i$ = 1 to $k$) {
    for ($j$ = 1 to $i-1$) {
        Ersetze jede Produktion $A_i \to A_j\gamma$ durch $A \to \delta_1\gamma\ |\ ...\ |\ \delta_n\gamma$,
                wobei $A_j \to \delta_1\ |\ ...\ |\ \delta_n$ alle $A_j$-Produktionen sind
    }
    Eliminiere direkte Linksrekursion aus $A_i$-Prod.
}
\end{lstlisting}

\Bsp $S = A_1, \quad A = A_2$
\begin{description}
 \item $i = 1 \qquad \checkmark$
 \item $i = 2 \qquad$ Ergebnis:
     \begin{align*}
      S &\to Aa\ |\ b
      A &\to Ac\ |\ Aad\ |\ bd \ |\ \varepsilon
     \end{align*}
 \item Eliminiere direkte Linksrekursion aus $A_i$-Prod.
     \begin{align*}
      S &\to Aa\ |\ b \\
      A &\to bdA'\ |\ A' \\
      A &\to cA'\ |\ adA'\ |\ \varepsilon
     \end{align*}
\end{description}

\subsection{Linksfaktorisierung}
\begin{itemize}
 \item Vermeiden gemeinsamer, nichttrivialer Präfixe
 \Bsp $S \to \textbf{if}\ E\ \textbf{then}\ S\ |\ \textbf{if}\ E\ \textbf{then}\ S\ \textbf{else} S\ |\ ...$
 \paragraph*{Lösung:} 
     \begin{align*}
          S &\to \textbf{if}\ E\ \textbf{then} SS' \\
         S' &\to \textbf{else}\ S\ |\ \varepsilon
     \end{align*}
 \paragraph*{allg. Verfahren:} Bestimme zu jedem $A$ den längsten Präfix $\alpha$, der in zwei oder mehr Produktionen zu $A$ vorkommt.
    Falls $\alpha \neq \varepsilon$, dann ersetzen wir $A \to \alpha\beta_1\ |\ ...\ |\ \alpha\beta_n\ |\ \gamma_1\ |\ ...\ |\ \gamma_m$ durch
    \begin{align*}
     A &\to \alpha A'\ |\ \gamma_1\ |\ ...\ |\ \gamma_m \\
     A' &\to \beta_1\ |\ ...\ |\ \beta_n
    \end{align*}
    Erschöpfende Anwendungen
\end{itemize}

\subsection{Behandlung der $\varepsilon$-Produktionen}
\begin{itemize}
 \item $A \to a\ |\ \beta$
 \item $a \in \operatorname{FIRST}(\alpha)$? $A \to \alpha$ 
 \item $\varepsilon \in \operatorname{FIRST}(\beta)$ \\
       $\Rightarrow$ Lookahead steht hinter einem "`leeren Wort"'
 \item $\operatorname{FIRST}(\alpha) = \{a\ |\ \exists\beta \in \mathcal{T} \cup \mathcal{N} \cup \varepsilon{:}\ \alpha \xrightarrow{*} a\beta\} \cup \{\varepsilon\ |\ \alpha \xrightarrow{*} \varepsilon\}$\\
       $\operatorname{FOLLOW}(A) = \{a\ |\ \exists\alpha, \beta \in \mathcal{T} \cup \mathcal{N} \cup \varepsilon{:}\ S \to \alpha Aa\beta\} \cup \{\$\ |\ \exists\alpha{:}\ S \xrightarrow{*} \alpha A\}$\\
       \emph{formale Definition}
\end{itemize}

\subsubsection{Verfahren zur Berechnung der FIRST- und FOLLOW-Mengen}
\begin{itemize}
 \item FIRST-Mengen zu allen rechten Regelseiten
 \item FOLLOW-Mengen zu allen Nichtterminalen
 \item Wende folgende Regel an, bis keine neuen  Elemente hinzutreten
     \begin{enumerate}
      \item $\forall a \in \mathcal{T}{:}\ FIRST(a) = \{a\}$
      \item Für jede Produktion $A \to X_1...X_n$ mit $n \geq 1$ und jedes $1 \leq i < n$ füge $\operatorname{FIRST}(X_i)$ zu $\operatorname{FIRST}(A)$ hinzu, falls alle $\operatorname{FIRST}(X_j)$ das Element $\varepsilon$ enthalten mit $1 \leq j < i$.
      Füge $\varepsilon$ zu $\operatorname{FIRST}(A)$ hinzu, falls alle $\operatorname{FIRST}(X_i)$ das Element $\varepsilon$ enthalten ($1 \leq i \leq n$)
      \item Füge $\varepsilon$ zu $\operatorname{FIRST}(A)$, falls $A \to \varepsilon$ in $\mathcal{P}$
     \end{enumerate}
     Zur Bestimmung von $\operatorname{FIRST}(\alpha)$ zu jeder rechten Regelseite, verfahre analog zu Regel 2 $\alpha = X_1...X_n$
  \Bsp  $\mathcal{T} = \{+,*,(,),\textbf{id}\}$,
        $\mathcal{N} = \{E, E', T, T', F\}$
        \begin{align*}
         \operatorname{FIRST}(E) &= \{{\color{blue}(, \textbf{id}}\} \\
         \operatorname{FIRST}(E') &= \{+, \varepsilon\} \\
         \operatorname{FIRST}(T) &= \{{\color{red}(, \textbf{id}}\} \\
         \operatorname{FIRST}(T') &= \{*, \varepsilon\} \\
         \operatorname{FIRST}(F) &= \{(, \textbf{id}\}
        \end{align*}
        (1. Durchlauf, {\color{red}2. Durchlauf}, {\color{blue}3. Durchlauf})
        \begin{align*}
         \operatorname{FIRST}(+TE') &= \{+\}  & \operatorname{FIRST}(\varepsilon) &= \{\varepsilon\} \\
         \operatorname{FIRST}(*FT') &= \{*\} \\
         \operatorname{FIRST}((E)) &= \{(\}   & \operatorname{FIRST}(\textbf{id}) &= \{\textbf{id}\}
        \end{align*}
 \item FOLLOW-Generieriung ("`Erschöpfende Anwendung folgender Regeln"'):
     \begin{enumerate}
      \item Füge $\$$ zu $\operatorname{FOLLOW}(S)$ hinzu.
      \item Für jede Produktion der Form $A \to \alpha B \beta$ füge $\operatorname{FIRST}(\beta) \setminus \{\varepsilon\}$ zu $\operatorname{FOLLOW}(B)$ hinzu.
      \item Für jede Produktion der Form $A \to \alpha B$ oder $A \to \alpha B \beta$ mit $\varepsilon \in \operatorname{FIRST}(\beta)$ füge $\operatorname{FOLLOW}(A)$ zu $\operatorname{FOLLOW}(B)$ hinzu
     \end{enumerate}
  \Bsp  \begin{align*}
         \operatorname{FOLLOW}(E) &= \{\$,)\} \\
         \operatorname{FOLLOW}(E') &= \{\$,)\} \\
         \operatorname{FOLLOW}(T) &= \{+,\$, )\} \\
         \operatorname{FOLLOW}(T') &= \{+, \$, )\} \\
         \operatorname{FOLLOW}(F) &= \{*, +, \$, )\}
        \end{align*}
\end{itemize}
\Defi Eine kontextfreie Grammatik $G = (\mathcal{N},\mathcal{T},S,\mathcal{P})$ heißt \emph{LL(1)-Grammatik}, genau dann wenn zu je zwei Produktionen $A \to \alpha$ und $A \to \beta$ gilt:
\begin{enumerate}
 \item $\operatorname{FIRST}(\alpha) \cap \operatorname{FIRST}(\beta) = \emptyset$
 \item Falls $\varepsilon \in \operatorname(\alpha)$ enthält, dann $\operatorname{FIRST}(\beta) \cap \operatorname{FOLLOW}(A) = \emptyset$
\end{enumerate}

\subsection{Tabellengesteuerter Kellerautomat}
Fehlersituation:
\begin{enumerate}
 \item Stackelement stimmt nicht mit gelesener Eingabe überein
 \item Eintrag in Parse-Tabelle ist leer
\end{enumerate}

Stelle zu gegebene Grammatik $G$ die Parsetabelle $\mathcal{M}$ wie folgt auf:
\begin{enumerate}
 \item Zu $A \to \alpha$ und $a \in \operatorname{FIRST}(\alpha), a \in \mathcal{T}$, trage $A \to \alpha$ in das Feld $\mathcal{M}[A,\alpha]$ ein.
 \item Wenn $\varepsilon \in \operatorname{FIRST}(\alpha)$ und $A \to \alpha \in \mathcal{P}$ mit $a \in \operatorname{FOLLOW}(A)$, trage $A \to \alpha$ in $\mathcal{M}[A,a]$ ein. Wenn $\varepsilon \in \operatorname{FIRST}(\alpha)$ und $\$ \in \operatorname{FOLLOW}(A)$, dann trage $A \to \alpha$ in $\mathcal{M}[A, \$]$ ein.
 \item Alle Felder, die leer bleiben werden als Fehlersituation interpretiert!
\end{enumerate}
\Bsp Parsetabellle zu Beispielgrammatik
\begin{center}
    \begin{tabular}{r||c|c|c|c|c|c|}
         & +                    & *            & (           & )                    & \textbf{id}         & \$                   \\\hline\hline
      E  &                      &              & $E \to TE'$ &                      & $E \to TE'$         &                      \\\hline
      E' & $E' \to +TE'$        &              &             &                      &                     &                      \\\hline
      T  &                      &              & $T \to FT'$ &                      & $E \to FT'$         &                      \\\hline
      T' & $T' \to \varepsilon$ & $T \to *FT'$ &             & $T' \to \varepsilon$ &                     & $T' \to \varepsilon$ \\\hline
      F  &                      &              & $F \to (E)$ &                      & $F \to \textbf{id}$ &                      \\\hline
    \end{tabular}
\end{center}
\Lemma $G$ ist eindeutig.
\Bew Parsetabelle zu $G$ enthält in jedem Feld höchstens \emph{einen} Eintrag.

 


\newcommand{\FIRST}{\operatorname{FIRST}}
\newcommand{\FOLLOW}{\operatorname{FOLLOW}}

\Satz Wenn Parsetabelle $M$ zu kontextfreier Grammatik $G$ höchstens \emph{einen} Eintrag in jedem Feld besitzt, dann ist $G$ eindeutig und eine LL(1)-Grammatik.
\Bsp Dangling \textbf{else}:
\begin{align*}
 S &\to iEtSS'\ |\ a \\
 S' &\to eS\ |\ \varepsilon \\
 E &\to b
\end{align*}
\begin{itemize}
 \item \textbf{Eingabe:} $ib\underbrace{\underbrace{ibta}ea}$
\end{itemize}
\begin{align*}
 \FOLLOW(S') &= \{\$, e\}
\end{align*}

\begin{center}
    \begin{tabular}{r|c|c|c|c|c|c|}
     $\mathcal{M}$ & $i$            & $t$ & $e$            & $a$       & $b$       & $\$$                 \\\hline
     $S$           & $S \to iEtSS'$ &     &                & $S \to a$ &           &                      \\\hline
     $S'$          &                &     & \begin{minipage}{2cm}
                                             $S' \to eS$ \\
                                             $S' \to \varepsilon$
                                            \end{minipage} &           &           & $S' \to \varepsilon$ \\\hline
     $E$           &                &     &                &           & $E \to b$ &                      \\\hline
    \end{tabular}
\end{center}
\begin{itemize}
 \item \textbf{Methode brutale:} Lösche aus dem Feld mit zwei Einträgen, die Regel $S' \to \varepsilon$
\end{itemize}

\paragraph*{Algorithmus} (Tabellengesteuerte, prädiktibve Syntaxanalyse)
\begin{algorithmic}
 \REQUIRE $w \in \mathcal{T}^*$ gefolgt von \$ mit $\$ \in \mathcal{T}$ und Parsetabelle $\mathcal{M}$ zu gegegebener Grammatik $G$
 \ENSURE  Wenn $w \in \mathcal{L}(G)$ Linksableitung von $w$, Fehler sonst.
 \STATE Anfangszustand: Eingabe enthält $w\$$, Stack enthält $S\$$ ($S$ über $\$$)
 \STATE $a \gets$ first symbol of $w$
 \STATE $X \gets$ top of the stack
 \WHILE{$X \neq \$$}
  \IF{$X = a$}
   \STATE pop den Stack
   \STATE $a \gets$ nächstes Eingabesymbol
  \ELSIF{$X$ ist Terminal}
   \STATE Fehlermeldung
  \ELSIF{$\mathcal{M}[X,a]$ ist leer}
   \STATE Fehlermeldung
  \ELSIF{$\mathcal{M}[X,a]$ enthält $X \to Y_1...Y_k$}
   \STATE pop Stack
   \STATE push $Y_k, ..., Y_1$ auf Stack
   \STATE gib $X \to Y_1...Y_k$ aus
  \ENDIF
  \STATE Setze $X$ auf top of stack
 \ENDWHILE
 \IF{$a = \$$}
  \STATE accept
 \ENDIF
\end{algorithmic}

\begin{center}
 \begin{tabular}{l|l|l}
  \textbf{Eingabe} & \textbf{Stack}        & \textbf{Ausgabe} \\\hline
  $ibtibtaea\$$    & $S\$$                 & $S \to iEtSS'$   \\
                   & $iEtSS'\$$            &                  \\
  $btibtaea\$$     & $EtSS'\$$             & $E \to b$        \\
                   & $btSS'\$$             &                  \\
  $tibtaea\$$      & $tSS'\$$              &                  \\
  $ibtaea\$$       & $SS'\$$               & $S \to iEtSS'$   \\
                   & $\cancel{i}EtSS'S'\$$ & $E \to b$        \\
                   & $\cancel{b}tSS'S'\$$  & $S \to a$        \\
                   & $\cancel{a}S'S'\$$    & $S' \to eS$      \\
                   & $\cancel{e}SS'\$$     & $S \to a$        \\
                   & $\cancel{a}S'\$$      & $S' \to \varepsilon$ \\
                   & $\$$                  & accept!
 \end{tabular}
\end{center}

\subsection{Fehlerbehandlung (panic mode) (Heuristiken)}
\begin{enumerate}
 \item Nimm alle Symbole aus $\FOLLOW(A)$ zu Synchronisationsmenge für $A$. \\
       Überspringe im Fehlerfall alle Eingabesymbole bis zum nächsten Synchronisationssymbol. \\
       Entferne $A$ vom Stack.
 \item (Reflektiere hierarchische Sprachstruktur)
 \item Füge $\FIRST(A)$ zur Synchronisationsmenge von $A$. Überspringe alle Eingabesymbole bist zu einem Terminal aus $\FIRST(A)$, entferne $A$ \emph{nicht} vom Stack!
 \item (Wende $\varepsilon$-Produktionen im Default-Fall an)
 \item Wenn Kellerspitze $b$ ein Terminal ungleich aktueller Eingabe ist, dann entferne $b$ vom Keller und gib Warnung aus: "`$b$ wurde eingefügt!"'
\end{enumerate}

Parsetabelle für $G$ (auf arithmetische Ausdrücke) mit Synchronisationseinträgen
\begin{align*}
 \FIRST(E) = \FIRST(T) = \FIRST(F) &= \{(,\textbf{id}\} \\
 \FIRST(E') &= \{+,\varepsilon\} \\
 \FIRST(T') &= \{*,\varepsilon\} \\\hline
 \FOLLOW(E) = \FOLLOW(E') &= \{),\$\} \\
 \FOLLOW(T) = \FOLLOW(T') &= \{+,),\$\} \\
 \FOLLOW(F) &= \{+, *, ), \$\}
\end{align*}

\begin{center}
    \newcommand{\synch}{\underline{\textbf{synch}}}
    \begin{tabular}{r|c|c|c|c|c|c|}
             & +                    & *             & (           & )                    & \textbf{id} & \$                   \\\hline
        $E$  &                      &               & $E \to TE'$ & \synch               & $E \to TE'$ & \synch               \\\hline
        $E'$ & $E' \to +TE'$        &               &             & $E' \to \varepsilon$ &             & $E' \to \varepsilon$ \\\hline
        $T$  & \synch               &               & $T \to FT'$ & \synch               & $T \to ET'$ & \synch               \\\hline
        $T'$ & $T' \to \varepsilon$ & $T' \to *FT'$ &             & $T' \to \varepsilon$ &             & $T' \to \varepsilon$ \\\hline
        $F$  & \synch               & \synch        & $F \to (E)$ & \synch               & $F \to \textbf{id}$ & \synch       \\\hline
    \end{tabular}
\end{center}
Wende Heuristiken 1, 3 und 5 an. Genügt im Allgemeinen für Ausdrücke.
\begin{center}
    \begin{tabular}{c|c|c}
        Stack & Eingabe & Ausgabe \\\hline
    \end{tabular}
\end{center}





\section{Bottom-Up-Syntaxanalyse}
\subsection{Einführung}
Naives Voegehen am Beispiel einfacher artihmetischer Ausdrücke
\begin{align*}
 E &\to E+T\ |\ T \tag{1. / 2.} \\
 T &\to T*F\ |\ F \tag{3. / 4.}\\
 F &\to (E)\ |\ i \tag{5. / 6.}
\end{align*}
\[                        \underline{i} * i 
\underset{6}{\rightarrow} \underline{F} * i 
\underset{4}{\rightarrow} T * \underline{i}
\underset{6}{\rightarrow} \underline{T * F}
\underset{3}{\rightarrow} \underline{T}
\underset{2}{\rightarrow} E
\]
in umgekehrter Reihenfolge ergibt sich eine Rechtsherleitung des analysierten Wortes. \\
alternativ:
\[                        \underline{T} * i
\underset{2}{\rightarrow} \underline{E} * i
\underset{6}{\rightarrow} E * \underline{F}
\underset{4}{\rightarrow} E * \underline{T}
\underset{2}{\rightarrow} E * E \tag{Sackgasse!}
\]
\paragraph*{Informelle Definition:} Ein \emph{Handle} in einer umgekehrten Rechtsherleitung ist der Teilstring $\beta$ der aktuellen Rechtssatzform, der als nächstes mit der Regel $A \to \beta$ auf $A$ zu reduzieren ist.
\paragraph*{Formale Definition:} Sei $S \underset{rm}{\xrightarrow{*}} \alpha Aw \underset{rm}{\rightarrow} \alpha \beta w$ eine Rechtsherleitung. Die Produktion $A \to \beta$ zusammen mit ihrem Vorkommen nach $\alpha$ heißt \emph{Handle} von $\alpha \beta w$
\Lemma Wenn $G$ eindeutig ist, dann gibt es zu jeder Rechtssatzform genau einen Handle.
\paragraph*{Konvention:} Wenn es zu $\beta$ nur eine rechte Regelseite der Form $\beta$ gibt und $\beta$ nur einmal in $\alpha \beta w$ vorkommt, dann spricht man: "`$\beta$ ist Handle von $\alpha \beta w$."'

\subsection{Bottom-Up-Syntaxanalyse durch Handle-Produktionen}
Sei $G$ eine kontextfreie Grammatik und $w \in \mathcal{L}(G)$. Sei $S = \gamma_0 \to \gamma_1 \to ... \gamma_n = w$ eine Rechtsherleitung von $w$.
\begin{enumerate}
 \item Beginne mit $i = n$
 \item \emph{Lokalisiere den Handle} in $\gamma_i$ und ersetze ihn durch das entsprechende Nichtterminal, um $\gamma_{i-1}$ zu erhalten.
 \item Wiederhole 2. bis man auf $\gamma_0 = S$ stößt.
\end{enumerate}
\paragraph*{Problem:} Wie lokalisiert man den Handle? (später)\\

\subsection{Stack-Implementierung der Bottom-Up-Analyse (Shift-Reduce-Parser)}
\begin{itemize}
 \item Verwende einen Stack, der Symbole $\mathcal{N} \cup \mathcal{T}$ enthält sowie \$ am Grund
 \item Eingabe $w \in \mathcal{T}^*$, mit $w\$$ auf Input (hier von links nach rechts)
 \item Schreibe top von Stack jeweils nach rechts.
 \item Anfangszustand: Stack: $\$$; Input: $w\$$
\end{itemize}

\paragraph*{Verfahren:}
\begin{itemize}
 \item Schreibe (shift) so lange Symbole vom Input auf den Stack, bis der Handle auf der Kellerspitze erscheint.
 \item Reduziere (reduce) am Handle auf der Kellerspitze
 \item Wiederhole diesen Prozess, bis entweder $\$S$ auf dem Keller erscheint mit Input $= \$$m dann akzeptiere $w$; oder ein Fehler auftritt, der gemeldet wird.
\end{itemize}

\begin{center}
 \begin{tabular}{|c|c|c|c|}
  Stack   & Input   & Aktion & Ausgabe     \\\hline
  $\$$    & $i*i\$$ & shift  &             \\
  $\$i$   & $*i\$$  & reduce & $F \to i$   \\
  $\$F$   & $*i\$$  & reduce & $T \to F$   \\
  $\$T$   & $*i\$$  & shift  &             \\
  $\$T*$  & $i\$$   & shift  &             \\
  $\$T*i$ & $\$$    & reduce & $F \to i$   \\
  $\$T*F$ & $\$$    & reduce & $T \to T*F$ \\
  $\$T$   & $\$$    & reduce & $E \to T$   \\
  $\$E$   & $\$$    & accept &
 \end{tabular}
\end{center}
Parsebaum:
\begin{center}
\pstree{\Tr{$E$}}{
    \pstree{\Tr{$T$}}{
        \pstree{\Tr{$T$}}{
            \pstree{\Tr{$F$}}{
                \Tr{$i$}
            }
        }
        \Tr{*}
        \pstree{\Tr{$F$}}{
            \Tr{$i$}
        }
    }
}
\end{center}
Nichtdeterministisch
\paragraph*{Konflikte:}
\begin{itemize}
 \item \emph{shift-reduce}-Konflikte
 \item \emph{reduce-reduce}-Konflikte
\end{itemize}

\subsection{Hierarchie innerhalb der Klasse der kontextfreien Sprachen}
\begin{description}
 \item[DPDA] Deterministischer Kellerautomat
 \item[NPDA] Nicht-deterministischer Kellerautomat
\end{description}
\[\mathcal{L}(\text{LL(1)}) \subsetneq \mathcal{L}(\text{DPDA}) \subsetneq \mathcal{L}(\text{DPDA}) = \text{alle kontextfreien Sprachen}\]
\[\text{LL(1)} \subseteq \text{LL($k$)}, \mathcal{L}(\text{LL(1)}) \subsetneq \mathcal{L}(\text{LL(2)}) ... \subsetneq \mathcal{L}(\text{LL($k$)}) \subsetneq \mathcal{L}(\text{DPDA})\]
\Defi Grammatik heißt \emph{LR(1)-Grammatik}, genau dann wenn es einen deterministischen Kellerautomaten gibt, der mit Lookahead $k$ eine Rechtsherleitung der eingabe in umgekehrter Reihenfolge liefert.
\Satz LR(1) ist mächtiger als $\textrm{LL} = \bigcup\limits_{k \in \mathbb{N}} \textrm{LL}(k)$




\subsection{Item-Mengen}
\paragraph*{Problem:} Erkennen den Handle auf der Kellerspitze!
Bottom-Up-Syntaxanalyse, LR-Parser
\paragraph*{Heute:} Wesentliche Grundlage fpr deterministisch einfache LR-Parser (simple LR-Parser). In der Praxis verwendet man Parsergeneratoren
\paragraph*{Erste Idee:} Verwende einen endl. Automaten (hier: LR(0)-Automat) zur Steuerung des Analyseprozesses.
\paragraph*{Zustandsmenge} des LR(0)-Automaten: Erweiterung der kontextfreien Grammatik $G$ um neues Startsymbol $S'$ und zusätzlichen Produktionen.
\begin{description}
 \item[Item:] $S' \to .S$
 \Defi Jede Produktion von $G'$, in die auf der rechten Regelseite ein Punkt eingefügt wurde, heißt \emph{Item}.
 \Bsp Zu einer Produktion der Form $A \to XYZ$ gehören vier Items:
     \[ A \to .XYZ, A \to X.YZ, A \to XY.Z\ \text{und}\ A \to XYZ. \]
\end{description}
Wir Konstruieren geeignete Item-Mengen als Zustände des LR(0)-Automaten. Wir benötigen zwei Funktionen: Closure und Goto.
\Defi (Closure) Sei $I$ eine Menge von Items. Closure($I$) verhält man unter erschöpfender Anwendung folgender Regeln.
\begin{enumerate}
 \item $I \subseteq \text{Closure}(I)$,
 \item Wenn $A \to \alpha.B\beta \in \text{Closure}(I)$ und $B \to \gamma \in P$, dann füge $B \to .\gamma$ zu $\text{Closure}(I)$ hinzu.
\end{enumerate}
\Bsp
\begin{align*}
 E' & \to E \\
 E  & \to E + T\ |\ T \\
 T  & \to T * F\ |\ F \\
 F  & \to ( E )\ |\ i
\end{align*}
$\text{Closure}(\{E' \to .E\}) = \{E' \to .E, E \to .E+T, E \to .T, T \to .T \times F, T \to .F, F \to .(E), F \to i\}$ ist der Anfangszustand des LR(0)-Automaten fpr unsere Beispielgrammatik.
\Defi (Goto) $\text{Goto}(I,X) := \text{Closure}(\{A \to \alpha X.\beta\ |\ A \to \alpha .X\beta \in I\})$
\Defi Zustandsmenge $\mathcal{C}$ (\emph{harmonische Sammlung von Item-Mengen}) des LR(0)-Automaten konstruiert man unter erschöpfender Anwendung folgender Regeln.
\begin{enumerate}
 \item $I_0 := \text{Closure}(\{S' \to .S\}) \in \mathcal{C}$
 \item Füge fpr jedes Symbos $X \in \mathcal{N} \cup \mathcal{T}$ der Grammatik $G$ und jedem $I$ aus $\mathcal{C}$ die Item-Menge $\text{Goto}(I,X)$ zu $\mathcal{C}$ hinzu, falls diese ungleich $\emptyset$ ist.
\end{enumerate}
\begin{center}
[Beispiel eines LR(0)-Automaten]
\end{center}
Lösung der shift/reduce-Konflikte durch shift, falls es einen entsprechend markierten Übergang gibt, reduce sonst.
\Bsp $i*i\$$
\begin{center}
    \begin{tabular}{c|c|c|c|c}
        Stack (Zustände) & Symbole (redundant) & Eingabe & Aktion & Ausgabe \\\hline
        0 & $\$$ & $i*i\$$ & shift & \\
        05 & $\$i$ & $*i\$$ & reduce & $F \to i$ \\
        03 & $\$F$ & $*i\$$ & reduce & $T \to F$ \\
        02 & $\$T$ & $*i\$$ & shift & \\
        027 & $\$T*$ & $i\$$ & shift & \\
        0275 & $\$T*i$ & $\$$ & reduce & $F \to i$ \\
        027\underline{10} & $\$T*F$ & $\$$ & reduce & $T \to T * F$ \\
        02 & $\$T$ & $\$$ & reduce & $E \to T$ \\
        01 & $\$E$ & $\$$ & accept
    \end{tabular}
\end{center}



\section{Zusammenfassung}
Aufbau eines SLR-Parsers (simple LR(1)-Parser)
\paragraph*{Struktur}
\begin{center}
    \begin{psmatrix}
        & Eingabe: \begin{tabular}{|c|c|c|c|c|c|}\hline $a_1$ & ... & \rnode{a_i}{$a_i$} & ... & $a_n$ & \$ \\\hline\end{tabular} $w \in G$ & \\
        Stack \begin{tabular}{|c|}$S_n$ \\\hline $\vdots$ \\\hline $S_1$ \\\hline 0 \\\hline\end{tabular} & \framebox{Parse-Programm}
        & \begin{minipage}{5cm}Folge von Produktionen, umgekehrte Rechtsherleitung\end{minipage} \\
        & \begin{tabular}{c|c|c|}\cline{2-3} 0 & \multirow{3}{*}{\rnode{ACTION}{\Large ACTION}} & \multirow{3}{*}{\rnode{GOTO}{\Large GOTO}} \\
                                           $\vdots$ & & \\
                                           $k$ & & \\\cline{2-3}
       \end{tabular}

    \end{psmatrix}
    \psset{arrows=->}
    \ncline{2,2}{a_i}\ncline{2,2}{2,1}\ncline{2,2}{2,3}\naput{Ausgabe}
    \ncline{2,2}{ACTION}\ncline{2,2}{GOTO}
\end{center}
Elemente auf dem Stack sind Nummern der Itemmenge aus der kanonischen LR(0)-Item-Sammlung. Zu jeder Item-Menge gehört genau ein Grammatiksymbol.\\
Die Steuerung erfolgt über eine Parse-Tabelle, deren Zeilen mit Zustandsnummern eines LR(0)-Automaten markiert sind. Der ACTION-Teil hat Spalten mit Terminalsymbolen und $\$$ und der GOTO-Teil hat Spalten mit Nichtterminalsymbolen. Die Einträge der Parse-Tabelle werden wie folgt bestimmt
\begin{enumerate}
 \item ACTION$[i,a]$ erhält "`shift $I_j$"' (kurz s$j$), wenn $[A \to \alpha.a\beta] \in I_i$ und GOTO$(I_i,a) = I_j$.
 \item ACTION$[i,a]$ erhält "`reduce $A \to \alpha$"' (kurz r$j$), wenn $[A \to \alpha.] \in I_i, A \neq S'$ und $a \in \text{FOLLOW}(A)$. Beachte: $j$ ist die Nummer der Produktion $A \to \alpha.$
 \item ACTION$[i,\$]$ erhält "`accept"' (kurz acc), wenn $[S' \to S.] \in I_i$.
 \item GOTO$[i,A]$ erhält den Eintrag $j$, wenn GOTO$(I_i, A) = I_j$
 \item Leere Felder werden als Fehler interpretiert.
\end{enumerate}
\begin{itemize}
 \item Mehrfacheinträge im ACTION-Teil, die durch Regeln 1. und 2. entstehen können, signalisieren, dass die Grammatik \emph{nicht} SLR(1) ist
\end{itemize}
\paragraph*{Parse-Tabelle für einfache Ausdrücke}
\begin{align*}
 1. E &\to E + T & \text{FOLLOW}(E) &= \{+,),\$\} \\
 2. E &\to T     & \text{FOLLOW}(T) &= \{+,*,),\$\} \\
 3. T &\to T * F & \text{FOLLOW}(F) &= \{+,*,),\$\} \\
 4. T &\to F   \\
 5. F &\to (E) \\
 6. F &\to i
\end{align*}
\begin{center}
    \begin{tabular}{c|c|c|c|c|c|c||c|c|c|}
            & $i$ & $+$ & $*$ & $($ & $)$  & $\$$ & $E$ & $T$ & $F$ \\\hline\hline
        0   & s5  &     &     & s4  &      &      & 1   & 2   & 3   \\\hline
        1   &     & s6  &     &     &      & acc  &     &     &     \\\hline
        2   &     & r2  & s7  &     & r2   & r2   &     &     &     \\\hline
        3   &     & r4  & r4  &     & r4   & r4   &     &     &     \\\hline
        4   & s5  &     & s4  &     &      &      & 8   & 2   & 3   \\\hline
        5   &     & r6  & r6  &     & r6   & r6   &     &     &     \\\hline
        6   & s5  &     & s4  &     &      &      &     & 9   & 3   \\\hline
        7   & s5  &     & s4  &     &      &      &     &     & 10  \\\hline
        8   &     & s6  &     &     & s11  &      &     &     &     \\\hline
        9   &     & r1  & s7  &     & r1   & r1   &     &     &     \\\hline
        10  &     & r3  & r3  &     & r3   & r3   &     &     &     \\\hline
        11  &     & r5  & r5  &     & r5   & r5   &     &     &     \\\hline
    \end{tabular}
\end{center}

\paragraph*{Konfiguration des LR-Parsers}
Der Anfangszustand ist $(0, a_1...a_n\$)$. Der jeweilige Folgezustand zu$(s_0...s_m, a_ia_{i+1}...a_n\$)$  berechnet sich wie folgt:
\begin{enumerate}
 \item Wenn ACTION$[s_m,a_i] = \text{s}j$, dann setze Folgezustand auf $(s_0...s_mj, a_{i+1}...a_n\$)$. Beachte: $a_i$ gehört zu $I_j$
 \item Denn ACTION$[s_m,a_i] = \text{r}j$ und die $j$-te Produktion die Form $A \to X_1 ... X_k$ und GOTO$[s_m, A] = t$ ist, dann setze Folgezustand auf $(s_0...s_{m-k}t, a_i...a_n\$)$.
 \item Wenn ACTION$[s_m,\$] = \text{acc}$, dann signalisiere erfolgreiches Parsen.
 \item Wenn ACTION$[s_m,a_i]$ leer ist, dann rufe Fehlerbehandlung auf.
\end{enumerate}

\paragraph*{Brauchbarer Präfix}
Im Keller steht immer eine Präfix einer Rechtssatzform. Die Konfiguration enthält immer eine Rechtssatzform (unter Konkatination).
\Defi (Brauchbarer Präfix) Ein Präfix einer Rechtssatzform erstreckt sich nicht über den Handle hinaus.
\Lemma SLR-Parser erkennen brauchbare Präfixe.
\chapter{Syntaxgerichtete Übersetzung}
Kern des Übersetzerbaus
\paragraph*{Grammatik:} Struktur des Quellprogramms steuert die Übersetzung.
\paragraph*{2 Ausprägungen:}
\begin{itemize}
 \item Allgemeine Methode: Syntaxgerichtete Definition (SDD)
 \item Spezielle Methode: Übersetzungsschema
\end{itemize}

\section{Syntaxgerichtete Definition}
\Defi Eine \emph{syntaxgerichtete Definition} ist eine kontextfreie Grammatik zusammen mit Attributen und semantischen Regeln. Jedem Symbol der Grammatik sind endlich viele Attribute zugeordnet. Jeder Produktion sind endlich viele semantische Regeln zugeordnet, wobei eine semantische Regel zu einer Produktion $A \to X_1 ... X_n$ eine Berechnungsvorschrift für \emph{ein} Atrribut von $A, X_1, ...$ oder $X_n$ darstellt. Typischer Weise schreiben wir semantische Regeln in der Implementierungssprache.
\paragraph*{Hier liegt die Kreativität des Übersetzerbauers}
\paragraph*{Wiederholung:}
In der Regel liefert die lexikalische Analyse Attributwerte (lexval) für die Terminalsymbole!
\begin{center}
    \begin{tabular}{l|l}
        Produktion             & semantische Regeln      \\\hline
        $L \to E\verb!\n!$     & $L.v = E.v$             \\\hline
        $E \to E_1 \texttt{+} T$&$E.v = E_1.v + T.v$     \\\hline
        $E \to T$              & $E.v = T.v$             \\\hline
        $T \to T_1 \texttt{*} F$&$T.v = T_1.v \cdot F.v$ \\\hline
        $T \to F$              & $T.v = F.v$             \\\hline
        $F \to (E)$            & $F.v = E.v$             \\\hline
        $F \to d$              & $F.v = d.\text{lexval}$
    \end{tabular}
\end{center}
\Bsp Für \emph{dekorierten Parsebaum}: \verb!3 * 5 + 4\n!
\begin{center}
\pstree{\Tr{$L$}\nput{10}{\pssucc}{\color{blue}$.v = 19$}}{
    \pstree{\Tr{$E$}\nput{170}{\pssucc}{\color{blue}$.v = 19$}}{
        \pstree{\Tr{$E$}\nput{170}{\pssucc}{\color{blue}$.v = 15$}}{
            \pstree{\Tr{$T$}\nput{170}{\pssucc}{\color{blue}$.v = 15$}}{
                \pstree{\Tr{$T$}\nput{170}{\pssucc}{\color{blue}$.v = 3$}}{
                    \pstree{\Tr{$F$}\nput{170}{\pssucc}{\color{blue}$.v = 3$}}{
                        \Tr{$d$}\nput{170}{\pssucc}{\color{blue}$.\text{lexval} = 3$}
                    }
                }
                \Tr{\texttt{*}}
                \pstree{\Tr{$F$}\nput{80}{\pssucc}{\color{blue}$.v = 5$}}{
                    \Tr{$d$}\nput{10}{\pssucc}{\color{blue}$.\text{lexval} = 5$}
                }
            }
        }
        \Tr{\texttt{+}}
        \pstree{\Tr{$T$}\nput{10}{\pssucc}{\color{blue}$.v = 4$}}{
            \pstree{\Tr{$F$}\nput{10}{\pssucc}{\color{blue}$.v = 4$}}{
                \Tr{$d$}\nput{10}{\pssucc}{\color{blue}$.\text{lexval} = 4$}
            }
        }
    }
    \Tr{\texttt{\textbackslash n}}
} 
\end{center}
Einbau in LR-Parser \emph{trivial.}

\subsection{Attribute}
Wir unterscheide zwei Sorten von Attributen: \emph{Synthetisierte} und \emph{ererbte}
\begin{itemize}
 \item Ein Attribut $a$ zu einem Nichtterminal $A$ heißt synthetisiert, wenn die zugehörige semantische Regel zu einer $A$-Produktion gehört.
 \item Ein Attribut $b$ zu einem Nichtterminal $B$ heißt ererbt, wenn die semantische Regel zur Berechnung von $B.b$ einer Produktion zugeordnet ist, in der $B$ auf der rechten Regelseite vorkommt.
\end{itemize}
\Defi Eine SDD heißt $S$-Attributierung, genau dann wenn jedes Attribut synthetisiert ist.
\paragraph*{Erweiterung:} Gewünschte Nebenwirkungen (\emph{side effects}):
\begin{center}
    \begin{tabular}{l|p{4cm}}
        Produktion             & semantische Regeln      \\\hline
        $L \to E\verb!\n!$     & $L.v = E.v$\texttt{;} \newline \color{red}\verb!print(!$L.v$\verb!);!             \\\hline
        $E \to E_1 \texttt{+} T$&$E.v = E_1.v + T.v$     \\\hline
        $E \to T$              & $E.v = T.v$             \\\hline
        $T \to T_1 \texttt{*} F$&$T.v = T_1.v \cdot F.v$ \\\hline
        $T \to F$              & $T.v = F.v$             \\\hline
        $F \to (E)$            & $F.v = E.v$             \\\hline
        $F \to d$              & $F.v = d.\text{lexval}$
    \end{tabular}
\end{center}
{\large Achtung: Reihenfolge!!}
\par Im Allgemeinen ist eine Attibutierung \emph{nicht} immer möglich, da Zyklen vorkommen können.
\begin{center}
    \begin{tabular}{l|p{3cm}}
        Prod. & sem. Regeln \\\hline
        $A \to B$ & $A.s = B.i$ \newline $B.i = A.s + 1$
    \end{tabular}
    \hspace*{3cm}
    \pstree{\Tr{A}\nput{10}{\pssucc}{\color{blue}.\rnode{s}{$s$}}}{\Tr{B}\nput{10}{\pssucc}{\color{blue}.\rnode{i}{$i$}}}
    \nccurve[linecolor=blue,angleA=180,angleB=180]{->}{s}{i}
    \nccurve[linecolor=blue,angleA=0,angleB=0]{->}{i}{s}
\end{center}
Nicht immer ist $S$-Attributierung möglich!
\par Betrachte kontextfreie Grammatik nach Eliminierung der Linksrekursion.
\begin{center}
    \begin{tabular}{l|p{3cm}}
        $T \to FT'$ & $T'.i = F.v$ \newline $T.v = T'.s$ \\
        $T' \to \texttt{*}FT'_1$ & $T'_1.i = T'.i \cdot F.v$ \newline $T'.s = T'_1.s$\\
        $T' \to \varepsilon$ & $T'.s = T'.i$\\
        $F \to d$ & $F.v = d.l$
    \end{tabular}
\end{center}
\verb!3*5!
\begin{center}
\psset{treesep=2cm}
\pstree{\Tr{$T$}\nput{10}{\pssucc}{\color{blue}$.v$}}{
    \pstree{\Tr{$E$}\nput{170}{\pssucc}{\color{blue}$.\rnode{v1}{v = 3}$}}{
        \Tr{$d$}\nput{170}{\pssucc}{\color{blue}$.l = 3$}
    }
    \pstree{\Tr{$T'$}\nput{10}{\pssucc}{\color{blue}$.\rnode{i1}{i = 3}$}\nput{45}{\pssucc}{\color{blue}$.\rnode{s1}{s = 15}$}}{
        \Tr{\texttt{*}}
        \pstree{\Tr{$F$}\nput{10}{\pssucc}{\color{blue}$.v = 5$}}{
            \Tr{$d$}\nput{10}{\pssucc}{\color{blue}$.l = 5$}
        }
        \pstree{\Tr{$T$}\nput{10}{\pssucc}{\color{blue}$.i = 15$}\nput{45}{\pssucc}{\color{blue}$.\rnode{s2}{s = 15}$}}{
            \Tr{$\varepsilon$}
        }
    }
}
\ncarc[linecolor=blue]{->}{v1}{i1}
\ncarc[linecolor=blue]{->}{s2}{s1}
\end{center}
Beim Teilbaum $T(\texttt{*},F,T)$ ist das ererbte Attibut $i$(nherited) notwendig.
\Defi Gegebein sei eine SDD und ein zugehöriger Parsebaum $P$.
\begin{enumerate}
 \item Zu jedem Knoten von $P$ und Symbol $X$ und jedem Attribut $a$ von $X$ erhält der \emph{Abhängigkeitsgraph} $\mathcal{A}_P$ einen Knoten mit der Marke $a$.
 \item Wenn eine semantische Regel zu Berechnung von $X.a$ den Wert $Y.b$ benötig, füge eine Kante $Y.b$ nach $X.a$ in $\mathcal{A}_P$ ein.
\end{enumerate}
Auswertungsreihenfolge kann durch \emph{topologische Sortierung} festgelegt werden.
\Defi Sei $N_1...N_k$ eine Aufzählung der Knoten in einem Abhängigkeitsgraphen $\mathcal{A}_P$. $N_1...N_k$ heißt \emph{topologische Sortierung}, genau dann wenn für jede Kante von $N_i$ nach $N_j$ gilt: $i < j$
\Lemma Jede topologische Sortierung bedingt eine Auswertungsstrategie.
\Satz Wenn die zugehörige SDD zyklenfrei ist, so erhält man eine topologische Sortierung wie folgt:
\Bew Starte mit beliebigem Knoten. Gehe jeweils zu einem seiner Vorgänger, bis es keinen Vorgänger mehr gibt. Setze diesen Knoten als $N_1$, entferne $N_1$ aus dem Graphen und fahre fort.

\paragraph*{allgemeine Methode:} Abhängigkeitsgraph zu Parsebaum $P$, topologische Sortierung \emph{aufwendig}!
\paragraph*{Wunsch:} \emph{Integrierte} Syntaxanalyse und Attributierung\\
$\Rightarrow$ Organisiere die SDD so, dass eine Berechnung der Attributwerte in der Riehenfolge möglich ist, in der die Knoten des Parsebaumes erzeugt (besucht) werden
\paragraph*{$S$-Attributierung:} Nur synthetisierte Attribute!
\begin{lstlisting}[language=Java,mathescape=True]
postorder ($N$) {
    for (jeden Nachfolger $C$ von links nach rechts) postorder(C);
    berechne Attributwerte von $N$;
}
\end{lstlisting}

\paragraph*{$L$-Attributierung} (Information darf im Parsebaum von \emph{l}inks nachr rechts fließen)
\begin{lstlisting}[language=Java,mathescape=True]
depthfirst ($N$) {
    for (jeden Nachfolger $C$ von $N$) {
        berechne ererbte Attribute von $C$;
        depthfirst($C$);
    }
    berechne synthetisierte Attributwerte von $N$;
}
\end{lstlisting}

\Defi Eine SDD heißt \emph{$L$-Attributierung}, wenn in jeder semantischen Regel zu einer Prod. $A \to X_1...X_n$ die Berechnung eines \emph{ererbten} Attributs von $X_j$ ($1 \leq j \leq n$) gilt:
\begin{itemize}
\item   Die Berechnung von diesem Attribut hängt nur ab von:
        \begin{enumerate}
        \item   ererbte Attribute von $A$.
        \item   Attributwerten der Symbole $X_1...X_{j-1}$.
        \item   Attributwerte von $X_j$, wenn dadurch keine Zyklen entstehen.
        \end{enumerate}
\end{itemize}

\paragraph*{Typische anwendung von SDD's} mit kontrollierten \emph{Nebenwirkungen}
\begin{itemize}
\item   (bereits gesehen) Drucke Ergebnis aus
\item   Trage Typinformationen von Variablen in Symboltabelle ein
        \begin{center}
            \begin{tabular}{|c|c|}
                    & Typ \\\cline{1-1}
            float   &     \\\cline{1-1}
            int     &     \\\cline{1-1}
            $\vdots$&     \\\hline
            $x$     & float    \\\hline
            $y$     & float    \\\hline
            $z$     & float    \\\hline
            \multicolumn{1}{c}{$\vdots$} & \multicolumn{1}{c}{}
            \end{tabular}
        \end{center}
        \paragraph*{Hilfsfunktion:} \lstinline!addType($v$,$t$)! trägt den Typ von $t$ für die Variable $v$ ein.
        \begin{center}
            \begin{tabular}{|l|l|}
                Produktion & Semantische Regeln \\\hline
                $D \to TL$ & $L.t = T.t$\\\hline
                $T \to \text{float}$ & $T.t = \text{float}$ \\\hline
                $T \to \text{int}$ & $T.t = \text{int}$ \\\hline 
                $T \to L_1, i$ & $L_1.t = L.t$; $i.t = L.t$; $\text{addType}(i.entry, i.t)$ \\\hline 
                $T \to i$ & $i.t = L.t$; $\text{addType}(i.entry, i.t)$ \\\hline
            \end{tabular}
        \end{center}
        \Bsp float $x, y, z$
        \begin{center}
            \psset{treesep=2cm,levelsep=1.5cm}
            \pstree{\Tr{$D$}}{
                \pstree{\Tr{$T$}\nput{10}{\pssucc}{\color{red}\small\rnode{T1_t}{.}t = float}}{
                    \Tr{float}
                }
                \pstree{\Tr{$L$}\nput{0}{\pssucc}{\color{red}\small\rnode{L1_t}{.}t = float}}{
                    \pstree{\Tr{$L$}\nput{0}{\pssucc}{\color{red}\small\rnode{L2_t}{.}t = float}}{
                        \pstree{\Tr{$L$}\nput{0}{\pssucc}{\color{red}\small\rnode{L3_t}{.}t = float}}{
                            \Tr{$i$}\nput{0}{\pssucc}{
                                \begin{minipage}{3cm}
                                    \color{red}\small
                                    \rnode{i3_t}{.}t = float\\
                                    .entry = $x$\\
                                    \rnode{i3_addType}{.}addType(i.entry, i.t)
                                \end{minipage}
                            }
                        }
                        \Tr{,}
                        \Tr{$i$}\nput{0}{\pssucc}{
                            \begin{minipage}{3cm}
                                \color{red}\small
                                \rnode{i2_t}{.}t = float\\
                                .entry = $x$\\
                                \rnode{i2_addType}{.}addType(i.entry, i.t)
                            \end{minipage}
                        }
                    }
                    \Tr{,}
                    \Tr{$i$}\nput{0}{\pssucc}{
                        \begin{minipage}{3cm}
                            \color{red}\small
                            \rnode{i1_t}{.}t = float\\
                            .entry = $x$\\
                            \rnode{i1_addType}{.}addType(i.entry, i.t)
                        \end{minipage}
                    }
                }
            }
            \ncarc[linecolor=red]{->}{T1_t}{L1_t}
            \ncarc[linecolor=red]{->}{L1_t}{L2_t}
            \ncarc[linecolor=red]{->}{L2_t}{L3_t}
            \ncarc[linecolor=red]{->}{L1_t}{i1_t}
            \ncarc[linecolor=red]{<-}{i1_addType}{i1_t}
            \ncarc[linecolor=red]{->}{L2_t}{i2_t}
            \ncarc[linecolor=red]{<-}{i2_addType}{i2_t}
            \ncarc[linecolor=red]{->}{L3_t}{i3_t}
            \ncarc[linecolor=red]{<-}{i3_addType}{i3_t}
        \end{center}
\end{itemize}

\section{Erzeugung (abstrakter) Syntaxbäume}
Verwende Hilfsfunktionen: Leaf(op, val) und Node(op, $c_1$, ..., $c_k$), wobei $k$ die Anzahl der unmittelbaren Komponenenten des Konstrukts op ist.
\begin{center}
    \begin{tabular}{l|l}
        Produktion      & semantische Regeln \\\hline
        $E \to E_1 + T$ & $E.k = $new Node('+', $E_1.k$, $E.k$)\\\hline
        $E \to T$       & $E.k = T.k$\\\hline
        $T \to T_1 * F$ & $T.k = $new Node('*', $T_1.k$, $F.k$)\\\hline
        $T \to F$       & $T.k = F.k$\\\hline
        $F \to (E)$     & \\\hline
        $F \to n$       & $F.k = $new Leaf(num, $n$.val)
    \end{tabular}
\end{center}
\Bsp 4+(2*5)
    \begin{center}
        \pstree{\Tr{$E$}}{
            \pstree{\Tr{$E$}}{
                \pstree{\Tr{$T$}}{
                    \pstree{\Tr{$F$}}{
                        \Tr{$n$}
                    }
                }
            }
            \Tr{$+$}
            \pstree{\Tr{$T$}}{
                \pstree{\Tr{$T$}}{
                    \pstree{\Tr{$F$}}{
                        \Tr{$n$}
                    }
                }
                \Tr{$*$}
                \pstree{\Tr{$F$}}{
                    \Tr{$n$}
                }
            }
        }
    \end{center}
    \begin{center}
        \color{red}
        \rnode{F1_leaf_num}{
        \begin{tabular}{|c|c|}
            \hline
            num & 4 \\\hline
        \end{tabular}} \hspace*{1cm}
        \rnode{F2_leaf_num}{
        \begin{tabular}{|c|c|}
            \hline
            num & 2 \\\hline
        \end{tabular}} \hspace*{1cm}
        \rnode{F3_leaf_num}{
        \begin{tabular}{|c|c|}
            \hline
            num & 5 \\\hline
        \end{tabular}}
    \end{center}
    \begin{center}
        \color{red}
        \rnode{F1_node}{
        \begin{tabular}{|c|c|c|}
            \hline
            * & & \\\hline
        \end{tabular}} \hspace*{1cm}
        \rnode{F2_node}{
        \begin{tabular}{|c|c|c|}
            \hline
            + & & \\\hline
        \end{tabular}}
    \end{center}
    Ein Attribut $k$ (Zeiger auf Knoten im Syntaxbaum)

\section{Übersetzungsschemata}
Syntax Directed Translation Schemes (SDT)
\Defi Eine kontextfreie Grammatik, in deren Produktionen auf der rechten Regelseite semantische Aktionen eingestreut sind, heißt \emph{Übersetzungsschema}, wobei semantische Aktionen, in geschweifte Klammern eingeschlossene Code-Fragmente der Implementierungssprache sind.
\begin{itemize}
 \item Semantische Aktionen tragen nicht zu der erzeugten Sprache bei. Beim Aufbau des Parsebaumes werden sie wie Terminale behandelt, jedoch mit gestrichelten Kanten zu deren Blättern (\emph{Pseudoterminale})
 \item Jedes Übersetzungsschema definiert eine Ausführung der Code-Fragmente, wie sie durch einen depth-first-left-to-right Durchlauf des Parsebaumes gegeben ist.
\end{itemize}
\paragraph*{Ziel:} nach Möglichkeit SDD $\rightarrow$ SDT. \\
2 Spezialfälle:
\begin{enumerate}
 \item LR-Grammatik uns $S$-Attributierung
 \item LL-Grammatik und $L$-Attributierung
\end{enumerate}
(Attributierung sind auf dem Parser-Stack möglich!)
\Bsp SDD:
\begin{center}
    \begin{tabular}{l|l}
        $L \to E$\verb!\n! & \verb!print(!$E.v$\verb!)! \\
        $E \to E_1 + T$ & $E.v = E_1.v + T.v$ \\
        $E \to T$ & $E.v = T.v$ \\
        $T \to T_1 * F$ & $T.v = T_1.v * F.v$ \\
        $T \to F$ & $T.v = F.v$\\
        $F \to d$ & $F.v = d.l$\\
    \end{tabular}
\end{center}
$S$-Attributierung.\\
Generiere äquivalentes Übersetzungsschema im Fall (1. s. o.) durch Anfügen von semantischen Aktionen nach jeder Produktion, die gene den semantischen Regeln entsprechen.
\paragraph*{} SDT zum gegebenen Beispiel:
\begin{align*}
    L &\to E\texttt{\textbackslash n} \{\texttt{print(}E.v\texttt{);}\} \\
    E &\to E_1 + T \{E.v = E_1.v + T.v; \}
\end{align*}
\begin{center}
    \psset{treesep=2.3cm}
    \def\dedge{\ncline[linestyle=dashed,linecolor=red]}
    \pstree{\Tr{$L$}}{
        \pstree{\Tr{$E$}\nput{0}{\pssucc}{\color{red}v = 7}}{
            \pstree{\Tr{$E$}\nput{0}{\pssucc}{\color{red}v = 3}}{
                \pstree{\Tr{$T$}\nput{0}{\pssucc}{\color{red}v = 3}}{
                    \pstree{\Tr{$F$}\nput{0}{\pssucc}{\color{red}v = 3}}{
                        \Tr{d}\nput{0}{\pssucc}{\color{red}l = 3}
                        \Tr[edge=\dedge]{\color{red}\{$F.v = d.l$\}}
                    }
                    \Tr[edge=\dedge]{...}
                    
                }
                \Tr[edge=\dedge]{\color{red}\{$E.v = T.v$\}}
            }
            \Tr{+}
            \pstree{\Tr{$E$}\nput{0}{\pssucc}{\color{red}v = 4}}{
                \pstree{\Tr{$T$}\nput{0}{\pssucc}{\color{red}v = 4}}{
                    \pstree{\Tr{$F$}\nput{0}{\pssucc}{\color{red}v = 4}}{
                        \Tr{d}\nput{0}{\pssucc}{\color{red}l = 4}
                        \Tr[edge=\dedge]{...}
                    }
                    \Tr[edge=\dedge]{...}
                }
                \Tr[edge=\dedge]{...}
            }
            \Tr[edge=\dedge]{\color{red}\{$E.v = E_1.v + T.v$\}}
        }
        \Tr{\texttt{\textbackslash n}}
        \Tr[edge=\dedge]{\color{red}\{\texttt{print ...}\}}
    }
\end{center}

\paragraph*{Idee zur Vermeidung des Aufbaus des Parsebaumes:}
Parse Stack:
\begin{center}
    \begin{tabular}{cc|c|c|l}\cline{1-4}
        ... & $x$   & $y$   & $z$   & top\\\cline{1-4}
            & $X.x$ & $Y.y$ & $Z.z$ & Attributwerte\\\cline{1-4}
    \end{tabular}
\end{center}
Aus den SDT, welche durch Anfügen der semantischen Aktionen entstanden ist, lässt sich ein äquivalentes SDT erzeugen, das auf dem Parser-Stack operiert!
\begin{align*}
    L &\to E\texttt{\textbackslash n} \{\texttt{print(stack[top-1]}.v\texttt{); top = top - 1;}\} \\
    E &\to E_1 + T \{\texttt{stack[top-2]}.v\texttt{ = stack[top-2]}.v\texttt{ + stack[top]}.v\texttt{; top = top - 2;}\} \\
    &\vdots
\end{align*}
\begin{center}
    \begin{tabular}{cc|c|c|}\hline
        ... & $E_1$   & $+$   & $T$\\\hline
            & 3 & $\cdot$ & 4 \\\hline
    \end{tabular}
\end{center}
Postfix-Übersetzungsschemata
\subsection{Übersetzungsschemata mit semantischen Aktionen innerhalb der Produktionen}
Sei $B \to X \{a\} Y$ eine Produktion des gegebenen Übersetzungsschemas.
\begin{itemize}
\item   Beim Bottom-Up-Parsen führe wir $\{a\}$ aus, sobald $X$ auf dem Stack erscheint.
\item   Beim Top-Down-Parsen führen wir $\{a\}$ aus, bevor $Y$ expandiert wird bzw. mit aktueller Eingabe gematcht wird
\end{itemize}
Nicht alle SDTs lassen sich beim Parsen implementieren
\Bsp Infix $\longleftarrow$ Präfix-Notation: $E \to \{\texttt{print($+$);}\} E_1 + T$
\paragraph*{Frage:} Lässt sich entscheiden, ob ein SDT auf dem Parser-Stack implementiert werden kann?
\paragraph*{Ja:}  Ersetze jede semantische Aktion durch ein neues Nichtterminal $M_1, ..., M_k$ und füge Produktionen \linebreak
$M_i \to \varepsilon, i \leq i \leq k$ hinzu. Erzeuge Parse-Tabelle.

\paragraph*{Implementierung von $L$-Attributierungen zu gegebenen $LL$-Grammatik}
\begin{enumerate}
 \item Schreibe semantische Aktionen zur Berechnung ererbter Attribute von $A$ unmittelbar vor das entsprechende Vorkommen von $A$
         \[B \to ... \{A.a = ...\} A ... \{B.b = ...\}\]
 \item Schreibe semantische Aktionen zur Berechnung von synthetisierten Attributen gant an das Ende der entsprechenden Produktionen
\end{enumerate}

\Bsp Ausschnitt aus Zwischencode-Erzeugung, Hier: Kontrollfluss
\begin{center}
    \begin{tabular}{l|p{10cm}}
        \textbf{Produktion} & \textbf{sem. Regeln} \\\hline
        $\vdots$ & \\
        $S \to \textbf{while}(C) S_1$ & $L1 = \text{new}(); L2 = \text{new}(); S_1.\text{next} = L1;$ \newline $ C.\text{false} = S.\text{next}; C.\text{true} = L2;$ \newline
        $S.\text{code} = \text{label} \| L1 \| C.\text{code} \| \text{label} \| L2 \| S_1.\text{code}$ \\ 
    \end{tabular}
\end{center}
$S \to \textbf{while}(\{L1 = \text{new}(); L2 = \text{new}(); C.\text{false} = S.\text{next}; C.\text{true} = L2;\}C)\{S_1.\text{next} = L1;\} S_1 \{S.\text{code} = ...\}$




\chapter{Zwischencode-Erzeugung}
;ethode wie bei semantischer Analyse SDD bzw. SDT
\begin{center}
    \begin{psmatrix}[colsep=1.5cm]
        Quellprogramm & \framebox{Lexer} & \framebox{Parser} & \framebox{
                        \begin{minipage}{1.4cm}
                         \centering
                         Static\\Checker
                        \end{minipage}} & \framebox{\begin{minipage}{2.2cm}
                         \centering
                         Zwischencode-Erzeugung
                        \end{minipage}} & ?
    \end{psmatrix}
    \ncline{->}{1,1}{1,2}\naput{String}
    \ncline{->}{1,2}{1,3}\naput{Tokens}
    \ncline{->}{1,3}{1,4}\naput{\begin{minipage}{1.5cm}\centering Parse"-baum\end{minipage}}
    \ncline{->}{1,4}{1,5}\naput{\begin{minipage}{1.5cm}\centering dek. Parse"-baum\end{minipage}}
    \ncline{->}{1,5}{1,6}\naput{\begin{minipage}{1.5cm}\centering Zwi"-schen"-code\end{minipage}}
\end{center}
Standard: Drei-Adress-Code.
\begin{description}
\item[Charakteristika:] Jeder Befehl enthält höchstens 3 Adressen
\end{description}

\begin{description}
\item[Gelegentlich:] Schrittweise Übersetzung im Zwischencode absteigender Ebenen. Höhere Ebene (Syntaxbäume, nahe der Quellsprache) in niedrigerer Ebene (Drei-Adress-Code: nahe Maschinensprache).
\end{description}
\section{Erste Optimierung auf der Ebene der (abstrakten) Syntaxbäume}
\begin{description}
\item[Idee:] Gemeindame Teilausdrücke sollen nur einmal dargestellt werden.
\item[Beispiel:] $a + a * (b - c) + (b - c) * d$
\item[GaG:] (Gerichteter azyklischer Graph) (DAG, Directed acyclic graph)\\
    Verwende zur Erstellung eines GaG's dieselbe SSD. die zur Erzeugung des Syntaxbaumes verwendet wurde, allerdings unter Verwendung modifizierter Hilfsfunktionen. leaf und node anstelle von new Leaf und new Node. \\
    Speichere Blätter und Knoten in einem Array. Die Funktionen leaf und node liefern vorh. Knoten und Blätter zurück.
    \begin{center}
        \begin{tabular}{l|l}
            \textbf{Produktion} & \textbf{semantische Regeln} \\
            $E \to E + T$       & $T.k = \text{node}(\texttt{'+'}, E_1.k, T.k)$\\
            $E \to E - T$       & $E.k = \text{node}(\texttt{'-'}, E_1.k, T.k)$\\
            $E \to T$           & $F.k = T.k$\\
            $T \to T_1 * F$     & $T.k = \text{node}(\texttt{'*'}, T_1.k, F.k)$\\
            $T \to F$           & $T.k = F.k$\\
            $F \to (E)$         & $F.k = T.k$\\
            $F \to i$           & $F.k = \text{leaf}(i,v)$\\
            $F \to n$           & $F.k = \text{leaf}(n, n.v)$
        \end{tabular}
    \end{center}
    $a * b + a$
    \begin{center}
        \psset{treesep=2cm}
        \pstree{\Tr{$E$}\nput{0}{\pssucc}{\color{red}$k = k_4$}}{
            \pstree{\Tr{$E$}\nput{0}{\pssucc}{\color{red}$k = k_3$}}{
                \pstree{\Tr{$T$}\nput{0}{\pssucc}{\color{red}$k = k_3$}}{
                    \pstree{\Tr{$T$}\nput{0}{\pssucc}{\color{red}$k = k_1$}}{
                        \pstree{\Tr{$F$}\nput{0}{\pssucc}{\color{red}$k = k_1$}}{
                            \Tr{$i$}\nput{0}{\pssucc}{\color{red}$v = a$}
                        }
                    }
                    \Tr{*}
                    \pstree{\Tr{$F$}\nput{0}{\pssucc}{\color{red}$k_2$}}{
                        \Tr{$i$}\nput{0}{\pssucc}{\color{red}$v = b$}
                    }
                }
            }
            \Tr{+}
            \pstree{\Tr{$T$}\nput{0}{\pssucc}{\color{red}$k k_1$}}{
                \pstree{\Tr{$F$}\nput{0}{\pssucc}{\color{red}$k = k_1$}}{
                    \Tr{$i$}\nput{0}{\pssucc}{\color{red}$v = a$}
                }
            }
        }
    \end{center}
\end{description}

\section{Implementierung von GaG's}
Wertnummermethode: Verwende Knotennummern anstelle der Verweise.
\Bsp
\begin{center}
    \begin{tabular}{|c|c|c|c|}
     Index & \multicolumn{3}{|c|}{ } \\\hline
     1 & $i$ & \multicolumn{2}{|c|}{$a$} \\\hline
     2 & $i$ & \multicolumn{2}{|c|}{$b$} \\\hline
     3 & * & 1 & 2 \\\hline
     4 & + & 3 & 1 \\\hline
    \end{tabular}
\end{center}

\paragraph*{Drei-Adress-Code}
Adressierungsarten:
\begin{enumerate}
 \item Hilfsvariablen (vom Übersetzer erzeugt)
 \item Variablen (Bez.) aus Quellprogramm
 \item Konstanten (unterschiedliche Typ, aus Quellprogramm, Adressen)
\end{enumerate}
Befehlsarten:
\begin{enumerate}
 \item $x = y \operatorname{op} z$: für alle binären Operationen der Quellsprache und Adressen $x$, $y$ und $z$
 \item $x = \operatorname{op} y$: für alle unären Operationen der Quellsprache und Adressen $x$, $y$ und $z$
 \item $x = y$: Kopierbefehl
 \item $\operatorname{goto} L$: unbedingter Sprung
 \item $\operatorname{if} x \operatorname{goto} L$ und $\operatorname{if} \operatorname{False} x \operatorname{goto} L$: bedingte Sprünge
 \item $\operatorname{if} x \operatorname{relop} y \operatorname{goto} L$: bedingte Sprünge unter Relationen $<, >, <=, ...$
 \item $\operatorname{param} x, \operatorname{call} p, n, \operatorname{return} y$ und return für Prozeduren\\
          z. B.
          \begin{algorithmic}
           \STATE param $x$
           \STATE param $y$
           \STATE param $z$
           \STATE call $p, 3$
          \end{algorithmic}
 \item Indizierte Kopierbefehle
     \[x = y[i] \operatorname{und} x[i] = y\]
 \item Adress- und Zeigeroperationen
     \[x = \&y, x = *y \operatorname{und} *x = y\]
\end{enumerate}



\section{Implementierung von Drei-Adress-Code}
\Bsp $x = y \operatorname{op} z \Rightarrow$ \emph{Quadrupeldarstellung}
\begin{itemize}
 \item Jeder Befehl hat die Struktur $\operatorname{op} \operatorname{arg}_1 \operatorname{arg}_2 \operatorname{result}$
 \item Unäre Operatoren machen keinen Gebrauch von $\operatorname{arg}_2$
 \item Kopierbefehle machen keinen Gebrauch von $\operatorname{arg}_2$
 \item Sprungbefehle schreiben Ziel in result
\end{itemize}
\Bsp 
\begin{align*}
 a &= b * -c + b * -c \\
 t_1 &= \operatorname{minus} c \\
 t_2 &= b * t_1 \\
 t_3 &= \operatorname{minus} c \\
 t_4 &= b * t_3 \\
 t_5 &= t_2 + t_4 \\
 a &= t_5
\end{align*}
Quadrupeldarstellung:
\begin{center}
    \begin{tabular}{r|c|c|c|c}
          & op    & $\operatorname{arg}_1$ & $\operatorname{arg}_2$ & result \\\hline
        0 & minus & $c$                    &                        & $t_1$  \\\hline
        1 & $*$   & $b$                    & $t_1$                  & $t_2$  \\\hline
        2 & minus & $c$                    &                        & $t_3$  \\\hline
        3 & $*$   & $b$                    & $t_3$                  & $t_4$  \\\hline
        4 & $+$   & $t_2$                  & $t_4$                  & $t_5$  \\\hline
        1 & $=$   & $t_5$                  &                        & $a$    \\\hline
    \end{tabular}
\end{center}

Platzminimierung $\rightarrow$ Tripeldarstellung \hspace{0.4\linewidth}\rnode{ref0_label}{Veweis auf 0-ten Befehl}
\begin{center}
    \begin{tabular}{r|c|c|c}
          & op    & $\operatorname{arg}_1$ & $\operatorname{arg}_2$ \\\hline
        0 & minus & $c$                    &                        \\\hline
        1 & $*$   & $b$                    & \rnode{ref0}{$(0)$}    \\\hline
        2 & minus & $c$                    &                        \\\hline
        3 & $*$   & $b$                    & $(2)$                  \\\hline
        4 & $+$   & $(1)$                  & $(3)$                  \\\hline
        1 & $=$   & $a$                    & $(4)$                  \\\hline
    \end{tabular}\ncline{->}{ref0_label}{ref0}
\end{center}
\textbf{Achtung:} Einige Befehle verwenden drei Adressen, sie werden durch je zwei Befehle dargestellt.
\[x[i] = y\]
Nachteil: \emph{Verschiebbarkeit}.
\paragraph*{Indirekte Tripel}
\begin{center}
    \begin{tabular}{r|c|c|c}
        33 & (0) \\\hline
        34 & (1) \\\hline
        35 & (2) \\\hline
        36 & (3) \\\hline
        37 & (4) \\\hline
        38 & (5) \\\hline
    \end{tabular}\ncline{->}{ref0_label}{ref0}
\end{center}

\section{Static-Single-Assignment (SSA)}
\Bsp
\begin{align*}
 p &= a+b & p_1 &= a+b \\
 q &= p-c & q_1 &= p_1-c\\
 p &= q*d & p_2 &= q_1*d\\
 p &= e-p & p_3 &= r-p_3\\
 q &= p+q & q_2 &= p_3+q_1
\end{align*}
Problem: \verb!if (flag) x = 0; else x = 1;! $\Rightarrow$ \verb!y = 3 * !$\underset{?}{\mathtt{x}}$\verb!;!\\
Neuer Operator $\Phi$ $\Rightarrow x_3 = \Phi(x_1,x_2)$

\section{Typen und Deklarationen}
\begin{itemize}
 \item Typüberprüfungen wichtig bei Fehlererkennung
 \item Typisierung dient effizienter Speicherung
\end{itemize}
Modernes Typkonzept enthält induktive Strukturen, d. h. Typausdrücke zur Definition von Typen.
\Defi (Typausdrücke, $\mathcal{TA}$)
\begin{enumerate}
 \item Elementare Typen
 \begin{itemize}
  \item Basistypen, z. B.: \textbf{float}, \textbf{int}, ...
  \item Typvariablen - Bezeichner
 \end{itemize}
 \item Mit $t \in \mathcal{TA}$ und $n \in \mathbb{N}$ ist auch \texttt{array($n$,$t$)} ein Typ.
 \item Mit Bezeichnern $f_1, ..., f_n$ und Typen $t_1, ..., t_n$ ist auch \texttt{record($f_1 t_1$, $...$, $f_n t_n$)} $\in \mathcal{TA}$
 \item Mit Typausdrücken $s$ und $t$ ist auch $s \to t \in \mathcal{TA}$
 \item Mit $s, t \in \mathcal{TA}$ ist auch $s \times t \in \mathcal{TA}$
\end{enumerate}

\Bsp zur Berechnung der Breite von Werten des Typs \texttt{array ....} \\
Auszug aus Syntax der Quellsprache zur Deklaration von Arrays
\begin{align*}
    T &\to BC \\
    B &\to \texttt{int}\ |\ \texttt{float} \\
    C &\to \texttt{[}\textit{num}\texttt{]}C\ |\ \varepsilon
\end{align*}
z. B. \texttt{int[2][3]}
\begin{center}
    \pstree{\Tr{\texttt{array}}}{
        \Tr{2}
        \pstree{\Tr{\texttt{array}}}{
            \Tr{3}
            \Tr{\texttt{int}}
        }
    }
\end{center}

SDD:
\begin{center}
    \psset{treesep=2cm}
    \pstree{\Tr{$T$}}{
        \pstree{\Tr{$B$}}{
            \Tr{\texttt{int}}
        }
        \pstree{\Tr{$C$}}{
            \Tr{\texttt{[}}
            \Tr{\textit{num}}\nput{0}{\pssucc}{\color{red} v = 2}
            \Tr{\texttt{]}}
            \pstree{\Tr{$C$}}{
                \Tr{\texttt{[}}
                \Tr{\textit{num}}\nput{0}{\pssucc}{\color{red} v = 2}
                \Tr{\texttt{]}}
                \pstree{\Tr{$C$}}{\Tr{$\varepsilon$}}
            }
        }
    }
\end{center}



\subsection{Deklarationen und Zuweisung relativer Adressen}
\begin{itemize}
 \item Die aktuelle \emph{Umgebung} (Symboltabelle) unter \emph{top}
 \item Die Ausführung top.put(id.lexeme, $T$.type, offset)
     \begin{align*}
      P &\to \{\text{offset} = 0;\}DS \\
      D &\to T \text{id}; \{\text{top.put(id.lexeme, $T$.type, offset); offset = offset + $T$.width;}\}D\ |\ \varepsilon \\
      S &\to ...
     \end{align*}
\end{itemize}

\subsection{Felder (Attribute, Komponenten) in Records und Klassen}
\begin{align*}
    T \to \text{record\ \{}D\text{\}}
\end{align*}
\begin{itemize}
\item Annahme: In jedmen Record kommen von einander verschiedene Variablen vor.
\item Die relativen Adressen werden in Bezug zur Adresse des Records fesgelegt.
\end{itemize}
\Bsp
\begin{verbatim}
float x;
record {float x; float y;} p;
record {int tag; float x; float y;} q;
\end{verbatim}
\begin{itemize}
 \item Es ist zweckmäßig, den Typkonstruktor record anzuwenden auf Umgebungen, z. B. record($s$) wenn $s$ eine Umgebung (Symboltabelle ist).
 \begin{align*}
  T &\to \text{record\ \{} \{\text{Env.push(top); top = new Env(); Stack.push(offset); offset = 0;}\} \\
  &D \text{\} \{$T$.type = record(top); $T$.width = offset; top = Env.pop(); offset = stack.pop();\}} 
 \end{align*}
 \item die Klasse Env implementiert Symboltabellen
 \item der Aufruf Env.push(top) legt die aktuelle Symboltabelle auf einen Stack.
 \item Stack zum Speichern des jeweiligen offsets
\end{itemize}

\section{Erzeugung von Drei-Adress-Code für Ausdrücke}
\begin{itemize}
 \item Hilfsfunktion zur Erzeugung jeweils neuer Hilfsvariablen: new T();
 \item Der Aufruf top.get(id.lexeme) liefert die zu verwendende Adresse für den Bezeichner id
 \item Die Hilfsfunktion gen generiert Drei-Adresss.Befehle aus den benötigten Komponenten
 \item Verwende das Attribut code für die Folge von Drei-Adress-Befehlen zur Ausführung einer Anweisung bzw. zur Berechnung des Wertes eine Ausdrucks.
 \item Verwende Attribut addr zur Speicherung der Adresse, in der der Wert eines Ausdrucks abgelegt wird
\end{itemize}
\begin{center}
\begin{tabular}{l|p{16cm}}
    Produktion & semantische Regel \\\hline
    $S \to \text{id} = E;$ & $S.\text{code} = E.\text{code} \| \underbrace{\text{top.get(id.lexeme) = $E$.addr}}_{\text{salopp\footnote{gen($E$.addr, \textquotesingle = \textquotesingle, $E_1$.addr)}}}$\\
    $E \to E_1 + E_2$ & $E.\text{addr} = \text{new}\ T();$ \newline $E.\text{code} = E_1.\text{code} \| E_2.\text{code} \| \underbrace{E.\text{addr} = E_1.\text{addr} + E_2.\text{addr}}$\\
    $E \to -E_1$ & $E.\text{addr} = \text{new}\ T(); E.\text{code} = E_1.\text{code} = E_1.\text{code} \| \underbrace{E.\text{addr} = \text{minus} E_1.\text{addr}}$\\
    $E \to (E_1)$ & $E.\text{addr} = E_1.\text{addr}; E.\text{code} = E_1.\text{code}$\\
    $E \to \text{id}$ & $E.\text{addr} = \text{top.get(id.lexeme)}; E.\text{code} = $\textquotesingle\textquotesingle
\end{tabular}
\end{center}





\section{Kontrollfluss}
\begin{itemize}
 \item Boolesche Ausdrücke -- Naheliegende Behandlung analog zu arithmetischen Ausdrücken (logische Operationen auf der Ebene des Drei-Adress-Codes). Für $b$ boolsche Variable, lässt sich $b = \underset{B}{............}$ übersetzen in Drei-Adress-Code
 \[B \to B \lor B\ |\ B \land B\ |\ \neg B\ |\ \textbf{true}\ |\ \textbf{false}\ |\ id\]
 log. Operationen $\lor$ und $\land$ sind linksassoziativ. Prioritäten aufsteigend: $\lor$, $\land$, $\neg$
 \item Steuerung der Kontrolle exemplarisch:
     \begin{align*}
      S &\to \textbf{if} (B) S \\
      S &\to \textit{if} (B) S \textbf{else} S \\
      S &\to \textbf{while} (B) S
     \end{align*}
 \item \emph{naiver Zugang:} Ein ererbtes Attribut next zum Nichtterminal $S$, dessen Wert ein Label ist, das hinter $S$ gesetzt wird.
 \item Idee: $S.\text{code}$
         \begin{verbatim}
begin   B.code
        ifFalse (B.addr) goto S.next
        C.code
        goto begin
S.next  ...
         \end{verbatim}
 \item \emph{Tiefere Einsicht:} Boolesche Ausdrücke im Kontext von Kontrollflussanweisungen spielen ausschließlich die Rolle der Steuerung des Kontrollflusses. 
 \item Anstelle von Wertberechnung berechnen wir Sprungadressen $B.\text{true}$ bzw. $B.\text{false}$ \emph{Short-Circuit-Code} (Jumping Code, Spring-Code)
 \Bsp  \lstinline[language=Java]$if (x < 100 || x > 200 && x != y) x = 0;$
     \begin{itemize}
      \item Unterlege \emph{nicht-strikte Semantik} der boolschen Operatoren, d. h. wenn $B_1$ \textbf{true} ergibt, dann ergibt auch $B_1$ \texttt{||} $B_2$ für beliebige $B_2$ den Wert \textbf{true} und aus $B_1$ ergibt \textbf{false} folgt $B_1$ \verb!&&! $B_2$ ergibt \textbf{false}
      \item in strikter Semantik:
      \begin{center}
       \begin{tabular}{r|c|c|c}
        \texttt{||}    & \textbf{true} & \textbf{false} & \textbf{undef} \\\hline\hline
        \textbf{true}  & \textbf{true} & \textbf{true}  & \textbf{true}  \\\hline
        \textbf{false} & \textbf{true} & \textbf{false} & \textbf{undef} \\\hline
        \textbf{undef} & \textbf{true} & \textbf{undef} & \textbf{undef} \\
       \end{tabular}
      \end{center}
     \end{itemize}
    Drei-Adress-Code zum Beispiel:
    \begin{verbatim}
    if x < 100 goto L2
    ifFalse x > 200 goto L1
    ifFalse x != y goto L1
L2: x = 0
L1: ...
    \end{verbatim}
\end{itemize}
\subsection{Schemata zur Übersetzung von Kontrollflussanweisungen unter Verwendung von Spring-Code}
\subsubsection{if-Anweisung}
\[S \to \textbf{if} (B) S_1\]
\begin{verbatim}
        B.code      // berechnet B.true und B.false
B.true  S.code
B.false ...
\end{verbatim}

\subsubsection{if-else-Anweisung}
\[S \to \textbf{if} (B) S_1\ \textbf{else}\ S_2\]
\begin{verbatim}
        B.code      // berechnet B.true und B.false
B.true  S_1.code
        goto S.next
B.false S_2.code
S.next  ...
\end{verbatim}

\subsubsection{while-Schleife}
\[S \to \textbf{while} (B) S_1\]
\begin{verbatim}
begin   B.code      // berechnet B.true und B.false
B.true  S_1.code
        goto begin
B.false ...
\end{verbatim}

\begin{itemize}
 \item Hilfsfunktionen newL und newT zur Erzeugung voin Labeln und Hilfsvariablen
 \item Das eererbte Attribut next von $S$ mit Wert eines Labels, das hinter $S$ steht.
 \item Die Attribute B.true und B.false erhalten die Sprungadressen ...
 \item Marken werden im Drei-Adress-Code inline geschrieben, d. h. \texttt{L: x=y+z} wird durch die Befehle $\underset{\texttt{x=y+z}}{\text{Label $L$}}$ kodiert.
\end{itemize}

Übersetzung einer einfachen höheren Programmiersprache in Drei-Adress-Code
\begin{center}
\begin{tabular}{l|p{12.5cm}}
\textbf{Produktion}   & \textbf{semantische Regeln} \\\hline
$P \to S$             & \lstinline$S.next = newL(); P.code = S.code ++ Label S.next ++ stop$ \\\hline
$S \to \text{id} + E$ & \lstinline$S.code = E.code ++ id.lexeme = E.addr$ \\\hline
... Prod. zu $E$ ...  & ... \\\hline
$S \to S_1 S_2$       & \lstinline$S_1.next = newL(); S_2.next = S.nextM;$ \newline
                        \lstinline$        S.code = S_1.code ++ Label S_1.next ++ S_2.code$ \\\hline
$S \to \textbf{if}(B) S_1$ & \lstinline$B.true = newL(); B.false = S.next;$ \newline
                        \lstinline$        S.code = B.code ++ Label B.true ++ S_1.code$ \\\hline
$S \to \textbf{if}(B) S_1\ \textbf{else}\ S_2$ & \lstinline$B.true = newL(); B.false = newL(); S_1.next = S_2.next = S.next;$ \newline
                        \lstinline$S.code = B.code ++ Label B.true ++ S_1.code ++ goto S.next$\newline
                        \lstinline$          ++ Label B.false ++ S_2.code$ \\\hline
$S \to \textbf{while}(B) S_1$ &  \lstinline$B.true = newL(); B.false = S.next; begin = newL(); S.next = begin;$ \newline
                        \lstinline$S.code = Label begin ++ B.code ++ Label B.true ++ S_1.code ++ goto begin$\\\hline
\end{tabular}
\end{center}
\subsubsection{Übersetzung boolescher Ausdrücke in Sprungcode}
\begin{center}
 \begin{tabular}{l|p{12.5cm}}
\textbf{Produktion}   & \textbf{semantische Regeln} \\\hline
$B \to B_1 \| B_2$    & \lstinline$B_1.true = B.true; B_1.false = newL();$\newline
                        \lstinline$B_2.true = B.true; B_2.false = B.false;$\newline
                        \lstinline$B.code = B_1.code ++ Label B_1.false ++ B_2.code$ \\\hline
$B \to B_1 \&\& B_2$  & \lstinline$B_1.true = newL(); B_1.false = B.false;$\newline
                        \lstinline$B_2.true = B.true; B_2.false = B.false;$\newline
                        \lstinline$B.code = B_1.code ++ Label B_1.true ++ B_2.code$ \\\hline
$B \to \neg B_1$      & \lstinline$B_1.true = newL(); B_1.false = B.true; B.code = B_1.code$\\
$B \to E_1 \text{rel} E_2$ &
 \end{tabular}

\end{center}



\chapter{Laufzeitumgebung}
\begin{itemize}
 \item Speicherverwaltung
     \begin{itemize}
     \item Der Übersetzer erzeugt Code für einen logischen Adressraum von $0$ bis $N$
     \item Grobstruktur des Speichers (logischer Adressraum)
        \begin{center}
        \begin{tabular}{r|c|l}\cline{2-2}
               0 & Code & ausführbares Maschinenprogram\\\cline{2-2}
        $\vdots$ & Static & Daten, deren Anzahl und Größe zur Compilezeit bekannt sind \\\cline{2-2}
         & Heap & Daten, die zur Laufzeit erzeugt werden \\\cline{2-2}
         & $\downarrow$ & \\
         & $\vdots$ & \\
         & $\uparrow$ & \\\cline{2-2}
        $N$ & Stack & Laufzeitkeller, für Aktivierungsrecords (frames) zu jedem Methodenaufruf \\\cline{2-2}
        \end{tabular}
        \end{center}
        \begin{center}
         statischer Speicher (zur Compilezeit) $\Leftrightarrow$ dynamischer Speicher (während der Ausführung)
        \end{center}
     \end{itemize}
\end{itemize}
\Bsp Sortierprogramm (fragmentarisch)
\begin{lstlisting}[language=C]
int a[11];
void read_a() { /* liest Werte in a[1] bis a[9] */
    int i;
    ...
}
int part(int m, int n) { /* waehlt v aus und teilt 
    a[m...n] auf, so dass a[m...p-1] <= v, a[p] - v, a[p+1...n] > v */
    ...
}
void qs(int m, int n) {
    int i;
    if (n > m) {
        i = part(m,n);
        qs(m,i-1);
        qs(i+1,n);
    }
}
void main() {
    read_a();
    a[0] = -9999;
    a[10] = 9999;
    qs(1,9);
}
\end{lstlisting}

\begin{verbatim}
Betrete main()
    Betrete read_a
    Verlasse read_a
    Betrete qs(1,9)
        Betrete part(1,9)
        Verlasse part(1,9)
        Betrete qs(1,3)
            ...
        Verlasse qs(1,3)
        Betrete qs(5,9)
            ...
        Verlasse qs(5,9)
    Verlasse qs(1,9)
Verlasse main()
\end{verbatim}
Ein \emph{Aktivierungsbaum} für unser Sortierprogramm:
\begin{center}
\pstree{\Tr{\texttt{main()}}}{
    \Tr{\texttt{read\_a()}}
    \pstree{\Tr{\texttt{qs(1,9)}}}{
        \Tr{\texttt{part(1,9)}}
        \pstree{\Tr{\texttt{qs(1,3)}}}{
            \Tr{\texttt{part(1,3)}}
            \Tr{\texttt{qs(1,0)}}
            \pstree{\Tr{\texttt{qs(2,3)}}}{
                \Tr{\texttt{part(2,3)}}
                \Tr{\texttt{qs(2,1)}}
                \Tr{\texttt{qs(3,3)}}
            }
        }
        \pstree{\Tr{\texttt{qs(5,9)}}}{
            \Tr{\texttt{part(5,9)}}
            \Tr{\texttt{qs(5,5)}}
            \pstree{\Tr{\texttt{qs(7,9)}}}{
                \Tr{\texttt{part(7,9)}}
                \Tr{\texttt{qs(7,7)}}
                \Tr{\texttt{qs(9,9)}}
            }
        }
    }
}
\end{center}
Zu jeder Programmausführung gehört genau ein Aktivierungsbaum.
\paragraph*{Struktur des Aktivierungscodes}
\begin{center}
\begin{tabular}{|c|l}\cline{1-1}
    aktuelle Parameter & Speicherbereich für Eingabewerte (ggf. Registerverwendung) \\\cline{1-1}
    Rückgabewerte & Speicherbereich für Ausgabewerte (ggf. Registerverwendung) \\\cline{1-1}
    Kontrollverweis & Zeigt auf unteres Aktivierungselement einschl. program counter \\\cline{1-1}
    Zugriffsverweis & Zugriff auf nicht-lokale Daten gemäß statischer Bindung \\\cline{1-1}
    saved machine status & Registerbelegung, ggf. program counter \\\cline{1-1}
    lokale Daten & \\\cline{1-1}
    Temporaries & Hilfsvariablen, die bei der Übersetzung erzeugt werden \\\cline{1-1}
\end{tabular}

\begin{tabular}{cc}
    \begin{minipage}{0.2\linewidth}
            \centering
            \Tr{\texttt{main()}}
        \end{minipage} &
        \begin{minipage}{0.5\linewidth}
            \begin{tabular}{|c|}\hline
                \texttt{main} \\\hline
            \end{tabular}
        \end{minipage} \\\hline
    \begin{minipage}{0.2\linewidth}
            \centering
            \pstree{\Tr{\texttt{main()}}}{
                \Tr{\texttt{read\_a()}}
            }
        \end{minipage} &
        \begin{minipage}{0.5\linewidth}
            \begin{tabular}{|c|}\hline
                \texttt{main} \\\hline
                \texttt{read\_a} \\
                \texttt{int i} \\\hline
            \end{tabular}
        \end{minipage} \\\hline
\end{tabular}


\end{center}

\chapter{Code-Erzeugung}
\subsection{Architekturarten}
\begin{tabular}{c|c|c}
 RISC & CISC & Stack-basierte Architektur \\\hline
 \begin{minipage}{0.3\linewidth}
  \begin{itemize}
   \item einfacher Befehlssatz
   \item Drei-Adress-Befehle
   \item Befehle fester Länge
   \item \emph{$\Rightarrow$ Beispielarchitektur für diese VL}
  \end{itemize}
 \end{minipage} &
 \begin{minipage}{0.3\linewidth}
  \begin{itemize}
   \item wenige Register
   \item Großer Befehlssatz
   \item Befehle variabler Länge
  \end{itemize}
 \end{minipage} &
 \begin{minipage}{0.3\linewidth}
  \begin{itemize}
   \item Operandenstack
   \item Operatoren auf Kellerspitze
   \item oberste Stackelemente in Registern
  \end{itemize}
 \end{minipage}
\end{tabular}

\subsection{Primäre Aufgaben der Code-Erzeugung}
\begin{itemize}
 \item Erzeuge \emph{semantische äquivalenten} Code
 \item Erzeuge effizienten Code (kurze Laufzeit), wenig Speicherbedarf
 \item Befehlsauswahl
 \item Registerzuweisung und -belegung
 \item Befehlsanordnung
\end{itemize}
Heute Algorithmen zur automatischen Code-Erzeugung auf exemplarischer RISC-Architektur
\begin{description}
\item[Idee 1] Verwende "`Code-Templates"'. Ergebnis: Schlechter Code mit hohem Optimierungspotential
    \Bsp Übersetze Befehle der Art \texttt{x = y + z} in folgende Code-Sequenz
        \begin{verbatim}
LD  R0,y
ADD R0,R0,z
ST  x,R0
        \end{verbatim}
    Intermediate Representation: \texttt{a = b + c; d = a + e} ergibt:
    \begin{verbatim}
LD  R0,b
ADD R0,R0,c
ST  a,R0     // vermeidbar, wenn der Wert in a keine spätere Verwendung hat
LD  R0,a     // redundant
ADD R0,R0,e
ST  d,R0
    \end{verbatim}
    Charakterisiere Besonderheiten mit Blich auf Befehlssatz der Zielarchitektur
    \Bsp \texttt{a = a + 1}
        \begin{verbatim}
LD  R0,a
ADD R0,R0,#1
ST  a,R0
        \end{verbatim}
        blöd, wenn \texttt{INC a} im Befehlssatz vorhanden ist!
\end{description}

\subsection{Probleme jenseits Templates}
\Bsp Es gibt Architekturen, die Gleitkommamultiplikation und -division ausschließlich auf Registerpaaren erlauben (gerade/ungerade Paare)
\begin{description}
 \item[Multiplikationsbefehl] \texttt{MUL x,y}, wobei \texttt{x} im ungeraden Teil eines Registerpaares steht und Ergebnis im gesamten Registerpaar
 \item[Divisionsbefehl] \texttt{DIV x,y}, wobei der Divident ein Registerpaar belegt. Das Ergebnis befindet sich im ungeraden Register, der Rest im geraden Register.
\end{description}
IR-Befehlssequenzen:
\renewcommand{\theenumi}{\alph{enumi}}
\renewcommand{\labelenumi}{\theenumi)}
\begin{enumerate}
 \item \begin{verbatim}
t = a + b
t = t * c
t = t / d
       \end{verbatim}
 \item \begin{verbatim}
t = a + b
t = t + c
t = t / d
       \end{verbatim}
\end{enumerate}
Optimaler Code zu a)
\begin{verbatim}
LD  R1,a
ADD R1,b
MUL R0,c
DIV R0,d
ST  R1,t
\end{verbatim}
Optimaler Code zu a)
\begin{verbatim}
LD   R0,a
ADD  R0,b
ADD  R0,c
SRDA R0,32
DIV  R0,d
ST   R1,t
\end{verbatim}

\subsection{Eine einfache RISC-Architektur zur Definintion des Codes}
Drei-Adress-Befehle, Speicher ist Byte-adressierbar, universelle Register \texttt{R0...Rn}, alle Werte sind Integer-Werte, Befehle können durch Label markiert sein.
\begin{itemize}
 \item Lode-Befehle:       \texttt{LD r,x}           wobei \texttt{r} ein Register und \texttt{x} Speicherplatz oder Register ist.
 \item Store-Befehle:      \texttt{ST x,r}           wobei \texttt{x} ein Speicherplatz ist 
 \item Operationsbefehle:  \texttt{OP dst,src1,src2} bzw. \texttt{OP dst,src} wobei \texttt{dst} Registster oder Speicherplatz ist, \texttt{src1},... Register, Speicherplatz oder Konstante \#7
 \item Unbedingte Sprünge: \texttt{BR L}             wobei \texttt{L} eine Marke ist
 \item Bedingte Sprünge:   \texttt{Bcond r,L}        wobei \texttt{cond} ein Bezeichner für typische Tests auf Registerinhalte ist (z. B. \texttt{BLTZ r,L}, \emph{L}ess \emph{T}hen \emph{Z}ero)
 \item Adressierungsarten: Symbolische Adressen, Register, Konstanten; zusätzlich: indizierte Adressen: z. B. \texttt{a(r)} wobei \texttt{a} ein Speicherplatz ist und \texttt{r} ein Register.
\end{itemize}
\begin{description}
 \item[Einfaches Kostenmaß] Jeder Befehl hat Kosten 1 plus Kosten seiner Adressen, wobei Kosten eines Register 0 ist und sonst 1.
 \item[Kosten Programm] Hinsichtlich Länge = Summe der Kosten aller Befehle
\end{description}
\section{Grundblöcke und Flussgraphen}
Optimierung der Grundblöcke
\begin{itemize}
 \item Zerlege das IR-Programm (Zwischencode) in Grundblöcke (einfache Befehlsfolgen, die stets gemeinsam, hintereinander ausgeführt werden)
\end{itemize}
\subsection{Algorithmus zur Zerlegun eines IR-Programmes im Grundblock}
\begin{enumerate}
 \item Bestimme alle \emph{Anführer} (jeweils erster Befehl in einen anderen Grundblock):
     \begin{itemize}
      \item Der erste Befehl ist ein Anführer
      \item Jeder Befehl, der Ziel einer Sprunganweisung ist, ist Anführer
      \item Jeder Befehl, der unmittelbar hinter einem Sprungbefehl steht, ist Anführer
     \end{itemize}
 \item Definiere die Grundblöcke jeweils als Anführer zusammen mit allen folgenden Befehlen bis zum nächsten Anführer
\end{enumerate}
(Anführer: \%)
 \begin{verbatim}
B1 | %    i = 1

B2 | %L1: j = 1

B3 | %L2: t1 = 10 * i
   |      t2 = t1 + j
   |      t3 = 8 * 12
   |      t4 = t3 - 88
   |      a[t4] = 0.0
   |      j = j+1
   |      if j <= 10 goto -L2- B3

B4 | %    i = i + 1
   |      if i <= 10 goto -L1- B2

B5 | %    i = 1

B6 | %L3: t5 = i - 1
   |      t6 = 88 * t5
   |      a[t6] = 1.0
   |      i = i + 1
   |      if i <= 10 goto -L3- B6
 \end{verbatim}

\subsection{Flussgraph für globale Analyse}
\begin{itemize}
\item Jeder Jeder Grundblock ist ein Knoten des Flussgraphen zu gegebenem IR-Programm
\item Es gibt eine gerichtete Kante von $B$ nach $C$, wenn die Kontrolle vom Block $B$ zum Block $C$ gehen kann
    \begin{itemize}
     \item Es gibt einen Sprungbefehl von $B$ nach $C$
     \item Block $C$ steht unmittelbar hinter $B$ und $B$ endet \emph{nicht} mit einem unbedingtem Sprung
    \end{itemize}
\item $B$ heißt \emph{Vorgänger} von $C$ und $C$ \emph{Nachfolger} von $B$
\item Man fügt einen Eingangsknoten \emph{Entry} und einen Ausgangsknoten \emph{Exit} hinzu
\item Eine Kante von \emph{Entry} nach $B_1$
\item Eine Kante von $B$ nach \emph{Exit}, wenn $B$ als letzter Block ausgeführt werden kann
\end{itemize}
\begin{center}
\begin{psmatrix}[rowsep=0.5cm]
    [name=Entry]\frame{Entry}\\
    [name=B1]\frame{$B_1$}\\
    [name=B2]\frame{$B_2$}\\
    [name=B3]\frame{$B_3$}\\
    [name=B4]\frame{$B_4$}\\
    [name=B5]\frame{$B_5$}\\
    [name=B6]\frame{$B_6$}\\
    [name=Exit]\frame{Exit}
    \ncline{->}{Entry}{B1}
    \ncline{->}{B1}{B2}
    \ncline{->}{B2}{B3}
    \nccurve[ncurv=2,angleA=-45,angleB=45]{->}{B3}{B3}
    \ncline{->}{B3}{B4}
    \ncline{->}{B4}{B5}
    \nccurve[angleA=-45,angleB=45]{->}{B4}{B2}
    \ncline{->}{B5}{B6}
    \nccurve[ncurv=2,angleA=-45,angleB=45]{->}{B6}{B6}
    \ncline{->}{B6}{Exit}
\end{psmatrix}
\end{center}

Implementiere Flussgraphen, z. B. mit Adjazenzmatrix \\
Implementiere Inf. in Knoten, z. B. durch verkettete Liste
\Defi Eine Menge $\mathcal{S}$ von Grundblöcke heißt Schleife, gdw. $\mathcal{S}$ einen Block $E$ (genannt Eingang) enthält, kein anderer Bloch außer $E$ von $\mathcal{S}$ kann von außen von außen betreten werden und für jeden Block $G$ in $\mathcal{S}$ gibt es einen nichtleeren Pfad von $G$ nach $E$.

\subsection{Information zu Lebendigkeit und nächste Verwendung innerhalb con Grundblöcken}
\paragraph{Algorithmus:} 
\begin{description}
 \item[Eingabe] Ein Grundblock $B$ und die Symboltabelle, in der zu jeder Variable "`lebendig (l.)"' eingetragen.
 \item[Ausgabe] Grundblock $B$, in dem zu jeder Anweisung und jeder enthaltenenen Variable, deren Info bzgl. Lebendigkeit und nächster Verweis enthalten ist
 \item[Verfahren]
     \begin{enumerate}
      \item Beginne mit dem \emph{letzten} Befehl von $B$ (Sei dieser $i{:} x = y \text{op} z$)
      \item Übernehme alle Infos zu den vorhandenen Variablen aus der Symboltabelle
      \item Trage in der Symboltabelle zu $x$ "`nicht lebendig (n.l.)"' ein.
      \item Trage in der Symboltabelle zu $y$ und $z$ "`lebendig (l.)"' ein und "`nächste Verwendung in $i$ (nV\_$i$)"'
      \item Fahre fort mit 2. un nächsten Befehl oberhalb von $i$, falls $i$ nicht der erste Befehl von $B$ ist.
     \end{enumerate}
\end{description}

\end{document}
