\chapter{Code-Erzeugung}
\subsection{Architekturarten}
\begin{tabular}{c|c|c}
 RISC & CISC & Stack-basierte Architektur \\\hline
 \begin{minipage}{0.3\linewidth}
  \begin{itemize}
   \item einfacher Befehlssatz
   \item Drei-Adress-Befehle
   \item Befehle fester Länge
   \item \emph{$\Rightarrow$ Beispielarchitektur für diese VL}
  \end{itemize}
 \end{minipage} &
 \begin{minipage}{0.3\linewidth}
  \begin{itemize}
   \item wenige Register
   \item Großer Befehlssatz
   \item Befehle variabler Länge
  \end{itemize}
 \end{minipage} &
 \begin{minipage}{0.3\linewidth}
  \begin{itemize}
   \item Operandenstack
   \item Operatoren auf Kellerspitze
   \item oberste Stackelemente in Registern
  \end{itemize}
 \end{minipage}
\end{tabular}

\subsection{Primäre Aufgaben der Code-Erzeugung}
\begin{itemize}
 \item Erzeuge \emph{semantische äquivalenten} Code
 \item Erzeuge effizienten Code (kurze Laufzeit), wenig Speicherbedarf
 \item Befehlsauswahl
 \item Registerzuweisung und -belegung
 \item Befehlsanordnung
\end{itemize}
Heute Algorithmen zur automatischen Code-Erzeugung auf exemplarischer RISC-Architektur
\begin{description}
\item[Idee 1] Verwende "`Code-Templates"'. Ergebnis: Schlechter Code mit hohem Optimierungspotential
    \Bsp Übersetze Befehle der Art \texttt{x = y + z} in folgende Code-Sequenz
        \begin{verbatim}
LD  R0,y
ADD R0,R0,z
ST  x,R0
        \end{verbatim}
    Intermediate Representation: \texttt{a = b + c; d = a + e} ergibt:
    \begin{verbatim}
LD  R0,b
ADD R0,R0,c
ST  a,R0     // vermeidbar, wenn der Wert in a keine spätere Verwendung hat
LD  R0,a     // redundant
ADD R0,R0,e
ST  d,R0
    \end{verbatim}
    Charakterisiere Besonderheiten mit Blich auf Befehlssatz der Zielarchitektur
    \Bsp \texttt{a = a + 1}
        \begin{verbatim}
LD  R0,a
ADD R0,R0,#1
ST  a,R0
        \end{verbatim}
        blöd, wenn \texttt{INC a} im Befehlssatz vorhanden ist!
\end{description}

\subsection{Probleme jenseits Templates}
\Bsp Es gibt Architekturen, die Gleitkommamultiplikation und -division ausschließlich auf Registerpaaren erlauben (gerade/ungerade Paare)
\begin{description}
 \item[Multiplikationsbefehl] \texttt{MUL x,y}, wobei \texttt{x} im ungeraden Teil eines Registerpaares steht und Ergebnis im gesamten Registerpaar
 \item[Divisionsbefehl] \texttt{DIV x,y}, wobei der Divident ein Registerpaar belegt. Das Ergebnis befindet sich im ungeraden Register, der Rest im geraden Register.
\end{description}
IR-Befehlssequenzen:
\renewcommand{\theenumi}{\alph{enumi}}
\renewcommand{\labelenumi}{\theenumi)}
\begin{enumerate}
 \item \begin{verbatim}
t = a + b
t = t * c
t = t / d
       \end{verbatim}
 \item \begin{verbatim}
t = a + b
t = t + c
t = t / d
       \end{verbatim}
\end{enumerate}
Optimaler Code zu a)
\begin{verbatim}
LD  R1,a
ADD R1,b
MUL R0,c
DIV R0,d
ST  R1,t
\end{verbatim}
Optimaler Code zu a)
\begin{verbatim}
LD   R0,a
ADD  R0,b
ADD  R0,c
SRDA R0,32
DIV  R0,d
ST   R1,t
\end{verbatim}

\subsection{Eine einfache RISC-Architektur zur Definintion des Codes}
Drei-Adress-Befehle, Speicher ist Byte-adressierbar, universelle Register \texttt{R0...Rn}, alle Werte sind Integer-Werte, Befehle können durch Label markiert sein.
\begin{itemize}
 \item Lode-Befehle:       \texttt{LD r,x}           wobei \texttt{r} ein Register und \texttt{x} Speicherplatz oder Register ist.
 \item Store-Befehle:      \texttt{ST x,r}           wobei \texttt{x} ein Speicherplatz ist 
 \item Operationsbefehle:  \texttt{OP dst,src1,src2} bzw. \texttt{OP dst,src} wobei \texttt{dst} Registster oder Speicherplatz ist, \texttt{src1},... Register, Speicherplatz oder Konstante \#7
 \item Unbedingte Sprünge: \texttt{BR L}             wobei \texttt{L} eine Marke ist
 \item Bedingte Sprünge:   \texttt{Bcond r,L}        wobei \texttt{cond} ein Bezeichner für typische Tests auf Registerinhalte ist (z. B. \texttt{BLTZ r,L}, \emph{L}ess \emph{T}hen \emph{Z}ero)
 \item Adressierungsarten: Symbolische Adressen, Register, Konstanten; zusätzlich: indizierte Adressen: z. B. \texttt{a(r)} wobei \texttt{a} ein Speicherplatz ist und \texttt{r} ein Register.
\end{itemize}
\begin{description}
 \item[Einfaches Kostenmaß] Jeder Befehl hat Kosten 1 plus Kosten seiner Adressen, wobei Kosten eines Register 0 ist und sonst 1.
 \item[Kosten Programm] Hinsichtlich Länge = Summe der Kosten aller Befehle
\end{description}