\chapter{Syntaxanalyse (Parser)}
\begin{description}
    \item[Eingabe] (nach lexik. Analyse): Folge von Symbolen (Terminalsymbole, Tokennamen)
    \item[Ausgabe] Parsebaum
\end{description}
\Bsp $E \to E+0\ |\ E+1\ |\ 0\ |\ 1$
\begin{description}
 \item[Eingabe] $1+0$
 \item[Ausgabe] \pstree{\Tr{$E$}}{
                    \pstree{\Tr{$E$}}{
                        \Tr{1}
                    }
                    \Tr{$+$}
                    \Tr{0}
                }
\end{description}
\Satz Kontextfreie Sprachen werden von \emph{nichtdeterministischen} Kellerautomaten erkannt.
\paragraph*{Beweisidee}
\begin{description}
 \item[Eingabe]
      \begin{center}
           \begin{pspicture}(0,0)(3.5,2.5)
                \psframe(0,2)(2.5,2.5)
                \psline(0.5,2)(0.5,2.5)
                \psline(1,2)(1,2.5)
                \psline(1.5,2)(1.5,2.5)
                \psline(2,2)(2,2.5)
                \psline{->}(0.25,1.75)(0.25,2.25)
                \uput{3pt}[-90](0.25,1.75){Lookahead}
                \psline(3,2.5)(3,0)(3.5,0)(3.5,2.5)
                \psline(3,2)(3.5,2)
                \psline(3,1.5)(3.5,1.5)
                \psline(3,1)(3.5,1)
                \psline(3,0.5)(3.5,0.5)
                \rput(3.25,0.25){$S$}
           \end{pspicture}
     \end{center}

     \begin{description}
     \item[Keller] $\mathcal{K} = \mathcal T \cup \mathcal{N}$
     \item[Startzustand] Zu analysierendes Wort $w = a_1...a_n$ auf Eingabe, Lookahead = $a_1$, Keller enthält nur $S$
     \item[Überführung] $\Delta{:}\ \mathcal T \times \mathcal K \to (\text{Aktion}, \text{Ausgabe})$; Ausgabe: Knoten $S$
         \begin{align*}
          \Delta(a, A) &= (\operatorname{push}(\bar \alpha).{}\operatorname{pop}, \operatorname{mkTree(\alpha)})
              \tag{$A \to \alpha \in P$, nichtdeterministisch aus allen $A$-Produktionen gewählt.}\\
          \Delta(a,a) &= (\operatorname{move}.{}\operatorname{pop}, \operatorname{nextNode})\\
          \Delta(a,b) &= (-, \text{error})
         \end{align*}
     \end{description}
    Erkennung mit leerer Eingabe und leerem Keller.
 \item[Ausgabe] \pstree{\Tr{$S$}}{\Tfan[linestyle=dashed]}
 \item[Laufzeit] $O(n^3)$ (inakzeptabe! + Stackoverflows sind recht einfach herstellbar)
\end{description}
$\Rightarrow$ Transformiere die Grammatik $G$, so dass lineare Laufzeit möglich ist.
\begin{itemize}
    \item Geht nicht immer, aber bei herkömmlichen höheren Programmiersprachen ist es machbar!
\end{itemize}
\paragraph*{Verfahren der Syntaxanalyse} Top-Down vs. Bottom-Up
\begin{center}
    \pstree{\Tr{$E$}}{
        \pstree{\Tr{$E$}}{
            \Tr{1}
        }\Tr{+}\Tr{0}
    }
\end{center}
\begin{center}
 \begin{tabular}{p{0.45\linewidth}|p{0.45\linewidth}}
  \textbf{Top-Down} & \textbf{Bottom-Up} \\\hline\hline
   Gut für Parser, die per Hand geschrieben werden & größere Sprachklasse, Verwendung bei \emph{Generatoren}, z. B. Yacc
 \end{tabular}
\end{center}

Gängiges Top-Down-Verfahren: Rekursiver Abstieg und Lookahead der Länge 1.

\section{Prädiktives Parsen}
\Bsp (Ausschnitt Kontrollstrukturen)
\begin{align*}
    \texttt{stmt} &\to \underline{\texttt{expr};}\ |\ \underline{\texttt{if}}\ (\underline{\texttt{expr}})\ \texttt{stmt}\ |\ \underline{\texttt{for}}\ (\texttt{optexpr}; \texttt{optexpr}; \texttt{optexpr})\ \texttt{stmt}\ |\ \underline{\text{other}} \\
     \texttt{optexpr} &\to \varepsilon\ |\ \underline{\texttt{expr}}
\end{align*}
Hilfsmethode
\begin{lstlisting}[language=Java]
void match(Terminal t) {
    if (t == loojahead) lookahead = nextTerminal;
    else report ("syntax error")
}
\end{lstlisting}
Schreibe zu jedem \emph{Nichtterminal} eine \emph{rekursive Prozedur}, die anhand des lookahead die Regel auswählt und anschließend "`die rechte Regelseite ausführt"'.

Ein Beispielparser für oben genannte Grammatik lautet:
\begin{lstlisting}[keywords={void, case, break},emphstyle=\underline,emph={expr,if,for,other}]
void stmt() {
    case expr:
        match(expr); match('j'); break;
    case if:
        match(if); match('('); match(expr); match(')'); stmt(); break;
    case for:
        match(for); match('('); optexpr(); match(';'); ... ; stmt(); break;
    case other:
        mathch(other); break;
    default:
        report("Syntax error")
}

void optexpr() {
    if (lookahead == expr) math(expr
    // Nichtterminal bedeuter $\varepsilon$-Produktion anwenden.
}
\end{lstlisting}
\begin{description}
\item[Beispieleingabe]\lstinline[keywords={void, case, break},emphstyle=\underline,emph={expr,if,for,other}]!for (;expr;expr) other!
\begin{center}
    \pstree{\Tr{\texttt{stmt}}}{
            \Tr{\underline{\texttt{for}}}
            \Tr{(}
            \Tr{\texttt{optexpr}}
            \Tr{;}
            \Tr{\texttt{optexpr}}
            \Tr{;}
            \Tr{\texttt{optexpr}}
            \Tr{)}
            \Tr{\texttt{stmt}}
        }
\end{center}

\end{description}








