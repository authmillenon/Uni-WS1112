\begin{enumerate}
 \item \emph{Syboltabellen}
 \item Semantische Analyse (kursorisch) -- Statische Überprüfungen \emph{nicht} kontextfreier Eigenschaften
 \item Zwischencode-Erzeugung (kursorisch)
\end{enumerate}

\begin{enumerate}
 \item Lexikalische Analyse liefert Token bestehend aus Tokenname und optionalen Attributen (Lexem, Index in Syboltabellen, $<$Konstante, Wert$>$, $<=>$, $<$IF$>$, ...)
           \begin{description}
            \item[Problem]  Gültigkeitsbereiche $\rnode{x_dek}{\{\textbf{int } x}; \rnode{y_dek}{\textbf{bool } y}; ... x ... \rnode{y}{y} \{\rnode{y2_dek}{\textbf{float } y}; ... \rnode{x}{x} ... \rnode{y2}{y} ...\} ... \{y ...\}\}$\\[0.2em]
 \ncangle[angleA=-90,angleB=-90,arm=0.5]{->}{x_dek}{x}
  \ncangle[angleA=-90,angleB=-90,arm=0.3]{->}{y_dek}{y}
  \ncangle[angleA=-90,angleB=-90,arm=0.3]{->}{y2_dek}{y2}
            \item[Idee] Strukturierte Symboltabelle
            \item[Erkenntnis] Lexikalische Analyse ist überfordert!
            \item[Ergebnis] Lexer liefert flache Struktur, d. h. eine Tabelle, die Gültigkeitsbereiche \emph{nicht} berücksichtigt.
           \end{description}
           Zusatzaufgabe des Parsers: Erstelle \emph{eigene} Symboltabelle für jeden Block, jedde Klasse, jede Methode, die in der lokalen Deklaration vorkommen. Organisation als Baumstruktur
           \begin{center}
            \pstree{\Tr*{
                \begin{minipage}{2cm}\centering
                 \begin{tabular}{|rl|}
                  \hline
                  $x$: & \textbf{int}\\\hline
                  $y$: & \textbf{bool}\\\hline
                 \end{tabular}
                \end{minipage}
            }}{
                \pstree{\Tr*{
                \begin{minipage}{2cm}\centering
                 \begin{tabular}{|rl|}
                  \hline
                  $y$: & \textbf{float}\\\hline
                 \end{tabular}
                \end{minipage}
                }}{
                    \Tr*{\begin{minipage}{2cm}\centering
                 \begin{tabular}{|c|}
                  \hline
                  ...\\\hline
                 \end{tabular}
                \end{minipage}}
                }
                 \Tr*{\begin{minipage}{2cm}\centering
                 \begin{tabular}{|c|}
                  \hline
                  ...\\\hline
                 \end{tabular}
                \end{minipage}}
            }
           \end{center}
\item  Statisch Überprüfung
    \begin{itemize}
     \item Typüberprüfungen, -herleitung, -konversion
     \item Überprüfung ob jede Variable deklariertt wurde
    \end{itemize}
    \emph{Methode:} Syntaxgerichtete Definition
\item  Zwischencodeerzeugung
    \begin{itemize}
     \item Bisher: Postfix-Notation, Befehle der abstrakten Stack-Maschine
     \item Typische Formen des Zwischencodes:
         \begin{enumerate}
          \item abstrakte Syntaxbäume
          \item Drei-Adress-Code (lineare Struktur, Befehlsfolge)
          Operatorbaum für 2 - (9+5)
          \begin{center}\psset{levelsep=1cm}
\pstree{\Tr{$-$}\nput{20}{\pssucc}{\color{blue}$t_1 = 2 - t_0$}}{
   \Tr{2}\nput{20}{\pssucc}{\color{blue}$2$}
   \pstree{\Tr{+}\nput{20}{\pssucc}{\color{blue}$t_0 = 9 + 5$}}{
   \Tr{9}\nput{20}{\pssucc}{\color{blue}$9$}\Tr{5}\nput{20}{\pssucc}{\color{blue}$5$}
}
}
          \end{center}
         \end{enumerate}
    \lstinline[language=Java]!while(E) S!
    \begin{center}
     \pstree{\Tr{\textbf{while}}}{
     \Ttri{E}\nput{160}{\pssucc}{\color{blue}Code für $E$ in $t$}
     \Ttri{S}\nput{20}{\pssucc}{\color{blue}Code von $S$}
    }
    \begin{minipage}{0.3\textwidth}
     \color{blue}
     \begin{tabular}{|c|l|c|}
      \hline
      \multicolumn{3}{|c|}{}\\\cline{2-2}
      M & Code für E & \\ \cline{2-2}
       & ifFalse t goto L& \\ \cline{2-2}
       & Code von S & \\ \cline{2-2}
       & goto M  &\\ \cline{2-2}
       \multicolumn{3}{|l|}{L}\\\hline
     \end{tabular}
    \end{minipage}
    \end{center}
    \end{itemize}
\item Optimierung\\
    \begin{tabular}{lcl}
     \lstset{language=Java}
     Beispiel: & \lstinline[language=Java]!if (peek = '\n') line = line + 1! & Quellprogramm \\
         & $\Downarrow$ & \\\cline{2-2}
         & \multicolumn{1}{|c|}{lexikalische Analyse} & \\\cline{2-2}
         & $\Downarrow$ & \\
         & $<\textbf{if}><(><\textbf{id},\lstinline!"peek"!><=><\textbf{char}, \lstinline!'\n'!><)>$ & \\
         & $<\textbf{id},\lstinline!"line"!><=><\textbf{id}, \lstinline!"line"!><+><\textbf{int},1><;>$ & \\
         & \rnode{start}{$\Downarrow$}
    \end{tabular}
    \vspace{10mm}
    \begin{center}
     \begin{tabular}{cc}
\begin{minipage}{0.4\textwidth}
\psset{levelsep=1cm}
\pstree{\Tr[name=var1]{\textbf{if}}}{
    \pstree{\Tr{\textbf{eq}}}{
        \Tr{peek}
        \pstree{\Tr{(\textbf{int})}}{
            \Tr{\lstinline!'\n'!}
        }
    }
    \pstree{\Tr{\textbf{assign}}}{
        \Tr{line}
        \pstree{\Tr{+}}{
            \Tr{line}
            \Tr{1}
        }
    }
}
\end{minipage}    &
     \rnode{var2}{
         \begin{minipage}{0.4\textwidth}
         \texttt{1. t1 = (int)'\textbackslash n'}\\
         \texttt{2. ifFalse (peek == t1) goto 4}\\
         \texttt{3. line = line + 1}\\
         \texttt{4. ...}
         \end{minipage}
     }
     \end{tabular}
     \ncline{->}{start}{var1}\nbput{Variante 1}
     \ncline{->}{start}{var2}\naput{Variante 2 unter Optimierung}
    \end{center}
\end{enumerate}
