\section{Traversierung "`depth-first-postorder"'}
\begin{lstlisting}
proc visit(Node N) {
	foreach (Child C of N, from left to right) {
		visit(C);
	}
	handle Node N;
}
\end{lstlisting}

\paragraph{Kritik:} Unnötiges Kopieren, Platzbedarf
\paragraph{Eigenschaft $\boldsymbol{E}$} Der Attributwert (Übersetzung) eines inneren Knotens ergibt sich aus der Konkatenation der Attributwerte aller Kinderknoten unter Beibehalt der Reihenfolge und mit möglichen Einfügungen.
\paragraph{Definition:} Ein Syntaxgerichtete Definition (SGD) heißt \emph{einfach}, wenn sie die Eigenschaft $E$ besitzt.
\paragraph{Idee:} Erzeuge Übersetzung inkrementell beim Traversieren des Baumes.
\begin{lstlisting}
print('9')
print('5')
print('2')
print('+')
print('-')
\end{lstlisting}
\subsection{Übersetungsschema}
Eine kontextfreie Grammatik, in deren rechte Produktionsseiten Programmfragmente (semantische Aktionen, hier Druckbefehle) eingestreut sind, heißt Übersetungsschema. \hfill$\square$
\paragraph{Konvention:} Programmfragmente seien in $\{\}$ eingeschlossen.

Die erzeugte Sprache bleibt unverändert!

Semantische Aktionen heißen auch \emph{Pseudoterminale}. Im Parsebaum mit gestrichelten Kanten.

\subsubsection*{Beispiel}
\begin{align*}
E &\to E + T \{\texttt{print('+')}\}\ |\ E + T \{\texttt{print('+')}\ |\ T\} \\
T &\to E * F \{\texttt{print('*')}\}\ |\ E / F \{\texttt{print('/')}\ |\ F\} \\
F &\to (E)\ |\ D\\
D &\to \{\texttt{print('0')}\}0\ |\ ...\ |\ \{\texttt{print('9')}\}9 
\end{align*}
\begin{center}
 \def\dedge{\ncline[linestyle=dashed]}
 \pstree{\Tr{$E$}}{
  \pstree{\Tr{$E$}}{
   \pstree{\Tr{$E$}}{
    \pstree{\Tr{$T$}}{
     \pstree{\Tr{$F$}}{
      \pstree{\Tr{$D$}}{
       \Tr[edge=\dedge]{\texttt{print('9')}}
       \Tr{9}
      }
     }
    }
    \Tr{$+$}
    \pstree{\Tr{$T$}}{
     \pstree{\Tr{$F$}}{
      \pstree{\Tr{$F$}}{
       \Tr[edge=\dedge]{\{\texttt{print('5')}\}}
       \Tr{5}
      }
     }
    }
    \Tr[edge=\dedge]{\{\texttt{print('+')}\}}
   }
  }
  \Tr{$-$}
  \pstree{\Tr{$T$}}{
   \pstree{\Tr{$F$}}{
    \pstree{\Tr{$D$}}{
     \Tr[edge=\dedge]{\{\texttt{print('2')}\}}
     \Tr{2}
    }
   }
  }
  \Tr[edge=\dedge]{\{\texttt{print('-')}\}}
 }
\end{center}
Die Übersetzung wird erzeugt durch Ausführung der semantischen Aktionen bei deren Besuch im Rahmen der Depth-First-Traversierung (left-to-right).

Maschinenprogramm der abstrakten Stapelmaschine

{\color{red} \texttt{print(\dq push 9\dq)...\texttt{print(\dq sub\dq)}}}

\subsection{Java Bytecode}
\begin{center}
\begin{tabular}{c|c|c|c|l}
 \textbf{Mnemonic} & \texttt{Op-Code} & Argumente & Stack vorher / nachher & Beschreibung \\\hline\hline
 ipush & \texttt{0x10} & 1 byte & $\to \texttt{value}$ & pushes a byte onto the stack as integer-value \\\hline
 iadd & \texttt{0x60} && $V_1, V_2 \to \texttt{result}$ & add two ints together \\\hline
 isub & \texttt{0x64} && $V_1, V_2 \to \texttt{result}$ & subtracts two ints \\\hline
 imul & \texttt{0x68} && $V_1, V_2 \to \texttt{result}$ & multiply two ints together \\\hline
 idiv & \texttt{0x6c} && $V_1, V_2 \to \texttt{result}$ & divide two ints \\\hline
\end{tabular}
\end{center}

\paragraph{Übersetungsschema:} Arithmetische Austrücke in Java Bytecode.
\begin{align*}
 E &\to E + T \{\texttt{print(\dq 60\dq)}\} \\
  &\vdots \\
 D &\to 0\{\texttt{print(\dq 1000\dq)}\}\\
  &\vdots\\
 D &\to 9\{\texttt{print(\dq 1009\dq)}\}\\
\end{align*}

