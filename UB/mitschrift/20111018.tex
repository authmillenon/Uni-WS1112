\chapter{Einführung}
\section{Kontakt}
\paragraph{Sprechstunde:} Di 13:30 -- 14:30 Uhr
\paragraph{E-Mail:}	\href{mailto:fehr@inf.fu-berlin.de}{fehr@inf.fu-berlin.de}

\section{Literatur}
Vorlesung basiert im wesentlichen auf dem Dragon-Book (Aho, Ullman, Sethi, Lam)

\section{Was ist Übersetzerbau?}
\begin{itemize}
 \item Übersetzung: Höhere Programmiersprache \emph{(Quellsprache)} in Maschinensprache \emph{(Zielsprache)}
 \item \emph{Aufgabe des Übersetzerbauers:} Schreibe ein Programm, deren Eingabe ein Programm der Quellsprache als Eingabe erhält und als Ausgabe ein \textbf{äquivalentes} Programm (also mit gleicher Bedeutung, gleicher Semantik) der Zielsprache liefert.
 \item natürliche Sprachen sind "`schillernd"' $\Rightarrow$ es gibt nicht immer eine genaue Übersetzung
 \item formale Sprachen haben meist eine genau spezifizierte Übersetzung der Semantik (mathematisch formalisierbar)
 \item moderne Programmiersprachen haben mitunter diese \emph{mathematische} Präzision nicht \\
		$\Rightarrow$ unterschiedliche Interpretation bei unterschiedlichen Übersetzern
\end{itemize}

\section{Geschichte}
\begin{psmatrix}[rowsep=0.5em]
 1940'er Jahre :-( & \framebox{Maschine} & \\
 1. Generation & & $\Rightarrow$ Assemblersprache & \framebox{\begin{minipage}{3cm} mnemotechnische \\ Bezeichnung \end{minipage}} :-$|$ \\
 & Maschinenbefehle &  Übersetzer (1:1) & \begin{minipage}{3cm} Makros (1:n) \\ mit Parametern \end{minipage}
\ncline{1,2}{3,2}\ncline{2,3}{3,3}\ncline{3,3}{3,4}
\end{psmatrix}\\
\hrule
\begin{psmatrix}[rowsep=0em]
 1950'er Jahre & $\underset{\text{wiss. Berechn.}}{\text{FORTRAN}}$, COBOL, LISP \\
 3. Generation \\
 & C, C++, Java, ...
\end{psmatrix}\\

\begin{psmatrix}[rowsep=0.5em]
 && Impl.-Sprache \\
 Programmierer :-) & [mnode=oval] \begin{minipage}{2cm} \centering höhere Pro"-gra"-mmier"-spra"-che \end{minipage} & \framebox{Übersetzer} & [mnode=oval] \begin{minipage}{2cm} \centering Maschinen-sprache \end{minipage}
 \ncline{1,3}{2,3}
 \ncline{->}{2,1}{2,2}\ncline{->}{2,2}{2,3}\ncline{->}{2,3}{2,4}
\end{psmatrix}

\section{Blick in die Box Übersetzer}
\begin{itemize}
 \item zwei Phasen: 
		\begin{enumerate}
		 \item Analysieren (lexikalisch, syntaktisch, Typüberprüfung, ...)\\
				\begin{psmatrix}[rowsep=0.5em]
					$\underbrace{\mathtt{173}}_{\text{Zahl}}$\verb! * (4 - 93)! \\
 					\psset{levelsep=1cm,nodesep=0.2em}
					\pstree{\Tdia{*}}{%
							\Tr*{173}
							\pstree{\Tdia{-}}{
 								\Tr*{4}
 								\Tr*{93}}}
				\end{psmatrix}
		 \item Synthetisieren:
				\begin{itemize}
				 \item Zwischencode
				 \item Optimierung
				\hrule
				 \item Codeerzeugung
				 \item Optimierung
				\end{itemize}
			\begin{psmatrix}[rowsep=0.5em]
			 [mnode=circle] $P_1$ && \framebox{$M_1$} \\
			 [mnode=circle] $P_2$ && \framebox{$M_2$} \\
			 $\vdots$ && $\vdots$ \\
			 [mnode=circle] $P_n$ && \framebox{$M_m$} \\
			 \ncline{->}{1,1}{1,3}\ncline{->}{1,1}{2,3}\ncline{->}{1,1}{4,3}
			 \ncline{->}{2,1}{1,3}\ncline{->}{2,1}{2,3}\ncline{->}{2,1}{4,3}
			 \ncline{->}{4,1}{1,3}\ncline{->}{4,1}{2,3}\ncline{->}{4,1}{4,3}
			 & $n \cdot m$ Übersetzer &
			\end{psmatrix}\hfill
			\begin{psmatrix}[rowsep=0.5em]
			 [mnode=circle] $P_1$ && \framebox{$M_1$} \\
			 $\vdots$ & \framebox{\begin{minipage}{1cm}abstr. Masch.\end{minipage}} & $\vdots$ \\
			 [mnode=circle] $P_n$ & (Zwischenspr.) & \framebox{$M_n$} \\
			 \ncline{->}{1,1}{2,2}\ncline{->}{2,2}{1,3}
			 \ncline{->}{3,1}{2,2}\ncline{->}{2,2}{3,3}
			 & $n + m$ Übersetzer &
			\end{psmatrix}
			\hrule
			\begin{center}
				Frontend (Quellsprache) $\Leftrightarrow$ Backend (Zielsprache)
			\end{center}
			\begin{description}
			 \item[Frontend] Analyse, Zwischencodesynthese- optimierung
			 \item[Backend] Codeerzeugung- und optimierung
			\end{description}
		\end{enumerate}
\end{itemize}

\begin{psmatrix}[rowsep=1em]
	[name=Quellprogramm] \verb!pos = pos + dauer * 60! & \emph{(Quellprogramm)}\\
	[name=lex_ana] \framebox{lexikalische Analyse} & \tiny\begin{minipage}{5cm}
										Symboltabelle \\
										\begin{tabular}{|r|c|}
											\multicolumn{2}{c}{\vdots}\\\hline
											14 & \verb!pos! \\\hline
											15 & \verb!dauer! \\\hline
											\multicolumn{2}{c}{\vdots}
										\end{tabular}
	                                   \end{minipage}\\
	[name=Tokenstrom] \verb!<id, 14><=><id,14><+><id,15><*><int,60>!& \emph{(Tokenstrom)}\\
	[name=syn_ana] \framebox{Syntaxanalyse} & \tiny\begin{minipage}{5cm}
								 $P \rightarrow D;A$\\
								 \begin{tabular}{|c|c|}
								  \pstree[levelsep=0.5cm, treesep=0.5cm, nodesep=0.2em]{\Tr*{$;$}}{\Tr*{$D$}\Tr*{$A$}} &
								  \pstree[levelsep=0.5cm, treesep=0.5cm, nodesep=0.2em]{\Tr*{$=$}}{\Tr*{$D$}\Tr*{$;$}\Tr*{$A$}} \\
									Syntaxbaum & Parsebaum
								 \end{tabular}
	                             \end{minipage} \\
	[name=Syntaxbaum] \psset{levelsep=1cm, nodesep=0.2em} \pstree{\Tdia{=}}{\Tr{\texttt{<id, 14>}} \pstree{\Tdia{+}}{\Tr{\texttt{<id, 14>}} \pstree{\Tdia{*}}{\Tr{\texttt{<id,15>}} \Tr{\texttt{<int,60>}}}}} & \emph{(Syntaxbaum)}\\
	[name=tmp_gen] \framebox{Zwischencodeerzeugnung} \\
	[name=Zwischencode] \begin{minipage}{5cm}
					\begin{lstlisting}[basicstyle=\tt]
t1 = inttofloat # 60
t2 = id15 * t1
t3 = id14 * t2
id14 = t3
					\end{lstlisting}
	               \end{minipage} & \emph{(Zwischencode)}\\
    [name=tmp_opt] \framebox{Codeoptimierung} \\
	[name=opt_Zwischencode] \begin{minipage}{5cm}
					\begin{lstlisting}[basicstyle=\tt]
t1 = id15 * 60
id14 = id14 + t1
					\end{lstlisting}
	               \end{minipage} & \emph{(opt. Zwischencode)}
	\psset{arrows=->}\ncline{Quellprogramm}{lex_ana}\ncline{lex_ana}{Tokenstrom}\ncline{Tokenstrom}{syn_ana}
	\ncline{syn_ana}{Syntaxbaum}\ncline{Syntaxbaum}{tmp_gen}\ncline{tmp_gen}{Zwischencode}\ncline{Zwischencode}{tmp_opt}
	\ncline{tmp_opt}{opt_Zwischencode}
\end{psmatrix}
