\chapter{Laufzeitumgebung}
\begin{itemize}
 \item Speicherverwaltung
     \begin{itemize}
     \item Der Übersetzer erzeugt Code für einen logischen Adressraum von $0$ bis $N$
     \item Grobstruktur des Speichers (logischer Adressraum)
        \begin{center}
        \begin{tabular}{r|c|l}\cline{2-2}
               0 & Code & ausführbares Maschinenprogram\\\cline{2-2}
        $\vdots$ & Static & Daten, deren Anzahl und Größe zur Compilezeit bekannt sind \\\cline{2-2}
         & Heap & Daten, die zur Laufzeit erzeugt werden \\\cline{2-2}
         & $\downarrow$ & \\
         & $\vdots$ & \\
         & $\uparrow$ & \\\cline{2-2}
        $N$ & Stack & Laufzeitkeller, für Aktivierungsrecords (frames) zu jedem Methodenaufruf \\\cline{2-2}
        \end{tabular}
        \end{center}
        \begin{center}
         statischer Speicher (zur Compilezeit) $\Leftrightarrow$ dynamischer Speicher (während der Ausführung)
        \end{center}
     \end{itemize}
\end{itemize}
\Bsp Sortierprogramm (fragmentarisch)
\begin{lstlisting}[language=C]
int a[11];
void read_a() { /* liest Werte in a[1] bis a[9] */
    int i;
    ...
}
int part(int m, int n) { /* waehlt v aus und teilt 
    a[m...n] auf, so dass a[m...p-1] <= v, a[p] - v, a[p+1...n] > v */
    ...
}
void qs(int m, int n) {
    int i;
    if (n > m) {
        i = part(m,n);
        qs(m,i-1);
        qs(i+1,n);
    }
}
void main() {
    read_a();
    a[0] = -9999;
    a[10] = 9999;
    qs(1,9);
}
\end{lstlisting}

\begin{verbatim}
Betrete main()
    Betrete read_a
    Verlasse read_a
    Betrete qs(1,9)
        Betrete part(1,9)
        Verlasse part(1,9)
        Betrete qs(1,3)
            ...
        Verlasse qs(1,3)
        Betrete qs(5,9)
            ...
        Verlasse qs(5,9)
    Verlasse qs(1,9)
Verlasse main()
\end{verbatim}
Ein \emph{Aktivierungsbaum} für unser Sortierprogramm:
\begin{center}
\pstree{\Tr{\texttt{main()}}}{
    \Tr{\texttt{read\_a()}}
    \pstree{\Tr{\texttt{qs(1,9)}}}{
        \Tr{\texttt{part(1,9)}}
        \pstree{\Tr{\texttt{qs(1,3)}}}{
            \Tr{\texttt{part(1,3)}}
            \Tr{\texttt{qs(1,0)}}
            \pstree{\Tr{\texttt{qs(2,3)}}}{
                \Tr{\texttt{part(2,3)}}
                \Tr{\texttt{qs(2,1)}}
                \Tr{\texttt{qs(3,3)}}
            }
        }
        \pstree{\Tr{\texttt{qs(5,9)}}}{
            \Tr{\texttt{part(5,9)}}
            \Tr{\texttt{qs(5,5)}}
            \pstree{\Tr{\texttt{qs(7,9)}}}{
                \Tr{\texttt{part(7,9)}}
                \Tr{\texttt{qs(7,7)}}
                \Tr{\texttt{qs(9,9)}}
            }
        }
    }
}
\end{center}
Zu jeder Programmausführung gehört genau ein Aktivierungsbaum.
\paragraph*{Struktur des Aktivierungscodes}
\begin{center}
\begin{tabular}{|c|l}\cline{1-1}
    aktuelle Parameter & Speicherbereich für Eingabewerte (ggf. Registerverwendung) \\\cline{1-1}
    Rückgabewerte & Speicherbereich für Ausgabewerte (ggf. Registerverwendung) \\\cline{1-1}
    Kontrollverweis & Zeigt auf unteres Aktivierungselement einschl. program counter \\\cline{1-1}
    Zugriffsverweis & Zugriff auf nicht-lokale Daten gemäß statischer Bindung \\\cline{1-1}
    saved machine status & Registerbelegung, ggf. program counter \\\cline{1-1}
    lokale Daten & \\\cline{1-1}
    Temporaries & Hilfsvariablen, die bei der Übersetzung erzeugt werden \\\cline{1-1}
\end{tabular}

\begin{tabular}{cc}
    \begin{minipage}{0.2\linewidth}
            \centering
            \Tr{\texttt{main()}}
        \end{minipage} &
        \begin{minipage}{0.5\linewidth}
            \begin{tabular}{|c|}\hline
                \texttt{main} \\\hline
            \end{tabular}
        \end{minipage} \\\hline
    \begin{minipage}{0.2\linewidth}
            \centering
            \pstree{\Tr{\texttt{main()}}}{
                \Tr{\texttt{read\_a()}}
            }
        \end{minipage} &
        \begin{minipage}{0.5\linewidth}
            \begin{tabular}{|c|}\hline
                \texttt{main} \\\hline
                \texttt{read\_a} \\
                \texttt{int i} \\\hline
            \end{tabular}
        \end{minipage} \\\hline
\end{tabular}


\end{center}
