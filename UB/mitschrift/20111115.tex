\psset{colsep=0.5cm,arrows=->}
\begin{psmatrix}
             & & & Analysephase \\
             & & \framebox{\begin{minipage}{3cm}
                               Fehlerbehandlung
                          \end{minipage}} && Start der Übersetzung \\
Quellprogramm & \framebox{Lexer} & \psframebox[linestyle=none,fillstyle=hlines,hatchcolor=red]{Tokenstrom} &
\framebox{Parser} & ... \\
              && \framebox{
                  \begin{minipage}{3cm}
                   Symboltabelle
                  \end{minipage}}
\end{psmatrix}\ncline{3,1}{3,2}\ncline{3,2}{3,3}\ncline{3,3}{3,4}\ncline{3,4}{3,5}
\ncline{<->}{2,3}{3,2}\ncline{<->}{2,3}{3,4}\ncline{<->}{2,3}{3,5}\ncline{<->}{4,3}{3,2}\ncline{<->}{4,3}{3,4}\ncline{<->}{4,3}{3,5}
\nccurve[angleA=-10,angleB=-170,linecolor=red]{3,2}{3,4}\nbput{\color{red} Token}\nccurve[angleA=170,angleB=10,linecolor=red]{3,4}{3,2}\nbput{\color{red} getNextToken}\ncline[doubleline=true]{2,5}{3,4}

\paragraph{Begriffserklärung}
\begin{itemize}
 \item Token
 \item Muster-Berschreibung zulässiger Lexeme
 \item Lexem
\end{itemize}
\begin{center}
 \texttt{\textvisiblespace\textvisiblespace'\textbackslash t'\textvisiblespace'\textbackslash n'} "`Leerzeichen"'
\end{center}
\begin{center}
 $\underbrace{\textvisiblespace\textvisiblespace}_{Leerraum}\underset{\overset{\uparrow}{<\textbf{num}, 34>}}{34} \overset{\underset{\downarrow}{\text{Beginnlexem}}}{+}\underbrace{\textvisiblespace\textvisiblespace}_{\textbf{id}, \texttt{'count'}}\textvisiblespace\textvisiblespace...$
\end{center}

\begin{enumerate}
 \item Folge von Leerzeichen -- Leerräume: werden entfernt
 \item Suche längsten Präfix der verbleibenden Eingabe, der auf eines der Muster, d. h. der Beschreibung von Strukturen passt
 \item Gib Token bestehend aus Tokenname (Terminalsymbol der kontextfreien Grammatik) ggf. zusammen mit Attributwert zurück, falls Suche erfolgreich. Fehlerbehandlung sonst.
\end{enumerate}
\paragraph*{Vorsicht:} Beispiel zu FORTRAN:
\begin{itemize}
 \lstset{language=fortran}
 \item \lstinline!DO 10 I = 1.25! $\rightsquigarrow$ \underline{\lstinline!DO10I!}\lstinline!=!\underline{\lstinline!1.25!} (Bezeichner, Zuweisung, Konstante)
 \item \lstinline!DO 10 I = 1,25! $\rightsquigarrow$ \lstset{language=} \underline{\lstinline!DO!}\lstinline!10!\underline{\lstinline!I!}\lstinline!=!\underline{\lstinline!1!}\lstinline!,!\underline{\lstinline!25!}
\end{itemize}
Scanner entfernt Leerräume. Unschön, in modernen Sprachen ausgemerzt
\begin{itemize}
 \item Leerzeichen als Trennzeichen
 \item Reservierung der Schlüsselwörter
\end{itemize}

\section{Fehlerbehandlung}
Nicht alle Schreibfehler können vom Lexer erkannt werden!
\paragraph*{Beispiel:} \lstinline!fi(x == f(y)) ...! vs \lstinline!if (x == f(y)) ....! \\
$<\textbf{id}, \texttt{'fi'}><(> ....$ kein lexik. Fehlerbehandlung\\
Welche Fehlersituation kann auftreten? \emph{Kein} Präfix der verbleibenden Eingabe passt auf einen der Muster.
\paragraph{Systematik zu Fehlerbehandlung}
\begin{itemize}
 \item "`Panic Recovery"' -- Entfenrne einzelne Zeichen am Anfang der verbleibenden Eingabe
 \item Vier primitive Operation werden angewendet, um korrekte Eingabe herzustellen
     \begin{itemize}
     \item Entferne ein Zeichen aus verbleibender Eingabe
     \item Füge ein Zeichen in verbleibender Eingabe ein
     \item Vertausche zwei benachbarte Zeichen in verbleibender Eingabe
     \item Ersetze ein Zeichen durch ein anderes in verbleibender Eingabe
     \end{itemize}
\end{itemize}

\paragraph{Optimale Strategie} Wende eine minimale Anzahl promitiver Operationen an, so dass verbleibebende Eingabe bis zum Ende erfolgreich lexikalisch analysiert werden kann.

Aus Effizienzgründen \emph{nicht praktikabel}!
\begin{itemize}
 \item Anwendung einer dieser Optionen ist meistens erfolgreich
\end{itemize}

\section{Eingabebehandlung mit zweigeteilten Puffer}
\begin{longtable}{|c|c|c|c|c|c|c|c|c|c||c|c|c|c|c|c|c|c|c|c|}
   &   & & & \multicolumn{4}{l}{beginLexem} & \multicolumn{4}{l}{$\downarrow$ forward} \\
 \hline
   &   & & & c & o & u & n & t & & = & & c & o & u & n & t & + & 1 & eof \\\hline
 0 & 1 & & & & & & & & $N-1$ & 0 & 1 & & & & & & & & $N-1$
\end{longtable}
$N$ ist Blocklänge, mit einem Lesebefehl wird jeweils eine Hälfte gelesen. Typisches $N = 4096$\\
\begin{itemize}
 \item Vereinfachung durch Einführung von Wächtern (Sentinel).
 \item Test auf Pufferende kann mit \emph{Lesen} des nächsten Zeichens verschmolzen werden.
\end{itemize}





