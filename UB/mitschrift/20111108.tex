\chapter{Lexikalische Analyse}
Alles kontextfrei:
\begin{align*}
 N &\to ND\ |\ D \tag{für natürliche Zahlen}\\
 D &\to 0\ |\ ...\ |\ 9
\end{align*}

\begin{align*}
 I &\to IL'\ |\ L \tag{für Bezeichner}\\
 L &\to \texttt{'a'}\ |\ ...\ |\ \texttt{'b'}\\
 L' &\to L\ |\ D\ |\ \texttt{'\_'}
\end{align*}

\paragraph{1. Überlegung:} Konkrete Ausprägung ist unerheblich für Syntaxanalyse zusätzlicher Regeln: "Ballast". Außerdem verschärfter Ballast: \texttt{\dq\textvisiblespace\textvisiblespace  count\textvisiblespace +\textvisiblespace 94'\textbackslash n'\dq}
\paragraph{Einsicht:} Erledige Aufgaben wie Erkennung von Konstanten und Bezeichnern und Eliminierung von Leerzeichen in der ersten Phase: \emph{Lexer}
\paragraph{Lexer:} Quellprogramm $\to$ lexikalische Grundbausteine (\emph{"`Token"'})
\[<\textit{Tokenname},\text{\textit{optionales Attribut} (irrelevant für Syntaxanalyse)}>\]
\emph{Tokenname:} Sorte, Terminalsymbol der Syntaxanalyse.
