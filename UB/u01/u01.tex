\documentclass[a4paper,10pt]{scrartcl}
\usepackage[utf8x]{inputenc}
\usepackage[T1]{fontenc}
\usepackage{amsmath,amsfonts,amssymb,amscd,amsthm,xspace}
\usepackage[ngerman]{babel}
\usepackage{listingsutf8}
\usepackage{color}
\usepackage{geometry}
\usepackage{graphicx}
\usepackage{multicol}
\usepackage{pst-tree}

\geometry{a4paper, left=2cm,right=2cm,top=2cm,bottom=2cm}

\title{Übersetzerbau - Übungsblatt 1}
\author{Christian Cikryt (4285814; Tutorium: Mo. 16-18 Uhr)\\Martin Lenders (4206090; Tutorium: Mi. 14-16 Uhr)}
\date{\today}

\newcommand{\changefont}[3]{\fontfamily{#1} \fontseries{#2} \fontshape{#3} \selectfont}

\renewcommand{\thesection}{Aufgabe \arabic{section}}
\renewcommand{\theenumi}{\alph{enumi})}

\definecolor{lgray}{gray}{0.95}
\definecolor{purple}{rgb}{0.498,0,0.3333}
\definecolor{identifier}{rgb}{0,0,0.1}
\definecolor{string}{rgb}{0.192,0,1}
\definecolor{comment}{rgb}{0.25,0.5,0.37}

\pagestyle{myheadings}
\oddsidemargin\oddsidemargin
\markright{Martin Lenders, Christian Cikryt}

\lstset{
	tabsize=4, 
	frame=tlrb, 
	basicstyle=\footnotesize\changefont{pcr}{m}{n},
	breaklines=true,
	numbers=left,
	emphstyle=\textit, 
	language=Python,
	keywordstyle=\color{purple}\textbf, 
	identifierstyle=\color{identifier},
	stringstyle=\color{string},
	backgroundcolor=\color{lgray},
	showstringspaces=false,
	commentstyle=\color{comment},
	extendedchars=true,
	inputencoding=utf8/latin1
}
\psset{nodesep=2pt,levelsep=2em,treesep=2em}

\begin{document}

\maketitle

\section{}
\begin{enumerate}
 \item	\begin{minipage}[t]{0.4\textwidth}
	  \begin{align*}
	    S & \to SS\star \tag{$S \to SS\star$} \\
	    SS\star & \to Sa\star \tag{$S \to a$} \\
	    Sa\star & \to SS+a\star \tag{$S \to SS+$}
	  \end{align*}
	\end{minipage}
	\begin{minipage}[t]{0.4\textwidth}
	  \begin{align*}
	    SS+a\star & \to Sa+a\star \tag{$S \to a$} \\
	    Sa+a\star & \to aa+a\star \tag{$S \to a$}
	  \end{align*}
	\end{minipage}
 \item	\begin{minipage}[t]{0.8\textwidth}
	  \centering
	  \pstree{\Tr*{$S$}}{
	    \pstree{\Tr*{$S$}}{
	      \pstree{\Tr*{$S$}}{
	      \Tr*{$a$}
	      }
	      \pstree{\Tr*{$S$}}{
	      \Tr*{$a$}
	      }
	      \Tr*{$+$}
	    }
	    \pstree{\Tr*{$S$}}{
	      \Tr*{$a$}
	    }
	    \Tr*{$\star$}
	  }
	\end{minipage}
 \item	Kurz gesagt lassen sich mit der Grammatik arithmetische Ausdrücke in Postfix-Notation mit $+$ und $*$ darstellen.
	Umschrieben: Die Grammatik erzeugt eine Sprache über dem Alphabet $\mathcal{T} = \{a,
	+, \star\}$, deren Wörter mit $aa$ beginnen und mit $+$ oder $\star$
  enden.
	Der Rest des Wortes besteht jeweils aus beliebigen Buchstaben des
  Alphabets.
	Das einzige Wort, das Element dieser Sprache ist, dass nicht dieser
Regel folgt ist $a$.
	Die Wörter müssen jeweils mit $aa$ beginnen, da $S \to a$ die einzige
	Produktion ist, die nur Terminalsymbole erzeugt ist und die beiden
	verbleibenden Produktionen immer zwei Nichtterminale am Anfang erzeugen.
	Daraus ergibt sich dann auch das gesonderte Wort $a$, wenn das
Startsymbol direkt zu $a$ umgewandelt wird. 
	Alle anderen Wörter müssen mit $+$ oder $\star$ enden, da $S \to SS+$
und $S \to SS\star$ die einzigen Produktionen mit nichtterminalen Symbolen sind
 und beide enden jeweils auf diesen Zeichen.
\end{enumerate}

\section{}
\begin{enumerate}
 \item	Sprache die aus Wörtern mit gleichvielen 0en wie 1en besteht, wobei die
0en zuerst kommen, gefolgt von den 1en. 
	$S \to 0S1$ erzeugt dabei beliebig viele $(0,1)$-Paare und die
Produktion wird dann mit $S \to 01$ beendet.
 \item	Sprache über dem Alphabet $\mathcal{T} = \{a,
	+, -\}$, deren Wörter mit $aa$ enden und mit $+$ oder $-$
  beginnen.
	Der Rest des Wortes besteht jeweils aus beliebigen Buchstaben des
  Alphabets.
	Das einzige Wort, das Element dieser Sprache ist, dass nicht dieser
Regel folgt ist $a$.
	Die Wörter müssen jeweils mit $aa$ enden, da $S \to a$ die einzige
	Produktion ist, die nur Terminalsymbole erzeugt ist und die beiden
	verbleibenden Produktionen immer zwei Nichtterminale am Ende erzeugen.
	Daraus ergibt sich dann auch das gesonderte Wort $a$, wenn das
Startsymbol direkt zu $a$ umgewandelt wird. 
	Präfixnotation arithmetischer Ausdrücke mit $+$ und $-$.
Alle anderen Wörter müssen mit $+$ oder $-$ beginnen, da $S \to +SS$
und $S \to -SS$ die einzigen Produktionen mit nichtterminalen Symbolen sind
 und beide beginnen jeweils auf diesen Zeichen.
 \item	Sprache aller gültigen Klammerausdrücke. $S \to S (S) S$ produziert
dabei jeveils einen gültigen Klammerausdruck. Ein weiteres $S \to S (S) S$ kann
in, links von oder recht von diesem Klammerausdruck einen weiteren gültigen
Klammerausdruck erzeugen. $S \to \varepsilon$ beendet die Produktion weiterer
Klammerausdrücke.
 \item	Sprache, in der alle Wörter eine gleiche Anzahl an $a$s und $b$s in
beliebiger Reihenfolge haben. Die gleiche Anzahl wird dabei durch die
Produktionen $S \to aSbS$ und $S \to bSaS$ garantiert, da diese dann aber
beliebig in die Ergebnisse wieder einsetzbar sind und die Produktionen nur mit
$S \to \varepsilon$ beendet werden, entsteht so eine beliebige Reihenfolge.
 \item	Die Sprache der regulären Ausdrücke (auf $*$, $+$ und die Klammern beschränkt).
\end{enumerate}

\section{}
c (der Baum lässt sich spiegeln), d (Teilbäume können rechts oder links angehangen werden) und e (der Baum lässt sich spiegeln).

\section{}
\begin{enumerate}
 \item	
    \begin{align*}
     E &\to EBE\ |\ I\ |\ N\\
     B &\to \mathtt{+}\ |\ \mathtt{-}\ |\ \mathtt{\times}\ |\ \mathtt{/}\\
     I &\to LI\ |\ LII\ |\ D\ |\ L \\
     N &\to D\ |\ DN \\
     L &\to \mathtt{A}\ |\ \dots\ |\ \mathtt{Z}\ |\ \mathtt{a}\ |\ \dots\ |\
\mathtt{z}\ |\ \mathtt{\_}\\
     D &\to \mathtt{0}\ |\ \dots\ |\ \mathtt{9} 
    \end{align*}
 \item	
    \begin{align*}
     E &\to EBE\ |\ UE\ |\ EU\ |\ I\ |\ N\\
     B &\to \mathtt{+}\ |\ \mathtt{-}\ |\ \mathtt{\times}\ |\ \mathtt{/}\\
     U &\to \mathtt{++}\ |\ \mathtt{--}\\
     I &\to LI\ |\ LII\ |\ D\ |\ L \\
     N &\to D\ |\ DN \\
     L &\to \mathtt{A}\ |\ \dots\ |\ \mathtt{Z}\ |\ \mathtt{a}\ |\ \dots\ |\
\mathtt{z}\ |\ \mathtt{\_}\\
     D &\to \mathtt{0}\ |\ \dots\ |\ \mathtt{9} 
    \end{align*}
\end{enumerate}

\section{}
\paragraph{Zu zeigen:}
Alle Binärzahlen, die durch folgende Grammatik erzeugt werden, sind durch 3 teilbar:

num $\to$ 11 | 1001 | num 0 | num num

\paragraph{Beweis:}
Der Beweis erfolgt durch vollständige Induktion über die Höhe Parsebaum.

\subsection*{Induktionsanfang}
Für einen Parsebaum der Höhe $n = 1$ , ergeben sich die folgenden zwei Parsebäume, die jeweils durch 3 teilbare Binärzahlen darstellen.
\begin{center}
 \pstree{\Tr{num}}{
  \Tr{1} \Tr{1}
 } \hspace*{3cm}
 \pstree{\Tr{num}}{
  \Tr{1} \Tr{0} \Tr{0} \Tr{1}
 }
\end{center}

\paragraph{Beweis:}
\begin{enumerate}
 \item $11$: entspricht 3 und ist durch 3 teilbar.
 \item $1001$: entspricht 9 und ist durch 3 teilbar.
\end{enumerate}

\subsection*{Induktionsvoraussetzung}
Es gilt für eine durch einen Parsebaum der Höhe $n$ repräsentierte Binärzahl, dass sie durch 3 teilbar ist, für ein $n \in \mathbb{N}$.
\subsection*{Induktionsbehauptung}
Dann gilt auch für eine durch einen Parsebaum der Höhe $n + 1$ repräsentierte Binärzahl, dass sie durch 3 teilbar ist.
\subsection*{Induktionsschritt}
Ein Parsebaum der Höhe $n + 1$ besteht nach obiger Grammatik entweder aus zwei Parsebäumen der Höhe $n$ oder einem Parsebaum der Höhe $n$ mit angehängter Null.
Laut Induktionsvoraussetzung ist die durch einen Parsebaum der Höhe $n$ (für ein $n \in \mathbb{N}$) repräsentierte Binärzahl durch 3 teilbar.
Nun gilt es nur noch zuzeigen, dass zwei hintereinander gehängte, durch 3 teilbare Binärzahlen ebenfalls durch 3 teilbar sind und dass das Anhängen einer Null die Teilbarkeit durch 3 erhält.

Wir beginnen mit dem Anhängen der Null. Dieses entspricht einer Mulitplikation mit 2, so dass die entstehende Zahl offensichtlich immer noch durch 3 teilbar ist.
Das Hintereinanderhängen von 2 Zahl gemäß, der Regel $num \Rightarrow num num$ ist ein wenig komplizierter. Dazu stützen wir uns auf den Satz, dass eine Binärzahl durch 3 teilbar ist, wenn ihre alternierende Quersumme es ist.
Von beiden Teilen wissen wir laut Induktionsvoraussetzung, dass deren alternierende Quersumme durch 3 teilbar ist. Die erste alternierende Quersumme sei $a * 3$ und die zweite $b * 3$. Durch das Hintereinanderhängen ergibt sich für die resultierende,
alternierende Quersumme entweder $a * 3 + b * 3 = 3 * (a + b)$ (falls die erste Binärzahl eine gerade Länge hat) oder $a * 3 - b * 3 = 3 * (a - b)$ (bei ungerader Länge). Da in jedem Fall die alternierende Quersumme durch 3 teilbar ist, haben wir bewiesen, dass
unter Rückgriff auf die Induktionsvoraussetzung die Induktionsbehauptung gilt. Aufgrund des Induktionsanfangs gilt somit für alle $n \in \mathbb{N}$ (Höhe des Parsebaums), dass die von der Grammatik erzeugten Binärzahlen durch 3 teilbar sind.
\end{document}