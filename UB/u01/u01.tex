\documentclass[a4paper,10pt]{scrartcl}
\usepackage[utf8x]{inputenc}
\usepackage[T1]{fontenc}
\usepackage{amsmath,amsfonts,amssymb,amscd,amsthm,xspace}
\usepackage[ngerman]{babel}
\usepackage{listingsutf8}
\usepackage{color}
\usepackage{geometry}
\usepackage{graphicx}
\usepackage{multicol}
\usepackage{pst-tree}

\geometry{a4paper, left=2cm,right=2cm,top=2cm,bottom=2cm}

\title{Übersetzerbau - Übungsblatt 1}
\author{Christian Cikryt (4285814), Martin Lenders (4206090)}
\date{\today}

\newcommand{\changefont}[3]{\fontfamily{#1} \fontseries{#2} \fontshape{#3} \selectfont}

\renewcommand{\thesection}{Aufgabe \arabic{section}}
\renewcommand{\theenumi}{\alph{enumi})}

\definecolor{lgray}{gray}{0.95}
\definecolor{purple}{rgb}{0.498,0,0.3333}
\definecolor{identifier}{rgb}{0,0,0.1}
\definecolor{string}{rgb}{0.192,0,1}
\definecolor{comment}{rgb}{0.25,0.5,0.37}

\pagestyle{myheadings}
\oddsidemargin\oddsidemargin
\markright{Martin Lenders, Christian Cikryt}

\lstset{
	tabsize=4, 
	frame=tlrb, 
	basicstyle=\footnotesize\changefont{pcr}{m}{n},
	breaklines=true,
	numbers=left,
	emphstyle=\textit, 
	language=Python,
	keywordstyle=\color{purple}\textbf, 
	identifierstyle=\color{identifier},
	stringstyle=\color{string},
	backgroundcolor=\color{lgray},
	showstringspaces=false,
	commentstyle=\color{comment},
	extendedchars=true,
	inputencoding=utf8/latin1
}
\psset{nodesep=2pt,levelsep=2em,treesep=2em}

\begin{document}

\maketitle

\section{}
\begin{enumerate}
 \item	\begin{minipage}[t]{0.4\textwidth}
	  \begin{align*}
	    S & \to SS\star \tag{$S \to SS\star$} \\
	    SS\star & \to Sa\star \tag{$S \to a$} \\
	    Sa\star & \to SS+a\star \tag{$S \to SS+$}
	  \end{align*}
	\end{minipage}
	\begin{minipage}[t]{0.4\textwidth}
	  \begin{align*}
	    SS+a\star & \to Sa+a\star \tag{$S \to a$} \\
	    Sa+a\star & \to aa+a\star \tag{$S \to a$}
	  \end{align*}
	\end{minipage}
 \item	\begin{minipage}[t]{0.8\textwidth}
	  \centering
	  \pstree{\Tr*{$S$}}{
	    \pstree{\Tr*{$S$}}{
	      \pstree{\Tr*{$S$}}{
	      \Tr*{$a$}
	      }
	      \pstree{\Tr*{$S$}}{
	      \Tr*{$a$}
	      }
	      \Tr*{$+$}
	    }
	    \pstree{\Tr*{$S$}}{
	      \Tr*{$a$}
	    }
	    \Tr*{$\star$}
	  }
	\end{minipage}
 \item	Die Grammatik erzeugt eine Sprache über dem Alphabet $\mathcal{T} = \{a,
	+, \star\}$, deren Wörter mit $aa$ beginnen und mit $+$ oder $\star$
  enden.
	Der Rest des Wortes besteht jeweils aus beliebigen Buchstaben des
  Alphabets.
	Das einzige Wort, das Element dieser Sprache ist, dass nicht dieser
Regel folgt ist $a$.
	Die Wörter müssen jeweils mit $aa$ beginnen, da $S \to a$ die einzige
	Produktion ist, die nur Terminalsymbole erzeugt ist und die beiden
	verbleibenden Produktionen immer zwei Nichtterminale am Anfang erzeugen.
	Daraus ergibt sich dann auch das gesonderte Wort $a$, wenn das
Startsymbol direkt zu $a$ umgewandelt wird. 
	Alle anderen Wörter müssen mit $+$ oder $\star$ enden, da $S \to SS+$
und $S \to SS\star$ die einzigen Produktionen mit nichtterminalen Symbolen sind
 und beide enden jeweils auf diesen Zeichen.
\end{enumerate}

\section{}
\begin{enumerate}
 \item	Sprache die aus Wörtern mit gleichvielen 0en wie 1en besteht, wobei die
0en zuerst kommen, gefolgt von den 1en. 
	$S \to 0S1$ erzeugt dabei beliebig viele $(0,1)$-Paare und die
Produktion wird dann mit $S \to 01$ beendet.
 \item	Sprache über dem Alphabet $\mathcal{T} = \{a,
	+, -\}$, deren Wörter mit $aa$ enden und mit $+$ oder $-$
  beginnen.
	Der Rest des Wortes besteht jeweils aus beliebigen Buchstaben des
  Alphabets.
	Das einzige Wort, das Element dieser Sprache ist, dass nicht dieser
Regel folgt ist $a$.
	Die Wörter müssen jeweils mit $aa$ enden, da $S \to a$ die einzige
	Produktion ist, die nur Terminalsymbole erzeugt ist und die beiden
	verbleibenden Produktionen immer zwei Nichtterminale am Ende erzeugen.
	Daraus ergibt sich dann auch das gesonderte Wort $a$, wenn das
Startsymbol direkt zu $a$ umgewandelt wird. 
	Alle anderen Wörter müssen mit $+$ oder $-$ beginnen, da $S \to +SS$
und $S \to -SS$ die einzigen Produktionen mit nichtterminalen Symbolen sind
 und beide beginnen jeweils auf diesen Zeichen.
 \item	Sprache aller gültigen Klammerausdrücke.
 \item	Sprache, in der alle Wörter eine gleiche Anzahl an $a$s und $b$s in
beliebiger Reihenfolge haben.
 \item	
\end{enumerate}

\section{}
b, e

\section{}
\begin{enumerate}
 \item	
    \begin{align*}
     E &\to EBE\ |\ I\ |\ N\\
     B &\to \mathtt{+}\ |\ \mathtt{-}\ |\ \mathtt{\times}\ |\ \mathtt{/}\\
     I &\to LI\ |\ LII\ |\ D\ |\ L \\
     N &\to D\ |\ DN \\
     L &\to \mathtt{A}\ |\ \dots\ |\ \mathtt{Z}\ |\ \mathtt{a}\ |\ \dots\ |\
\mathtt{z}\ |\ \mathtt{\_}\\
     D &\to \mathtt{0}\ |\ \dots\ |\ \mathtt{9} 
    \end{align*}
 \item	
    \begin{align*}
     E &\to EBE\ |\ UE\ |\ EU\ |\ I\ |\ N\\
     B &\to \mathtt{+}\ |\ \mathtt{-}\ |\ \mathtt{\times}\ |\ \mathtt{/}\\
     U &\to \mathtt{++}\ |\ \mathtt{--}\\
     I &\to LI\ |\ LII\ |\ D\ |\ L \\
     N &\to D\ |\ DN \\
     L &\to \mathtt{A}\ |\ \dots\ |\ \mathtt{Z}\ |\ \mathtt{a}\ |\ \dots\ |\
\mathtt{z}\ |\ \mathtt{\_}\\
     D &\to \mathtt{0}\ |\ \dots\ |\ \mathtt{9} 
    \end{align*}
\end{enumerate}

\section{}
\subsection*{Induktionsanfang}
Die Parsebäume für einen Produktionsschritt haben eine ungerade Anzahl an
Knoten.
\paragraph{Beweis:}
\begin{center}
 \begin{tabular}[b]{c|c|c}
  \textbf{Parsebaum} & \textbf{\#Knoten} & \textbf{mod} \\\hline\hline
  \pstree{\Tr*{num}}{
    \Tr*{1} \Tr*{1}
  } & 3 & $3 \mod 2 = 1$ \\\hline
  \pstree{\Tr*{num}}{
    \Tr*{1} \Tr*{0} \Tr*{0} \Tr*{1}
  } & 5 & $5 \mod 2 = 1$ \\\hline
  \pstree{\Tr*{num}}{
    \Tr*{num} \Tr*{0}
  } & 3 & $3 \mod 2 = 1$ \\\hline
  \pstree{\Tr*{num}}{
    \Tr*{num} \Tr*{num}
  } & 3 & $3 \mod 2 = 1$
 \end{tabular} \hspace{0.3\textwidth}$\square$
\end{center}
\subsection*{Induktionsvoraussetzung}
Die Parsebäume für $n \in \mathbb{N}$ Produktionsschritte haben eine ungerade
Anzahl an Knoten.
\subsection*{Induktionsbehauptung}
Die Parsebäume für $n+1 \in \mathbb{N}$ Produktionsschritt haben eine ungerade
Anzahl an Knoten.
\subsection*{Induktionsschritt} Erweitern wir einen Parsebaum, der nach IV eine
ungerade Anzahl an Knoten hat um einen weiteren Produktionsschritt, werden alle
num um einen weiteren Subbaum für einen Produktionsschritt erweitert. Diese
Subbäume haben nach IA eine ungerade Anzahl an Knoten, wodurch, wenn wir die
Wurzel abziehen (da diese durch num ja bereits im Baum ist) eine gerade Anzahl
an Knoten jeweils zum Baum hinzufügen. Addieren wir eine gerade Zahl mit einer
ungeraden, so ist die Summe ungerade.\hfill$\square$
 

\end{document}