\documentclass[a4paper,10pt]{scrartcl}
\usepackage[utf8x]{inputenc}
\usepackage[T1]{fontenc}
\usepackage{amsmath,amsfonts,amssymb,amscd,amsthm,xspace}
\usepackage[ngerman]{babel}
\usepackage{listingsutf8}
\usepackage{color}
\usepackage{geometry}
\usepackage{graphicx}
\usepackage{multicol}
\usepackage{pst-tree}

\geometry{a4paper, left=2cm,right=2cm,top=2cm,bottom=2cm}

\title{Übersetzerbau - Übungsblatt 9}
\author{Christian Cikryt (4285814; Tutorium: Mo. 16-18 Uhr)\\
  Robert Fels (4210496; Tutorium: Mi. 8-10 Uhr)\\
  Martin Lenders (4206090; Tutorium: Mi. 10-12 Uhr)
  }
\date{\today}

\newcommand{\changefont}[3]{\fontfamily{#1} \fontseries{#2} \fontshape{#3} \selectfont}

\renewcommand{\thesection}{Aufgabe \arabic{section}}
\renewcommand{\labelenumi}{\theenumi)}
\renewcommand{\theenumi}{\alph{enumi}}

\definecolor{lgray}{gray}{0.95}
\definecolor{purple}{rgb}{0.498,0,0.3333}
\definecolor{identifier}{rgb}{0,0,0.1}
\definecolor{string}{rgb}{0.192,0,1}
\definecolor{comment}{rgb}{0.25,0.5,0.37}

\pagestyle{myheadings}
\oddsidemargin\oddsidemargin
\markright{Christian Cikrit, Robert Fels, Martin Lenders}

\lstset{
	tabsize=4, 
	frame=tlrb, 
	basicstyle=\footnotesize\changefont{pcr}{m}{n},
	breaklines=true,
	numbers=left,
	emphstyle=\textit, 
	language=Python,
	keywordstyle=\color{purple}\textbf, 
	identifierstyle=\color{identifier},
	stringstyle=\color{string},
	backgroundcolor=\color{lgray},
	showstringspaces=false,
	commentstyle=\color{comment},
	extendedchars=true,
	inputencoding=utf8/latin1
}
\psset{nodesep=2pt,levelsep=2em,treesep=2em}

\makeatletter
\newcommand{\Rm}[1]{\textrm{\expandafter\@slowromancap\romannumeral #1@}}
\makeatother
\newcommand{\print}[1]{\{\texttt{print(\dq #1\dq})\}}
\newcommand{\add}[1]{\{\texttt{s += #1})\}}
\newcommand{\FIRST}{\operatorname{FIRST}}
\newcommand{\FOLLOW}{\operatorname{FOLLOW}}

\begin{document}

\maketitle

\section{}
\begin{enumerate}
\item   Wir konstruieren zunächst eine geeignete Grammatik (mit $d$ als beliebige Ziffer)
        \begin{align*}
            E &\to E + T\ |\ T &    T &\to T * F\ |\ F \\
            F &\to (E)         &    F &\to d
        \end{align*}
        Die syntaxgerichtete Definition ist dann:
        \[
            \begin{array}{l|l}
                \textbf{Produktion} & \textbf{semantische Regel}                \\\hline\hline
                E \to E_1 + T       & E.v = E_1.v + T.v                         \\
                E \to T             & E.v = T.v                                 \\
                T \to T_1 * F       & T.v = T_1.v \cdot F.v                     \\
                T \to F             & T.v = F.v                                 \\
                F \to (E)           & F.v = E.v                                 \\
                F \to d             & F.v = d.\textit{lexval}
            \end{array}
        \]
\item   Die syntaxgerichtete Definition enthält nur synthetisierte Attribute und ist somit eine $S$-Attributierung
\item   \hfill
        \begin{center}
            \psset{treesep=2.5cm,levelsep=1cm}
            \pstree[linestyle=dotted,linecolor=gray]{\Tr{$E$}\nput{0}{\pssucc}{\rnode{E_1_v}{$v = 38$}}\nput{175}{\pssucc}{$N_{11}$}}{
                \pstree[linestyle=dotted,linecolor=gray]{\Tr{$E$}\nput{0}{\pssucc}{\rnode{E_2_v}{$v = 3$}}\nput{175}{\pssucc}{$N_4$}}{
                    \pstree[linestyle=dotted,linecolor=gray]{\Tr{$T$}\nput{0}{\pssucc}{\rnode{T_2_v}{$v = 3$}}\nput{175}{\pssucc}{$N_3$}}{
                        \pstree[linestyle=dotted,linecolor=gray]{\Tr{$F$}\nput{0}{\pssucc}{\rnode{F_2_v}{$v = 3$}}\nput{175}{\pssucc}{$N_2$}}{
                            \Tr{$d$}\nput{0}{\pssucc}{\rnode{d_2_lexval}{$\textit{lexval} = 3$}\nput{175}{\pssucc}{$N_1$}}
                        }
                    }
                }
                \Tr{$+$}
                \pstree[linestyle=dotted]{\Tr{$T$}\nput{0}{\pssucc}{\rnode{T_1_v}{$v = 35$}}\nput{175}{\pssucc}{$N_{10}$}}{
                    \pstree[linestyle=dotted,linecolor=gray]{\Tr{$T$}\nput{0}{\pssucc}{\rnode{T_3_v}{$v = 5$}}\nput{175}{\pssucc}{$N_7$}}{
                        \pstree[linestyle=dotted,linecolor=gray]{\Tr{$F$}\nput{0}{\pssucc}{\rnode{F_3_v}{$v = 5$}}\nput{175}{\pssucc}{$N_6$}}{
                            \Tr{$d$}\nput{0}{\pssucc}{\rnode{d_3_lexval}{$\textit{lexval} = 5$}\nput{175}{\pssucc}{$N_5$}}
                        }
                    }
                    \Tr{$*$}
                    \pstree[linestyle=dotted,linecolor=gray]{\Tr{$F$}\nput{0}{\pssucc}{\rnode{F_1_v}{$v = 7$}}\nput{175}{\pssucc}{$N_9$}}{
                        \Tr{$d$}\nput{0}{\pssucc}{\rnode{d_1_lexval}{$\textit{lexval} = 7$}}\nput{175}{\pssucc}{$N_8$}
                    }
                }
            }
            \psset{arrows=->}
            \ncline{d_1_lexval}{F_1_v}
            \ncline{d_2_lexval}{F_2_v}
            \ncline{d_3_lexval}{F_3_v}
            \ncline{E_2_v}{E_1_v}
            \ncline{F_1_v}{T_1_v}
            \ncline{F_2_v}{T_2_v}
            \ncline{F_3_v}{T_3_v}
            \ncline{T_3_v}{T_1_v}
            \ncline{T_2_v}{E_2_v}
            \ncline{T_1_v}{E_1_v}
        \end{center}
        Topologische Sortierung: $N_1N_2N_3N_4N_5N_6N_7N_8N_9N_{10}N_{11}$
\end{enumerate}

\section{}
\begin{enumerate}
\item   
\item   
\item   
\item   
\end{enumerate}

\section{}


\end{document}