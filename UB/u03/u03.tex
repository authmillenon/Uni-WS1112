\documentclass[a4paper,10pt]{scrartcl}
\usepackage[utf8x]{inputenc}
\usepackage[T1]{fontenc}
\usepackage{amsmath,amsfonts,amssymb,amscd,amsthm,xspace}
\usepackage[ngerman]{babel}
\usepackage{listingsutf8}
\usepackage{color}
\usepackage{geometry}
\usepackage{graphicx}
\usepackage{multicol}
\usepackage{pst-tree}

\geometry{a4paper, left=2cm,right=2cm,top=2cm,bottom=2cm}

\title{Übersetzerbau - Übungsblatt 2}
\author{Christian Cikryt (4285814; Tutorium: Mo. 16-18 Uhr)\\
  Robert Fels (4210496; Tutorium: Mi. 10-12 Uhr)\\
  Martin Lenders (4206090; Tutorium: Mi. 10-12 Uhr)
  }
\date{\today}

\newcommand{\changefont}[3]{\fontfamily{#1} \fontseries{#2} \fontshape{#3} \selectfont}

\renewcommand{\thesection}{Aufgabe \arabic{section}}
\renewcommand{\labelenumi}{\theenumi)}
\renewcommand{\theenumi}{\alph{enumi}}

\definecolor{lgray}{gray}{0.95}
\definecolor{purple}{rgb}{0.498,0,0.3333}
\definecolor{identifier}{rgb}{0,0,0.1}
\definecolor{string}{rgb}{0.192,0,1}
\definecolor{comment}{rgb}{0.25,0.5,0.37}

\pagestyle{myheadings}
\oddsidemargin\oddsidemargin
\markright{Martin Lenders, Christian Cikryt}

\lstset{
	tabsize=4, 
	frame=tlrb, 
	basicstyle=\footnotesize\changefont{pcr}{m}{n},
	breaklines=true,
	numbers=left,
	emphstyle=\textit, 
	language=Python,
	keywordstyle=\color{purple}\textbf, 
	identifierstyle=\color{identifier},
	stringstyle=\color{string},
	backgroundcolor=\color{lgray},
	showstringspaces=false,
	commentstyle=\color{comment},
	extendedchars=true,
	inputencoding=utf8/latin1
}
\psset{nodesep=2pt,levelsep=2em,treesep=2em}

\makeatletter
\newcommand{\Rm}[1]{\textrm{\expandafter\@slowromancap\romannumeral #1@}}
\makeatother

\begin{document}

\maketitle

\section{}
\begin{enumerate}
 \item  
 \item  
 \item	
\end{enumerate}

\section{}
\begin{enumerate}
 \item  
 \item	
 \item	
\end{enumerate}

\section{}


\section{}
\begin{enumerate}
 \item  Startsymbol der Grammatik sei $\mu$
        \begin{align*}
         \mu &\to \Rm{1000}\ |\ \Rm{1000}\mu\ |\ \Rm{900}\ |\ \Rm{900}\kappa\ |\ \delta && \\
         \delta &\to \Rm{500}\ |\ \Rm{500}\theta\ |\ \Rm{400}\ |\ \Rm{400}\kappa\ |\ \theta &
            \kappa &\to \Rm{90}\chi\ |\ \Rm{90}\ |\ \lambda \\
        \theta &\to \gamma\ |\ \gamma\lambda\ |\ \Rm{90}\ |\ \Rm{90}\chi\ |\ \gamma \Rm{90}\ |\ \gamma \Rm{90}\chi\ |\ \lambda &
            \gamma &\to \Rm{100}\ |\ \Rm{200}\ |\ \Rm{300} \\
        \lambda &\to \Rm{50}\psi\ |\ \Rm{50}\ |\ \Rm{40}\chi\ |\ \Rm{40}\ |\ \psi &
            \chi &\to \Rm{9}\ |\ \upsilon \\
        \psi &\to \xi\ |\ \xi\upsilon\ |\ \xi \Rm{9}\ |\ \Rm{9}\ |\ \upsilon &
            \xi &\to \Rm{10}\ |\ \Rm{20}\ |\ \Rm{30} \\
        \upsilon &\to \Rm{5}\ |\ \Rm{5}\iota\ |\ \Rm{4}\ |\ \iota &
            \iota &\to \Rm{1}\ |\ \Rm{2}\ |\ \Rm{3}
        \end{align*}
 \item  
\end{enumerate}

\section{}
\begin{enumerate}
 \item  
 \item  
\end{enumerate}
\end{document}