\documentclass[a4paper,10pt]{scrartcl}
\usepackage[utf8x]{inputenc}
\usepackage[T1]{fontenc}
\usepackage{amsmath,amsfonts,amssymb,amscd,amsthm,xspace}
\usepackage[ngerman]{babel}
\usepackage{listingsutf8}
\usepackage{color}
\usepackage{geometry}
\usepackage{graphicx}
\usepackage{multicol}
\usepackage{pstricks,pstricks-add,pst-node}
\usepackage{lscape}
\usepackage{textcomp}
\usepackage{arydshln}
\usepackage{multicol}

\geometry{a4paper, left=2cm,right=2cm,top=2cm,bottom=2cm}

\title{Übersetzerbau - Übungsblatt 14}
\author{Christian Cikryt (4285814; Tutorium: Mo. 16-18 Uhr)\\
  Robert Fels (4210496; Tutorium: Mi. 8-10 Uhr)\\
  Martin Lenders (4206090; Tutorium: Mi. 10-12 Uhr)
  }
\date{\today}

\newcommand{\changefont}[3]{\fontfamily{#1} \fontseries{#2} \fontshape{#3} \selectfont}

\renewcommand{\thesection}{Aufgabe \arabic{section}}
\renewcommand{\labelenumi}{\theenumi)}
\renewcommand{\theenumi}{\alph{enumi}}

\definecolor{lgray}{gray}{0.95}
\definecolor{purple}{rgb}{0.498,0,0.3333}
\definecolor{identifier}{rgb}{0,0,0.1}
\definecolor{string}{rgb}{0.192,0,1}
\definecolor{comment}{rgb}{0.25,0.5,0.37}

\pagestyle{myheadings}
\oddsidemargin\oddsidemargin
\markright{Christian Cikrit, Robert Fels, Martin Lenders}

\lstset{
	tabsize=4, 
	basicstyle=\changefont{pcr}{m}{n},
	breaklines=true,
	emphstyle=\textit, 
	keywordstyle=\color{purple}\textbf, 
	identifierstyle=\color{identifier},
	stringstyle=\color{string},
	backgroundcolor=\color{lgray},
	showstringspaces=false,
	commentstyle=\color{comment},
	extendedchars=true,
	keywords={LD,MUL,ADD,ST,SUB,BLTZ,BR},
	inputencoding=utf8/latin1
}

\begin{document}

\maketitle

\section{}
\begin{enumerate}
\item   \begin{lstlisting}
LD  R0,b
LD  R1,c
MUL R0,R0,R1
LD  R1,a
ADD R1,R1,R0
ST  y,R1
\end{lstlisting}

\item   \begin{lstlisting}
LD  R0,i
LD  R1,a(R0)
LD  R2,j
LD  R3,a(R2)
ST  a(R0),R3
ST  a(R2),R1
\end{lstlisting}

\item   \begin{lstlisting}
LD  R0,i
LD  R1,a(R0)
LD  R2,b(R0)
LD  R3,a(R2)
ST  a(R0),R3
ST  a(R2),R1
\end{lstlisting}

\item   \begin{lstlisting}
LD  R0,q
LD  R1,*R0
ADD R0,R0,#4
ST  q,R0
LD  R0,p
ST  *R0,R1
ADD R0,R0,#4
ST  p,R0
\end{lstlisting}

\item   \begin{lstlisting}[emph={L1,L2}]
    SUB  R0,x,y
    BLTZ R0,L1
    LD   R0,#0
    ST   z,R0
    BR   L2
L1: LD   R0,#1
    ST   z,R0
\end{lstlisting}
\end{enumerate}

\section{}
\begin{enumerate}
\item   \begin{itemize}
        \item   \texttt{LD R0, y}: 2 = 1+0+1
        \item   \texttt{LD R1, z}: 2 = 1+0+1
        \item   \texttt{ADD R0,R0,R1}: 1 = 1+0+0+0
        \item   \texttt{ST x,R0}: 2 = 1+1+0
        \item   $\Rightarrow$ Gesamtkosten: 7
        \end{itemize}
\item   \begin{itemize}
        \item   \texttt{LD R0, i}: 2 = 1+0+1
        \item   \texttt{MUL R0, R0, \#8}: 2 = 1+0+0+1
        \item   \texttt{LD R1,a(R0)}: 2 = 1+0+1+(0)
        \item   \texttt{ST b,R1}: 2 = 1+1+0
        \item   $\Rightarrow$ Gesamtkosten: 8
        \end{itemize}
\item   \begin{itemize}
        \item   \texttt{LD R0, x}: 2 = 1+0+1
        \item   \texttt{LD R1, y}: 2 = 1+0+1
        \item   \texttt{SUB R0, R0, R1}: 1 = 1+0+0+0
        \item   \texttt{BLTZ *R3,R0}: 2 = 1+0+0
        \item   $\Rightarrow$ Gesamtkosten: 7
        \end{itemize}
\end{enumerate}

\section{}
\begin{itemize}
 \item Eingabe:
\begin{center}
\begin{tabular}{c|c|c}
 Symbol     & Lebendigkeit & nächste Verwendung \\\hline
 \texttt{x} & l            &                    \\
 \texttt{y} & l            &                    \\
 \texttt{z} & l            &                    \\
\end{tabular}
\end{center}
 \item $i = 16$:
\begin{center}
\begin{tabular}{c|c|c}
 Symbol     & Lebendigkeit & nächste Verwendung \\\hline
 \texttt{x} & nl           &                    \\
 \texttt{y} & l            & 16                 \\
 \texttt{z} & l            & 16                 \\
\end{tabular}
\end{center}
 \item $i = 15$:
\begin{center}
\begin{tabular}{c|c|c}
 Symbol     & Lebendigkeit & nächste Verwendung \\\hline
 \texttt{x} & nl           &                    \\
 \texttt{y} & l            & 15                 \\
 \texttt{z} & nl           & 16                 \\
\end{tabular}
\end{center}
 \item $i = 14$:
\begin{center}
\begin{tabular}{c|c|c}
 Symbol     & Lebendigkeit & nächste Verwendung \\\hline
 \texttt{x} & nl           &                    \\
 \texttt{y} & l            & 14                 \\
 \texttt{z} & nl           & 16                 \\
\end{tabular}
\end{center}
 \item $i = 13$:
\begin{center}
\begin{tabular}{c|c|c}
 Symbol     & Lebendigkeit & nächste Verwendung \\\hline
 \texttt{x} & nl           &                    \\
 \texttt{y} & l            & 13                 \\
 \texttt{z} & nl           & 16                 \\
\end{tabular}
\end{center}
\end{itemize}

\section{}
\lstset{keywords={goto,ifFalse,stop}}
\begin{enumerate}
\item   \hspace{0em}\lstinputlisting[numbers=left]{3addr4.txt}
\item   Anführer: Befehle in Zeile 1, 3 und 9
		\begin{center}
			\begin{psmatrix}[rowsep=0.5cm]
				[name=Entry] \fbox{Entry} \\
				[name=B1] { $B_1$\hspace{1em}\begin{minipage}[t]{6cm}%
						\lstinputlisting[linerange={1-2},frame=tblr]{3addr4.txt}%
					\end{minipage}}\\
				[name=B2] { $B_2$\hspace{1em}\begin{minipage}[t]{6cm}%
						\lstinputlisting[linerange={3-9},frame=tblr]{3addr4.txt}%
					\end{minipage}} \\
				[name=B3] { $B_3$\hspace{1em}\begin{minipage}[t]{6cm}%
						\lstinputlisting[linerange={10},frame=tblr]{3addr4.txt}%
					\end{minipage}} \\
				[name=Exit] \fbox{Exit}
			\end{psmatrix}
			\ncline{->}{Entry}{B1}
			\ncline{->}{B1}{B2}
			\nccurve[angleA=-45,angleB=0,arm=2,offsetB=5px]{->}{B2}{B2}
			\ncline{->}{B2}{B3}
			\ncline{->}{B3}{Exit}
		\end{center}
\item   $L = \{B_2\}$ ist eine Schleife
\end{enumerate}

\section{}
\renewcommand{\labelenumi}{\theenumi.}
\renewcommand{\theenumi}{\arabic{enumi}}
\renewcommand{\theenumii}{\arabic{enumi}.\arabic{enumii}}
\begin{enumerate}
\item   Beginne mit \emph{letzten} Befehl in $B$
\item   Übernehme alle Informationen zu den vorhandenen Variablen aus der Symboltabelle
\item   Wenn Befehl $i$ der Form \texttt{i:\ x = y op z}:
	\begin{enumerate}
	\item   Trage in der Symboltabelle zu \texttt{x} "`nicht lebendig (n. l.)"' ein.
	\item   Trage in der Symboltabelle zu \texttt{y} und \texttt{z} "`lebendig (n. l.)"' und "`nächste Verwendung in $i$ (nV\_$i$)"' ein.
	\end{enumerate}
\item   Wenn Befehl $i$ der Form \texttt{i:\ a[j] = b}:
	\begin{enumerate}
	\item   Trage in der Symboltabelle zu \texttt{a}, \texttt{b} und \texttt{j} "`lebendig (n. l.)"' und "`nächste Verwendung in $i$ (nV\_$i$)"' ein.
	\end{enumerate}
\item   Wenn Befehl $i$ der Form \texttt{i:\ a = b[j]}:
	\begin{enumerate}
	\item   Trage in der Symboltabelle zu \texttt{a} "`nicht lebendig (n. l.)"' ein.
	\item   Trage in der Symboltabelle zu \texttt{b} und \texttt{j} "`lebendig (n. l.)"' und "`nächste Verwendung in $i$ (nV\_$i$)"' ein.
	\end{enumerate}
\item   Wenn Befehl $i$ der Form \texttt{i:\ a = *b}:
	\begin{enumerate}
	\item   Trage in der Symboltabelle zu \texttt{a} "`nicht lebendig (n. l.)"' ein.
	\item   Trage in der Symboltabelle zu \texttt{b} "`lebendig (n. l.)"' und "`nächste Verwendung in $i$ (nV\_$i$)"' ein.
	\end{enumerate}
\item   Wenn Befehl $i$ der Form \texttt{i:\ *a = b}:
	\begin{enumerate}
	\item   Trage in der Symboltabelle zu \texttt{a} und \texttt{b} "`lebendig (n. l.)"' und "`nächste Verwendung in $i$ (nV\_$i$)"' ein.
	\end{enumerate}
\item   Fahre fort mit 3. für den nächsten Befehl oberhalb von $i$, falls $i$ nicht erster Befehl von $B$
\end{enumerate}


\end{document}
