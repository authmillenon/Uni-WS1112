\documentclass[a4paper,10pt]{scrartcl}
\usepackage[utf8x]{inputenc}
\usepackage[T1]{fontenc}
\usepackage{amsmath,amsfonts,amssymb,amscd,amsthm,xspace}
\usepackage[ngerman]{babel}
\usepackage{listingsutf8}
\usepackage{color}
\usepackage{geometry}
\usepackage{graphicx}
\usepackage{multicol}
\usepackage{pst-tree}

\geometry{a4paper, left=2cm,right=2cm,top=2cm,bottom=2cm}

\title{Übersetzerbau - Übungsblatt 4}
\author{Christian Cikryt (4285814; Tutorium: Mo. 16-18 Uhr)\\
  Robert Fels (4210496; Tutorium: Mi. 8-10 Uhr)\\
  Martin Lenders (4206090; Tutorium: Mi. 10-12 Uhr)
  }
\date{\today}

\newcommand{\changefont}[3]{\fontfamily{#1} \fontseries{#2} \fontshape{#3} \selectfont}

\renewcommand{\thesection}{Aufgabe \arabic{section}}
\renewcommand{\labelenumi}{\theenumi)}
\renewcommand{\theenumi}{\alph{enumi}}

\definecolor{lgray}{gray}{0.95}
\definecolor{purple}{rgb}{0.498,0,0.3333}
\definecolor{identifier}{rgb}{0,0,0.1}
\definecolor{string}{rgb}{0.192,0,1}
\definecolor{comment}{rgb}{0.25,0.5,0.37}

\pagestyle{myheadings}
\oddsidemargin\oddsidemargin
\markright{Martin Lenders, Christian Cikryt}

\lstset{
	tabsize=4, 
	frame=tlrb, 
	basicstyle=\footnotesize\changefont{pcr}{m}{n},
	breaklines=true,
	numbers=left,
	emphstyle=\textit, 
	language=Python,
	keywordstyle=\color{purple}\textbf, 
	identifierstyle=\color{identifier},
	stringstyle=\color{string},
	backgroundcolor=\color{lgray},
	showstringspaces=false,
	commentstyle=\color{comment},
	extendedchars=true,
	inputencoding=utf8/latin1
}
\psset{nodesep=2pt,levelsep=2em,treesep=2em}

\makeatletter
\newcommand{\Rm}[1]{\textrm{\expandafter\@slowromancap\romannumeral #1@}}
\makeatother
\newcommand{\print}[1]{\{\texttt{print(\dq #1\dq})\}}
\newcommand{\add}[1]{\{\texttt{s += #1})\}}
\newcommand{\FIRST}{\operatorname{FIRST}}
\newcommand{\FOLLOW}{\operatorname{FOLLOW}}

\begin{document}

\maketitle

\section{}
$B \to Se$ ist eine indirekte Rekursion, alle anderen Regeln bleiben erhalten.
\begin{align*}
 B &\to Se \\
 B &\to Aae\ |\ Bbe\ |\ ce \\
 B &\to Bdae\ |\ dae\ |\ Bbe\ |\ ce
\end{align*}
Wir haben nun eine direkte Linksrekusion erzeugt, die wir nun, durch hinzufügen eines neuen Nichtterminals, auflösen:
\begin{align*}
 S &\to Aa\ |\ Bb\ |\ c \\
 A &\to Bd\ |\ d \\
 B &\to ceR\ |\ daeR \\
 R &\to daeR\ |\ beR\ |\ \varepsilon
\end{align*}

\section{}
\begin{align*}
 S &\to aS' \\
 S' &\to aS''\ |\ Bc \\
 S'' &\to S\ |\ Ab \\
 A &\to aAb\ |\ c \\
 B &\to aBa\ |\ c
\end{align*}

\section{}
Beseitigung der Linksrekursion:
\begin{align*}
 B &\to TB' \\
 B' &\to \underline{\textrm{or}} TB'\ |\ \varepsilon \\
 T &\to FT' \\
 T' &\to \underline{\textrm{and}} FT'\ |\ \varepsilon
 F &\to \underline{\textrm{not}} F\ |\ (B)\ |\ \underline{\textrm{true}}\ |\ \underline{\textrm{false}}
\end{align*}

Bestimmung der FIRST- und FOLLOW-Mengen + Erstellung der Parsetabelle:
\begin{align*}
 \FIRST(B) = \FIRST(T) = \FIRST(F) &= \{\underline{\textrm{not}}, \underline{\textrm{true}}, \underline{\textrm{false}}\} \\
 \FIRST(B') &= \{\underline{\textrm{or}}, \varepsilon\} \\
 \FIRST(T') &= \{\underline{\textrm{and}}, \varepsilon\} \\\hline
 \FOLLOW(B) &= \FOLLOW(B') = \{\$, )\}\\
 \FOLLOW(T) &= \FOLLOW(T') = \{\$, ), \underline{\textrm{or}}\}\\
 \FOLLOW(F) &= \{\$, ), \underline{\textrm{or}}, \underline{\textrm{and}}\}
\end{align*}
    \renewcommand{\not}{\underline{\textrm{not}}}
    \newcommand{\true}{\underline{\textrm{true}}}
    \newcommand{\false}{\underline{\textrm{false}}}
    \newcommand{\e}{\varepsilon}
\begin{center}
    $\begin{array}{r|c|c|c|c|c|c|c|c|}
           & \$        & \underline{\textrm{or}}           & \underline{\textrm{and}}            & \not            & (         & )         & \true       & \false     \\\hline
        B  &           &               &                 & B \to TB'       & B \to TB' &           & B \to TB'   & B \to TB'  \\\hline
        B' & B' \to \e & B \to \underline{\textrm{or}} TB' &                 &                 &           & B' \to \e &             &            \\\hline
        T  &           &               &                 & T \to FT'       & T \to FT' &           & T \to FT'   & T \to FT'  \\\hline
        T' & T \to \e  &               & T' \to \underline{\textrm{and}} FT' &                 &           & T' \to \e &             &            \\\hline
        F  &           &               &                 & F \to \underline{\textrm{and}} FT'  & F \to (B) &           & F \to \true & F \to \false \\\hline 
    \end{array}$
\end{center}



\section{}

\section{}
\begin{enumerate}
 \item 
 \item   
\end{enumerate}
\end{document}
